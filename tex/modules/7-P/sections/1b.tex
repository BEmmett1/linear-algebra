
\begin{applicationActivities}

\begin{definition}
  In chemistry, we learn that when the two substances
  \begin{itemize}
    \item Hydrochloric acid \(HCl\) (formed from 1 \(H\) and 1 \(Cl\) atom)
    \item Sodium \(Na\) (formed from 1 \(Na\) atom)
  \end{itemize}
  react, their atoms rearrange to form the substances
  \begin{itemize}
    \item Salt \(NaCl\) (formed from 1 \(Na\) and 1 \(Cl\) atom)
    \item Hydrogen gas \(H_2\) (formed from 2 \(H\) atoms).
  \end{itemize}

  This may be represented by the \term{chemical equation}
  \[
    \unknown HCl + \unknown Na
      \rightarrow
    \unknown NaCl + \unknown H_2
  \]
  where each \(\unknown\) represents the amount of that substance before/after
  the reaction.
\end{definition}

\begin{activity}{5}
  The \term{law of conservation of mass} states that the quantity of
  atoms before and after a chemical reaction must remain the same.

  \vspace{1em}

  Find positive integers so that both sides of the chemical equation represent
  the same amount of matter:
  \[
    \unknown HCl + \unknown Na
      \rightarrow
    \unknown NaCl + \unknown H_2
  \]
\end{activity}

\begin{definition}
  A chemical equation is \term{balanced} if the given quantities of each
  substance before and after the reaction are equal and minimal positive
  integers:
  \[
    2 HCl + 2 Na
      \rightarrow
    2 NaCl + H_2
  \]
\end{definition}

\begin{observation}
  For example, the following equation isn't balanced because all the integers
  may be divided by three:
  \[
    6 HCl + 6 Na
      \rightarrow
    6 NaCl + 3 H_2
  \]
  Therefore if a chemical equation can be balanced, there is exactly one
  correct solution.
\end{observation}

\begin{activity}{15}
  Balance the following chemical equations:
  \[
    \unknown Fe + \unknown Cl_2
      \rightarrow
    \unknown FeCl_3
  \]
  \[
    \unknown Ca(OH)_2 + \unknown H_3PO_4
      \rightarrow
    \unknown Ca_3(PO_4)_2 + \unknown H_2O
  \]
  \[
    \unknown K_4Fe(CN)_6 + \unknown H_2SO_4 + \unknown H_2O
      \rightarrow
    \unknown K_2SO_4 + \unknown FeSO_4 + \unknown (NH_4)_2SO_4 + \unknown CO
  \]

  (Note that \((NH_4)_2SO_4\) represents 2 \(N\), 8 \(H\), 1 \(S\),
  and 4 \(O\).)
\end{activity}

\begin{observation}
  For the purposes of balancing chemical equations, the set
  \[
    L
      =
    \setBuilder{\vec{A}}{\vec{A}\text{ is combination of elements}}
  \]
  may be treated as a kind of \textbf{vector space}.
  This means that balancing the chemical equation
  \[
    \unknown Fe + \unknown Cl_2
      \rightarrow
    \unknown FeCl_3
  \]
  may be acheived by finding a solution \(\begin{bmatrix}x\\y\\z\end{bmatrix}\)
  to the vector equation
  \[
    x\vec{Fe} + y(2\vec{Cl})
      =
    z(\vec{Fe}+3\vec{Cl})
  .\]
\end{observation}

\begin{activity}{5}
  To solve the vector equation
  \[
    x\vec{Fe} + y(2\vec{Cl})
      =
    z(\vec{Fe}+3\vec{Cl})
  \]
  we are only concerned with the subspace
  \(W=\vspan\setList{\vec{Cl},\vec{Fe}}\)
  of \(L\). Since the element \(\vec{Fe}\) cannot be created from the element
  \(\vec{Cl}\) in a chemical reaction and vice versa, the set
  \(\setList{\vec{Cl},\vec{Fe}}\):
  \begin{enumerate}[a)]
    \item spans \(W\), but is linearly dependent.
    \item is linearly independent, but does not span \(W\).
    \item is a basis for \(W\).
  \end{enumerate}
\end{activity}

\begin{observation}
  \(W=\vspan\setList{\vec{Cl},\vec{Fe}}\) is a two-dimensional subspace of
  \(L\), so as usual we'd rather work with its isomorphic Euclidean space
  \(\IR^2\).

  \vspace{1em}

  Thus we should assign a transformation of bases such as:
  \[
    \vec{Cl}
      \leftrightarrow
    \vec e_1
      =
    \begin{bmatrix}
      1\\0
    \end{bmatrix}
    \hspace{3em}
    \vec{Fe}
      \leftrightarrow
    \vec e_2
      =
    \begin{bmatrix}
      0\\1
    \end{bmatrix}
  \]
\end{observation}

\begin{activity}{10}
  Rewrite the \(W=\vspan\setList{\vec{Cl},\vec{Fe}}\) vector equation
  \[
    x\vec{Fe} + y(2\vec{Cl})
      =
    z(\vec{Fe}+3\vec{Cl})
  \]
  using the transformation of bases
  \[
    \vec{Cl}
      \leftrightarrow
    \vec e_1
      =
    \begin{bmatrix}
      1\\0
    \end{bmatrix}
    \hspace{3em}
    \vec{Fe}
      \leftrightarrow
    \vec e_2
      =
    \begin{bmatrix}
      0\\1
    \end{bmatrix}
  \]
  and show how it may be simplifed to
  \[
    x\begin{bmatrix}0\\1\end{bmatrix}
      +
    y\begin{bmatrix}2\\0\end{bmatrix}
      -
    z\begin{bmatrix}3\\1\end{bmatrix}
      =
    \begin{bmatrix}0\\0\end{bmatrix}
  .\]
\end{activity}

\begin{activity}{10}
  Consider the Euclidean vector equation
  \[
    x\begin{bmatrix}0\\1\end{bmatrix}
      +
    y\begin{bmatrix}2\\0\end{bmatrix}
      -
    z\begin{bmatrix}3\\1\end{bmatrix}
      =
    \begin{bmatrix}0\\0\end{bmatrix}
  .\]
  \begin{subactivity}
    Find its solution set.
  \end{subactivity}
  \begin{subactivity}
    Find a vector in the solution space that consists of minimal
    positive integers.
  \end{subactivity}
  \begin{subactivity}
    Balance the chemical equation
    \[
      \unknown Fe + \unknown Cl_2
        \rightarrow
      \unknown FeCl_3
    .\]
  \end{subactivity}
\end{activity}

\begin{activity}{10}
  Balance the chemical equation
  \[
    \unknown Ca(OH)_2 + \unknown H_3PO_4
      \rightarrow
    \unknown Ca_3(PO_4)_2 + \unknown H_2O
  \]
  by first converting it into an \(\IR^4\) vector equation
  and finding its solution set.
\end{activity}

\end{applicationActivities}
