
\begin{applicationActivities}

\begin{definition}
In geology, a \term{phase} is any physically separable material in the system, such as various minerals or liquids.

\vspace{1em}

A \term{component} is a chemical compound necessary to make up the phases; for historical reasons these are usually oxides such as Calcium Oxide (${\rm CaO}$) or Silicone Dioxide (${\rm SiO_2}$).

\vspace{1em}

In a typical problem, a geologist knows how to build each phase from the components, and is interested in determining reactions among the different phases.
\end{definition}

\begin{activity}{5}
Consider the 3 components $\vec{c}_1={\rm CaO}$, $\vec{c}_2={\rm MgO}$, and $\vec{c}_3={\rm SiO_2}$, and the 5 phases
\begin{align*}
\vec{p}_1 &= {\rm Ca_3MgSi_2O_8} & \vec{p}_2 &= {\rm CaMgSiO_4} & \vec{p}_3 &= {\rm CaSiO_3} \\
\vec{p}_4 &= {\rm CaMgSi_2O_6} & \vec{p}_5 &= {\rm Ca_2MgSi_2O_7}
\end{align*}

Geologists will know
\begin{align*}
\vec{p}_1 &= 3\vec{c}_1 + \vec{c}_2 + 2 \vec{c}_3 & \vec{p}_2 &= \vec{c}_1 +\vec{c}_2 + \vec{c}_3  &
\vec{p}_3 &= \vec{c}_1 + 0\vec{c}_2 +  \vec{c}_3 \\
\vec{p}_4 &= \vec{c}_1 +\vec{c}_2 + 2\vec{c}_3 &  \vec{p}_5 &= 2\vec{c}_1 + \vec{c}_2 + 2 \vec{c}_3 
\end{align*}
or more compactly,
$$ \vec{p}_1 = \begin{bmatrix} 3 \\ 1 \\ 2 \end{bmatrix},
\vec{p}_2 = \begin{bmatrix} 1 \\ 1 \\ 1 \end{bmatrix},
\vec{p}_3 = \begin{bmatrix} 1 \\ 0 \\ 1 \end{bmatrix},
\vec{p}_4 = \begin{bmatrix} 1 \\ 1 \\ 2 \end{bmatrix},
\vec{p}_5 = \begin{bmatrix} 2 \\ 1 \\ 2 \end{bmatrix}.$$

Determine if the 5 phases are linearly dependent or linearly independent.
\end{activity}

\begin{activity}{15}
Recall our five phases:
\begin{align*}
\vec{p}_1 &= 3\vec{c}_1 + \vec{c}_2 + 2 \vec{c}_3 & \vec{p}_2 &= \vec{c}_1 +\vec{c}_2 + \vec{c}_3  &
\vec{p}_3 &= \vec{c}_1 + 0\vec{c}_2 +  \vec{c}_3 \\
\vec{p}_4 &= \vec{c}_1 +\vec{c}_2 + 2\vec{c}_3 &  \vec{p}_5 &= 2\vec{c}_1 + \vec{c}_2 + 2 \vec{c}_3 
\end{align*}
or more compactly,
$$ \vec{p}_1 = \begin{bmatrix} 3 \\ 1 \\ 2 \end{bmatrix},
\vec{p}_2 = \begin{bmatrix} 1 \\ 1 \\ 1 \end{bmatrix},
\vec{p}_3 = \begin{bmatrix} 1 \\ 0 \\ 1 \end{bmatrix},
\vec{p}_4 = \begin{bmatrix} 1 \\ 1 \\ 2 \end{bmatrix},
\vec{p}_5 = \begin{bmatrix} 2 \\ 1 \\ 2 \end{bmatrix}.$$

Geologists want to find chemical reactions among the 5 phases; that is, they want to find numbers $x_1,x_2,x_3,x_4,x_5$ such that $$x_1\vec{p}_1+x_2\vec{p}_2+x_3\vec{p_3}+x_4\vec{p}_4+x_5\vec{p}_5 = 0.$$

\begin{subactivity}
Set up  a system of equations that gives these chemical equations.
\end{subactivity}
\begin{subactivity}
Find a basis for the solution set.
\end{subactivity}
\begin{subactivity}
Interpret each basis vector as a chemical equation. 
\end{subactivity}

\end{activity}

\begin{activity}{10}
We found two basis vector $\begin{bmatrix} -1 \\ 2 \\ 2 \\ -1 \\ 0 \end{bmatrix}$ and $\begin{bmatrix} 0 \\ 1 \\ 1 \\ 0 \\ -1 \end{bmatrix}$, corresponding to two chemical equations
\begin{align*}
2\vec{p}_2 + 2 \vec{p}_3 &= \vec{p}_1 + \vec{p}_4 & 2{\rm CaMgSiO_4}+2{\rm CaSiO_3}&={\rm Ca_3MgSi_2O_8}+{\rm CaMgSi_2O_6}\\
\vec{p}_2 +\vec{p}_3 &= \vec{p}_5 &  {\rm CaMgSiO_4} + {\rm CaSiO_3} &= {\rm Ca_2MgSi_2O_7}
\end{align*}

Find a chemical equation among the five phases that does not involve $\vec{p}_2 = {\rm CaMgSiO_4}$.
\end{activity}




\end{applicationActivities}
