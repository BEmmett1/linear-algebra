\begin{applicationActivities}

\begin{definition}
\term{Cryptography} is the practice and study of encoding messages
so that only the intended receiver can decode them.

For example, the \texttt{ROT13} cipher both encodes and decodes
messages by shifting each letter thirteen places in the alphabet,
cycling from \(Z\) back to \(A\). This may be accomplished by
converting each letter to a number 
\[
\mathtt{A}\equiv 1,\mathtt{B}\equiv 2,\dots,
\mathtt{Y}\equiv 25,\mathtt{Z}\equiv 0
\]
and adding \(13\) (modulo \(26\)):

\[
\mathtt{HELLO} \equiv
\begin{bmatrix}8\\5\\12\\12\\15\end{bmatrix}
\underset{ROT13}{\leftrightarrow}
\begin{bmatrix}21\\18\\25\\25\\2\end{bmatrix}
\equiv
\mathtt{URYYB}
\]
\end{definition}
 
\begin{activity}{10}
Suppose your instructor saw another student passing a note that said
\begin{center}\texttt{MFUT DIFBU PO UIF UFTU}\end{center}
%LETS CHEAT ON THE TEST (rot1)

How could the instructor decode this message, taking advantage of the 
fact that \texttt{THE} is one of the most commonly used words
in the English language?
\end{activity}

\begin{observation}
\term{Frequency analysis} is a common tool used in breaking
\term{substitution ciphers}
that simply substitute letters for other letters. In the message 
\begin{center}\texttt{MFUT DIFBU PO UIF UFTU}\end{center}
the common word \texttt{THE} is encoded as \texttt{UIF}, 
and the most common letters in the English language \texttt{E,T}
match the most common letters used in this message: \texttt{F,U}.

\vspace{1em}

This suggests the following partial decryption:
\begin{center}\texttt{-ET- -HE-T -- THE TE-T}\end{center}

By considering the context, or perhaps by an analysis of other messages
sent using the same code, the completed message may be revealed:
\begin{center}\texttt{LETS CHEAT ON THE TEST}\end{center}
\end{observation}

\begin{remark}
To defeat naive frequency analysis attacks, one method that may be
used is to create a rule that converts groups of letters into
new groups of letters, rather than converting single letters
individually.

\vspace{1em}

So to send the message 
\begin{center}\texttt{LETS CHEAT ON THE TEST}\end{center}
one might first break it into three-letter pieces.
\begin{center}\texttt{LET SCH EAT ONT HET EST}\end{center}
\end{remark}

\begin{remark}
Each piece then may be converted to a Euclidean vector in \(\IR^3\),
which may be linearly transformed by multiplying by a matrix \(A\)
with \(\det(A)=1=\det(A^{-1})\).

\vspace{1em}

For \(A=\begin{bmatrix}3&-2&-3\\-2&3&0\\-1&0&2\end{bmatrix}\):

\[
\mathtt{LET}\equiv
\begin{bmatrix}12\\5\\20\end{bmatrix}\to
\begin{bmatrix}3&-2&-3\\-2&3&0\\-1&0&2\end{bmatrix}\begin{bmatrix}12\\5\\20\end{bmatrix}=
\begin{bmatrix}-34\\-9\\28\end{bmatrix}
\]
\end{remark}

\begin{remark}
The resulting vector may be converted back into English letters by
adding multiples of \(26\) to each component to obtain numbers between
\(0\) and \(25\).
\[
\begin{bmatrix}-34\\-9\\28\end{bmatrix}
\equiv
\begin{bmatrix}-34+52\\-9+26\\28-26\end{bmatrix}
=
\begin{bmatrix}18\\17\\2\end{bmatrix}
\equiv
\mathtt{RPB}
\]
\end{remark}

\begin{observation}
This process may be done all at once by converting the entire message into a matrix:
\[
\text{\texttt{LET SCH ...}}
\equiv
\begin{bmatrix}12&19\\5&3&\dots\\20&8\end{bmatrix}
\]
\[
\to
\begin{bmatrix}3&-2&-3\\-2&3&0\\-1&0&2\end{bmatrix}\begin{bmatrix}12&19\\5&3&\dots\\20&8\end{bmatrix}=
\begin{bmatrix}-34&27\\-9&-29&\dots\\28&-3\end{bmatrix}
\]
\[
\equiv
\begin{bmatrix}18&1\\17&23&\dots\\2&23\end{bmatrix}
\equiv
\text{\texttt{RPB AVV ...}}
\]
\end{observation}

\begin{activity}{10}
Complete the following encoding of the entire message given below,
using the encoding matrix \(A=\begin{bmatrix}3&-2&-3\\-2&3&0\\-1&0&2\end{bmatrix}\).
\[
\text{\texttt{LET SCH EAT ONT HET EST}}
\equiv
\begin{bmatrix}12&19\\5&3&\dots\\20&8\end{bmatrix}
\]
\[
\to
\begin{bmatrix}3&-2&-3\\-2&3&0\\-1&0&2\end{bmatrix}\begin{bmatrix}12&19\\5&3&\dots\\20&8\end{bmatrix}=
\begin{bmatrix}-34&27\\-9&-29&\dots\\28&-3\end{bmatrix}
\]
\[
\equiv
\begin{bmatrix}18&1\\17&23&\dots\\2&23\end{bmatrix}
\equiv
\text{\texttt{RPB AVV ESK ILY JYF UUK}}
\]
\end{activity}

\begin{activity}{10}
Reverse this process by using the decoding matrix,
\(A^{-1}=\begin{bmatrix}6&4&9\\4&3&6\\3&2&5\end{bmatrix}\).
\[
\text{\texttt{RPB AVV ESK ILY JYF UUK}}
\equiv
\begin{bmatrix}18&1\\17&23&\dots\\2&23\end{bmatrix}
\]
\[
\to
\begin{bmatrix}6&4&9\\4&3&6\\3&2&5\end{bmatrix}\begin{bmatrix}18&1\\17&23&\dots\\2&23\end{bmatrix}
=
\begin{bmatrix}194&305\\135&211&\dots\\98&164\end{bmatrix}
\]
\[
\equiv
\begin{bmatrix}12&19\\5&3&\dots\\20&8\end{bmatrix}
\equiv
\text{\texttt{LET SCH EAT ONT HET EST}}
\]
\end{activity}
\end{applicationActivities}
