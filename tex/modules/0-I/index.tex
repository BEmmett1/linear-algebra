\begin{module}{I}{Introduction}

\begin{remark}
  This brief module gives an overview for the course.
\end{remark}


\begin{applicationActivities}

\begin{remark} \textbf{What is Linear Algebra?}

  Linear algebra is the study of \textbf{linear maps}.

  \begin{itemize}
  \item In Calculus, you learn how to approximate any function by a
        linear function.
  \item In Linear Algebra, we learn about how linear maps behave.
  \item Combining the two, we can approximate how any function behaves.
  \end{itemize}
\end{remark}

\begin{remark} \textbf{What is Linear Algebra good for?}
  \begin{itemize}
  \item Linear algebra is used throughout several fields
        in higher mathematics.
  \item In computer graphics, linear algebra is used to help represent
        3D objects in a 2D grid of pixels.
  \item Linear algebra is used to approximate
        differential equation solutions in a vast number of engineering
        applications (e.g. fluid flows, vibrations, heat transfer) whose
        solutions are very difficult (or impossible) to find precisely.
  \item Google's search engine is based on its Page Rank algorithm, which
        ranks websites by computing an eigenvector of a matrix.
  \end{itemize}
\end{remark}

\begin{remark} \textbf{What will I learn in this class?}

  By the end of this class, you will be able to:
  \begin{itemize}
  \item Solve systems of linear equations.
        (Module E)
  \item Identify vector spaces and their properties.
        (Module V)
  \item Analyze the structure of vector spaces and sets of vectors.
        (Module S)
  \item Use and apply the algebraic properties of linear transformations.
        (Module A)
  \item Perform fundamental operations in the algebra of matrices.
        (Module M)
  \item Use and apply the geometric properties of linear transformations.
        (Module G)
  \end{itemize}
\end{remark}

\end{applicationActivities}
\end{module}
