%!TEX root =../../../course-notes.tex
% ^ leave for LaTeXTools build functionality


\begin{definition}
A \term{linear equation} is an equation of the variables \(x_i\) of the form
\[
a_1x_1+a_2x_2+\dots+a_nx_n=b
.\]
A \term{solution} for a linear equation is a Euclidean vector
\[
  \begin{bmatrix}
    s_1 \\
    s_2 \\
    \vdots \\
    s_n
  \end{bmatrix}
\]
that satisfies
\[
a_1s_1+a_2s_2+\dots+a_ns_n=b
\]
(that is, a Euclidean vector that can be plugged into the equation).
\end{definition}



\begin{remark}
In previous classes you likely used the variables \(x,y,z\) in equations.
However, since this course often deals with equations of four or more
variables, we will often write our variables as \(x_i\), and assume
\(x=x_1,y=x_2,z=x_3,w=x_4\) when convenient.
\end{remark}

\begin{definition}
A \term{system of linear equations} (or a \term{linear system} for short)
is a collection of one or more linear equations.
  \begin{alignat*}{5}
    a_{11}x_1 &\,+\,& a_{12}x_2 &\,+\,& \dots  &\,+\,& a_{1n}x_n &\,=\,& b_1 \\
    a_{21}x_1 &\,+\,& a_{22}x_2 &\,+\,& \dots  &\,+\,& a_{2n}x_n &\,=\,& b_2 \\
     \vdots&  &\vdots&   &&  &\vdots&&\vdots  \\
    a_{m1}x_1 &\,+\,& a_{m2}x_2 &\,+\,& \dots  &\,+\,& a_{mn}x_n &\,=\,& b_m
  \end{alignat*}
Its \term{solution set} is given by
\[
  \setBuilder
  {
    \begin{bmatrix}
      s_1 \\
      s_2 \\
      \vdots \\
      s_n
    \end{bmatrix}
  }{
    \begin{bmatrix}
      s_1 \\
      s_2 \\
      \vdots \\
      s_n
    \end{bmatrix}
    \text{ is a solution to all equations in the system}
  }
.\]
\end{definition}

\begin{remark}
  When variables in a large linear system are missing, we prefer to
  write the system in one of the following standard forms:

  \begin{multicols}{3}\noindent
    Original linear system:
    \begin{alignat*}{2}
       x_1 + 3x_3 &\,=\,& 3 \\
      3x_1 - 2x_2 + 4x_3 &\,=\,& 0 \\
      -x_2 +  x_3 &\,=\,& -2
    \end{alignat*}
    Verbose standard form:
    \begin{alignat*}{4}
      1x_1 &\,+\,& 0x_2 &\,+\,& 3x_3 &\,=\,& 3 \\
      3x_1 &\,-\,& 2x_2 &\,+\,& 4x_3 &\,=\,& 0 \\
      0x_1 &\,-\,& 1x_2 &\,+\,& 1x_3 &\,=\,& -2
    \end{alignat*}
    Concise standard form:
    \begin{alignat*}{4}
       x_1 &     &      &\,+\,& 3x_3 &\,=\,& 3 \\
      3x_1 &\,-\,& 2x_2 &\,+\,& 4x_3 &\,=\,& 0 \\
           &\,-\,&  x_2 &\,+\,&  x_3 &\,=\,& -2
    \end{alignat*}
  \end{multicols}
\end{remark}

\begin{definition}
  A linear system is \term{consistent} if its solution set
  is non-empty (that is, there exists a solution for the
  system). Otherwise it is \term{inconsistent}.
\end{definition}

\begin{fact}
  All linear systems are one of the following:
  \begin{itemize}
    \item \textbf{Consistent with one solution}:
          its solution set contains a single vector, e.g.
          \(\setList{\begin{bmatrix}1\\2\\3\end{bmatrix}}\)
    \item \textbf{Consistent with infinitely-many solutions}:
          its solution set contains infinitely many vectors, e.g.
          \(
            \setBuilder
            {
              \begin{bmatrix}1\\2-3a\\a\end{bmatrix}
            }{
              a\in\IR
            }
          \)
    \item \textbf{Inconsistent}:
          its solution set is the empty set \(\{\}=\emptyset\)
  \end{itemize}
\end{fact}

\begin{activity}{10}
  All inconsistent linear systems contain a logical \term{contradiction}.
  Find a contradiction in this system to show that its solution set
  is \(\emptyset\).
  \begin{align*}
  -x_1+2x_2  &=  5 \\
  2x_1-4x_2  &=  6
  \end{align*}
\end{activity}

\begin{activity}{10}
  Consider the following consistent linear system.
  \begin{align*}
  -x_1+2x_2  &= -3 \\
  2x_1-4x_2  &=  6
  \end{align*}
\begin{subactivity}
  Find three different solutions
  for this system.
\end{subactivity}
\begin{subactivity}
  Let \(x_2=a\) where \(a\) is an arbitrary real number, then find an
  expression for \(x_1\) in terms of \(a\). Use this to write
  the solution set
  \(
    \setBuilder
    {
    	\begin{bmatrix}
        \unknown \\
        a
      \end{bmatrix}
    }{
      a \in \IR
	  }
  \)
  for the linear system.
\end{subactivity}
\end{activity}

\begin{activity}{10}
  Consider the following linear system.
  \begin{alignat*}{5}
    x_1 &\,+\,& 2x_2 &\, \,&     &\,-\,&  x_4 &\,=\,& 3 \\
        &\, \,&      &\, \,& x_3 &\,+\,& 4x_4 &\,=\,& -2
  \end{alignat*}
  Describe the solution set
  \[
  	\setBuilder
    {
    	\begin{bmatrix}
          \unknown \\
          a \\
          \unknown \\
          b
    	\end{bmatrix}
    }{
      a,b \in \IR
    }
  \]
  to the linear system
  by setting \(x_2=a\) and \(x_4=b\), and then solving for \(x_1\) and
  \(x_3\).
\end{activity}

\begin{observation}
  Solving linear systems of two variables by graphing or substitution is
  reasonable for two-variable systems, but these simple techniques
  won't usually cut it for equations with
  more than two variables or more than two equations. For example,
  \begin{alignat*}{5}
    -2x_1 &\,-\,& 4x_2 &\,+\,&  x_3 &\,-\,&  4x_4 &\,=\,& -8 \\
      x_1 &\,+\,& 2x_2 &\,+\,& 2x_3 &\,+\,& 12x_4 &\,=\,& -1 \\
      x_1 &\,+\,& 2x_2 &\,+\,&  x_3 &\,+\,&  8x_4 &\,=\,&  1 \\
  \end{alignat*}
  has the exact same solution set as the system in the previous
  activity, but we'll want to learn new techniques
  to compute these solutions efficiently.
\end{observation}





\begin{remark}
  The only important information in a linear system are its coefficients and
  constants.

  \begin{multicols}{3}\noindent
    Original linear system:
    \begin{alignat*}{2}
       x_1 + 3x_3 &\,=\,& 3 \\
      3x_1 - 2x_2 + 4x_3 &\,=\,& 0 \\
      -x_2 +  x_3 &\,=\,& -2
    \end{alignat*}
    Verbose standard form:
    \begin{alignat*}{4}
      1x_1 &\,+\,& 0x_2 &\,+\,& 3x_3 &\,=\,& 3 \\
      3x_1 &\,-\,& 2x_2 &\,+\,& 4x_3 &\,=\,& 0 \\
      0x_1 &\,-\,& 1x_2 &\,+\,& 1x_3 &\,=\,& -2
    \end{alignat*}
    Coefficients/constants:
    \begin{alignat*}{4}
       1 &     &  0 &\,\,& 3 &\,|\,& 3 \\
       3 &\, \,& -2 &\,\,& 4 &\,|\,& 0 \\
       0 &\, \,& -1 &\,\,& 1 &\,|\,& -2
    \end{alignat*}
  \end{multicols}
\end{remark}

\begin{definition}
  A system of \(m\) linear equations with \(n\) variables is often represented
  by writing its coefficients and constants in an \term{augmented matrix}.
  \begin{multicols}{2}\noindent
  \begin{alignat*}{5}
    a_{11}x_1 &\,+\,& a_{12}x_2 &\,+\,& \dots  &\,+\,& a_{1n}x_n &\,=\,& b_1 \\
    a_{21}x_1 &\,+\,& a_{22}x_2 &\,+\,& \dots  &\,+\,& a_{2n}x_n &\,=\,& b_2 \\
     \vdots&  &\vdots&   &&  &\vdots&&\vdots  \\
    a_{m1}x_1 &\,+\,& a_{m2}x_2 &\,+\,& \dots  &\,+\,& a_{mn}x_n &\,=\,& b_m
  \end{alignat*}
  \[
    \begin{bmatrix}[cccc|c]
      a_{11} & a_{12} & \cdots & a_{1n} & b_1\\
      a_{21} & a_{22} & \cdots & a_{2n} & b_2\\
      \vdots & \vdots & \ddots & \vdots & \vdots\\
      a_{m1} & a_{m2} & \cdots & a_{mn} & b_m
    \end{bmatrix}
  \]
  \end{multicols}
\end{definition}

\begin{example}
  The corresponding augmented matrix for this system is obtained by
  simply writing the coefficients and constants in matrix form.

  \begin{multicols}{2}
    Linear system:
    \begin{alignat*}{2}
       x_1 + 3x_3 &\,=\,& 3 \\
      3x_1 - 2x_2 + 4x_3 &\,=\,& 0 \\
      -x_2 +  x_3 &\,=\,& -2
    \end{alignat*}

    \columnbreak

    Augmented matrix:
    \[
      \begin{bmatrix}[ccc|c]
        1 & 0 & 3 & 3 \\
        3 & -2 & 4 & 0 \\
        0 & -1 & 1 & -2
      \end{bmatrix}
    \]
  \end{multicols}
\end{example}

\begin{definition}
  Two systems of linear equations (and their corresponding augmented
  matrices) are said to be \term{equivalent} if they have the same
  solution set.

  For example, both of these systems share the same solution set
  \(\setList{ \begin{bmatrix} 1 \\ 1\end{bmatrix} }\).

  \begin{multicols}{2}\noindent
    \begin{alignat*}{3}
      3x_1 &\,-\,& 2x_2 &\,=\,& 1 \\
      x_1 &\,+\,& 4x_2 &\,=\,& 5 \\
    \end{alignat*}
    \begin{alignat*}{3}
      3x_1 &\,-\,& 2x_2 &\,=\,& 1 \\
      4x_1 &\,+\,& 2x_2 &\,=\,& 6 \\
    \end{alignat*}
  \end{multicols}

  Therefore these augmented matrices are equivalent,
  which we denote with \(\sim\):

    \[
      \begin{bmatrix}[cc|c]
        3 & -2 & 1\\
        1 &  4 & 5\\
      \end{bmatrix}
    \sim
      \begin{bmatrix}[cc|c]
        3 & -2 & 1\\
        4 &  2 & 6\\
      \end{bmatrix}
    \]
\end{definition}

\begin{activity}{10}
  Following are seven procedures used to manipulate an augmented matrix.
  Label the procedures that would result in an equivalent augmented
  matrix as \textbf{valid}, and label the procedures that might
  change the solution set of the corresponding linear system as
  \textbf{invalid}.
  \begin{multicols}{2}
    \begin{enumerate}[a)]
      \item Swap two rows.
      \item Swap two columns.
      \item Add a constant to every term in a row.
      \item Multiply a row by a nonzero constant. \columnbreak
      \item Add a constant multiple of one row to another row.
      \item Replace a column with zeros.
      \item Replace a row with zeros.
    \end{enumerate}
  \end{multicols}
  \begin{instructorNote}
    This activity could be ran as a card sort.  Allow 5 additional minutes for intra team discussion.
  \end{instructorNote}
\end{activity}


\begin{definition}
  The following \term{row operations} produce equivalent
  augmented matrices:
  \begin{enumerate}
    \item Swap two rows, for example, \(R_1\leftrightarrow R_2\):
    \[
      \begin{bmatrix}[cc|c] 1 & 2 & 3 \\ 4 & 5 & 6 \end{bmatrix}
    \sim
      \begin{bmatrix}[cc|c] 4 & 5 & 6 \\ 1 & 2 & 3 \end{bmatrix}
    \]
    \item Multiply a row by a nonzero constant, for example, \(2R_1\rightarrow R_1\):
    \[
      \begin{bmatrix}[cc|c] 1 & 2 & 3 \\ 4 & 5 & 6 \end{bmatrix}
    \sim
      \begin{bmatrix}[cc|c] 2(1) & 2(2) & 2(3) \\ 4 & 5 & 6 \end{bmatrix}
    \]
    \item Add a constant multiple of one row to another row,
          for example, \(R_2-4R_1\rightarrow R_2\):
    \[
      \begin{bmatrix}[cc|c] 1 & 2 & 3 \\ 4 & 5 & 6 \end{bmatrix}
    \sim
      \begin{bmatrix}[cc|c] 1 & 2 & 3 \\ 4-4(1) & 5-4(2) & 6-4(3) \end{bmatrix}
    \]
  \end{enumerate}
  Whenever two matrices \(A,B\) are equivalent (so whenever we do any of
  these operations), we write \(A\sim B\).
\end{definition}

\begin{activity}{10}
  Consider the following (equivalent) linear systems.
  \begin{multicols}{3}%
  \begin{enumerate}[(A)]
    \item 
	\[
		\begin{alignedat}{4}
   		  x &\,+\,& 2y  &\,+\,& z &\,=\,& 3 \\  
   		  -x &\,-\,& y  &\,+\,& z &\,=\,& 1 \\  
   		  2x &\,+\,& 5y  &\,+\,& 3z &\,=\,& 7   
   		\end{alignedat}
	\]

    \item 
	\[
		\begin{alignedat}{4}
   		  2x &\,+\,& 5y  &\,+\,& 3z &\,=\,& 7 \\  
   		  -x &\,-\,& y  &\,+\,& z &\,=\,& 1 \\  
   		  x &\,+\,& 2y  &\,+\,& z &\,=\,& 3  
   		\end{alignedat}
	\]

    \item \[
		\begin{alignedat}{4}
   		  x & &   &\,-\,& z &\,=\,& 1 \\  
   		   & & y  &\,+\,& z &\,=\,& 1 \\  
   		   & & y  &\,+\,& 2z &\,=\,& 4   
   		\end{alignedat}
	\]

    \item \[
		\begin{alignedat}{4}
   		  x &\,+\,& 2y  &\,+\,& z &\,=\,& 3 \\  
   		   & & y  &\,+\,& z &\,=\,& 1 \\  
   		  2x &\,+\,& 5y  &\,+\,& 3z &\,=\,& 7   
   		\end{alignedat}
	\]

    \item \[
		\begin{alignedat}{4}
   		  x & &   &\,-\,& z &\,=\,& 1 \\  
   		   & & y  &\,+\,& z &\,=\,& 1 \\  
   		   & &   & & z &\,=\,& 3   
   		\end{alignedat}
	\]

    \item \[
		\begin{alignedat}{4}
   		  x &\,+\,& 2y  &\,+\,& z &\,=\,& 3 \\  
   		   & & y  &\,+\,& z &\,=\,& 1 \\  
   		   & & y  &\,+\,& 2z &\,=\,& 4   
   		\end{alignedat}
	\]
	\end{enumerate}
    \end{multicols}
   Rank the six linear systems from most complicated to simplest.
\end{activity}


\begin{activity}{5}
We can rewrite the previous in terms of  equivalences of augmented matrices
\[
	\begin{alignedat}{3}
		\begin{bmatrix}[ccc|c] 2 & 5 & 1 3 & 7 \\ -1 & -1 & 1 & 1 \\ 1 & 2 & 1 & 3 \end{bmatrix} & \sim &
		\begin{bmatrix}[ccc|c] \circledNumber{1} & 2 & 1 & 3 \\ -1 & -1 & 1 & 1 \\ 2 & 5 & 1 & 3  \end{bmatrix} & \sim &
		\begin{bmatrix}[ccc|c] \circledNumber{1} & 2 & 1 & 3 \\ 0 & 1 & 1 & 1 \\ 2 & 5 & 1 & 3  \end{bmatrix} \sim  \\
		\begin{bmatrix}[ccc|c] \circledNumber{1} & 2 & 1 & 3 \\ 0 & \circledNumber{1} & 1 & 1 \\ 0 & 1 & 2 & 4  \end{bmatrix} & \sim &
		\begin{bmatrix}[ccc|c] \circledNumber{1} & 0 & -1 & 1 \\ 0 & \circledNumber{1} & 1 & 1 \\ 0 & 1 & 2 & 4  \end{bmatrix} & \sim &
		\begin{bmatrix}[ccc|c] \circledNumber{1} & 0 & -1 & 1 \\ 0 & \circledNumber{1} & 1 & 1 \\ 0 & 0 & \circledNumber{1} & 3  \end{bmatrix} 
 	\end{alignedat}
\]

		Determine the row operation(s) necessary in each step to transform the most complicated
    system's augmented matrix into the simplest.

\end{activity}

\begin{definition}
A matrix is in \term{reduced row echelon form} (\term{RREF}) if
\begin{enumerate}
\item The leading term (first nonzero term) of each nonzero row is a 1.
      Call these terms \term{pivots}.
\item Each pivot is to the right of every higher pivot.
\item Each term above or below a pivot is zero.
\item All rows of zeroes are at the bottom of the matrix.
\end{enumerate}
\vfill
Every matrix has a unique reduced row echelon form. If \(A\) is a matrix, we write \(\RREF(A)\) for the reduced row echelon form of that matrix.
\end{definition}

\begin{activity}{15}
Recall that a matrix is in \term{reduced row echelon form} (\term{RREF}) if
\begin{enumerate}
\item The leading term (first nonzero term) of each nonzero row is a 1.
      Call these terms \term{pivots}.
\item Each pivot is to the right of every higher pivot.
\item Each term above or below a pivot is zero.
\item All rows of zeroes are at the bottom of the matrix.
\end{enumerate}
\vfill
%Consider the following matrices

\begin{multicols}{3}
\begin{enumerate}[(A)]
\item \[ \begin{bmatrix}[ccc|c] 1 & 0 & 0 & 3 \\ 0 & 0 & 1 & -1 \\ 0 & 0 & 0 & 0 \end{bmatrix}\]
\item \[ \begin{bmatrix}[ccc|c] 1 & 2 & 4 & 3 \\ 0 & 0 & 1 & -1 \\ 0 & 0 & 0 & 0 \end{bmatrix}\]
\item \[ \begin{bmatrix}[ccc|c] 0 & 0 & 0 & 0 \\ 1 & 2 & 0 & 3 \\ 0 & 0 & 1 & -1  \end{bmatrix}\]
\item \[ \begin{bmatrix}[ccc|c] 1 & 0 & 2 & -3 \\ 0 & 3 & 3 & -3 \\ 0 & 0 & 0 & 0 \end{bmatrix}\]
\item \[ \begin{bmatrix}[ccc|c] 0 & 1 & 0 & 7 \\ 1 & 0 & 0 & 4 \\ 0 & 0 & 0 & 0 \end{bmatrix}\]
\item \[ \begin{bmatrix}[ccc|c] 1 & 0 & 0 & 4 \\ 0 & 1 & 0 & 7 \\ 0 & 0 & 1 & 0 \end{bmatrix}\]
\end{enumerate}
\end{multicols}
%\begin{subactivity}
For each matrix, circle the leading terms, and label it as RREF or not RREF.
%\end{subactivity}
%\begin{subactivity}
For the ones not in RREF, find their RREF.
%\end{subactivity}
\end{activity}

\begin{remark}
In practice, if we simply need to convert a matrix into reduced row echelon form, 
we use technology to do so.
\vfill
However, it is also important to understand the \term{Gauss-Jordan elimination} algorithm
that a computer or calculator uses to convert a matrix (augmented or not) into reduced row echelon form.
Understanding this algorithm will help us better understand how to interpret the results
in many applications we use it for in Module V.
\end{remark}

\begin{activity}{8}
Consider the matrix
\[\begin{bmatrix}[cccc]2 & 6 & -1 & 6  \\ 1 & 3 & -1 & 2 \\ -1 & -3 & 2 & 0 \end{bmatrix}.\]
Which row operation is the best choice for the first move in converting to RREF?
\begin{enumerate}[(a)]
\item Add row 3 to row 2 (\(R_2+R_3 \rightarrow R_2\))
\item Add row 2 to row 3 (\(R_3+R_2 \rightarrow R_3\))
\item Swap row 1 to row 2 (\(R_1 \leftrightarrow R_2\))
\item Add -2 row 2 to row 1 (\(R_1-2R_2 \rightarrow R_1\))
\end{enumerate}
\end{activity}

\begin{activity}{7}
Consider the matrix
\[\begin{bmatrix}[cccc] \circledNumber{1} & 3 & -1 & 2 \\ 2 & 6 & -1 & 6 \\ -1 & -3 & 2 & 0 \end{bmatrix}.\]
Which row operation is the best choice for the next move in converting to RREF?
\begin{enumerate}[(a)]
\item Add row 1 to  row 3 (\(R_3+R_1 \rightarrow R_3\))
\item Add -2 row 1 to  row 2 (\(R_2-2R_1 \rightarrow R_2\))
\item Add 2 row 2 to row 3 (\(R_3+2R_2 \rightarrow R_3\))
\item Add 2 row 3 to row 2 (\(R_2+2R_3 \rightarrow R_2\))
\end{enumerate}
\end{activity}

\begin{activity}{5}
Consider the matrix
\[\begin{bmatrix}[cccc]\circledNumber{1} & 3 & -1 & 2  \\ 0 & 0 & 1 & 2 \\ 0 & 0 & 1 & 2 \end{bmatrix}.\]
Which row operation is the best choice for the next move in converting to RREF?
\begin{enumerate}[(a)]
\item Add row 1 to row 2 (\(R_2+R_1 \rightarrow R_2\))
\item Add -1 row 3 to  row 2 (\(R_2-R_3 \rightarrow R_2\))
\item Add -1 row 2 to  row 3 (\(R_3-R_2 \rightarrow R_3\))
\item Add row 2 to row 1 (\(R_1+R_2 \rightarrow R_1\))
\end{enumerate}
\end{activity}

\begin{activity}{10}
Consider the matrix
\[\begin{bmatrix}[ccc]2 & 1 & 0 \\ 1 & 0 & 0 \\ 3 & -1 & 1 \end{bmatrix}.\]
\vfill
\begin{subactivity}
Perform three row operations to produce a matrix closer to RREF.
\end{subactivity}
\begin{subactivity}
Finish putting it in RREF.
\end{subactivity}
\end{activity}

\begin{activity}{10}
Consider the matrix
\[A=\begin{bmatrix}[cccc]2 & 3 & 2 & 3 \\ -2 & 1 & 6 & 1 \\ -1 & -3 & -4 & 1 \end{bmatrix}.\]
\vfill
Compute \(\RREF(A)\).
\end{activity}

\begin{activity}{10}
Consider the matrix
\[A=\begin{bmatrix}[cccc] 2 & 4 & 2 & -4 \\ -2 & -4 & 1 & 1 \\ 3 & 6 & -1 & -4 \end{bmatrix}.\]
\vfill
Compute \(\RREF(A)\).
\end{activity}

\begin{remark}
A video example of how to perform the Gauss-Jordan Elimination algorithm
by hand is available at \url{https://youtu.be/Cq0Nxk2dhhU}.
\vfill
Practicing several exercises on your own using this method is strongly
recommended.
\end{remark}





\begin{activity}{10}
Free browser-based technologies for mathematical computation
are available online.
\begin{itemize}
\item Go to \url{https://octave-online.net}.
\item Type \texttt{A=sym([1 3 4 ; 2 5 7])} and press \texttt{Enter}
      to store the matrix
      \(\begin{bmatrix} 1 & 3 & 2 \\ 2 & 5 & 7 \end{bmatrix}\)
      in the variable \(A\).
  \begin{itemize}
    \item The symbolic function \texttt{sym} is used to calculate precise answers 
          rather than floating-point approximations.
    \item The vertical bar in an augmented matrix does not affect
          row operations, so the \(\RREF\) of
      \(\begin{bmatrix}[cc|c] 1 & 3 & 2 \\ 2 & 5 & 7 \end{bmatrix}\)
          may be computed in the same way.
  \end{itemize}
\item Type \texttt{rref(A)} and press \texttt{Enter}
      to compute the reduced row echelon form of \(A\).
\end{itemize}
\end{activity}

\begin{remark}
We will frequently need to know the reduced row echelon form of matrices 
during class, so feel free to use Octave-Online.net to compute RREF
efficiently.

\vspace{1em}

You may alternatively use the calculator you will use during assessments.
Be sure to use fractions mode to compute exact solutions rather
than floating-point approximations. 
\end{remark}

\begin{activity}{10}
Consider the system of equations.
 \[
		\begin{alignedat}{4}
   		  3x_1 &\,-\,& 2x_2 &\,+\,& 13x_3 &\,=\,& 6 \\
   		  2x_1 &\,-\,& 2x_2 &\,+\,& 10x_3 &\,=\,& 2 \\
   		  -x_1 &\,+\,& 3x_2 &\,-\,&  6x_3 &\,=\,& 11
   		\end{alignedat}
\]

\begin{subactivity}
Convert this to an augmented matrix and use technology to compute its
reduced row echelon form:
\[
  \RREF
  \begin{bmatrix}[ccc|c]
    \unknown&\unknown&\unknown&\unknown\\ 
    \unknown&\unknown&\unknown&\unknown\\ 
    \unknown&\unknown&\unknown&\unknown\\ 
  \end{bmatrix}
=
  \begin{bmatrix}[ccc|c]
    \unknown&\unknown&\unknown&\unknown\\ 
    \unknown&\unknown&\unknown&\unknown\\ 
    \unknown&\unknown&\unknown&\unknown\\ 
  \end{bmatrix}
\]
\end{subactivity}
\begin{subactivity}
Use the \(\RREF\) matrix to write a linear system equivalent
to the original system. Then find its solution set.
\end{subactivity}
\end{activity}

\begin{activity}{10}
Consider the system of equations.
 \[
		\begin{alignedat}{4}
   		  3x_1 &\,-\,& 2x_2 &\,+\,& 13x_3 &\,=\,& 6 \\
   		  2x_1 &\,-\,& 2x_2 &\,+\,& 10x_3 &\,=\,& 2 \\
   		  -x_1 &\,\,&  &\,-\,&  3x_3 &\,=\,&1
   		\end{alignedat}
\]

\begin{subactivity}
Convert this to an augmented matrix and use technology to compute its
reduced row echelon form:
\[
  \RREF
  \begin{bmatrix}[ccc|c]
    \unknown&\unknown&\unknown&\unknown\\ 
    \unknown&\unknown&\unknown&\unknown\\ 
    \unknown&\unknown&\unknown&\unknown\\ 
  \end{bmatrix}
=
  \begin{bmatrix}[ccc|c]
    \unknown&\unknown&\unknown&\unknown\\ 
    \unknown&\unknown&\unknown&\unknown\\ 
    \unknown&\unknown&\unknown&\unknown\\ 
  \end{bmatrix}
\]
\end{subactivity}
\begin{subactivity}
Use the \(\RREF\) matrix to write a linear system equivalent
to the original system. Then find its solution set.
\end{subactivity}
\end{activity}

\begin{activity}{10}
  Consider the following linear system.
  	\begin{alignat*}{4}
  		x_1 &+ 2x_2 &+ 3x_3 &= 1\\
  	  2x_1 &+ 4x_2 &+ 8x_3 &= 0
  	\end{alignat*}
  \begin{subactivity}
    Find its corresponding augmented matrix \(A\) and
    use technology to find \(\RREF(A)\).
  \end{subactivity}
  \begin{subactivity}
    How many solutions do these linear systems have?
  \end{subactivity}
\end{activity}

\begin{activity}{10}
Consider the simple linear system equivalent to the system
from the previous activity:
	\begin{alignat*}{3}
		x_1 &+ 2x_2 & &= 4\\
	     	 & &\phantom{+}x_3 &= -1
	\end{alignat*}
\begin{subactivity}
Let \(x_1=a\) and write the solution set in the form
\(
  \setBuilder
  {
    \begin{bmatrix} a \\ \unknown \\ \unknown \end{bmatrix}
  }{
    a \in \IR
  }
\).
\end{subactivity}
\begin{subactivity}
Let \(x_2=b\) and write the solution set in the form
\(
  \setBuilder
  {
    \begin{bmatrix} \unknown \\ b \\ \unknown \end{bmatrix}
  }{
    b \in \IR
  }
\).
\end{subactivity}
\begin{subactivity}
Which of these was easier?  What features of the RREF matrix
\(
  \begin{bmatrix}[ccc|c]
    \circledNumber{1} & 2 & 0 & 4 \\
    0 & 0 & \circledNumber{1} & -1
  \end{bmatrix}
\) caused this?
\end{subactivity}
\end{activity}

\begin{definition}
Recall that the pivots of a matrix in \(\RREF\) form are the leading
\(1\)s in each non-zero row.

\vspace{0.2in}

The pivot columns in an augmented matrix correspond to the
\term{bound variables} in the system of equations (\(x_1,x_3\) below).
The remaining variables are called \term{free variables} (\(x_2\) below).

\[
  \begin{bmatrix}[ccc|c]
    \circledNumber{1} & 2 & 0 & 4 \\
    0 & 0 & \circledNumber{1} & -1
  \end{bmatrix}
\]

To efficiently solve a system in RREF form, assign letters to the free
variables, and then solve for the bound variables.
\end{definition}

\begin{activity}{10}
Find the solution set for the system
\begin{alignat*}{6}
2x_1&\,-\,&2x_2&\,-\,&6x_3&\,+\,&x_4&\,-\,&x_5&\,=\,&3 \\
-x_1&\,+\,&x_2&\,+\,&3x_3&\,-\,&x_4&\,+\,&2x_5 &\,=\,& -3 \\
x_1&\,-\,&2x_2&\,-\,&x_3&\,+\,&x_4&\,+\,&x_5 &\,=\,& 2
\end{alignat*}
by row-reducing its augmented matrix, and then
assigning letters to the free variables (given by non-pivot columns)
and solving for the bound variables (given by pivot columns) in
the corresponding linear system.
\end{activity}

\begin{observation}
The solution set to the system
\begin{alignat*}{6}
2x_1&\,-\,&2x_2&\,-\,&6x_3&\,+\,&x_4&\,-\,&x_5&\,=\,&3 \\
-x_1&\,+\,&x_2&\,+\,&3x_3&\,-\,&x_4&\,+\,&2x_5 &\,=\,& -3 \\
x_1&\,-\,&2x_2&\,-\,&x_3&\,+\,&x_4&\,+\,&x_5 &\,=\,& 2
\end{alignat*}
may be written as
\[
  \setBuilder
  {
    \begin{bmatrix}
      1+5a+2b \\
      1+2a+3b \\
      a \\
      3+3b \\
      b
    \end{bmatrix}
  }{
    a,b\in \IR
  }
.\]
\end{observation}

\begin{remark}
Don't forget to correctly express the solution set of a linear system,
using set-builder notation for consistent systems with
infintely many solutions.
  \begin{itemize}
  \item \textbf{Consistent with one solution}: e.g.
        \(\setList{ \begin{bmatrix}1\\2\\3\end{bmatrix} }\)
  \item \textbf{Consistent with infinitely-many solutions}: e.g.
        \(
          \setBuilder
          {
            \begin{bmatrix}1\\2-3a\\a\end{bmatrix}
          }{
            a\in\IR
          }
        \)
  \item \textbf{Inconsistent}: \(\emptyset\) or \(\{\}\)
  \end{itemize}
\end{remark}

