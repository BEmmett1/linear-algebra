%!TEX root =../../../course-notes.tex
% ^ leave for LaTeXTools build functionality

\begin{applicationActivities}



\begin{remark}
  The only important information in a linear system are its coefficients and
  constants.

  \begin{multicols}{3}\noindent
    Original linear system:
    \begin{alignat*}{2}
       x_1 + 3x_3 &\,=\,& 3 \\
      3x_1 - 2x_2 + 4x_3 &\,=\,& 0 \\
      -x_2 +  x_3 &\,=\,& -2
    \end{alignat*}
    Verbose standard form:
    \begin{alignat*}{4}
      1x_1 &\,+\,& 0x_2 &\,+\,& 3x_3 &\,=\,& 3 \\
      3x_1 &\,-\,& 2x_2 &\,+\,& 4x_3 &\,=\,& 0 \\
      0x_1 &\,-\,& 1x_2 &\,+\,& 1x_3 &\,=\,& -2
    \end{alignat*}
    Coefficients/constants:
    \begin{alignat*}{4}
       1 &     &  0 &\,\,& 3 &\,|\,& 3 \\
       3 &\, \,& -2 &\,\,& 4 &\,|\,& 0 \\
       0 &\, \,& -1 &\,\,& 1 &\,|\,& -2
    \end{alignat*}
  \end{multicols}
\end{remark}

\begin{definition}
  A system of \(m\) linear equations with \(n\) variables is often represented
  by writing its coefficients and constants in an \term{augmented matrix}.
  \begin{multicols}{2}\noindent
  \begin{alignat*}{5}
    a_{11}x_1 &\,+\,& a_{12}x_2 &\,+\,& \dots  &\,+\,& a_{1n}x_n &\,=\,& b_1 \\
    a_{21}x_1 &\,+\,& a_{22}x_2 &\,+\,& \dots  &\,+\,& a_{2n}x_n &\,=\,& b_2 \\
     \vdots&  &\vdots&   &&  &\vdots&&\vdots  \\
    a_{m1}x_1 &\,+\,& a_{m2}x_2 &\,+\,& \dots  &\,+\,& a_{mn}x_n &\,=\,& b_m
  \end{alignat*}
  \[
    \begin{bmatrix}[cccc|c]
      a_{11} & a_{12} & \cdots & a_{1n} & b_1\\
      a_{21} & a_{22} & \cdots & a_{2n} & b_2\\
      \vdots & \vdots & \ddots & \vdots & \vdots\\
      a_{m1} & a_{m2} & \cdots & a_{mn} & b_m
    \end{bmatrix}
  \]
  \end{multicols}
\end{definition}

\begin{example}
  The corresopnding augmented matrix for this system is obtained by
  simply writing the coefficients and constants in matrix form.

  \begin{multicols}{2}
    Linear system:
    \begin{alignat*}{2}
       x_1 + 3x_3 &\,=\,& 3 \\
      3x_1 - 2x_2 + 4x_3 &\,=\,& 0 \\
      -x_2 +  x_3 &\,=\,& -2
    \end{alignat*}

    \columnbreak

    Augmented matrix:
    \[
      \begin{bmatrix}[ccc|c]
        1 & 0 & 3 & 3 \\
        3 & -2 & 4 & 0 \\
        0 & -1 & 1 & -2
      \end{bmatrix}
    \]
  \end{multicols}
\end{example}

\begin{definition}
  Two systems of linear equations (and their corresponding augmented
  matrices) are said to be \term{equivalent} if they have the same
  solution set.

  For example, both of these systems share the same solution set
  \(\setList{ \begin{bmatrix} 1 \\ 1\end{bmatrix} }\).

  \begin{multicols}{2}\noindent
    \begin{alignat*}{3}
      3x_1 &\,-\,& 2x_2 &\,=\,& 1 \\
      x_1 &\,+\,& 4x_2 &\,=\,& 5 \\
    \end{alignat*}
    \begin{alignat*}{3}
      3x_1 &\,-\,& 2x_2 &\,=\,& 1 \\
      4x_1 &\,+\,& 2x_2 &\,=\,& 6 \\
    \end{alignat*}
  \end{multicols}

  Therefore these augmented matrices are equivalent:

  \begin{multicols}{2}\noindent
    \[
      \begin{bmatrix}[cc|c]
        3 & -2 & 1\\
        1 &  4 & 5\\
      \end{bmatrix}
    \]
    \[
      \begin{bmatrix}[cc|c]
        3 & -2 & 1\\
        4 &  2 & 6\\
      \end{bmatrix}
    \]
  \end{multicols}
\end{definition}

\begin{activity}{10}
  Following are seven procedures used to manipulate an augmented matrix.
  Label the procedures that would result in an equivalent augmented
  matrix as \textbf{valid}, and label the procedures that might
  change the solution set of the corresponding linear system as
  \textbf{invalid}.
  \begin{multicols}{2}
    \begin{enumerate}[a)]
      \item Swap two rows.
      \item Swap two columns.
      \item Add a constant to every term in a row.
      \item Multiply a row by a nonzero constant. \columnbreak
      \item Add a constant multiple of one row to another row.
      \item Replace a column with zeros.
      \item Replace a row with zeros.
    \end{enumerate}
  \end{multicols}
  \begin{instructorNote}
    This activity could be ran as a card sort.  Allow 5 additional minutes for intra team discussion.
  \end{instructorNote}
\end{activity}


\begin{definition}
  The following \term{row operations} produce equivalent
  augmented matrices:
  \begin{enumerate}
    \item Swap two rows.
    % \begin{itemize}
    %   \item (swap the order of two equations in the system)
    % \end{itemize}
    \item Multiply a row by a nonzero constant.
    % \begin{itemize}
    %   \item (multiply both sides of an equation by a nonzero constant)
    % \end{itemize}
    \item Add a constant multiple of one row to another row.
    % \begin{itemize}
    %   \item (add a multiple of one equation to another)
    % \end{itemize}
  \end{enumerate}
  Whenever two matrices \(A,B\) are equivalent (so whenever we do any of
  these operations), we write \(A\sim B\).
\end{definition}

\begin{activity}{10}
  Consider the following (equivalent) linear systems.
  \begin{multicols}{3}%
  \begin{enumerate}[(A)]
    \item \[
		\begin{alignedat}{4}
   		  -2x_1 &\,+\,& 4x_2 &\,-\,& 2x_3 &\,=\,& -8 \\
   		  x_1 &\,-\,& 2x_2 &\,+\,& 2x_3 &\,=\,& 7 \\
   		  3x_1 &\,-\, & 6x_2 &\,+\,&  4x_3 &\,=\,&  15
   		\end{alignedat}
	\]

    \item \[
		\begin{alignedat}{4}
   		  x_1 &\,-\,& 2x_2 &\,+\,& 2x_3 &\,=\,& 7 \\
   		  -2x_1 &\,+\,& 4x_2 &\,-\,& 2x_3 &\,=\,& -8 \\
   		  3x_1 &\,-\, & 6x_2 &\,+\,&  4x_3 &\,=\,&  15
   		\end{alignedat}
	\]

    \item \[
		\begin{alignedat}{4}
   		  x_1 &\,-\,& 2x_2 &\,+\,& 2x_3 &\,=\,& 7 \\
   		   & &  & & 2x_3 &\,=\,& 6 \\
   		   & & &\,-\,&2x_3 &\,=\,& -6
   		\end{alignedat}
	\]

    \item \[
		\begin{alignedat}{4}
   		  x_1 &\,-\,& 2x_2 &\,+\,& 2x_3 &\,=\,& 7 \\
   		   & &  & & x_3 &\,=\,& 3 \\
   		   & & &\,-\,&2x_3 &\,=\,& -6
   		\end{alignedat}
	\]

    \item \[
		\begin{alignedat}{4}
   		  x_1 &\,-\,& 2x_2 &&  &\,=\,& 1 \\
   		   & &  &\,\,\, & x_3 &\,=\,& 3 \\
   		   & & &&0 &\,=\,& 0
   		\end{alignedat}
	\]

    \item \[
		\begin{alignedat}{4}
   		  x_1 &\,-\,& 2x_2 &\,+\,& 2x_3 &\,=\,& 7 \\
   		   &\,&  && 2x_3 &\,=\,& 6 \\
   		  3x_1 &\,-\, & 6x_2 &\,+\,&  4x_3 &\,=\,&  15
   		\end{alignedat}
	\]
	\end{enumerate}
    \end{multicols}
  \begin{subactivity}
  	Find a solution to one of these systems.
	\end{subactivity}
  \begin{subactivity}
    Rank the six linear systems from most complicated to simplest.
  \end{subactivity}
\end{activity}


\begin{activity}{5}
We can rewrite the previous in terms of  equivalences of augmented matrices
\[
	\begin{alignedat}{3}
		\begin{bmatrix}[ccc|c] -2 & 4 & -2 & -8 \\ 1 & -2 & 2 & 7 \\ 3 & -6 & 4 & 15 \end{bmatrix} & \sim &
		\begin{bmatrix}[ccc|c] \circledNumber{1} & -2 & 2 & 7 \\ -2 & 4 & -2 & -8 \\ 3 & -6 & 4 & 15 \end{bmatrix} &\sim&
		\begin{bmatrix}[ccc|c] \circledNumber{1} & -2 & 2 & 7 \\ 0 & 0 & 2 & 6 \\ 3 & - 6 & 4 & 15 \end{bmatrix} \\
 \sim
		\begin{bmatrix}[ccc|c] \circledNumber{1} & -2 & 2 & 7 \\ 0 & 0 & 2 & 6 \\ 0 & 0 & -2 & -6 \end{bmatrix}  &\sim&
		\begin{bmatrix}[ccc|c] \circledNumber{1} & -2 & 2 & 7 \\ 0 & 0 & \circledNumber{1} & 3 \\ 0 & 0 & -2 & -6 \end{bmatrix}  &\sim&
		\begin{bmatrix}[ccc|c] \circledNumber{1} & -2 & 0 & 1 \\ 0 & 0 & \circledNumber{1} & 3 \\ 0 & 0 & 0 & 0 \end{bmatrix}
 	\end{alignedat}
\]

		Determine the row operation(s) necessary in each step to transform the most complicated
    system's augmented matrix into the simplest.

\end{activity}

\begin{activity}{10}
A matrix is in \term{reduced row echelon form} (\term{RREF}) if
\begin{enumerate}
\item The leading term (first nonzero term) of each nonzero row is a 1.
      Call these terms \term{pivots}.
\item Each pivot is to the right of every higher pivot.
\item Each term above or below a pivot is zero.
\item All rows of zeroes are at the bottom of the matrix.
\end{enumerate}

Circle the leading terms in each example, and label it as RREF or not RREF.
\begin{multicols}{3}
\begin{enumerate}[(A)]
\item \[ \begin{bmatrix}[ccc|c] 1 & 0 & 0 & 3 \\ 0 & 0 & 1 & -1 \\ 0 & 0 & 0 & 0 \end{bmatrix}\]
\item \[ \begin{bmatrix}[ccc|c] 1 & 2 & 4 & 3 \\ 0 & 0 & 1 & -1 \\ 0 & 0 & 0 & 0 \end{bmatrix}\]
\item \[ \begin{bmatrix}[ccc|c] 0 & 0 & 0 & 0 \\ 1 & 2 & 0 & 3 \\ 0 & 0 & 1 & -1  \end{bmatrix}\]
\item \[ \begin{bmatrix}[ccc|c] 1 & 0 & 2 & -3 \\ 0 & 3 & 3 & -3 \\ 0 & 0 & 0 & 0 \end{bmatrix}\]
\item \[ \begin{bmatrix}[ccc|c] 0 & 1 & 0 & 7 \\ 1 & 0 & 0 & 4 \\ 0 & 0 & 0 & 0 \end{bmatrix}\]
\item \[ \begin{bmatrix}[ccc|c] 1 & 0 & 0 & 4 \\ 0 & 1 & 0 & 7 \\ 0 & 0 & 1 & 0 \end{bmatrix}\]
\end{enumerate}
\end{multicols}
\end{activity}

\begin{remark}
It is important to understand the \term{Gauss-Jordan elimination} algorithm
that converts a matrix into reduced row echelon form.

\vspace{0.5in}

A video outlining how to perform the Gauss-Jordan Elimination algorithm
by hand is available at \url{https://youtu.be/Cq0Nxk2dhhU}.
Practicing several exercises outside of class using this method is
recommended.

\vspace{0.5in}

In the next section, we will learn to use technology to perform this
operation for us, as will be expected when applying row-reduced
matrices to solve other problems.
\end{remark}




\end{applicationActivities}
