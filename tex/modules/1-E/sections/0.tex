%!TEX root =../../../course-notes.tex
% ^ leave for LaTeXTools build functionality

\begin{applicationActivities}

\begin{definition}
A \term{linear equation} is an equation of the variables \(x_i\) of the form
\[
a_1x_1+a_2x_2+\dots+a_nx_n=b
.\]
A \term{solution} for a linear equation is a Euclidean vector
\[
  \begin{bmatrix}
    s_1 \\
    s_2 \\
    \vdots \\
    s_n
  \end{bmatrix}
\]
that satisfies
\[
a_1s_1+a_2s_2+\dots+a_ns_n=b
\]
(that is, a Euclidean vector that can be plugged into the equation).
\end{definition}



\begin{remark}
In previous classes you likely used the variables \(x,y,z\) in equations.
However, since this course often deals with equations of four or more
variables, we will often write our variables as \(x_i\), and assume
\(x=x_1,y=x_2,z=x_3,w=x_4\) when convenient.
\end{remark}

\begin{definition}
A \term{system of linear equations} (or a \term{linear system} for short)
is a collection of one or more linear equations.
  \begin{alignat*}{5}
    a_{11}x_1 &\,+\,& a_{12}x_2 &\,+\,& \dots  &\,+\,& a_{1n}x_n &\,=\,& b_1 \\
    a_{21}x_1 &\,+\,& a_{22}x_2 &\,+\,& \dots  &\,+\,& a_{2n}x_n &\,=\,& b_2 \\
     \vdots&  &\vdots&   &&  &\vdots&&\vdots  \\
    a_{m1}x_1 &\,+\,& a_{m2}x_2 &\,+\,& \dots  &\,+\,& a_{mn}x_n &\,=\,& b_m
  \end{alignat*}
Its \term{solution set} is given by
\[
  \setBuilder
  {
    \begin{bmatrix}
      s_1 \\
      s_2 \\
      \vdots \\
      s_n
    \end{bmatrix}
  }{
    \begin{bmatrix}
      s_1 \\
      s_2 \\
      \vdots \\
      s_n
    \end{bmatrix}
    \text{ is a solution to all equations in the system}
  }
.\]
\end{definition}

\begin{remark}
  When variables in a large linear system are missing, we prefer to
  write the system in one of the following standard forms:

  \begin{multicols}{3}\noindent
    Original linear system:
    \begin{alignat*}{2}
       x_1 + 3x_3 &\,=\,& 3 \\
      3x_1 - 2x_2 + 4x_3 &\,=\,& 0 \\
      -x_2 +  x_3 &\,=\,& -2
    \end{alignat*}
    Verbose standard form:
    \begin{alignat*}{4}
      1x_1 &\,+\,& 0x_2 &\,+\,& 3x_3 &\,=\,& 3 \\
      3x_1 &\,-\,& 2x_2 &\,+\,& 4x_3 &\,=\,& 0 \\
      0x_1 &\,-\,& 1x_2 &\,+\,& 1x_3 &\,=\,& -2
    \end{alignat*}
    Concise standard form:
    \begin{alignat*}{4}
       x_1 &     &      &\,+\,& 3x_3 &\,=\,& 3 \\
      3x_1 &\,-\,& 2x_2 &\,+\,& 4x_3 &\,=\,& 0 \\
           &\,-\,&  x_2 &\,+\,&  x_3 &\,=\,& -2
    \end{alignat*}
  \end{multicols}
\end{remark}

\begin{definition}
  A linear system is \term{consistent} if its solution set
  is non-empty (that is, there exists a solution for the
  system). Otherwise it is \term{inconsistent}.
\end{definition}

\begin{fact}
  All linear systems are one of the following:
  \begin{itemize}
    \item \textbf{Consistent with one solution}:
          its solution set contains a single vector, e.g.
          \(\setList{\begin{bmatrix}1\\2\\3\end{bmatrix}}\)
    \item \textbf{Consistent with infinitely-many solutions}:
          its solution set contains infinitely many vectors, e.g.
          \(
            \setBuilder
            {
              \begin{bmatrix}1\\2-3a\\a\end{bmatrix}
            }{
              a\in\IR
            }
          \)
    \item \textbf{Inconsistent}:
          its solution set is the empty set \(\{\}=\emptyset\)
  \end{itemize}
\end{fact}

\begin{activity}{10}
  All inconsistent linear systems contain a logical \term{contradiction}.
  Find a contradiction in this system to show that its solution set
  is \(\emptyset\).
  \begin{align*}
  -x_1+2x_2  &=  5 \\
  2x_1-4x_2  &=  6
  \end{align*}
\end{activity}

\begin{activity}{10}
  Consider the following consistent linear system.
  \begin{align*}
  -x_1+2x_2  &= -3 \\
  2x_1-4x_2  &=  6
  \end{align*}
\begin{subactivity}
  Find three different solutions
  for this system.
\end{subactivity}
\begin{subactivity}
  Let \(x_2=a\) where \(a\) is an arbitrary real number, then find an
  expression for \(x_1\) in terms of \(a\). Use this to write
  the solution set
  \(
    \setBuilder
    {
    	\begin{bmatrix}
        \unknown \\
        a
      \end{bmatrix}
    }{
      a \in \IR
	  }
  \)
  for the linear system.
\end{subactivity}
\end{activity}

\begin{activity}{10}
  Consider the following linear system.
  \begin{alignat*}{5}
    x_1 &\,+\,& 2x_2 &\, \,&     &\,-\,&  x_4 &\,=\,& 3 \\
        &\, \,&      &\, \,& x_3 &\,+\,& 4x_4 &\,=\,& -2
  \end{alignat*}
  Describe the solution set
  \[
  	\setBuilder
    {
    	\begin{bmatrix}
          \unknown \\
          a \\
          \unknown \\
          b
    	\end{bmatrix}
    }{
      a,b \in \IR
    }
  \]
  to the linear system
  by setting \(x_2=a\) and \(x_4=b\), and then solving for \(x_1\) and
  \(x_3\).
\end{activity}

\begin{observation}
  Solving linear systems of two variables by graphing or substitution is
  reasonable for two-variable systems, but these simple techniques
  won't usually cut it for equations with
  more than two variables or more than two equations. For example,
  \begin{alignat*}{5}
    -2x_1 &\,-\,& 4x_2 &\,+\,&  x_3 &\,-\,&  4x_4 &\,=\,& -8 \\
      x_1 &\,+\,& 2x_2 &\,+\,& 2x_3 &\,+\,& 12x_4 &\,=\,& -1 \\
      x_1 &\,+\,& 2x_2 &\,+\,&  x_3 &\,+\,&  8x_4 &\,=\,&  1 \\
  \end{alignat*}
  has the exact same solution set as the system in the previous
  activity, but we'll want to learn new techniques
  to compute these solutions efficiently.
\end{observation}
\end{applicationActivities}
