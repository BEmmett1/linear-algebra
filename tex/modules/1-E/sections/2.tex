%!TEX root =../../../course-notes.tex
% ^ leave for LaTeXTools build functionality

\begin{applicationActivities}


\begin{activity}{10}
Free browser-based technologies for mathematical computation
are available online.
\begin{itemize}
\item Go to \url{https://octave-online.net}.
\item Type \texttt{A=sym([1 3 4 ; 2 5 7])} and press \texttt{Enter}
      to store the matrix
      \(\begin{bmatrix} 1 & 3 & 2 \\ 2 & -9 & -1 \end{bmatrix}\)
      in the variable \(A\).
  \begin{itemize}
    \item The symbolic form is used to calculate precise answers 
          rather than floating-point approximations.
    \item The vertical bar in an augmented matrix does not affect
          row operations, so it does not need to be entered into
          a calculator.
  \end{itemize}
\item Type \texttt{rref(A)} and press \texttt{Enter}
      to compute the reduced row echelon form of \(A\).
\end{itemize}
\end{activity}

\begin{remark}
We will frequently need to know the reduced row echelon form of matrices 
during class, so feel free to use Octave-Online.net to compute RREF
efficiently.

\vspace{1em}

You may alternatively use the calculator you will use during assessments.
Be sure to use fractions mode to compute exact solutions rather
than floating-point approximations. 
\end{remark}

\begin{activity}{10}
Consider the system of equations.
 \[
		\begin{alignedat}{4}
   		  3x_1 &\,-\,& 2x_2 &\,+\,& 13x_3 &\,=\,& 6 \\
   		  2x_1 &\,-\,& 2x_2 &\,+\,& 10x_3 &\,=\,& 2 \\
   		  -x_1 &\,+\,& 3x_2 &\,-\,&  6x_3 &\,=\,& 11
   		\end{alignedat}
\]

\begin{subactivity}
Convert this to an augmented matrix and use technology to compute its
reduced row echelon form:
\[
  \RREF
  \begin{bmatrix}[ccc|c]
    \unknown&\unknown&\unknown&\unknown\\ 
    \unknown&\unknown&\unknown&\unknown\\ 
    \unknown&\unknown&\unknown&\unknown\\ 
  \end{bmatrix}
=
  \begin{bmatrix}[ccc|c]
    \unknown&\unknown&\unknown&\unknown\\ 
    \unknown&\unknown&\unknown&\unknown\\ 
    \unknown&\unknown&\unknown&\unknown\\ 
  \end{bmatrix}
\]
\end{subactivity}
\begin{subactivity}
Use the \(\RREF\) matrix to write a linear system equivalent
to the original system. Then find its solution set.
\end{subactivity}
\end{activity}

\begin{activity}{10}
Consider the system of equations.
 \[
		\begin{alignedat}{4}
   		  3x_1 &\,-\,& 2x_2 &\,+\,& 13x_3 &\,=\,& 6 \\
   		  2x_1 &\,-\,& 2x_2 &\,+\,& 10x_3 &\,=\,& 2 \\
   		  -x_1 &\,\,&  &\,-\,&  3x_3 &\,=\,&1
   		\end{alignedat}
\]

\begin{subactivity}
Convert this to an augmented matrix and use technology to compute its
reduced row echelon form:
\[
  \RREF
  \begin{bmatrix}[ccc|c]
    \unknown&\unknown&\unknown&\unknown\\ 
    \unknown&\unknown&\unknown&\unknown\\ 
    \unknown&\unknown&\unknown&\unknown\\ 
  \end{bmatrix}
=
  \begin{bmatrix}[ccc|c]
    \unknown&\unknown&\unknown&\unknown\\ 
    \unknown&\unknown&\unknown&\unknown\\ 
    \unknown&\unknown&\unknown&\unknown\\ 
  \end{bmatrix}
\]
\end{subactivity}
\begin{subactivity}
Use the \(\RREF\) matrix to write a linear system equivalent
to the original system. Then find its solution set.
\end{subactivity}
\end{activity}

\begin{activity}{10}
  Consider the following linear system.
  	\begin{alignat*}{4}
  		x_1 &+ 2x_2 &+ 3x_3 &= 1\\
  	  2x_1 &+ 4x_2 &+ 8x_3 &= 0
  	\end{alignat*}
  \begin{subactivity}
    Find its corresponding augmented matrix \(A\) and
    use technology to find \(\RREF(A)\).
  \end{subactivity}
  \begin{subactivity}
    How many solutions do these linear systems have?
  \end{subactivity}
\end{activity}

\begin{activity}{10}
Consider the simple linear system equivalent to the system
from the previous activity:
	\begin{alignat*}{3}
		x_1 &+ 2x_2 & &= 4\\
	     	 & &\phantom{+}x_3 &= -1
	\end{alignat*}
\begin{subactivity}
Let \(x_1=a\) and write the solution set in the form
\(
  \setBuilder
  {
    \begin{bmatrix} a \\ \unknown \\ \unknown \end{bmatrix}
  }{
    a \in \IR
  }
\).
\end{subactivity}
\begin{subactivity}
Let \(x_2=b\) and write the solution set in the form
\(
  \setBuilder
  {
    \begin{bmatrix} \unknown \\ b \\ \unknown \end{bmatrix}
  }{
    b \in \IR
  }
\).
\end{subactivity}
\begin{subactivity}
Which of these was easier?  What features of the RREF matrix
\(
  \begin{bmatrix}[ccc|c]
    \circledNumber{1} & 2 & 0 & 4 \\
    0 & 0 & \circledNumber{1} & -1
  \end{bmatrix}
\) caused this?
\end{subactivity}
\end{activity}

\begin{definition}
Recall that the pivots of a matrix in \(\RREF\) form are the leading
\(1\)s in each non-zero row.

\vspace{0.2in}

The pivot columns in an augmented matrix correspond to the
\term{bound variables} in the system of equations (\(x_1,x_3\) below).
The remaining variables are called \term{free variables} (\(x_2\) below).

\[
  \begin{bmatrix}[ccc|c]
    \circledNumber{1} & 2 & 0 & 4 \\
    0 & 0 & \circledNumber{1} & -1
  \end{bmatrix}
\]

To efficiently solve a system in RREF form, assign letters to the free
variables, and then solve for the bound variables.
\end{definition}

\begin{activity}{10}
Find the solution set for the system
\begin{alignat*}{6}
2x_1&\,-\,&2x_2&\,-\,&6x_3&\,+\,&x_4&\,-\,&x_5&\,=\,&3 \\
-x_1&\,+\,&x_2&\,+\,&3x_3&\,-\,&x_4&\,+\,&2x_5 &\,=\,& -3 \\
x_1&\,-\,&2x_2&\,-\,&x_3&\,+\,&x_4&\,+\,&x_5 &\,=\,& 2
\end{alignat*}
by row-reducing its augmented matrix, and then
assigning letters to the free variables (given by non-pivot columns)
and solving for the bound variables (given by pivot columns) in
the corresponding linear system.
\end{activity}

\begin{observation}
The solution set to the system
\begin{alignat*}{6}
2x_1&\,-\,&2x_2&\,-\,&6x_3&\,+\,&x_4&\,-\,&x_5&\,=\,&3 \\
-x_1&\,+\,&x_2&\,+\,&3x_3&\,-\,&x_4&\,+\,&2x_5 &\,=\,& -3 \\
x_1&\,-\,&2x_2&\,-\,&x_3&\,+\,&x_4&\,+\,&x_5 &\,=\,& 2
\end{alignat*}
may be written as
\[
  \setBuilder
  {
    \begin{bmatrix}
      1+5a+2b \\
      1+2a+3b \\
      a \\
      3+3b \\
      b
    \end{bmatrix}
  }{
    a,b\in \IR
  }
.\]
\end{observation}

\begin{remark}
Don't forget to correctly express the solution set of a linear system,
using set-builder notation for consistent systems with
infintely many solutions.
  \begin{itemize}
  \item \textbf{Consistent with one solution}: e.g.
        \(\setList{ \begin{bmatrix}1\\2\\3\end{bmatrix} }\)
  \item \textbf{Consistent with infinitely-many solutions}: e.g.
        \(
          \setBuilder
          {
            \begin{bmatrix}1\\2-3a\\a\end{bmatrix}
          }{
            a\in\IR
          }
        \)
  \item \textbf{Inconsistent}: \(\emptyset\) or \(\{\}\)
  \end{itemize}
\end{remark}

\end{applicationActivities}
