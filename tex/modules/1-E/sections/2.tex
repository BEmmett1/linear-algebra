%!TEX root =../../../course-notes.tex
% ^ leave for LaTeXTools build functionality

\begin{applicationActivities}


\begin{activity}{10}
Free browser-based technologies for mathematical computation
are available online.
\begin{itemize}
\item Go to \url{http://www.cocalc.com} and create an account.
\item Create a project titled ``Linear Algebra Team X'' with your appropriate
      team number.  Add all team members as collaborators.
\item Open the project and click on ``New''
\item Give it an appropriate name such as ``Class E.2 workbook''.  Make a new
      Jupyter notebook.
\item Click on ``Kernel'' and make sure ``Octave'' is selected.
\item Type \texttt{A=[1 3 4 ; 2 5 7]} and press \texttt{Shift+Enter}
      to store the matrix
      \(\begin{bmatrix} 1 & 3 & 4 \\ 2 & 5 & 7\end{bmatrix}\)
      in the variable $A$.
\item Type \texttt{rref(A)} and press \texttt{Shift+Enter}
      to compute the reduced row echelon form of $A$.
\end{itemize}
\end{activity}

\begin{remark}
If you need to find the reduced row echelon form of a matrix during class, you
are encouraged to use CoCalc's Octave interpreter.

You can change a cell from ``Code'' to ``Markdown'' or ``Raw'' to put comments
around your calculations such as Activity numbers.
\end{remark}

\begin{activity}{10}
Consider the system of equations.
 \[
		\begin{alignedat}{4}
   		  3x_1 &\,-\,& 2x_2 &\,+\,& 13x_3 &\,=\,& 6 \\
   		  2x_1 &\,-\,& 2x_2 &\,+\,& 10x_3 &\,=\,& 2 \\
   		  -x_1 &\,+\,& 3x_2 &\,-\,&  6x_3 &\,=\,& 11
   		\end{alignedat}
\]

Convert this to an augmented matrix and use CoCalc to compute its
reduced row echelon form. Write these on your whiteboard, and use
them to write a simpler yet equivalent linear system of equations.
Then find its solution set.
\end{activity}

\begin{activity}{10}
Consider our system of equations from above.
 \[
		\begin{alignedat}{4}
   		  3x_1 &\,-\,& 2x_2 &\,+\,& 13x_3 &\,=\,& 6 \\
   		  2x_1 &\,-\,& 2x_2 &\,+\,& 10x_3 &\,=\,& 2 \\
   		  -x_1 &\,\,&  &\,-\,&  3x_3 &\,=\,&1
   		\end{alignedat}
\]

Convert this to an augmented matrix and use CoCalc to compute its
reduced row echelon form. Write these on your whiteboard, and use
them to write a simpler yet equivalent linear system of equations.
Then find its solution set.
\end{activity}

\begin{activity}{10}
  Consider the following linear system.
  	\begin{alignat*}{4}
  		x_1 &+ 2x_2 &+ 3x_3 &= 1\\
  	  2x_1 &+ 4x_2 &+ 8x_3 &= 0
  	\end{alignat*}
  \begin{subactivity}
    Find its corresponding augmented matrix \(A\) and
    use CoCalc to find \(\RREF(A)\).
  \end{subactivity}
  \begin{subactivity}
    How many solutions does the corresponding linear system have?
  \end{subactivity}
\end{activity}

\begin{activity}{10}
Consider the simple linear system equivalent to the system
from the previous problem:
	\begin{alignat*}{3}
		x_1 &+ 2x_2 & &= 4\\
	     	 & &x_3 &= -1
	\end{alignat*}
\begin{subactivity}
Let \(x_1=a\) and write the solution set in the form
\(
  \setBuilder
  {
    \begin{bmatrix} a \\ \unknown \\ \unknown \end{bmatrix}
  }{
    a \in \IR
  }
\).
\end{subactivity}
\begin{subactivity}
Let \(x_2=b\) and write the solution set in the form
\(
  \setBuilder
  {
    \begin{bmatrix} \unknown \\ b \\ \unknown \end{bmatrix}
  }{
    b \in \IR
  }
\).
\end{subactivity}
\begin{subactivity}
Which of these was easier?  What features of the RREF matrix
\(
  \begin{bmatrix}[ccc|c]
    \circledNumber{1} & 2 & 0 & 4 \\
    0 & 0 & \circledNumber{1} & -1
  \end{bmatrix}
\) caused this?
\end{subactivity}
\end{activity}

\begin{definition}
Recall that the pivots of a matrix in \(\RREF\) form are the leading
\(1\)s in each non-zero row.

The pivot columns in an augmented matrix correspond to the
\term{bound variables} in the system of equations.
The remaining variables are called \term{free variables}.

To efficiently solve a system in RREF form, we may assign letters to free
variables and solve for the bound variables.
\end{definition}

\begin{activity}{10}
Find the solution set for the system
\begin{alignat*}{6}
2x_1&\,-\,&2x_2&\,-\,&6x_3&\,+\,&x_4&\,-\,&x_5&\,=\,&3 \\
-x_1&\,+\,&x_2&\,+\,&3x_3&\,-\,&x_4&\,+\,&2x_5 &\,=\,& -3 \\
x_1&\,-\,&2x_2&\,-\,&x_3&\,+\,&x_4&\,+\,&x_5 &\,=\,& 2
\end{alignat*}
by assigning letters to the free variables and solving for the bound variables
in the simplified system given by row-reducing its augmented matrix.
\end{activity}

\begin{observation}
The solution set to the system
\begin{alignat*}{6}
2x_1&\,-\,&2x_2&\,-\,&6x_3&\,+\,&x_4&\,-\,&x_5&\,=\,&3 \\
-x_1&\,+\,&x_2&\,+\,&3x_3&\,-\,&x_4&\,+\,&2x_5 &\,=\,& -3 \\
x_1&\,-\,&2x_2&\,-\,&x_3&\,+\,&x_4&\,+\,&x_5 &\,=\,& 2
\end{alignat*}
may be written as
\[
  \setBuilder
  {
    \begin{bmatrix}
      1+5a+2b \\
      1+2a+3b \\
      a \\
      3+3b \\
      b
    \end{bmatrix}
  }{
    a,b\in \IR
  }
.\]
\end{observation}

\begin{remark}
Don't forget to correctly express the solution set of a linear system,
using set-builder notation for consistent systems with
infintely many solutions.
  \begin{itemize}
  \item \textbf{Consistent with one solution}: e.g.
        \(\setList{ \begin{bmatrix}1\\2\\3\end{bmatrix} }\)
  \item \textbf{Consistent with infinitely-many solutions}: e.g.
        \(
          \setBuilder
          {
            \begin{bmatrix}1\\2-3a\\a\end{bmatrix}
          }{
            a\in\IR
          }
        \)
  \item \textbf{Inconsistent}: \(\emptyset\)
  \end{itemize}
\end{remark}

\end{applicationActivities}
