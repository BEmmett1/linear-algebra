\begin{applicationActivities}


\begin{activity}{10}
\begin{itemize}
\item Go to {\tt http://www.cocalc.com} and create an account.
\item Create a project titled ``Linear Algebra Team X'' with your appropriate team number.  Add all team members as collaborators.
\item Open the project and click on ``New''
\item Give it an appropriate name such as ``Class E2 workbook''.  Make a new Jupyter notebook.
\item Click on ``Kernel'' and make sure ``Octave'' is selected.
\item Type {\tt A=[1 3 4 ; 2 5 7]} to store the matrix $\begin{bmatrix} 1 & 3 & 4 \\ 2 & 5 & 7\end{bmatrix}$ in the variable $A$; hold shift when you press enter.
\item Type {\tt rref(A)} to compute the reduced row echelon form of $A$.
\end{itemize}
\end{activity}

\begin{remark}
If you need to find the reduced row echelon form of a matrix during class, you should feel free to use CoCalc/Octave.
\ \\
\ \\
You can change a cell from ``Code'' to ``Markdown'' or ``Raw'' to put comments around your calculations such as Activity numbers.
\end{remark}

\begin{activity}{8}
Consider the system of equations.
 \[
		\begin{alignedat}{4}
   		  3x_1 &\,-\,& 2x_2 &\,+\,& 13x_3 &\,=\,& 6 \\
   		  2x_1 &\,-\,& 2x_2 &\,+\,& 10x_3 &\,=\,& 2 \\
   		  -x_1 &\,+\,& 3x_2 &\,-\,&  6x_3 &\,=\,& 11
   		\end{alignedat}
\]

Convert this to an augmented matrix, use CoCalc to compute the reduced row echelon form, and convert back to a simpler system of equations to solve this system.  Write your solution on your whiteboard.
\end{activity}

\begin{activity}{7}
Consider our system of equations from above.
 \[
		\begin{alignedat}{4}
   		  3x_1 &\,-\,& 2x_2 &\,+\,& 13x_3 &\,=\,& 6 \\
   		  2x_1 &\,-\,& 2x_2 &\,+\,& 10x_3 &\,=\,& 2 \\
   		  -x_1 &\,\,&  &\,-\,&  3x_3 &\,=\,&1
   		\end{alignedat}
\]

Convert this to an augmented matrix, use CoCalc to compute the reduced row echelon form, and convert back to a simpler system of equations to solve this system.  Write your solution on your whiteboard.
\end{activity}

\begin{activity}{10}
  Consider the following matrix.
  \[
    A = \begin{bmatrix}[ccc|c]
      1 & 2 & 3 & 1\\
      2 & 4 & 8 & 0
    \end{bmatrix}
  \]
  \begin{subactivity}
    Find \(\RREF(A)\) (Use CoCalc).
  \end{subactivity}
  \begin{subactivity}
    How many solutions does the corresponding linear system have?
  \end{subactivity}
\end{activity}

\begin{activity}{10}
Consider the (simpler) system from the previous problem:
	\begin{alignat*}{3}
		x_1 &+ 2x_2 & &= 4\\
	     	 & &x_3 &= -1
	\end{alignat*}
\begin{subactivity}
Let $x_1=a$ and write the solution set in the form
\( \left\{ \begin{bmatrix} a \\ ? \\ ? \end{bmatrix} \,\middle|\, a \in \IR \right\} \)
\end{subactivity}
\begin{subactivity}
Let $x_2=b$ and write the solution set in the form
\( \left\{ \begin{bmatrix} ? \\ b \\ ? \end{bmatrix} \,\middle|\, b \in \IR \right\} \)
\end{subactivity}
\begin{subactivity}
Which of these was easier?  What features of the RREF matrix \[\begin{bmatrix}[ccc|c] 1 & 2 & 0 & 4 \\ 0 & 0 & 1 & -1 \end{bmatrix}\] cause this?
\end{subactivity}
\end{activity}

\begin{definition}
If a matrix is in reduced row echelon form, a \term{pivot} is an entry satisfying
\begin{enumerate}[1.]
\item It is $1$
\item Everything else in the same row but to the left of it is zero
\item Everything else in the same column is zero.
\end{enumerate}

For example, the pivots are circled in

\[\begin{bmatrix}[ccc|c] \circledNumber{1} & 2 & 0 & 4 \\ 0 & 0 & \circledNumber{1} & -1 \end{bmatrix}\]
\end{definition}


\begin{activity}{5}
Circle the pivots in each matrix below.
\begin{multicols}{2}
\[ \begin{bmatrix}[cccc|c] 1 & 1 & 0 & 0 & 2 \\ 0 & 0 & 1 & 0 &  0 \\ 0 & 0 & 0 &  1 & 1 \end{bmatrix} \]
\[ \begin{bmatrix}[cccc|c] 1 & 1 & 0 & 1 & 0 \\ 0 & 0 & 1 & 1 & 0 \\ 0 & 0 & 0 & 0  & 0 \end{bmatrix} \]
\[ \begin{bmatrix}[ccc|c] 1 & 0 & 1 & 2 \\ 0 & 1 & 0 & 1 \end{bmatrix} \]
\[ \begin{bmatrix}[ccc|c] 1 & 0 & 0 & 2 \\ 0 & 1 & 1 & 0 \\ 0 & 0 & 0 & 0 \end{bmatrix} \]
\end{multicols}
\end{activity}

\begin{definition}
The pivots in a matrix correspond to \term{bound variables} in the system of equations.  The remaining variables are called \term{free variables}.

To efficiently solve a system in RREF form, assign letters to free variables and solve for the bound variables.
\end{definition}

\begin{activity}{10}
Find the solution set for the system
\begin{alignat*}{6}
2x_1&\,-\,&2x_2&\,-\,&6x_3&\,+\,&x_4&\,-\,&x_5&\,=\,&3 \\
-x_1&\,+\,&x_2&\,+\,&3x_3&\,-\,&x_4&\,+\,&2x_5 &\,=\,& -3 \\
x_1&\,-\,&2x_2&\,-\,&x_3&\,+\,&x_4&\,+\,&x_5 &\,=\,& 2
\end{alignat*}
by assigning letters to the free variables and solving for the bounded variables.
\end{activity}

\begin{observation}
The solution set to the system
\begin{alignat*}{6}
2x_1&\,-\,&2x_2&\,-\,&6x_3&\,+\,&x_4&\,-\,&x_5&\,=\,&3 \\
-x_1&\,+\,&x_2&\,+\,&3x_3&\,-\,&x_4&\,+\,&2x_5 &\,=\,& -3 \\
x_1&\,-\,&2x_2&\,-\,&x_3&\,+\,&x_4&\,+\,&x_5 &\,=\,& 2
\end{alignat*}
is \[\left\{ \begin{bmatrix} 1+5a+2b \\ 1+2a+3b \\ a \\3+3b \\ b \end{bmatrix}\,\middle|\, a,b\in \IR\right\}.\]
\end{observation}

\begin{remark}
You should always use set-builder notation to describe the solution set of a linear system.
\end{remark}

\end{applicationActivities}
