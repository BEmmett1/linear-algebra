%!TEX root =../../../course-notes.tex
% ^ leave for LaTeXTools build functionality
\begin{applicationActivities}
\begin{remark}
  Last time, we defined a \textbf{vector space} \(V\) to be any collection of
  mathematical objects with
  associated addition and scalar multiplication operations that satisfy
  the following properties. Let \(\vect u,\vect v,\vect w\) belong to \(V\),
  and let \(a,b\) be scalar numbers.

  \vectorSpaceProperties
\end{remark}

\begin{remark}
  The following sets are examples of vector spaces, with the usual/natural
  operations for addition and scalar multiplication.
  \begin{itemize}
    \item \(\IR^n\): Euclidean vectors with \(n\) components.
    \item \(\IR^\infty\): Sequences of real numbers \((v_1,v_2,\dots)\).
    \item \(M_{m,n}\): Matrices of real numbers with \(m\) rows and
          \(n\) columns.
    \item \(\IC\): Complex numbers.
    \item \(\P^n\): Polynomials of degree \(n\) or less.
    \item \(\P\): Polynomials of any degree.
    \item \(C(\IR)\): Real-valued continuous functions.
  \end{itemize}
\end{remark}


\begin{observation}
  Consider the following set that models motion along the curve
  \(y=e^x\). Let \(V=\setBuilder{(x,y)}{y=e^x}\) have operations defined by
  \[
    (x,y)\oplus (z,w)=(x+z,yw)
      \hspace{3em}
    c\odot (x,y)=(cx,y^c)
  .\]

  It may be proven that \(V\) satisifes all eight properties of a vector space.
    \begin{multicols}{2}
    \begin{itemize}
      \item \textbf{Addition associativity.}

            \(\vect u\oplus(\vect v\oplus\vect w)=
            (\vect u\oplus\vect v)\oplus\vect w\).
      \item \textbf{Addition commutivity.}

            \(\vect u\oplus\vect v=
            \vect v\oplus\vect u\).
      \item \textbf{Addition identity.}

            There exists some \(\vect z\)
            where \(\vect v\oplus\vect z=\vect v\).
      \item \textbf{Addition inverse.}

            There exists some \(-\vect v\)
            where \(\vect v\oplus(-\vect v)=\vect z\).
      \item \textbf{Scalar multiplication associativity.}

            \(a\odot(b\odot\vect v)=(ab)\odot\vect v\).
      \item \textbf{Scalar multiplication identity.}

            \(1\odot\vect v=\vect v\).
      \item \textbf{Scalar distribution.}

            \(a\odot(\vect u\oplus\vect v)=
            (a\odot\vect u)\oplus (a\odot\vect v)\).
      \item \textbf{Vector distribution.}

            \((a+b)\odot\vect v=(a\odot\vect v)\oplus (b\odot\vect v)\).
    \end{itemize}
    \end{multicols}
\end{observation}


\begin{activity}{10}
  Show that \(V=\setBuilder{(x,y)}{y=e^x}\) with operations defined by
  \[
    (x,y)\oplus (z,w)=(x+z,yw)
      \hspace{3em}
    c\odot (x,y)=(cx,y^c)
  \]
  satisfies the distribution rule
  \[(a+b)\odot\vect v=(a\odot\vect v)\oplus (b\odot\vect v)\]
  by letting \(\vec v=(x,y)\) and simplifying both sides.
\end{activity}


\begin{activity}{15}
  Let \(V=\setBuilder{(x,y)}{x,y\in\IR}\) have operations defined by
  \[
    (x,y)\oplus (z,w)=(x+y+z+w,x^2+z^2)
      \hspace{3em}
    c\odot (x,y)=(x^c,y+c-1)
  .\]

  \begin{subactivity}
    Show that the scalar multiplication identity holds by
    simplifying \(1\odot(x,y)\) to \((x,y)\).
  \end{subactivity}

  \begin{subactivity}
    Show that the addition identity property fails by
    showing that \((0,-1)\oplus\vec z=(0,-1)\oplus(z_1,z_2)\)
    can never equal \((0,-1)\).
  \end{subactivity}

  \begin{subactivity}
    Can \(V\) be a vector space?
  \end{subactivity}
\end{activity}

\begin{definition}
  A \term{linear combination} of a set of vectors
  \(\{\vect v_1,\vect v_2,\dots,\vect v_m\}\) is given by
  \(c_1\vect v_1+c_2\vect v_2+\dots+c_m\vect v_m\) for any choice of
  scalar multiples \(c_1,c_2,\dots,c_m\).

	\vspace{2em}

  For example, we can say $\begin{bmatrix}3 \\0 \\ 5\end{bmatrix}$ is a linear combination of the vectors $\begin{bmatrix} 1 \\ -1 \\ 2 \end{bmatrix}$ and $\begin{bmatrix} 1 \\ 2 \\ 1 \end{bmatrix}$ since $$\begin{bmatrix} 3 \\ 0 \\ 5 \end{bmatrix} = 2 \begin{bmatrix} 1 \\ -1 \\ 2 \end{bmatrix} + 1\begin{bmatrix} 1 \\ 2 \\ 1 \end{bmatrix}$$
\end{definition}

\begin{definition}
  The \term{span} of a set of vectors is the collection of all linear
  combinations of that set:
  \[
    \vspan\{\vect v_1,\vect v_2,\dots,\vect v_m\} =
    \setBuilder{c_1\vect v_1+c_2\vect v_2+\dots+c_m\vect v_m}{
    c_i\in\IR}.
  \]
\end{definition}

\begin{activity}{10}
  Consider \(\vspan\left\{\begin{bmatrix}1\\2\end{bmatrix}\right\}\).
  \begin{subactivity}
    Sketch
    \(1\begin{bmatrix}1\\2\end{bmatrix}\),
    \(3\begin{bmatrix}1\\2\end{bmatrix}\),
    \(0\begin{bmatrix}1\\2\end{bmatrix}\),
    and \(-2\begin{bmatrix}1\\2\end{bmatrix}\) in the \(xy\) plane.
  \end{subactivity}
  \begin{subactivity}
    Sketch a representation of all the vectors belonging to
    \(\vspan\left\{\begin{bmatrix}1\\2\end{bmatrix}\right\}\)
    in the \(xy\) plane.
  \end{subactivity}
\end{activity}





\begin{activity}{10}
  Consider \(\vspan\left\{\begin{bmatrix}1\\2\end{bmatrix},
  \begin{bmatrix}-1\\1\end{bmatrix}\right\}\).
  \begin{subactivity}
    Sketch the following linear combinations in the \(xy\) plane.
    \[
    1\begin{bmatrix}1\\2\end{bmatrix}+
    0\begin{bmatrix}-1\\1\end{bmatrix}\hspace{3em}
    0\begin{bmatrix}1\\2\end{bmatrix}+
    1\begin{bmatrix}-1\\1\end{bmatrix}\hspace{3em}
    1\begin{bmatrix}1\\2\end{bmatrix}+
    1\begin{bmatrix}-1\\1\end{bmatrix}
    \]
    \[
    2\begin{bmatrix}1\\2\end{bmatrix}+
    0\begin{bmatrix}-1\\1\end{bmatrix}\hspace{3em}
    2\begin{bmatrix}1\\2\end{bmatrix}+
    1\begin{bmatrix}-1\\1\end{bmatrix}
    \]
  \end{subactivity}
  \begin{subactivity}
    Sketch a representation of all the vectors belonging to
    \(\vspan\left\{\begin{bmatrix}1\\2\end{bmatrix},
     \begin{bmatrix}-1\\1\end{bmatrix}\right\}\)
    in the \(xy\) plane.
  \end{subactivity}
\end{activity}

\begin{activity}{5}
    Sketch a representation of all the vectors belonging to
    \(\vspan\left\{\begin{bmatrix}6\\-4\end{bmatrix},
     \begin{bmatrix}-2\\3\end{bmatrix}\right\}\)
    in the \(xy\) plane.
\end{activity}


\end{applicationActivities}
