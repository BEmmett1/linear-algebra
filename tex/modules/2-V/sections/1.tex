%!TEX root =../../../course-notes.tex
% ^ leave for LaTeXTools build functionality
\begin{applicationActivities}
\begin{remark}
  Last time, we defined a \textbf{vector space} \(V\) to be any collection of
  mathematical objects with
  associated addition and scalar multiplication operations that satisfy
  the following eight properties for all \(\vec u,\vec v,\vec w\) in \(V\),
  and all scalars (i.e. real numbers) \(a,b\).

  \vecorSpaceProperties
\end{remark}

\begin{remark}
  The following sets are examples of vector spaces, with the usual/natural
  operations for addition and scalar multiplication.
  \begin{itemize}
    \item \(\IR^n\): Euclidean vectors with \(n\) components.
    \item \(\IR^\infty\): Sequences of real numbers \((v_1,v_2,\dots)\).
    \item \(M_{m,n}\): Matrices of real numbers with \(m\) rows and
          \(n\) columns.
    \item \(\IC\): Complex numbers.
    \item \(\P^n\): Polynomials of degree \(n\) or less.
    \item \(\P\): Polynomials of any degree.
    \item \(C(\IR)\): Real-valued continuous functions.
  \end{itemize}
\end{remark}

\begin{activity}{20}
  Consider the set \(V=\setBuilder{(x,y)}{y=e^x}\) with operations defined by
  \[
    (x,y)\oplus (z,w)=(x+z,yw)
      \hspace{3em}
    c\odot (x,y)=(cx,y^c)
  \]
  \begin{subactivity}
  Show that $V$ satisfies the vector distributive property
  \[(a+b)\odot\vec v=(a\odot\vec v)\oplus (b\odot\vec v)\]
  by letting \(\vec v=(x,y)\) and showing both sides simplify to the same
  expression.
  \end{subactivity}
  \begin{subactivity}%optional
  Show that $V$ contains an additive identity element by choosing
  \(\vec{z}=(\unknown,\unknown)\) such that
  \(\vec{v}\oplus\vec{z}=(x,y)\oplus(\unknown,\unknown)=\vec{v}\)
  for any \(\vec{v}=(x,y)\in V\).
  \end{subactivity}
\end{activity}


\begin{remark}
  It turns out \(V=\setBuilder{(x,y)}{y=e^x}\) with operations defined by
  \[
    (x,y)\oplus (z,w)=(x+z,yw)
      \hspace{3em}
    c\odot (x,y)=(cx,y^c)
  \]
  satisifes all eight properties.
    \begin{multicols}{2}
    \begin{itemize}
      \item \textbf{Addition associativity.}

            \(\vec u\oplus(\vec v\oplus\vec w)=
            (\vec u\oplus\vec v)\oplus\vec w\).
      \item \textbf{Addition commutivity.}

            \(\vec u\oplus\vec v=
            \vec v\oplus\vec u\).
      \item \textbf{Addition identity.}

            There exists some \(\vec z\)
            where \(\vec v\oplus\vec z=\vec v\).
      \item \textbf{Addition inverse.}

            There exists some \(-\vec v\)
            where \(\vec v\oplus(-\vec v)=\vec z\).
      \item \textbf{Scalar multiplication associativity.}

            \(a\odot(b\odot\vec v)=(ab)\odot\vec v\).
      \item \textbf{Scalar multiplication identity.}

            \(1\odot\vec v=\vec v\).
      \item \textbf{Scalar distribution.}

            \(a\odot(\vec u\oplus\vec v)=
            (a\odot\vec u)\oplus (a\odot\vec v)\).
      \item \textbf{Vector distribution.}

            \((a+b)\odot\vec v=(a\odot\vec v)\oplus (b\odot\vec v)\).
    \end{itemize}
    \end{multicols}
	Thus, \(V\) is a vector space.
\end{remark}





\begin{activity}{15}
  Let \(V=\setBuilder{(x,y)}{x,y\in\IR}\) have operations defined by
  \[
    (x,y)\oplus (z,w)=(x+y+z+w,x^2+z^2)
      \hspace{3em}
    c\odot (x,y)=(x^c,y+c-1)
  .\]

  \begin{subactivity}
    Show that the scalar multiplication identity holds by
    simplifying \(1\odot(x,y)\) to \((x,y)\).
  \end{subactivity}

  \begin{subactivity}
    Show that the addition identity property fails by
    showing that \((0,-1)\oplus\vec z\not=(0,-1)\) no matter how
    \(\vec z=(z_1,z_2)\) is chosen.
  \end{subactivity}

  \begin{subactivity}
    Can \(V\) be a vector space?
  \end{subactivity}
\end{activity}

\begin{definition}
  A \term{linear combination} of a set of vectors
  \(\{\vec v_1,\vec v_2,\dots,\vec v_m\}\) is given by
  \(c_1\vec v_1+c_2\vec v_2+\dots+c_m\vec v_m\) for any choice of
  scalar multiples \(c_1,c_2,\dots,c_m\).

	\vspace{2em}

  For example, we can say \(\begin{bmatrix}3 \\0 \\ 5\end{bmatrix}\) is a linear combination of the vectors \(\begin{bmatrix} 1 \\ -1 \\ 2 \end{bmatrix}\) and \(\begin{bmatrix} 1 \\ 2 \\ 1 \end{bmatrix}\) since \[\begin{bmatrix} 3 \\ 0 \\ 5 \end{bmatrix} = 2 \begin{bmatrix} 1 \\ -1 \\ 2 \end{bmatrix} + 1\begin{bmatrix} 1 \\ 2 \\ 1 \end{bmatrix}\]
\end{definition}

\begin{definition}
  The \term{span} of a set of vectors is the collection of all linear
  combinations of that set:
  \[
    \vspan\{\vec v_1,\vec v_2,\dots,\vec v_m\} =
    \setBuilder{c_1\vec v_1+c_2\vec v_2+\dots+c_m\vec v_m}{
    c_i\in\IR}.
  \]

	\vspace{2em}

  For example:

  \[
    \vspan\setList
    {
      \begin{bmatrix} 1 \\ -1 \\ 2 \end{bmatrix},
      \begin{bmatrix} 1 \\ 2 \\ 1 \end{bmatrix}
    } = \setBuilder
    {
      a\begin{bmatrix} 1 \\ -1 \\ 2 \end{bmatrix}+
      b\begin{bmatrix} 1 \\ 2 \\ 1 \end{bmatrix}
    }{
      a,b\in\IR
    }
  \]
\end{definition}

\begin{activity}{10}
  Consider \(\vspan\left\{\begin{bmatrix}1\\2\end{bmatrix}\right\}\).
  \begin{subactivity}
    Sketch
    \(1\begin{bmatrix}1\\2\end{bmatrix}\),
    \(3\begin{bmatrix}1\\2\end{bmatrix}\),
    \(0\begin{bmatrix}1\\2\end{bmatrix}\),
    and \(-2\begin{bmatrix}1\\2\end{bmatrix}\) in the \(xy\) plane.
  \end{subactivity}
  \begin{subactivity}
    Sketch a representation of all the vectors belonging to
    \(
      \vspan\setList{\begin{bmatrix}1\\2\end{bmatrix}}
        =
      \setBuilder{a\begin{bmatrix}1\\2\end{bmatrix}}{a\in\IR}
    \)
    in the \(xy\) plane.
  \end{subactivity}
\end{activity}





\begin{activity}{10}
  Consider \(\vspan\left\{\begin{bmatrix}1\\2\end{bmatrix},
  \begin{bmatrix}-1\\1\end{bmatrix}\right\}\).
  \begin{subactivity}
    Sketch the following linear combinations in the \(xy\) plane.
    \[
    1\begin{bmatrix}1\\2\end{bmatrix}+
    0\begin{bmatrix}-1\\1\end{bmatrix}\hspace{3em}
    0\begin{bmatrix}1\\2\end{bmatrix}+
    1\begin{bmatrix}-1\\1\end{bmatrix}\hspace{3em}
    1\begin{bmatrix}1\\2\end{bmatrix}+
    1\begin{bmatrix}-1\\1\end{bmatrix}
    \]
    \[
    -2\begin{bmatrix}1\\2\end{bmatrix}+
    1\begin{bmatrix}-1\\1\end{bmatrix}\hspace{3em}
    -1\begin{bmatrix}1\\2\end{bmatrix}+
    -2\begin{bmatrix}-1\\1\end{bmatrix}
    \]
  \end{subactivity}
  \begin{subactivity}
    Sketch a representation of all the vectors belonging to
    \(\vspan\left\{\begin{bmatrix}1\\2\end{bmatrix},
     \begin{bmatrix}-1\\1\end{bmatrix}\right\}\)
    in the \(xy\) plane.
  \end{subactivity}
\end{activity}

\begin{activity}{5}
    Sketch a representation of all the vectors belonging to
    \(\vspan\left\{\begin{bmatrix}6\\-4\end{bmatrix},
     \begin{bmatrix}-3\\2\end{bmatrix}\right\}\)
    in the \(xy\) plane.
\end{activity}


\end{applicationActivities}
