%!TEX root =../../../course-notes.tex
% ^ leave for LaTeXTools build functionality
\begin{applicationActivities}
\begin{remark}
  Previously, we defined a \textbf{vector space} \(V\) to be any collection of
  mathematical objects with
  associated addition and scalar multiplication operations that satisfy
  the following eight properties for all \(\vec u,\vec v,\vec w\) in \(V\),
  and all scalars (i.e. real numbers) \(a,b\).

  \vectorSpaceProperties
\end{remark}

\begin{remark}
  Every Euclidean space \(\IR^n\) is a vector space, but there are other
  examples of vector spaces as well. 

  \vspace{1em}
  
  For example, consider the
  set \(\IC\) of complex numbers with the usual defintions of
  addition and scalar multiplication, and let 
  \(\vec u=a+b\mathbf{i}\), \(\vec v=c+d\mathbf{i}\), and \(\vec w=e+f\mathbf{i}\). Then

  \begin{align*}
    \vec u+(\vec v+\vec w)
      &=
    (a+b\mathbf{i})+((c+d\mathbf{i})+(e+f\mathbf{i}))
      \\&=
    a+b+c+d\mathbf{i}+e\mathbf{i}+f\mathbf{i}
      \\&=
    ((a+b\mathbf{i})+(c+d\mathbf{i}))+(e+f\mathbf{i})
      \\&=
    (\vec u+\vec v)+\vec w
  \end{align*}

  All eight properties can be verified in this way.
\end{remark}

\begin{remark}
  The following sets are just a few examples of vector spaces, with the usual/natural
  operations for addition and scalar multiplication.
  \begin{itemize}
    \item \(\IR^n\): Euclidean vectors with \(n\) components.
    \item \(\IC\): Complex numbers.
    \item \(M_{m,n}\): Matrices of real numbers with \(m\) rows and
          \(n\) columns.
    \item \(\P^n\): Polynomials of degree \(n\) or less.
    \item \(\P\): Polynomials of any degree.
    \item \(C(\IR)\): Real-valued continuous functions.
  \end{itemize}
\end{remark}

\begin{activity}{20}
  Consider the set \(V=\setBuilder{(x,y)}{y=e^x}\) with operations defined by
  \[
    (x,y)\oplus (z,w)=(x+z,yw)
      \hspace{3em}
    c\odot (x,y)=(cx,y^c)
  \]
  \begin{subactivity}
  Show that \(V\) satisfies the distribution property
  \[(a+b)\odot\vec v=(a\odot\vec v)\oplus (b\odot\vec v)\]
  by substituting \(\vec v=(x,y)\) and showing both sides simplify to the same
  expression.
  \end{subactivity}
  \begin{subactivity}%optional
  Show that \(V\) contains an additive identity element satisfying
  \[(x,y)\oplus\vec{z}=(x,y)\]
  for all \((x,y)\in V\)
  by choosing appropriate values for \(\vec{z}=(\unknown,\unknown)\). 
  \end{subactivity}
\end{activity}


\begin{remark}
  It turns out \(V=\setBuilder{(x,y)}{y=e^x}\) with operations defined by
  \[
    (x,y)\oplus (z,w)=(x+z,yw)
      \hspace{3em}
    c\odot (x,y)=(cx,y^c)
  \]
  satisifes all eight properties.

  \vectorSpaceProperties

  Thus, \(V\) is a vector space.
\end{remark}





\begin{activity}{15}
  Let \(V=\setBuilder{(x,y)}{x,y\in\IR}\) have operations defined by
  \[
    (x,y)\oplus (z,w)=(x+y+z+w,x^2+z^2)
      \hspace{3em}
    c\odot (x,y)=(x^c,y+c-1)
  .\]

  \begin{subactivity}
    Show that \(1\) is the scalar multiplication identity element
	by simplifying \(1\odot(x,y)\) to \((x,y)\).
  \end{subactivity}

  \begin{subactivity}
    Show that \(V\) does not have an additive identity element by showing that 
	\((0,-1)\oplus\vec z\not=(0,-1)\) no matter how
    \(\vec z=(z_1,z_2)\) is chosen.
  \end{subactivity}

  \begin{subactivity}
    Is \(V\) a vector space?
  \end{subactivity}
\end{activity}

\begin{definition}
  A \term{linear combination} of a set of vectors
  \(\{\vec v_1,\vec v_2,\dots,\vec v_m\}\) is given by
  \(c_1\vec v_1+c_2\vec v_2+\dots+c_m\vec v_m\) for any choice of
  scalar multiples \(c_1,c_2,\dots,c_m\).

	\vspace{2em}

  For example, we can say \(\begin{bmatrix}3 \\0 \\ 5\end{bmatrix}\) 
  is a linear combination of the vectors \(\begin{bmatrix} 1 \\ -1 \\ 2 \end{bmatrix}\) 
  and \(\begin{bmatrix} 1 \\ 2 \\ 1 \end{bmatrix}\) since 
  \[
    \begin{bmatrix} 3 \\ 0 \\ 5 \end{bmatrix} = 
    2 \begin{bmatrix} 1 \\ -1 \\ 2 \end{bmatrix} + 
    1\begin{bmatrix} 1 \\ 2 \\ 1 \end{bmatrix}
  \]
\end{definition}

\begin{definition}
  The \term{span} of a set of vectors is the collection of all linear
  combinations of that set:
  \[
    \vspan\{\vec v_1,\vec v_2,\dots,\vec v_m\} =
    \setBuilder{c_1\vec v_1+c_2\vec v_2+\dots+c_m\vec v_m}{
    c_i\in\IR}.
  \]

	\vspace{2em}

  For example:

  \[
    \vspan\setList
    {
      \begin{bmatrix} 1 \\ -1 \\ 2 \end{bmatrix},
      \begin{bmatrix} 1 \\ 2 \\ 1 \end{bmatrix}
    } = \setBuilder
    {
      a\begin{bmatrix} 1 \\ -1 \\ 2 \end{bmatrix}+
      b\begin{bmatrix} 1 \\ 2 \\ 1 \end{bmatrix}
    }{
      a,b\in\IR
    }
  \]
\end{definition}

\begin{activity}{10}
  Consider \(\vspan\left\{\begin{bmatrix}1\\2\end{bmatrix}\right\}\).
  \begin{subactivity}
    Sketch

    \(1\begin{bmatrix}1\\2\end{bmatrix}=\begin{bmatrix}1\\2\end{bmatrix}\),
    \hfill\(3\begin{bmatrix}1\\2\end{bmatrix}=\begin{bmatrix}3\\6\end{bmatrix}\),
    \hfill\(0\begin{bmatrix}1\\2\end{bmatrix}=\begin{bmatrix}0\\0\end{bmatrix}\),
    \hfill and \(-2\begin{bmatrix}1\\2\end{bmatrix}=\begin{bmatrix}-2\\-4\end{bmatrix}\) 

    in the \(xy\) plane.
  \end{subactivity}
  \begin{subactivity}
    Sketch a representation of all the vectors belonging to
    \(
      \vspan\setList{\begin{bmatrix}1\\2\end{bmatrix}}
        =
      \setBuilder{a\begin{bmatrix}1\\2\end{bmatrix}}{a\in\IR}
    \)
    in the \(xy\) plane.
  \end{subactivity}
\end{activity}





\begin{activity}{10}
  Consider \(\vspan\left\{\begin{bmatrix}1\\2\end{bmatrix},
  \begin{bmatrix}-1\\1\end{bmatrix}\right\}\).
  \begin{subactivity}
    Sketch the following linear combinations in the \(xy\) plane.
    \[
    1\begin{bmatrix}1\\2\end{bmatrix}+
    0\begin{bmatrix}-1\\1\end{bmatrix}\hspace{3em}
    0\begin{bmatrix}1\\2\end{bmatrix}+
    1\begin{bmatrix}-1\\1\end{bmatrix}\hspace{3em}
    1\begin{bmatrix}1\\2\end{bmatrix}+
    1\begin{bmatrix}-1\\1\end{bmatrix}
    \]
    \[
    -2\begin{bmatrix}1\\2\end{bmatrix}+
    1\begin{bmatrix}-1\\1\end{bmatrix}\hspace{3em}
    -1\begin{bmatrix}1\\2\end{bmatrix}+
    -2\begin{bmatrix}-1\\1\end{bmatrix}
    \]
  \end{subactivity}
  \begin{subactivity}
    Sketch a representation of all the vectors belonging to
    \(\vspan\left\{\begin{bmatrix}1\\2\end{bmatrix},
     \begin{bmatrix}-1\\1\end{bmatrix}\right\}\)
    in the \(xy\) plane.
  \end{subactivity}
\end{activity}

\begin{activity}{5}
    Sketch a representation of all the vectors belonging to
    \(\vspan\left\{\begin{bmatrix}6\\-4\end{bmatrix},
     \begin{bmatrix}-3\\2\end{bmatrix}\right\}\)
    in the \(xy\) plane.
\end{activity}


\end{applicationActivities}
