\begin{applicationActivities}
\begin{definition}
  A \textbf{vector space} \(V\) is any collection of mathematical objects with
  associated addition and scalar multiplication operations that satisfy
  the following properties. Let \(\vect u,\vect v,\vect w\) belong to \(V\),
  and let \(a,b\) be scalar numbers.

  \vectorSpaceProperties
\end{definition}

\begin{remark}
  The following sets are examples of vector spaces, with the usual/natural
  operations for addition and scalar multiplication.
  \begin{itemize}
    \item \(\IR^n\): Euclidean vectors with \(n\) components.
    \item \(\IR^\infty\): Sequences of real numbers \((v_1,v_2,\dots)\).
    \item \(\IR^{m\times n}\): Matrices of real numbers with \(m\) rows and
          \(n\) columns.
    \item \(\IC\): Complex numbers.
    \item \(\P^n\): Polynomials of degree \(n\) or less.
    \item \(\P\): Polynomials of any degree.
    \item \(C(\IR)\): Real-valued continuous functions.
  \end{itemize}
\end{remark}


\begin{activity}{25}
  Consider the following set that models motion along the curve
  \(y=e^x\). Let \(V=\{(x,y):y=e^x\}\). Let vector addition be defined by
  \((x_1,y_1)\oplus (x_2,y_2)=(x_1+x_2,y_1y_2)\), and
  let scalar multiplication be defined by
  \(c\odot (x,y)=(cx,y^c)\).

  \begin{subactivity}
    Which of the vector space properties are satisfied by \(V\) paired with
    these operations?
    \begin{multicols}{2}
    \begin{itemize}
      \item \textbf{Addition associativity.}

            \(\vect u\oplus(\vect v\oplus\vect w)=
            (\vect u\oplus\vect v)\oplus\vect w\).
      \item \textbf{Addition commutivity.}

            \(\vect u\oplus\vect v=
            \vect v\oplus\vect u\).
      \item \textbf{Addition identity.}

            There exists some \(\vect z\)
            where \(\vect v\oplus\vect z=\vect v\).
      \item \textbf{Addition inverse.}

            There exists some \(-\vect v\)
            where \(\vect v\oplus(-\vect v)=\vect z\).
      \item \textbf{Scalar multiplication associativity.}

            \(a\odot(b\odot\vect v)=(ab)\odot\vect v\).
      \item \textbf{Scalar multiplication identity.}

            \(1\odot\vect v=\vect v\).
      \item \textbf{Scalar distribution.}

            \(a\odot(\vect u\oplus\vect v)=
            (a\odot\vect u)\oplus (a\odot\vect v)\).
      \item \textbf{Vector distribution.}

            \((a+b)\odot\vect v=(a\odot\vect v)\oplus (b\odot\vect v)\).
    \end{itemize}
    \end{multicols}
  \end{subactivity}

  \begin{subactivity}
    Is \(V\) a vector space?
  \end{subactivity}
\end{activity}


\begin{activity}{10}
  Let \(V=\setBuilder{(a,b)}{a,b\in\IR}\), where
  \(\begin{bmatrix} a_1 \\ a_2 \end{bmatrix})\oplus \begin{bmatrix}b_1 \\ b_2\end{bmatrix}=
  \begin{bmatrix}a_1+b_1+a_2+b_2 \\ a_1^2+b_1^2\end{bmatrix}\) and \(c\odot \begin{bmatrix} a_1 \\ a_2\end{bmatrix}=\begin{bmatrix} a_1^c \\ a_2+c\end{bmatrix}\). Show that
  this is not a vector space by finding a counterexample
  that does not satisfy one of the vector space properties.

  \vectorSpacePropertiesO
\end{activity}

\begin{definition}
  A \term{linear combination} of a set of vectors
  \(\{\vect v_1,\vect v_2,\dots,\vect v_m\}\) is given by
  \(c_1\vect v_1+c_2\vect v_2+\dots+c_m\vect v_m\) for any choice of
  scalar multiples \(c_1,c_2,\dots,c_m\).

	\ \\
	\ \\

  For example, we say $\begin{bmatrix}3 \\0 \\ 5\end{bmatrix}$ is a linear combination of the vectors $\begin{bmatrix} 1 \\ -1 \\ 2 \end{bmatrix}$ and $\begin{bmatrix} 1 \\ 2 \\ 1 \end{bmatrix}$ since $$\begin{bmatrix} 3 \\ 0 \\ 5 \end{bmatrix} = 2 \begin{bmatrix} 1 \\ -1 \\ 2 \end{bmatrix} + 1\begin{bmatrix} 1 \\ 2 \\ 1 \end{bmatrix}$$
\end{definition}

\begin{definition}
  The \term{span} of a set of vectors is the collection of all linear
  combinations of that set:
  \[
    \vspan\{\vect v_1,\vect v_2,\dots,\vect v_m\} =
    \setBuilder{c_1\vect v_1+c_2\vect v_2+\dots+c_m\vect v_m}{ 
    c_i\text{ is a real number}}
  \]
\end{definition}

\begin{activity}{10}
  Consider \(\vspan\left\{\begin{bmatrix}1\\2\end{bmatrix}\right\}\).
  \begin{subactivity}
    Sketch
    \(c\begin{bmatrix}1\\2\end{bmatrix}\) in the \(xy\) plane
    for \(c=1,3,0,-2\).
  \end{subactivity}
  \begin{subactivity}
    Sketch a representation of all the vectors given by
    \(\vspan\left\{\begin{bmatrix}1\\2\end{bmatrix}\right\}\)
    in the \(xy\) plane.
  \end{subactivity}
\end{activity}





\begin{activity}{10}
  Consider \(\vspan\left\{\begin{bmatrix}1\\2\end{bmatrix},
  \begin{bmatrix}-1\\1\end{bmatrix}\right\}\).
  \begin{subactivity}
    Sketch the following linear combinations in the \(xy\) plane:
    \(1\begin{bmatrix}1\\2\end{bmatrix}+
    0\begin{bmatrix}-1\\1\end{bmatrix}\),
    \(0\begin{bmatrix}1\\2\end{bmatrix}+
    1\begin{bmatrix}-1\\1\end{bmatrix}\),
    \(2\begin{bmatrix}1\\2\end{bmatrix}+
    0\begin{bmatrix}-1\\1\end{bmatrix}\),
    \(2\begin{bmatrix}1\\2\end{bmatrix}+
    1\begin{bmatrix}-1\\1\end{bmatrix}\).
  \end{subactivity}
  \begin{subactivity}
    Sketch a representation of all the vectors given by
    \(\vspan\left\{\begin{bmatrix}1\\2\end{bmatrix},
     \begin{bmatrix}-1\\1\end{bmatrix}\right\}\)
    in the \(xy\) plane.
  \end{subactivity}
\end{activity}

\begin{activity}{5}
    Sketch a representation of all the vectors given by
    \(\vspan\left\{\begin{bmatrix}6\\-4\end{bmatrix},
     \begin{bmatrix}-2\\3\end{bmatrix}\right\}\)
    in the \(xy\) plane.
\end{activity}


\end{applicationActivities}
