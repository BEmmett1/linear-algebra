%!TEX root =../../../course-notes.tex
% ^ leave for LaTeXTools build functionality

\begin{applicationActivities}

% have students capture 8 properties of R2 from list of 11
% include ca=b, unique equidistant, nonzero orthogonal

\begin{observation}
  Several properites of the real numbers (the
  \term{one-dimensional Euclidean vectors} \(\IR^1\)) also work for
  \term{two-dimensional Euclidean vectors} (\(\IR^2\)) and
  \term{three-dimensional Euclidean vectors} (\(\IR^3\)),
  such as \(\vec u+\vec v=\vec v+\vec u\) and \(1\vec{v}=\vec{v}\):
  \[
    x+y=y+x
      \hspace{3em}
    \begin{bmatrix}x_1\\x_2\end{bmatrix}+\begin{bmatrix}y_1\\y_2\end{bmatrix}
      =
    \begin{bmatrix}y_1\\y_2\end{bmatrix}+\begin{bmatrix}x_1\\x_2\end{bmatrix}
      \hspace{3em}
    \begin{bmatrix}x_1\\x_2\\x_3\end{bmatrix}
      +
    \begin{bmatrix}y_1\\y_2\\y_3\end{bmatrix}
      =
    \begin{bmatrix}y_1\\y_2\\y_3\end{bmatrix}
      +
    \begin{bmatrix}x_1\\x_2\\x_3\end{bmatrix}
  \]
  \[
    1x=x
      \hspace{3em}
    1\begin{bmatrix}x\\y\end{bmatrix}=\begin{bmatrix}x\\y\end{bmatrix}
      \hspace{3em}
    1\begin{bmatrix}x\\y\\z\end{bmatrix}=\begin{bmatrix}x\\y\\z\end{bmatrix}
  \]
\end{observation}

\begin{activity}{20}\smallSlideText
Consider each of the following properties of vectors \(\vec u,\vec v, \vec w\)
in \(\IR^1\) (real numbers). Label each property
as ``VALID`` if it also holds for Euclidean vectors \(\vec u,\vec v, \vec w\)
in \(\IR^2\), and ``INVALID'' if it does not.
\begin{multicols}{2}
\begin{enumerate}
  \item \textbf{Addition associativity.}

        \(\vec u+(\vec v+\vec w)=
        (\vec u+\vec v)+\vec w\).
  \item \textbf{Addition commutivity.}

        \(\vec u+\vec v=
        \vec v+\vec u\).
  \item \textbf{Addition identity.}

        There exists some \(\vec z\)
        where \(\vec v+\vec z=\vec v\).
  \item \textbf{Addition inverse.}

        There exists some \(-\vec v\)
        where \(\vec v+(-\vec v)=\vec z\).
  \item \textbf{Addition midpoint uniqueness.}

        If \(\vec u\not=\vec v\), then
        \(\frac{1}{2}(\vec u+\vec v)\) is the only vector
        equally distant from \(\vec u\) and \(\vec v\).
  \item \textbf{Scalar multiplication associativity.}

        \(a(b\vec v)=(ab)\vec v\).
  \item \textbf{Scalar multiplication identity.}

        \(1\vec v=\vec v\).
  \item \textbf{Scalar multiplication relativity.}

        If \(\vec u\not=\vec z\), there exists a scalar \(c\)
        satisfying \(c\vec u=\vec v\).
  \item \textbf{Scalar distribution.}

        \(a(\vec u+\vec v)=a\vec u+a\vec v\).
  \item \textbf{Vector distribution.}

        \((a+b)\vec v=a\vec v+b\vec v\).
\end{enumerate}
\end{multicols}
\end{activity}

\begin{definition}
  A \term{vector space} \(V\) is any collection of mathematical objects with
  associated addition and scalar multiplication operations that satisfy
  the following properties. Let \(\vect u,\vect v,\vect w\) belong to \(V\),
  and let \(a,b\) be scalar numbers.

  \vectorSpaceProperties

  Any \term{Euclidean vector space} \(\IR^n\) satisfies all eight
  requirements regardless of the value of \(n\),
  but we will also study other types of vector spaces.
\end{definition}


\end{applicationActivities}
