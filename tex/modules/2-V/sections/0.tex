%!TEX root =../../../course-notes.tex
% ^ leave for LaTeXTools build functionality

\begin{applicationActivities}

\begin{observation}
Several properties of the real numbers, such as commutivity:
\[
  x + y = y + x
\]
also hold for Eudlicean vectors with multiple components:
\[
\begin{bmatrix}x_1\\x_2\end{bmatrix}
+
\begin{bmatrix}y_1\\y_2\end{bmatrix}
=
\begin{bmatrix}y_1\\y_2\end{bmatrix}
+
\begin{bmatrix}x_1\\x_2\end{bmatrix}
\]
\end{observation}

\begin{activity}{20}\smallSlideText
Consider each of the following properties of the real numbers
\(\IR^1\). Label each property as \textbf{valid} if the property also
holds for two-dimensional Euclidean vectors 
\(\vec u,\vec v,\vec w\in\IR^2\) and scalars \(a,b\in\IR\),
and \textbf{invalid} if it does not.
\begin{multicols}{2}
\begin{enumerate}
  \item \(\vec u+(\vec v+\vec w)=
        (\vec u+\vec v)+\vec w\).
  \item \(\vec u+\vec v=
        \vec v+\vec u\).
  \item There exists some \(\vec z\)
        where \(\vec v+\vec z=\vec v\).
  \item There exists some \(-\vec v\)
        where \(\vec v+(-\vec v)=\vec z\).
  \item If \(\vec u\not=\vec v\), then \(\frac{1}{2}(\vec u+\vec v)\)
        is the only vector equally distant from both \(\vec u\) and \(\vec v\)
  \item \(a(b\vec v)=(ab)\vec v\).
  \item \(1\vec v=\vec v\).
  \item If \(\vec u\not=\vec 0\), then there exists some scalar \(c\) 
        such that \(c\vec u=\vec v\).
  \item \(a(\vec u+\vec v)=a\vec u+a\vec v\).
  \item \((a+b)\vec v=a\vec v+b\vec v\).
\end{enumerate}
\end{multicols}
\end{activity}

\begin{definition}
  A \term{vector space} \(V\) is any collection of mathematical objects with
  associated addition \(\oplus\) and scalar multiplication \(\odot\)
  operations that satisfy the following properties. 
  Let \(\vec u,\vec v,\vec w\) belong to \(V\), and let \(a,b\) be scalar numbers.

  \vectorSpaceProperties

  Every \term{Euclidean vector space}
  \[
    \IR^n=\setBuilder{\begin{bmatrix}x_1\\x_2\\\vdots\\x_n\end{bmatrix}}{x_1,x_2,\dots,x_n\in\IR}
  \] 
  satisfies all eight requirements for the usual definitions of addition
  and scalar multiplication,
  but we will also study other types of vector spaces.
\end{definition}


\end{applicationActivities}
