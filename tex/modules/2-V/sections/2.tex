


\begin{applicationActivities}



\begin{activity}{15}
  The vector
  \(\begin{bmatrix}-1\\-6\\1\end{bmatrix}\) belongs to
  \(\vspan\left\{\begin{bmatrix}1\\0\\-3\end{bmatrix},
  \begin{bmatrix}-1\\-3\\2\end{bmatrix}\right\}\) exactly when
  the vector equation
  \(x_1\begin{bmatrix}1\\0\\-3\end{bmatrix}+
  x_2\begin{bmatrix}-1\\-3\\2\end{bmatrix}
  =\begin{bmatrix}-1\\-6\\1\end{bmatrix}\) holds for some scalars
  \(x_1,x_2\).
  \begin{subactivity}
    Reinterpret this vector equation as a system of linear equations.
  \end{subactivity}

  \begin{subactivity}
    Solve this system. (Remember, you should use a calculator to help
    find \(\RREF\).)

  \end{subactivity}
  \begin{subactivity}
    Given this solution, does
    \(\begin{bmatrix}-1\\-6\\1\end{bmatrix}\) belong to
    \(\vspan\left\{\begin{bmatrix}1\\0\\-3\end{bmatrix},
    \begin{bmatrix}-1\\-3\\2\end{bmatrix}\right\}\)?
  \end{subactivity}
\end{activity}

\begin{fact}
  A vector \(\vect b\) belongs to
  \(\vspan\{\vect v_1,\dots,\vect v_n\}\) if and only if
  the linear system corresponding to
  \([\vect v_1\,\dots\,\vect v_n \,|\, \vect b]\)
  is consistent.
\end{fact}

\begin{remark}
  To determine if \(\vect b\) belongs to
  \(\vspan\{\vect v_1,\dots,\vect v_n\}\), find
  \(\RREF[\vect v_1\,\dots\,\vect v_n \,|\, \vect b]\).
\end{remark}

\begin{activity}{10}
  Determine if
  \(\begin{bmatrix}3\\-2\\1 \\ 5\end{bmatrix}\) belongs to
  \(\vspan\left\{\begin{bmatrix}1\\0\\-3 \\ 2\end{bmatrix},
  \begin{bmatrix}-1\\-3\\2 \\ 2\end{bmatrix}\right\}\)
  by row-reducing an appropriate matrix.
\end{activity}

\begin{activity}{5}
  Determine if
  \(\begin{bmatrix}-1\\-9\\0\end{bmatrix}\) belongs to
  \(\vspan\left\{\begin{bmatrix}1\\0\\-3\end{bmatrix},
  \begin{bmatrix}-1\\-3\\2\end{bmatrix}\right\}\)
  by row-reducing an appropriate matrix.
\end{activity}


\begin{observation}
  So far we've only discussed linear combinations of Euclidean vectors.
  Fortunately, many vector spaces of interest can be reinterpreted as an
  \term{isomorphic} Euclidean space \(\IR^n\); that is, a Euclidean space
  that mirrors the behavior of the vector space exactly.
\end{observation}

\begin{activity}{5}
  We previously checked that \(\begin{bmatrix}3\\-2\\1 \\ 5\end{bmatrix}\)
  does not belong to
  \(\vspan\left\{\begin{bmatrix}1\\0\\-3 \\ 2\end{bmatrix},
  \begin{bmatrix}-1\\-3\\2 \\ 2\end{bmatrix}\right\}\).
  Does \(f(x)=3x^3-2x^2+x+5\) belong to
  \(\vspan\{x^3-3x+2,-x^3-3x^2+2+2x+2\}\)?
\end{activity}

\begin{activity}{10}
  Does the matrix \(\begin{bmatrix}3&-2\\1&5\end{bmatrix}\) belong to
  \(\vspan\left\{\begin{bmatrix}1&0\\-3&2\end{bmatrix},
  \begin{bmatrix}-1&-3\\2&2\end{bmatrix}\right\}\)?
\end{activity}

\begin{activity}{5}
  Does the complex number \(2i\) belong to
  \(\vspan\{-3+i,6-2i\}\)?
\end{activity}

\end{applicationActivities}
