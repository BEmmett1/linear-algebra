%!TEX root =../../../course-notes.tex
% ^ leave for LaTeXTools build functionality
\begin{applicationActivities}

\begin{remark}
	Recall these definitions from last class:
\begin{itemize}
	\item
  A \term{linear combination} of vectors is given by adding scalar
  multiples of those vectors, such as:
  \[
    \begin{bmatrix} 3 \\ 0 \\ 5 \end{bmatrix} =
    2 \begin{bmatrix} 1 \\ -1 \\ 2 \end{bmatrix} +
    1\begin{bmatrix} 1 \\ 2 \\ 1 \end{bmatrix}
  \]

\item The \term{span} of a set of vectors is the collection of all linear
  combinations of that set, such as:
  \[
    \vspan\setList
    {
      \begin{bmatrix} 1 \\ -1 \\ 2 \end{bmatrix},
      \begin{bmatrix} 1 \\ 2 \\ 1 \end{bmatrix}
    } = \setBuilder
    {
      a\begin{bmatrix} 1 \\ -1 \\ 2 \end{bmatrix}+
      b\begin{bmatrix} 1 \\ 2 \\ 1 \end{bmatrix}
    }{
      a,b\in\IR
    }
  \]
 \end{itemize}
\end{remark}


\begin{activity}{15}
  The vector
  \(\begin{bmatrix}-1\\-6\\1\end{bmatrix}\) belongs to
  \(\vspan\left\{\begin{bmatrix}1\\0\\-3\end{bmatrix},
  \begin{bmatrix}-1\\-3\\2\end{bmatrix}\right\}\) exactly when
  there exists a solution to the vector equation
  \(x_1\begin{bmatrix}1\\0\\-3\end{bmatrix}+
  x_2\begin{bmatrix}-1\\-3\\2\end{bmatrix}
  =\begin{bmatrix}-1\\-6\\1\end{bmatrix}\).

  \begin{subactivity}
    Reinterpret this vector equation as a system of linear equations.
  \end{subactivity}

  \begin{subactivity}
    Find its solution set, using CoCalc.com to find \(\RREF\) of its
    corresponding augmented matrix.
  \end{subactivity}

  \begin{subactivity}
    Given this solution set, does
    \(\begin{bmatrix}-1\\-6\\1\end{bmatrix}\) belong to
    \(\vspan\left\{\begin{bmatrix}1\\0\\-3\end{bmatrix},
    \begin{bmatrix}-1\\-3\\2\end{bmatrix}\right\}\)?
  \end{subactivity}
\end{activity}

\begin{fact}
  A vector \(\vect b\) belongs to
  \(\vspan\{\vect v_1,\dots,\vect v_n\}\) if and only if
  the linear system corresponding to
  \([\vect v_1\,\dots\,\vect v_n \,|\, \vect b]\)
  is consistent.

  \vspace{1em}

  Put another way, \(\vect b\) belongs to
  \(\vspan\{\vect v_1,\dots,\vect v_n\}\) exactly when
  \(\RREF[\vect v_1\,\dots\,\vect v_n \,|\, \vect b]\)
  doesn't have a row \([0\,\cdots\,0\,|\,1]\)
  representing the contradiction \(0=1\).
\end{fact}

\begin{activity}{10}
  Determine if
  \(\begin{bmatrix}3\\-2\\1 \\ 5\end{bmatrix}\) belongs to
  \(\vspan\left\{\begin{bmatrix}1\\0\\-3 \\ 2\end{bmatrix},
  \begin{bmatrix}-1\\-3\\2 \\ 2\end{bmatrix}\right\}\)
  by row-reducing an appropriate matrix.
\end{activity}

\begin{activity}{5}
  Determine if
  \(\begin{bmatrix}-1\\-9\\0\end{bmatrix}\) belongs to
  \(\vspan\left\{\begin{bmatrix}1\\0\\-3\end{bmatrix},
  \begin{bmatrix}-1\\-3\\2\end{bmatrix}\right\}\)
  by row-reducing an appropriate matrix.
\end{activity}


\begin{activity}{10}
  Does the third-degree polynomial \(3y^3-2y^2+y+5\) in \(\P^3\) belong to
  \(\vspan\{y^3-3y+2,-y^3-3y^2+2y+2\}\)?
  \begin{subactivity}
  	Reinterpret this question as an equivalent exercise involving Euclidean
    vectors in \(\IR^4\). (Hint: What four numbers must you know to write
    a \(\P^3\) polynomial?)
  \end{subactivity}
  \begin{subactivity}
  	Solve this equivalent exercise, and use its solution to answer the original
    question.
  \end{subactivity}
\end{activity}

\begin{activity}{5}
  Does the matrix \(\begin{bmatrix}3&-2\\1&5\end{bmatrix}\) belong to
  \(\vspan\left\{\begin{bmatrix}1&0\\-3&2\end{bmatrix},
  \begin{bmatrix}-1&-3\\2&2\end{bmatrix}\right\}\)?
\end{activity}

\begin{activity}{5}
  Does the complex number \(2i\) belong to
  \(\vspan\{-3+i,6-2i\}\)?
\end{activity}

\end{applicationActivities}
