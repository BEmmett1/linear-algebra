\begin{problem}{V1}
Let \(V\) be the  set of all real numbers together with the operations \(\oplus\) and \(\odot\) defined by, for any \(x,y\in V\) and \(c\in \IR\),
\begin{align*}
x\oplus y  &= x+y \\
c \odot x &= cx-3(c-1)
\end{align*}
\begin{enumerate}[(a)]
\item Show that scalar multiplication is associative: 
      \(a\odot(b\odot x)=(ab)\odot x\) for all scalars \(a,b \in \IR\) and \(x \in V\).
\item Explain why \(V\) nonetheless isn't a vector space.
\end{enumerate}
\end{problem}


\begin{problem}{V1}
Let \(V\) be the set of all pairs of real numbers with the operations, for any \((x_1,x_2), (y_1,y_2) \in V\), \(c\in \IR\),
\begin{align*}
(x_1,x_2) \oplus (y_1,y_2) &= (x_1+y_1,x_2+y_2+2x_1y_1) \\
c \odot (x_1,x_2) &= (cx_1, cx_2)
\end{align*}
\begin{enumerate}[(a)]
\item Show that the vector addition \(\oplus\) is associative:
      \((x_1,x_2) \oplus ((y_1,y_2) \oplus (z_1,z_2))=((x_1,x_2)\oplus (y_1,y_2))\oplus (z_1,z_2)\) for all \((x_1,x_2), (y_1,y_2), (z_1,z_2) \in V\).
\item Explain why \(V\) nonetheless isn't a vector space.
\end{enumerate}
\end{problem}



\begin{problem}{V1}
Let \(V\) be the set of all pairs of real numbers with the operations, for any \((x_1,x_2), (y_1,y_2) \in V\), \(c\in \IR\),
\begin{align*}
(x_1,x_2) \oplus (y_1,y_2) &= (x_1+y_1-1,x_2+y_2-1) \\
c \odot (x_1,x_2) &= (cx_1, cx_2)
\end{align*}
\begin{enumerate}[(a)]
\item Show that this vector space has an additive identity element:
      there exists \(\vec{z} \in V\) satisfying \((x,y)\oplus\vec{z}=(x,y)\) for every \((x,y) \in V\).
\item Explain why \(V\) nonetheless isn't a vector space.
\end{enumerate}
\end{problem}

\begin{problem}{V1}
Let \(V\) be the set of all pairs of real numbers with the operations, for any \((x_1,x_2), (y_1,y_2) \in V\), \(c\in \IR\),
\begin{align*}
(x_1,x_2) \oplus (y_1,y_2) &= (x_1+y_1,x_2+y_2) \\
c \odot (x_1,x_2) &= (0, cx_2)
\end{align*}
\begin{enumerate}[(a)]
\item Show that scalar multiplication distributes over scalar addition:
      \((c+d)\odot(x_1,x_2)=c\odot(x_1,x_2) \oplus d\odot(x_1,x_2)\) for every \(c,d \in \IR\) and \( (x_1,x_2) \in V\).
\item Explain why \(V\) nonetheless isn't a vector space.
\end{enumerate}
\end{problem}


\begin{problem}{V1}
 Let \(V\) be the set of all pairs of real numbers with the operations, for any \((x_1,x_2), (y_1,y_2) \in V\), \(c\in \IR\),
 \begin{align*}
 (x_1,x_2) \oplus (y_1,y_2) &= (x_1+y_1,x_2+y_2) \\
 c \odot (x_1,x_2) &= (c^2x_1, c^3x_2)
 \end{align*}
 \begin{enumerate}[(a)]
 \item Show that scalar multiplication distributes over vector addition:
       \(c\odot((x_1,x_2) \oplus (y_1,y_2))=c\odot(x_1,x_2) \oplus c\odot(y_1,y_2)\) for all \(c \in \IR\) and \( (x_1,x_2), (y_1,y_2) \in V\).
\item Explain why \(V\) nonetheless isn't a vector space.
 \end{enumerate}
\end{problem}
 
\begin{problem}{V1}
 Let \(V\) be the set of all real numbers with the operations, for any \(x,y\in V\), \(c\in \IR\),
 \begin{align*}
 x \oplus y &= \sqrt{x^2+y^2} \\
 c \odot x &= c x
 \end{align*}
 \begin{enumerate}[(a)]
 \item Show that the vector addition \(\oplus\) is associative:
       \(x \oplus (y \oplus z)=(x\oplus y)\oplus z\) for all \(x,y,z \in V\).
\item Explain why \(V\) nonetheless isn't a vector space.
 \end{enumerate}
\end{problem}

\begin{problem}{V1}
 Let \(V\) be the set of all pairs of real numbers with the operations, for any \((x_1,x_2), (y_1,y_2) \in V\), \(c\in \IR\),
 \begin{align*}
 (x_1,x_2) \oplus (y_1,y_2) &= (x_1+y_1,x_2y_2) \\
 c \odot (x_1,x_2) &= (cx_1, cx_2)
 \end{align*}
 \begin{enumerate}[(a)]
 \item Show that there is an additive identity element:
       there exists an element \(\vec{z} \in V\) such that \((x_1,x_2)\oplus\vec{z}= (x_1,x_2)\) for any \( (x_1,x_2) \in V\).
\item Explain why \(V\) nonetheless isn't a vector space.
 \end{enumerate}
\end{problem}
