%IF-AT Form B002 21-30
\begin{readinessAssuranceTest}
\setcounter{enumi}{30}
\item Which of the following is a solution to the system of linear equations
      \begin{align*}
      x+3y-z    &=   2 \\
      2x+8y+3z  &=  -1 \\
      -x-y+9z   &= -10
      \end{align*}

%B
\begin{multicols}{4}
\begin{readinessAssuranceTestChoices}
\item $\begin{bmatrix} 1 \\ 1 \\ 0 \end{bmatrix}$
\item $\begin{bmatrix} 1 \\ 0 \\ -1 \end{bmatrix}$%correct
\item $\begin{bmatrix} 0 \\ 1 \\ -1 \end{bmatrix}$
\item $\begin{bmatrix} 1 \\ -1 \\ 1 \end{bmatrix}$
\end{readinessAssuranceTestChoices}
\end{multicols}


\item Find a basis for the solution set of the following homogeneous system of
      linear equations
      \begin{align*}
      x+2y+-z-w    &= 0 \\
      -2x-4y+3z+5w &= 0
      \end{align*}

%A
\begin{multicols}{4}
\begin{readinessAssuranceTestChoices}
\item $\left\{ \begin{bmatrix} -2 \\ 1 \\ 0 \\ 0 \end{bmatrix}, \begin{bmatrix} -2 \\ 0 \\ -3 \\ 1 \end{bmatrix} \right\}$%correct
\item $\left\{ \begin{bmatrix} 1 \\ 2 \\ 0 \\ 2 \end{bmatrix}, \begin{bmatrix} 0 \\ 0 \\ 1 \\ 3 \end{bmatrix} \right\}$
\item $\left\{ \begin{bmatrix} 2 \\ -1 \\ 3 \\ -1 \end{bmatrix} \right\}$
\item $\left\{ \begin{bmatrix} 1 \\ 2 \\ 1 \\ 5 \end{bmatrix} \right\}$
\end{readinessAssuranceTestChoices}
\end{multicols}


\item Determine which property applies to the set of vectors $$\left\{ \begin{bmatrix}  1 \\ 0 \\ 0 \end{bmatrix}, \begin{bmatrix} 0 \\ 1 \\ 0 \end{bmatrix} \right\} \subset \IR^3.$$
%C
\begin{readinessAssuranceTestChoices}
\item It does not span \(\IR^3\) and is linearly dependent
\item It spans \(\IR^3\) but it is linearly dependent
\item It does not span \(\IR^3\) and is linearly independent %correct
\item It is a basis of \(\IR^3\).
\end{readinessAssuranceTestChoices}


\item Determine which property applies to the set of vectors $$\left\{ \begin{bmatrix}  1 \\ 0 \\ 0 \end{bmatrix}, \begin{bmatrix} 2 \\ 1 \\ 0 \end{bmatrix} , \begin{bmatrix} 1 \\ 1 \\ 3 \end{bmatrix} \right\}\subset \IR^3.$$
%C
\begin{readinessAssuranceTestChoices}
\item It does not span \(\IR^3\) and is linearly dependent
\item It does not span \(\IR^3\) and is linearly independent
\item It is a basis of \(\IR^3\). %Correct
\item It spans \(\IR^3\) but it is linearly dependent
\end{readinessAssuranceTestChoices}


\item Determine which property applies to the set of vectors $$\left\{ \begin{bmatrix}  1 \\ 0 \\ 0 \end{bmatrix}, \begin{bmatrix} -2 \\ 0 \\ -2 \end{bmatrix} , \begin{bmatrix} 1 \\ 1 \\ 0 \end{bmatrix} , \begin{bmatrix} 3 \\ 3 \\ -3 \end{bmatrix}\right\}\subset \IR^3.$$
%B
\begin{readinessAssuranceTestChoices}
\item It does not span \(\IR^3\) and is linearly dependent
\item It spans \(\IR^3\) but it is linearly dependent %correct
\item It does not span \(\IR^3\) and is linearly independent
\item It is a basis of \(\IR^3\).
\end{readinessAssuranceTestChoices}


\item Determine which property applies to the set of vectors $$\left\{ \begin{bmatrix}  2 \\ 2 \\ -1 \end{bmatrix}, \begin{bmatrix} -3 \\ 1 \\ -2 \end{bmatrix} , \begin{bmatrix} 1 \\ 5 \\ -4 \end{bmatrix}\right\}\subset \IR^3.$$
%D
\begin{readinessAssuranceTestChoices}
\item It spans \(\IR^3\) but it is linearly dependent
\item It is a basis of $\IR^3$.
\item It does not span \(\IR^3\) and is linearly independent
\item It does not span \(\IR^3\) and is linearly dependent %correct
\end{readinessAssuranceTestChoices}


\item Suppose \(S\) is a set of \(\IR^5\) vectors, and you know that every vector in
      \(\vspan S\) can be written \textit{uniquely} as a linear combination of the vectors
      in \(S\).  What can you conclude about \(S\)?
%B
\begin{readinessAssuranceTestChoices}
\item \(S\) has exactly \(5\) vectors
\item \(S\) has at most \(5\) vectors
\item \(S\) has at least \(5\) vectors
\item \(S\) could have any number of vectors
\end{readinessAssuranceTestChoices}

\item Suppose you know that every vector in \(\IR^5\) can be written as a linear combination of 
      the vectors in a set \(S\).  What can you conclude about \(S\)?
%C
\begin{readinessAssuranceTestChoices}
\item \(S\) has exactly \(5\) vectors
\item \(S\) has at most \(5\) vectors
\item \(S\) has at least \(5\) vectors
\item \(S\) could have any number of vectors
\end{readinessAssuranceTestChoices}

\item Suppose you know that every vector in \(\IR^5\) can be \textit{uniquely}
      written as a linear combination of 
      the vectors in a set \(S\).  What can you conclude about \(S\)?
%A
\begin{readinessAssuranceTestChoices}
\item \(S\) has exactly \(5\) vectors
\item \(S\) has at most \(5\) vectors
\item \(S\) has at least \(5\) vectors
\item \(S\) could have any number of vectors
\end{readinessAssuranceTestChoices}

\item What else can you conclude about \(S\) from the previous question?
%A
\begin{readinessAssuranceTestChoices}
\item \(S\) is a basis of \(\IR^5\).
\item \(S\) does not span \(\IR^5\) and is linearly dependent
\item \(S\) does not span \(\IR^5\) and is linearly independent
\item \(S\) spans \(\IR^5\) but it is linearly dependent
\end{readinessAssuranceTestChoices}

\end{readinessAssuranceTest}
