\begin{applicationActivities}

\begin{remark}
Recall that a linear map \(T:V\rightarrow W\)
satisfies
\begin{enumerate}
\item \(T(\vec{v}+\vec{w}) = T(\vec{v})+T(\vec{w})\)
      for any \(\vec{v},\vec{w} \in V\).
\item \(T(c\vec{v}) = cT(\vec{v})\)
      for any \(c \in \IR,\vec{v} \in V\).
\end{enumerate}
In other words, a map is linear when vecor space operations
can be applied before or after the transformation without affecting the result.
\end{remark}

\begin{activity}{5}
Suppose \(T: \IR^3 \rightarrow \IR^2\) is a linear map, and you know
\(
  T\left(\begin{bmatrix} 1 \\ 0 \\ 0 \end{bmatrix} \right)
=
  \begin{bmatrix} 2 \\ 1 \end{bmatrix}
\)
and
\(
  T\left(\begin{bmatrix} 0 \\ 0 \\ 1 \end{bmatrix} \right)
=
  \begin{bmatrix} -3 \\ 2 \end{bmatrix}
\).
Compute \(T\left(\begin{bmatrix} 3 \\ 0 \\ 0 \end{bmatrix}\right)\).
\begin{multicols}{2}
\begin{enumerate}[(a)]
\item \(\begin{bmatrix} 6 \\ 3\end{bmatrix}\)
\item \(\begin{bmatrix} -9 \\ 6 \end{bmatrix}\)
\item \(\begin{bmatrix} -4 \\ -2 \end{bmatrix}\)
\item \(\begin{bmatrix} 6 \\ -4 \end{bmatrix}\)
\end{enumerate}
\end{multicols}
\end{activity}

\begin{activity}{3}
Suppose \(T: \IR^3 \rightarrow \IR^2\) is a linear map, and you know
\(
  T\left(\begin{bmatrix} 1 \\ 0 \\ 0 \end{bmatrix} \right)
=
  \begin{bmatrix} 2 \\ 1 \end{bmatrix}
\)
and
\(
  T\left(\begin{bmatrix} 0 \\ 0 \\ 1 \end{bmatrix} \right)
=
  \begin{bmatrix} -3 \\ 2 \end{bmatrix}
\).
Compute \(T\left(\begin{bmatrix} 1 \\ 0 \\ 1 \end{bmatrix}\right)\).
\begin{multicols}{2}
\begin{enumerate}[(a)]
\item \(\begin{bmatrix} 2 \\ 1\end{bmatrix}\)
\item \(\begin{bmatrix} 3 \\ -1 \end{bmatrix}\)
\item \(\begin{bmatrix} -1 \\ 3 \end{bmatrix}\)
\item \(\begin{bmatrix} 5 \\ -8 \end{bmatrix}\)
\end{enumerate}
\end{multicols}
\end{activity}

\begin{activity}{2}
Suppose \(T: \IR^3 \rightarrow \IR^2\) is a linear map, and you know
\(
  T\left(\begin{bmatrix} 1 \\ 0 \\ 0 \end{bmatrix} \right)
=
  \begin{bmatrix} 2 \\ 1 \end{bmatrix}
\)
and
\(
  T\left(\begin{bmatrix} 0 \\ 0 \\ 1 \end{bmatrix} \right)
=
  \begin{bmatrix} -3 \\ 2 \end{bmatrix}
\).
Compute \(T\left(\begin{bmatrix} -2 \\ 0 \\ -3 \end{bmatrix}\right)\).
\begin{multicols}{2}
\begin{enumerate}[(a)]
\item \(\begin{bmatrix} 2 \\ 1\end{bmatrix}\)
\item \(\begin{bmatrix} 3 \\ -1 \end{bmatrix}\)
\item \(\begin{bmatrix} -1 \\ 3 \end{bmatrix}\)
\item \(\begin{bmatrix} 5 \\ -8 \end{bmatrix}\)
\end{enumerate}
\end{multicols}
\end{activity}

\begin{activity}{5}
Suppose \(T: \IR^3 \rightarrow \IR^2\) is a linear map, and you know
\(
  T\left(\begin{bmatrix} 1 \\ 0 \\ 0 \end{bmatrix} \right)
=
  \begin{bmatrix} 2 \\ 1 \end{bmatrix}
\)
and
\(
  T\left(\begin{bmatrix} 0 \\ 0 \\ 1 \end{bmatrix} \right)
=
  \begin{bmatrix} -3 \\ 2 \end{bmatrix}
\).
Do you have enough information to compute
\(T(\vec{v})\) for \textit{any} \(\vec v\in\IR^3\)?
\begin{enumerate}[(a)]
\item Yes.
\item No, exactly one more piece of information is needed.
\item No, an infinite amount of information would be necessary to compute
      the transformation of infinitely-many vectors.
\end{enumerate}
\end{activity}

\begin{fact}
Consider any basis \(\{\vec b_1,\dots,\vec b_n\}\) for $V$.  Since every
vector \(\vec v\) can be written \textit{uniquely} as a linear combination of
basis vectors, \(x_1\vec b_1+\dots+ x_n\vec b_n\), we may compute
\(T(\vec v)\) as follows:

\[
  T(\vec v)=T(x_1\vec b_1+\dots+ x_n\vec b_n)=
  x_1T(\vec b_1)+\dots+x_nT(\vec b_n)
.\]

Therefore any linear transformation \(T:V \rightarrow W\) can be defined
by just describing the values of \(T(\vec b_i)\).

Put another way, the images of the basis vectors \term{determine} the transformation \(T\).
\end{fact}

\begin{definition}
Since linear transformation \(T:\IR^n\to\IR^m\) is determined by
the standard basis \(\{\vec e_1,\dots,\vec e_n\}\), it's convenient to
store this information in the \(m\times n\) \term{standard matrix}
\([T(\vec e_1) \,\cdots\, T(\vec e_n)]\).

\vspace{1em}

For example,
let \(T: \IR^3 \rightarrow \IR^2\) be the linear map determined by
the following values for \(T\) applied to the standard basis of \(\IR^3\).

\[\scriptsize
  T\left(\vec e_1 \right)
=
  T\left(\begin{bmatrix} 1 \\ 0 \\ 0 \end{bmatrix} \right)
=
  \begin{bmatrix} 3 \\ 2\end{bmatrix}
\hspace{2em}
  T\left(\vec e_2 \right)
=
  T\left(\begin{bmatrix} 0 \\ 1 \\ 0 \end{bmatrix} \right)
=
  \begin{bmatrix} -1 \\ 4\end{bmatrix}
\hspace{2em}
  T\left(\vec e_3 \right)
=
  T\left(\begin{bmatrix} 0 \\ 0 \\ 1 \end{bmatrix} \right)
=
  \begin{bmatrix} 5 \\ 0\end{bmatrix}
\]

Then the standard matrix corresponding to \(T\) is
\[
  \begin{bmatrix}T(\vec e_1) & T(\vec e_2) & T(\vec e_3)\end{bmatrix}
=
  \begin{bmatrix}3 & -1 & 5 \\ 2 & 4 & 0 \end{bmatrix}
.\]
\end{definition}

\begin{activity}{3}
  Let $T: \IR^4 \rightarrow \IR^3$ be the linear transformation given by
  \[
    T\left(\vec e_1 \right)
  =
    \begin{bmatrix} 0 \\ 3 \\ -2\end{bmatrix}
  \hspace{2em}
    T\left(\vec e_2 \right)
  =
    \begin{bmatrix} -3 \\ 0 \\ 1\end{bmatrix}
  \hspace{2em}
    T\left(\vec e_3 \right)
  =
    \begin{bmatrix} 4 \\ -2 \\ 1\end{bmatrix}
  \hspace{2em}
    T\left(\vec e_4 \right)
  =
    \begin{bmatrix} 2 \\ 0 \\ 0\end{bmatrix}
  \]
Write the standard matrix \([T(\vec e_1) \,\cdots\, T(\vec e_n)]\) for \(T\).
\end{activity}

\begin{activity}{5}
  Let $T: \IR^3 \rightarrow \IR^2$ be the linear transformation given by
$$T\left(\begin{bmatrix} x\\ y \\ z \end{bmatrix} \right) = \begin{bmatrix} x+3z \\ 2x-y-4z \end{bmatrix}$$
Find the standard matrix for \(T\).
\end{activity}

\begin{fact}
  Because every linear map \(T:\IR^m\to\IR^n\) has a linear combination
  of the variables in each component, and thus
  \(T(\vec e_i)\) yields exactly the coefficients of \(x_i\),
  the standard matrix for \(T\) is simply an ordered list of
  the coefficients of the \(x_i\):
  \[
    T\left(\begin{bmatrix}x\\y\\z\\w\end{bmatrix}\right)
      =
    \begin{bmatrix}
      ax+by+cz+dw \\
      ex+fy+gz+hw
    \end{bmatrix}
  \hspace{2em}
    A
      =
    \begin{bmatrix}
      a & b & c & d \\
      e & f & g & h
    \end{bmatrix}
  \]
\end{fact}

\begin{activity}{5}
  Let $T: \IR^3 \rightarrow \IR^3$ be the linear transformation given by the standard matrix
  \[
    \begin{bmatrix} 3  & -2 & -1  \\ 4 & 5 & 2 \\ 0 & -2 & 1 \end{bmatrix}
  .\]

Compute $T\left(\begin{bmatrix} x\\ y \\ z \end{bmatrix} \right) $.
\end{activity}

\begin{activity}{5}
  Let $T: \IR^3 \rightarrow \IR^3$ be the linear transformation given by the standard matrix
  \[
    \begin{bmatrix} 3  & -2 & -1  \\ 4 & 5 & 2 \\ 0 & -2 & 1 \end{bmatrix}
  .\]

Compute $T\left(\begin{bmatrix} 1\\ 2 \\ 3 \end{bmatrix} \right) $.
\end{activity}

\begin{fact}
  To quickly compute \(T(\vec v)\) from its standard matrix \(A\),
  compute the \textbf{dot product} (defined in Calculus 3) of each matrix row
  with the vector. For example, if \(T\) has the standard matrix
  \[
    A = \begin{bmatrix} 1 & 2 & 3 \\ 0 & 1 & -2 \\ 2 & -1 & 0 \end{bmatrix}
  \]
  then for \(\vec v=\begin{bmatrix}x\\y\\z\end{bmatrix}\) we will write
  \[
    T(\vec v)
      =
    A\vec v
      =
    \begin{bmatrix} 1 & 2 & 3 \\ 0 & 1 & -2 \\ 2 & -1 & 0 \end{bmatrix}
    \begin{bmatrix}x\\y\\z\end{bmatrix}
      =
    \begin{bmatrix}
      1x+2y+3z\\
      0x+1y-2z\\
      2x-1y+0z
    \end{bmatrix}
  \]
  and for \(\vec v=\begin{bmatrix}3\\0\\-2\end{bmatrix}\) we will write
  \[
    T(\vec v)
      =
    A\vec v
      =
    \begin{bmatrix} 1 & 2 & 3 \\ 0 & 1 & -2 \\ 2 & -1 & 0 \end{bmatrix}
    \begin{bmatrix}3\\0\\-2\end{bmatrix}
      =
    \begin{bmatrix}
      1(3)+2(0)+3(-2)\\
      0(3)+1(0)-2(-2)\\
      2(3)-1(0)+0(-2)
    \end{bmatrix}
      =
    \begin{bmatrix}
      -3\\
      4\\
      6
    \end{bmatrix}
  .\]
\end{fact}

\begin{activity}{15}
  Compute the following linear transformations of vectors given their
  standard matrices.
  \[
    T_1\left(\begin{bmatrix}1\\2\end{bmatrix}\right)
    \text{ for the standard matrix }
    A_1=\begin{bmatrix}4&3\\0&-1\\1&1\\3&0\end{bmatrix}
  \]
  \[
    T_2\left(\begin{bmatrix}1\\1\\0\\-3\end{bmatrix}\right)
    \text{ for the standard matrix }
    A_2=\begin{bmatrix}4&3&0&-1\\1&1&3&0\end{bmatrix}
  \]
  \[
    T_3\left(\begin{bmatrix}0\\-2\\0\end{bmatrix}\right)
    \text{ for the standard matrix }
    A_3=\begin{bmatrix}4&3&0\\0&-1&3\\5&1&1\\3&0&0\end{bmatrix}
  \]
\end{activity}

\end{applicationActivities}
