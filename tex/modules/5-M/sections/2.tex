\begin{applicationActivities}

\begin{observation}
Recall that if \(T: \IR^n \rightarrow \IR^k\) is a linear map with standard matrix \(B \in M_{k,n}\) and \(S: \IR^k \rightarrow \IR^m\) is a linear map with standard matrix \(A \in M_{m,k}\), the product matrix \(AB \in M_{m,n}\) is defined to be the standard matrix of the composition map \[S \circ T: \IR^n \rightarrow \IR^m.\]
\end{observation}


\begin{activity}{5}
Matrix multiplication only makes sense if the first matrix has as many
columns as the second matrix has rows. Label each of these matrices
with \textbf{rows \(\times\) columns}, and then figure out which of
the products \(AB,AC,BA,BC,CA,CB\) can actually be computed.

\[
  A = \begin{bmatrix}1&0&-3\\3&2&1\end{bmatrix}
    \hspace{2em}
  B = \begin{bmatrix}2&2&1&0&1\\1&1&1&-1&0\\0&0&3&2&1\\-1&5&7&2&1\end{bmatrix}
    \hspace{2em}
  C = \begin{bmatrix}2&2\\0&-1\\3&1\\4&0\end{bmatrix}
\]
\end{activity}




\begin{activity}{10}
Let \(B=\begin{bmatrix}1 & 0  & 2 \\ 0 & 1 & 0 \\ 0 & 0 & 1 \end{bmatrix}\), and let \(A=\begin{bmatrix} 2 & 7 & -1 \\ 0 & 3 & 2 \\ 1 & 1 & -1 \end{bmatrix}\).  Compute the product \(BA\).
\end{activity}

\begin{activity}{5}
Let $A=\begin{bmatrix} 2 & 7 & -1 \\ 0 & 3 & 2 \\ 1 & 1 & -1 \end{bmatrix}$.  Find a $3 \times 3$ matrix $I$ such that $IA=A$, that is,
\[
  \begin{bmatrix} \unknown & \unknown & \unknown \\ \unknown & \unknown & \unknown \\ \unknown & \unknown & \unknown \end{bmatrix}
  \begin{bmatrix} 2 & 7 & -1 \\ 0 & 3 & 2 \\ 1 & 1 & -1 \end{bmatrix}
=
  \begin{bmatrix} 2 & 7 & -1 \\ 0 & 3 & 2 \\ 1 & 1 & -1 \end{bmatrix}
\]
\end{activity}

\begin{definition}
The identity matrix $I_n$ (or just $I$ when $n$ is obvious from context) is  the $n \times n$ matrix $$I_n = \begin{bmatrix} 1 & 0  & \hdots & 0 \\ 0 & 1 & \ddots & \vdots  \\ \vdots & \ddots & \ddots & 0 \\ 0 & \hdots & 0 & 1 \end{bmatrix}.$$
It has a $1$ on each diagonal element and a $0$ in every other position.
\end{definition}

\begin{fact}
  For any square matrix \(A\), \(IA=AI=A\):

  \[
    \begin{bmatrix} 1 & 0 & 0 \\ 0 & 1 & 0 \\ 0 & 0 & 1 \end{bmatrix}
    \begin{bmatrix} 2 & 7 & -1 \\ 0 & 3 & 2 \\ 1 & 1 & -1 \end{bmatrix}
  =
    \begin{bmatrix} 2 & 7 & -1 \\ 0 & 3 & 2 \\ 1 & 1 & -1 \end{bmatrix}
      \begin{bmatrix} 1 & 0 & 0 \\ 0 & 1 & 0 \\ 0 & 0 & 1 \end{bmatrix}
  =
    \begin{bmatrix} 2 & 7 & -1 \\ 0 & 3 & 2 \\ 1 & 1 & -1 \end{bmatrix}
  \]
\end{fact}

\begin{activity}{20}
Each row operation can be interpreted as a type of matrix multiplication.
\begin{subactivity}
Tweak the identity matrix slightly to create a matrix that
doubles the third row of $A$:
\[
 \begin{bmatrix} \unknown & \unknown & \unknown \\ \unknown & \unknown & \unknown \\ \unknown & \unknown & \unknown \end{bmatrix}
 \begin{bmatrix} 2 & 7 & -1 \\ 0 & 3 & 2 \\ 1 & 1 & -1 \end{bmatrix}
=
 \begin{bmatrix} 2 & 7 & -1 \\ 0 & 3 & 2 \\ 2 & 2 & -2 \end{bmatrix}
\]
\end{subactivity}
\begin{subactivity}
  Create a matrix that swaps the second and third rows of $A$:
  \[
   \begin{bmatrix} \unknown & \unknown & \unknown \\ \unknown & \unknown & \unknown \\ \unknown & \unknown & \unknown \end{bmatrix}
   \begin{bmatrix} 2 & 7 & -1 \\ 0 & 3 & 2 \\ 1 & 1 & -1 \end{bmatrix}
  =
  \begin{bmatrix} 2 & 7 & -1 \\ 1 & 1 & -1 \\ 0 & 3 & 2 \end{bmatrix}
  \]
\end{subactivity}
\begin{subactivity}
Create a matrix that adds $5$ times the third row of $A$ to the first row:
\[
 \begin{bmatrix} \unknown & \unknown & \unknown \\ \unknown & \unknown & \unknown \\ \unknown & \unknown & \unknown \end{bmatrix}
 \begin{bmatrix} 2 & 7 & -1 \\ 0 & 3 & 2 \\ 1 & 1 & -1 \end{bmatrix}
=
 \begin{bmatrix} 2+5(1) & 7+5(1) & -1+5(-1) \\ 0 & 3 & 2 \\ 1 & 1 & -1 \end{bmatrix}
\]
\end{subactivity}
\end{activity}

\begin{fact}
If \(R\) is the result of applying a row operation to \(I\), then
\(RA\) is the result of applying the same row operation to \(A\).

This means that for any matrix $A$, we can find a series of matrices $R_1, \ldots, R_k$ corresponding to the row operations such that $$R_1 R_2 \cdots R_k A = \RREF(A).$$ That is, row reduction can be thought of as the result
of matrix multiplication.
\end{fact}






\end{applicationActivities}
