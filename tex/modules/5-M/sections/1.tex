\begin{applicationActivities}


\begin{activity}{5}
Let $T: \IR^3 \rightarrow \IR^2$ be given by the \(2\times 3\) standard matrix $B=\begin{bmatrix} 2 & 1 & -3 \\ 5 & -3 & 4 \end{bmatrix}$ and $S: \IR^2 \rightarrow \IR^4$ be given by the \(4\times 2\) standard matrix $A=\begin{bmatrix} 1 & 2 \\ 0 & 1 \\ 3 & 5 \\ -1 & -2 \end{bmatrix}$.

What is the domain of the composition map $S \circ T$?
\begin{enumerate}[(a)]
\item $\IR$
\item $\IR^2$
\item $\IR^3$
\item $\IR^4$
\end{enumerate}
\end{activity}

\begin{activity}{3}
Let $T: \IR^3 \rightarrow \IR^2$ be given by the \(2\times 3\) standard matrix $B=\begin{bmatrix} 2 & 1 & -3 \\ 5 & -3 & 4 \end{bmatrix}$ and $S: \IR^2 \rightarrow \IR^4$ be given by the \(4\times 2\) standard matrix $A=\begin{bmatrix} 1 & 2 \\ 0 & 1 \\ 3 & 5 \\ -1 & -2 \end{bmatrix}$.

What is the codomain of the composition map $S \circ T$?
\begin{enumerate}[(a)]
\item $\IR$
\item $\IR^2$
\item $\IR^3$
\item $\IR^4$
\end{enumerate}
\end{activity}

\begin{activity}{2}
Let $T: \IR^3 \rightarrow \IR^2$ be given by the \(2\times 3\) standard matrix $B=\begin{bmatrix} 2 & 1 & -3 \\ 5 & -3 & 4 \end{bmatrix}$ and $S: \IR^2 \rightarrow \IR^4$ be given by the \(4\times 2\) standard matrix $A=\begin{bmatrix} 1 & 2 \\ 0 & 1 \\ 3 & 5 \\ -1 & -2 \end{bmatrix}$.

What size will the standard matrix of $S \circ T:\IR^3\to\IR^4$ be?
(Rows \(\times\) Columns)
\begin{multicols}{3}
\begin{enumerate}[(a)]
\item $4 \times 3$
\item $4 \times 2$
\item $3 \times 4$
\item $3 \times 2$
\item $2 \times 4$
\item $2 \times 3$
\end{enumerate}
\end{multicols}
\end{activity}

\begin{activity}{15}
Let $T: \IR^3 \rightarrow \IR^2$ be given by the \(2\times 3\) standard matrix $B=\begin{bmatrix} 2 & 1 & -3 \\ 5 & -3 & 4 \end{bmatrix}$ and $S: \IR^2 \rightarrow \IR^4$ be given by the \(4\times 2\) standard matrix $A=\begin{bmatrix} 1 & 2 \\ 0 & 1 \\ 3 & 5 \\ -1 & -2 \end{bmatrix}$.

\begin{subactivity}
Compute
\[
  (S \circ T)(\vec{e}_1)
    =
  S(T(\vec{e}_1))
    =
  S\left(\begin{bmatrix} 2 \\ 5\end{bmatrix}\right)
        =
  \begin{bmatrix}\unknown\\\unknown\\\unknown\\\unknown\end{bmatrix}.
\]

\end{subactivity}
\begin{subactivity}
Compute
\(
  (S \circ T)(\vec{e}_2)
\).
\end{subactivity}
\begin{subactivity}
Compute
\(
  (S \circ T)(\vec{e}_3)
\).
\end{subactivity}
\begin{subactivity}
Find the \(4\times 3\) standard matrix of $S \circ T:\IR^3\to\IR^4$.
\end{subactivity}
\end{activity}

\begin{definition}
We define the \term{product} \(AB\) of a $m \times n$ matrix $A$ and a
$n \times k$
matrix $B$ to be the $m \times k$ standard matrix of the composition map of the
two corresponding linear functions.

\vspace{1em}

For the previous activity, \(S\) had a \(4\times \circledNumber{2}\) matrix and
\(T\) had a \(\circledNumber{2}\times 3\) matrix, so \(S\circ T\) had a
\(4\times 3\) standard matrix:
\[
  AB
    =
  \begin{bmatrix} 1 & 2 \\ 0 & 1 \\ 3 & 5 \\ -1 & -2 \end{bmatrix}
  \begin{bmatrix} 2 & 1 & -3 \\ 5 & -3 & 4 \end{bmatrix}
\]
\[
    =
  \left[
  (S \circ T)(\vec{e}_1) (S\circ T)(\vec{e}_2) (S \circ T)(\vec{e}_3)
  \right]
    =
  \begin{bmatrix}
    12 & -5 & 5 \\
    5 & -3 & 4 \\
    31 & -12 & 11 \\
    -12 & 5 & -5
  \end{bmatrix}
\].
\end{definition}



\begin{activity}{10}
Let $T: \IR^2 \rightarrow \IR^3$ be given by the matrix $B=\begin{bmatrix} 2 & 3 \\ 1 & -1 \\ 0 & -1 \end{bmatrix}$ and $S: \IR^3 \rightarrow \IR^2$ be given by the matrix $A=\begin{bmatrix} -4 & -2 & 3 \\ 0 & 1 & 1 \end{bmatrix}$.

Find the standard matrix \(AB\) of $S \circ T$.
\end{activity}

\begin{activity}{5}
Let $T: \IR^2 \rightarrow \IR^3$ be given by the matrix $B=\begin{bmatrix} 2 & 3 \\ 1 & -1 \\ 0 & -1 \end{bmatrix}$ and $S: \IR^3 \rightarrow \IR^2$ be given by the matrix $A=\begin{bmatrix} -4 & -2 & 3 \\ 0 & 1 & 1 \end{bmatrix}$.

Find the standard matrix \(BA\) of $T \circ S$.
\end{activity}



\begin{activity}{5}
Matrix multiplication only makes sense if the first matrix has as many
columns as the second matrix has rows. Label each of these matrices
with \textbf{rows \(\times\) columns}, and then figure out which of
the products \(AB,AC,BA,BC,CA,CB\) can actually be computed.

\[
  A = \begin{bmatrix}1&0&-3\\3&2&1\end{bmatrix}
    \hspace{2em}
  B = \begin{bmatrix}2&2&1&0&1\\1&1&1&-1&0\\0&0&3&2&1\\-1&5&7&2&1\end{bmatrix}
    \hspace{2em}
  C = \begin{bmatrix}2&2\\0&-1\\3&1\\4&0\end{bmatrix}
\]
\end{activity}



\begin{observation}
Note that an \(\IR^n\) vector acts exactly the same as an \(n\times 1\)
matrix, so we will use them interchangablely, as follows.

\begin{align*}
A&=\begin{bmatrix} 3 & 1 & -1 \\ 2 & 0 & 4 \\ -1 & 3 & 5  \end{bmatrix} & X=\vec{x}&=\begin{bmatrix} x  \\ y \\ z  \end{bmatrix} & B=\vec{b} &= \begin{bmatrix} 5 \\ -7 \\ 2 \end{bmatrix}
\end{align*}

So we may study the linear system
\begin{align*}
3x+y-z &= 5 \\ 2x+4z &= -7 \\ -x+3y+5z &=2
\end{align*}
as both a vector equation \(A\vec{x}=\vec{b}\) and a matrix equation
\(AX=B\):
\[
  \begin{bmatrix} 3 & 1 & -1 \\ 2 & 0 & 4 \\ -1 & 3 & 5  \end{bmatrix}
  \begin{bmatrix} x  \\ y \\ z  \end{bmatrix}
    =
  \begin{bmatrix} 5 \\ -7 \\ 2 \end{bmatrix}
\]
\end{observation}



\end{applicationActivities}
