\begin{applicationActivities}

\begin{definition}
  Let $T: \IR^n \rightarrow \IR^n$ be a linear map with standard matrix $A$.
\begin{itemize}
\item  If $T$ is a bijection and $B$ is any $\IR^n$ vector, then $T(X)=AX=B$ has a unique solution $X$.
\item So we may define an \term{inverse map} $T^{-1} : \IR^n \rightarrow \IR^n$ by setting $T^{-1}(B)=X$ to be this unique solution.
\item Let $A^{-1}$ be the standard matrix for $T^{-1}$. We call $A^{-1}$ the
\term{inverse matrix} of $A$, so we also say that $A$ is \term{invertible}.
\end{itemize}
\end{definition}

\begin{activity}{10}
  Let \(T:\IR^2\to\IR^2\) be the bijective linear map defined by
  \(T\left(\begin{bmatrix}x\\y\end{bmatrix}\right)=\begin{bmatrix} 2x -3y \\ -3x + 5y\end{bmatrix}\). It can be shown that \(T\) is bijective and
  has the inverse map
  \(T^{-1}\left(\begin{bmatrix}x\\y\end{bmatrix}\right)=\begin{bmatrix} 5x+ 3y \\ 3x + 2y\end{bmatrix}\).
  \begin{subactivity}
    Compute \((T^{-1}\circ T)\left(\begin{bmatrix}-2\\1\end{bmatrix}\right)\).
  \end{subactivity}
  \begin{subactivity}
    If \(A\) is the standard matrix for \(T\) and \(A^{-1}\) is the
    standard matrix for \(T^{-1}\), what must \(A^{-1}A\) be?
  \end{subactivity}
\end{activity}

\begin{observation}
  \(T^{-1}\circ T=T\circ T^{-1}\) is the identity map for any bijective
  linear transformation \(T\). Therefore
  \(A^{-1}A=AA^{-1}=I\) is the identity matrix for any invertible matrix
  \(A\).
\end{observation}


\begin{activity}{20}
  Let \(T: \IR^3 \rightarrow \IR^3\) be given by the matrix
  \(A=\begin{bmatrix} 2 & -1 & -6 \\ 2 & 1 & 3 \\ 1 & 1 & 4 \end{bmatrix}\).
  \begin{subactivity}
  Solve $T(X)=\vec{e}_1$ to find $T^{-1}(\vec{e}_1)$.
  \end{subactivity}
  \begin{subactivity}
   Solve $T(X)=\vec{e}_2$ to find $T^{-1}(\vec{e}_2)$.
  \end{subactivity}
  \begin{subactivity}
   Solve $T(X)=\vec{e}_3$ to find $T^{-1}(\vec{e}_3)$.
  \end{subactivity}
  \begin{subactivity}
  Compute $A^{-1}$, the standard matrix for $T^{-1}$.
  \end{subactivity}
\end{activity}

\begin{observation}
We could have solved these three systems simultaneously
by row reducing the matrix \([A\,|\,I]\) at once.
\[
  A=\begin{bmatrix}[ccc|ccc]
    2 & -1 & -6 & 1 & 0 & 0 \\
    2 & 1 & 3 & 0 & 1 & 0 \\
    1 & 1 & 4 & 0 & 0 & 1
  \end{bmatrix} \sim
  \begin{bmatrix}[ccc|ccc]
    1 & 0 & 0 & 1 & -2 & 3 \\
    0 & 1 & 0 & -5 & 14 & -18 \\
    0 & 0 & 1 & 1 & -3 & 4
  \end{bmatrix}
\]
\end{observation}


\begin{activity}{10}
  Find the inverse \(A^{-1}\) of the matrix
  \(A=\begin{bmatrix} 1 & 3 \\ 0 & -2 \end{bmatrix}\)
  by row-reducing \([A\,|\,I]\).
\end{activity}

\begin{activity}{10}
Is the matrix $\begin{bmatrix} 2 & 3 & 1 \\ -1 & -4 & 2 \\ 0 & -5 & 5 \end{bmatrix}$ invertible? Give a reason for your answer.
\end{activity}

\begin{observation}
 A matrix $A \in \IR^{n \times n}$ is invertible if and only if $\RREF(A) = I_n$.
\end{observation}

\end{applicationActivities}
