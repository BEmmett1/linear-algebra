\begin{applicationActivities}

\begin{activity}{15}
Let $T: \IR^n \rightarrow \IR^m$ be a linear map with standard matrix $A$.
Sort the following items into three groups of statements: a group that means
\(T\) is \textbf{injective}, a group that means \(T\) is \textbf{surjective},
and a group that means \(T\) is \textbf{bijective}.
\begin{multicols}{2}
\begin{enumerate}[(a)]
\item \(A\vec x=\vec b\) has a solution for all \(\vec b\in\IR^m\)
\item \(A\vec x=\vec b\) has a unique solution for all \(\vec b\in\IR^m\)
\item \(A\vec x=\vec 0\) has a unique solution.
\item The columns of $A$ span $\IR^m$
\item The columns of $A$ are linearly independent
\item The columns of $A$ are a basis of $\IR^m$
\item Every column of $\RREF(A)$ has a pivot
\item Every row of $\RREF(A)$ has a pivot
\item $m=n$ and $\RREF(A)=I$
\end{enumerate}
\end{multicols}
\end{activity}

\begin{definition}
  Let $T: \IR^n \rightarrow \IR^n$ be a linear map with standard matrix $A$.
\begin{itemize}
\item  If $T$ is a bijection and \(\vec b\) is any $\IR^n$ vector, 
  then $T(\vec x)=A\vec x=\vec b$ has a unique solution.
\item So we may define an \term{inverse map} $T^{-1} : \IR^n \rightarrow \IR^n$ 
  by setting $T^{-1}(\vec b)$ to be this unique solution.
\item Let $A^{-1}$ be the standard matrix for $T^{-1}$. We call $A^{-1}$ the
\term{inverse matrix} of $A$, so we also say that $A$ is \term{invertible}.
\end{itemize}
\end{definition}


\begin{activity}{20}
  Let \(T: \IR^3 \rightarrow \IR^3\) be the linear transformation given by the standard matrix
  \(A=\begin{bmatrix} 2 & -1 & -6 \\ 2 & 1 & 3 \\ 1 & 1 & 4 \end{bmatrix}\).
  \begin{subactivity}
  Write an augmented matrix representing the system of equations given by \(T(\vec x)=\vec{e}_1\),
  that is, \(A\vec x=\begin{bmatrix}1 \\ 0 \\ 0 \end{bmatrix}\).
  Then solve \(T(\vec x)=\vec{e}_1\) to find \(T^{-1}(\vec{e}_1)\).
  \end{subactivity}
  \begin{subactivity}
   Solve \(T(\vec x)=\vec{e}_2\) to find \(T^{-1}(\vec{e}_2)\).
  \end{subactivity}
  \begin{subactivity}
   Solve \(T(\vec x)=\vec{e}_3\) to find \(T^{-1}(\vec{e}_3)\).
  \end{subactivity}
  \begin{subactivity}
   Write \(A^{-1}\), the standard matrix for \(T^{-1}\).
  \end{subactivity}
\end{activity}

\begin{observation}
We could have solved these three systems simultaneously
by row reducing the matrix \([A\,|\,I]\) at once.
\[
  \begin{bmatrix}[ccc|ccc]
    2 & -1 & -6 & 1 & 0 & 0 \\
    2 & 1 & 3 & 0 & 1 & 0 \\
    1 & 1 & 4 & 0 & 0 & 1
  \end{bmatrix} \sim
  \begin{bmatrix}[ccc|ccc]
    1 & 0 & 0 & 1 & -2 & 3 \\
    0 & 1 & 0 & -5 & 14 & -18 \\
    0 & 0 & 1 & 1 & -3 & 4
  \end{bmatrix}
\]
\end{observation}


\begin{activity}{5}
  Find the inverse \(A^{-1}\) of the matrix
  \(A=\begin{bmatrix} 1 & 3 \\ 0 & -2 \end{bmatrix}\)
  by row-reducing \([A\,|\,I]\).
\end{activity}

\begin{activity}{5}
Is the matrix $\begin{bmatrix} 2 & 3 & 1 \\ -1 & -4 & 2 \\ 0 & -5 & 5 \end{bmatrix}$ invertible? Give a reason for your answer.
\end{activity}

\begin{observation}
    An \(n\times n\) matrix $A$ is invertible if and only if $\RREF(A) = I_n$.
\end{observation}

\begin{activity}{10}
  Let \(T:\IR^2\to\IR^2\) be the bijective linear map defined by
  \(T\left(\begin{bmatrix}x\\y\end{bmatrix}\right)=\begin{bmatrix} 2x -3y \\ -3x + 5y\end{bmatrix}\),
  with the inverse map
  \(T^{-1}\left(\begin{bmatrix}x\\y\end{bmatrix}\right)=\begin{bmatrix} 5x+ 3y \\ 3x + 2y\end{bmatrix}\).
  \begin{subactivity}
    Compute \((T^{-1}\circ T)\left(\begin{bmatrix}-2\\1\end{bmatrix}\right)\).
  \end{subactivity}
  \begin{subactivity}
    If \(A\) is the standard matrix for \(T\) and \(A^{-1}\) is the
    standard matrix for \(T^{-1}\), find the \(2\times 2\) matrix 
    \[A^{-1}A=\begin{bmatrix}\unknown&\unknown\\\unknown&\unknown\end{bmatrix}.\]
  \end{subactivity}
\end{activity}

\begin{observation}
  \(T^{-1}\circ T=T\circ T^{-1}\) is the identity map for any bijective
  linear transformation \(T\). Therefore
  \(A^{-1}A=AA^{-1}=I\) is the identity matrix for any invertible matrix
  \(A\).
\end{observation}


\end{applicationActivities}
