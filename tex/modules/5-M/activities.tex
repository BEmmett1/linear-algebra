\begin{activity}{5}
Let $T: \IR^3 \rightarrow \IR^2$ be given by the \(2\times 3\) standard matrix $B=\begin{bmatrix} 2 & 1 & -3 \\ 5 & -3 & 4 \end{bmatrix}$ and $S: \IR^2 \rightarrow \IR^4$ be given by the \(4\times 2\) standard matrix $A=\begin{bmatrix} 1 & 2 \\ 0 & 1 \\ 3 & 5 \\ -1 & -2 \end{bmatrix}$.

What is the domain of the composition map $S \circ T$?
\begin{enumerate}[(a)]
\item $\IR$
\item $\IR^2$
\item $\IR^3$
\item $\IR^4$
\end{enumerate}
\end{activity}

\begin{activity}{3}
Let $T: \IR^3 \rightarrow \IR^2$ be given by the \(2\times 3\) standard matrix $B=\begin{bmatrix} 2 & 1 & -3 \\ 5 & -3 & 4 \end{bmatrix}$ and $S: \IR^2 \rightarrow \IR^4$ be given by the \(4\times 2\) standard matrix $A=\begin{bmatrix} 1 & 2 \\ 0 & 1 \\ 3 & 5 \\ -1 & -2 \end{bmatrix}$.

What is the codomain of the composition map $S \circ T$?
\begin{enumerate}[(a)]
\item $\IR$
\item $\IR^2$
\item $\IR^3$
\item $\IR^4$
\end{enumerate}
\end{activity}

\begin{activity}{2}
Let $T: \IR^3 \rightarrow \IR^2$ be given by the \(2\times 3\) standard matrix $B=\begin{bmatrix} 2 & 1 & -3 \\ 5 & -3 & 4 \end{bmatrix}$ and $S: \IR^2 \rightarrow \IR^4$ be given by the \(4\times 2\) standard matrix $A=\begin{bmatrix} 1 & 2 \\ 0 & 1 \\ 3 & 5 \\ -1 & -2 \end{bmatrix}$.

What size will the standard matrix of $S \circ T:\IR^3\to\IR^4$ be?
(Rows \(\times\) Columns)
\begin{multicols}{3}
\begin{enumerate}[(a)]
\item $4 \times 3$
\item $4 \times 2$
\item $3 \times 4$
\item $3 \times 2$
\item $2 \times 4$
\item $2 \times 3$
\end{enumerate}
\end{multicols}
\end{activity}

\begin{activity}{15}
Let $T: \IR^3 \rightarrow \IR^2$ be given by the \(2\times 3\) standard matrix $B=\begin{bmatrix} 2 & 1 & -3 \\ 5 & -3 & 4 \end{bmatrix}$ and $S: \IR^2 \rightarrow \IR^4$ be given by the \(4\times 2\) standard matrix $A=\begin{bmatrix} 1 & 2 \\ 0 & 1 \\ 3 & 5 \\ -1 & -2 \end{bmatrix}$.

\begin{subactivity}
Compute
\[
  (S \circ T)(\vec{e}_1)
    =
  S(T(\vec{e}_1))
    =
  S\left(\begin{bmatrix} 2 \\ 5\end{bmatrix}\right)
        =
  \begin{bmatrix}\unknown\\\unknown\\\unknown\\\unknown\end{bmatrix}.
\]

\end{subactivity}
\begin{subactivity}
Compute
\(
  (S \circ T)(\vec{e}_2)
\).
\end{subactivity}
\begin{subactivity}
Compute
\(
  (S \circ T)(\vec{e}_3)
\).
\end{subactivity}
\begin{subactivity}
Write the \(4\times 3\) standard matrix of $S \circ T:\IR^3\to\IR^4$.
\end{subactivity}
\end{activity}

\begin{definition}
We define the \term{product} \(AB\) of a \(m \times n\) matrix \(A\) and a
\(n \times k\)
matrix \(B\) to be the \(m \times k\) standard matrix of the composition map of the
two corresponding linear functions.

\vspace{1em}

For the previous activity, \(S\) had a \(4\times \circledNumber{2}\) matrix and
\(T\) had a \(\circledNumber{2}\times 3\) matrix, so \(S\circ T\) had a
\(4\times 3\) standard matrix:
\[
  AB
    =
  \begin{bmatrix} 1 & 2 \\ 0 & 1 \\ 3 & 5 \\ -1 & -2 \end{bmatrix}
  \begin{bmatrix} 2 & 1 & -3 \\ 5 & -3 & 4 \end{bmatrix}
\]
\[
    =
  \left[
  (S \circ T)(\vec{e}_1) \hspace{1em}
  (S\circ T)(\vec{e}_2) \hspace{1em}
  (S \circ T)(\vec{e}_3)
  \right]
    =
  \begin{bmatrix}
    12 & -5 & 5 \\
    5 & -3 & 4 \\
    31 & -12 & 11 \\
    -12 & 5 & -5
  \end{bmatrix}
.\]
\end{definition}



\begin{activity}{15}
Let \(S: \IR^3 \rightarrow \IR^2\) be given by the matrix 
\(A=\begin{bmatrix} -4 & -2 & 3 \\ 0 & 1 & 1 \end{bmatrix}\)
and \(T: \IR^2 \rightarrow \IR^3\) be given by the matrix
\(B=\begin{bmatrix} 2 & 3 \\ 1 & -1 \\ 0 & -1 \end{bmatrix}\).

\begin{subactivity}
Write the dimensions (rows \(\times\) columns)
for \(A\), \(B\), \(AB\), and \(BA\).
\end{subactivity}
\begin{subactivity}
Find the standard matrix \(AB\) of \(S \circ T\).
\end{subactivity}
\begin{subactivity}
FInd the standard matrix \(BA\) of \(T \circ S\).
\end{subactivity}
\end{activity}


\begin{activity}{10}
Consider the following three matrices.

\[
  A = \begin{bmatrix}1&0&-3\\3&2&1\end{bmatrix}
    \hspace{2em}
  B = \begin{bmatrix}2&2&1&0&1\\1&1&1&-1&0\\0&0&3&2&1\\-1&5&7&2&1\end{bmatrix}
    \hspace{2em}
  C = \begin{bmatrix}2&2\\0&-1\\3&1\\4&0\end{bmatrix}
\]

\begin{subactivity}
Label each of these matrices
with its number of rows \(\times\) columns.
\end{subactivity}
\begin{subactivity} 
Only one of the matrix products
\(AB,AC,BA,BC,CA,CB\) can actually be computed.
Compute it.
\end{subactivity}
\end{activity}



\begin{remark}
Recall that the \term{product} \(AB\) of a 
\(m \times n\) matrix \(A\) and an \(n \times k\)
matrix \(B\) is the \(m \times k\) 
standard matrix of the composition map of the
two corresponding linear functions.

\vspace{1em}

For example, if \(S\) has a 
\(4\times \circledNumber{2}\) matrix \(A\) and
\(T\) has a \(\circledNumber{2}\times 3\) matrix 
\(B\), then \(S\circ T\) has a
\(4\times 3\) standard matrix:
\[
  AB
    =
  \begin{bmatrix} 1 & 2 \\ 0 & 1 \\ 3 & 5 \\ -1 & -2 \end{bmatrix}
  \begin{bmatrix} 2 & 1 & -3 \\ 5 & -3 & 4 \end{bmatrix}
\]
\[
    =
  \left[
  (S \circ T)(\vec{e}_1) \hspace{1em}
  (S\circ T)(\vec{e}_2) \hspace{1em}
  (S \circ T)(\vec{e}_3)
  \right]
    =
  \begin{bmatrix}
    12 & -5 & 5 \\
    5 & -3 & 4 \\
    31 & -12 & 11 \\
    -12 & 5 & -5
  \end{bmatrix}
.\]
\end{remark}






\begin{activity}{15}
Let \(B=\begin{bmatrix} 3 & -4 & 0 \\ 2 & 0 & -1 \\ 0 & -3 & 3 \end{bmatrix}\), 
and let \(A=\begin{bmatrix} 2 & 7 & -1 \\ 0 & 3 & 2 \\ 1 & 1 & -1 \end{bmatrix}\).  
\begin{subactivity}
  Compute the product \(BA\) by hand.
\end{subactivity}
\begin{subactivity}
  Check your work using technology. Using Octave:
  \begin{itemize}
    \item \texttt{ B = sym([3 -4 0 ; 2 0 -1 ; 0 -3 3]) }
    \item \texttt{ A = sym([2 7 -1 ; 0 3 2  ; 1 1 -1]) }
    \item \texttt{ B*A }
  \end{itemize}
\end{subactivity}
\end{activity}

\begin{activity}{5}
Let \(A=\begin{bmatrix} 2 & 7 & -1 \\ 0 & 3 & 2 \\ 1 & 1 & -1 \end{bmatrix}\).  
Find a \(3 \times 3\) matrix \(B\) such that \(BA=A\), that is,
\[
  \begin{bmatrix} \unknown & \unknown & \unknown \\ 
  \unknown & \unknown & \unknown 
  \\ \unknown & \unknown & \unknown \end{bmatrix}
  \begin{bmatrix} 2 & 7 & -1 \\ 0 & 3 & 2 \\ 1 & 1 & -1 \end{bmatrix}
=
  \begin{bmatrix} 2 & 7 & -1 \\ 0 & 3 & 2 \\ 1 & 1 & -1 \end{bmatrix}
\]
Check your guess using technology.
\end{activity}

\begin{definition}
The identity matrix $I_n$ (or just $I$ when $n$ is obvious from context) is  the $n \times n$ matrix $$I_n = \begin{bmatrix} 1 & 0  & \hdots & 0 \\ 0 & 1 & \ddots & \vdots  \\ \vdots & \ddots & \ddots & 0 \\ 0 & \hdots & 0 & 1 \end{bmatrix}.$$
It has a $1$ on each diagonal element and a $0$ in every other position.
\end{definition}

\begin{fact}
  For any square matrix \(A\), \(IA=AI=A\):

  \[
    \begin{bmatrix} 1 & 0 & 0 \\ 0 & 1 & 0 \\ 0 & 0 & 1 \end{bmatrix}
    \begin{bmatrix} 2 & 7 & -1 \\ 0 & 3 & 2 \\ 1 & 1 & -1 \end{bmatrix}
  =
    \begin{bmatrix} 2 & 7 & -1 \\ 0 & 3 & 2 \\ 1 & 1 & -1 \end{bmatrix}
      \begin{bmatrix} 1 & 0 & 0 \\ 0 & 1 & 0 \\ 0 & 0 & 1 \end{bmatrix}
  =
    \begin{bmatrix} 2 & 7 & -1 \\ 0 & 3 & 2 \\ 1 & 1 & -1 \end{bmatrix}
  \]
\end{fact}

\begin{activity}{20}
Tweaking the identity matrix slightly allows us to write row operations
in terms of matrix multiplication.
\begin{subactivity}
Create a matrix that doubles the third row of \(A\):
\[
 \begin{bmatrix} \unknown & \unknown & \unknown \\ \unknown & \unknown & \unknown \\ \unknown & \unknown & \unknown \end{bmatrix}
 \begin{bmatrix} 2 & 7 & -1 \\ 0 & 3 & 2 \\ 1 & 1 & -1 \end{bmatrix}
=
 \begin{bmatrix} 2 & 7 & -1 \\ 0 & 3 & 2 \\ 2 & 2 & -2 \end{bmatrix}
\]
\end{subactivity}
\begin{subactivity}
  Create a matrix that swaps the second and third rows of \(A\):
  \[
   \begin{bmatrix} \unknown & \unknown & \unknown \\ \unknown & \unknown & \unknown \\ \unknown & \unknown & \unknown \end{bmatrix}
   \begin{bmatrix} 2 & 7 & -1 \\ 0 & 3 & 2 \\ 1 & 1 & -1 \end{bmatrix}
  =
  \begin{bmatrix} 2 & 7 & -1 \\ 1 & 1 & -1 \\ 0 & 3 & 2 \end{bmatrix}
  \]
\end{subactivity}
\begin{subactivity}
Create a matrix that adds \(5\) times the third row of \(A\) to the first row:
\[
 \begin{bmatrix} \unknown & \unknown & \unknown \\ \unknown & \unknown & \unknown \\ \unknown & \unknown & \unknown \end{bmatrix}
 \begin{bmatrix} 2 & 7 & -1 \\ 0 & 3 & 2 \\ 1 & 1 & -1 \end{bmatrix}
=
 \begin{bmatrix} 2+5(1) & 7+5(1) & -1+5(-1) \\ 0 & 3 & 2 \\ 1 & 1 & -1 \end{bmatrix}
\]
\end{subactivity}
\end{activity}

\begin{fact}
If \(R\) is the result of applying a row operation to \(I\), then
\(RA\) is the result of applying the same row operation to \(A\).
\begin{itemize}
\item Scaling a row: \(R=
  \begin{bmatrix}
  c & 0 & 0 \\
  0 & 1 & 0 \\
  0 & 0 & 1
  \end{bmatrix}
\)
\item Swapping rows: \(R=
  \begin{bmatrix}
  0 & 1 & 0 \\
  1 & 0 & 0 \\
  0 & 0 & 1
  \end{bmatrix}
\)
\item Adding a row multiple to another row: \(R=
  \begin{bmatrix}
  1 & 0 & c \\
  0 & 1 & 0 \\
  0 & 0 & 1
  \end{bmatrix}
\)
\end{itemize}

Such matrices can be chained together to emulate multiple row operations.
In particular,
\[\RREF(A)=R_k\dots R_2R_1A\]
for some sequence of matrices \(R_1,R_2,\dots,R_k\).
\end{fact}

\begin{activity}{10}
Consider the two row operations 
\(R_2\leftrightarrow R_3\) and \(R_1+R_2\to R_1\)
applied as follows to show \(A\sim B\):
\begin{align*}
A
  =
\begin{bmatrix}
-1&4&5\\
0&3&-1\\
1&2&3\\
\end{bmatrix}
  &\sim
\begin{bmatrix}
-1&4&5\\
1&2&3\\
0&3&-1\\
\end{bmatrix}
  \\&\sim
\begin{bmatrix}
-1+1&4+2&5+3\\
1&2&3\\
0&3&-1\\
\end{bmatrix}
  =
\begin{bmatrix}
0&6&8\\
1&2&3\\
0&3&-1\\
\end{bmatrix}
  = 
B
\end{align*}
Express these row operations as matrix multiplication
by expressing \(B\) as the product of two matrices and \(A\):
\[
B =
\begin{bmatrix}
\unknown&\unknown&\unknown\\
\unknown&\unknown&\unknown\\
\unknown&\unknown&\unknown
\end{bmatrix}
\begin{bmatrix}
\unknown&\unknown&\unknown\\
\unknown&\unknown&\unknown\\
\unknown&\unknown&\unknown
\end{bmatrix}
A
\]
Check your work using technology.
\end{activity}




\begin{activity}{15}
Let $T: \IR^n \rightarrow \IR^m$ be a linear map with standard matrix $A$.
Sort the following items into three groups of statements: a group that means
\(T\) is \textbf{injective}, a group that means \(T\) is \textbf{surjective},
and a group that means \(T\) is \textbf{bijective}.
\begin{multicols}{2}
\begin{enumerate}[(a)]
\item \(A\vec x=\vec b\) has a solution for all \(\vec b\in\IR^m\)
\item \(A\vec x=\vec b\) has a unique solution for all \(\vec b\in\IR^m\)
\item \(A\vec x=\vec 0\) has a unique solution.
\item The columns of $A$ span $\IR^m$
\item The columns of $A$ are linearly independent
\item The columns of $A$ are a basis of $\IR^m$
\item Every column of $\RREF(A)$ has a pivot
\item Every row of $\RREF(A)$ has a pivot
\item $m=n$ and $\RREF(A)=I$
\end{enumerate}
\end{multicols}
\end{activity}

\begin{definition}
  Let $T: \IR^n \rightarrow \IR^n$ be a linear map with standard matrix $A$.
\begin{itemize}
\item  If $T$ is a bijection and \(\vec b\) is any $\IR^n$ vector, 
  then $T(\vec x)=A\vec x=\vec b$ has a unique solution.
\item So we may define an \term{inverse map} $T^{-1} : \IR^n \rightarrow \IR^n$ 
  by setting $T^{-1}(\vec b)$ to be this unique solution.
\item Let $A^{-1}$ be the standard matrix for $T^{-1}$. We call $A^{-1}$ the
\term{inverse matrix} of $A$, so we also say that $A$ is \term{invertible}.
\end{itemize}
\end{definition}


\begin{activity}{20}
  Let \(T: \IR^3 \rightarrow \IR^3\) be the linear transformation given by the standard matrix
  \(A=\begin{bmatrix} 2 & -1 & -6 \\ 2 & 1 & 3 \\ 1 & 1 & 4 \end{bmatrix}\).
  \begin{subactivity}
  Write an augmented matrix representing the system of equations given by \(T(\vec x)=\vec{e}_1\),
  that is, \(A\vec x=\begin{bmatrix}1 \\ 0 \\ 0 \end{bmatrix}\).
  Then solve \(T(\vec x)=\vec{e}_1\) to find \(T^{-1}(\vec{e}_1)\).
  \end{subactivity}
  \begin{subactivity}
   Solve \(T(\vec x)=\vec{e}_2\) to find \(T^{-1}(\vec{e}_2)\).
  \end{subactivity}
  \begin{subactivity}
   Solve \(T(\vec x)=\vec{e}_3\) to find \(T^{-1}(\vec{e}_3)\).
  \end{subactivity}
  \begin{subactivity}
   Write \(A^{-1}\), the standard matrix for \(T^{-1}\).
  \end{subactivity}
\end{activity}

\begin{observation}
We could have solved these three systems simultaneously
by row reducing the matrix \([A\,|\,I]\) at once.
\[
  \begin{bmatrix}[ccc|ccc]
    2 & -1 & -6 & 1 & 0 & 0 \\
    2 & 1 & 3 & 0 & 1 & 0 \\
    1 & 1 & 4 & 0 & 0 & 1
  \end{bmatrix} \sim
  \begin{bmatrix}[ccc|ccc]
    1 & 0 & 0 & 1 & -2 & 3 \\
    0 & 1 & 0 & -5 & 14 & -18 \\
    0 & 0 & 1 & 1 & -3 & 4
  \end{bmatrix}
\]
\end{observation}


\begin{activity}{5}
  Find the inverse \(A^{-1}\) of the matrix
  \(A=\begin{bmatrix} 1 & 3 \\ 0 & -2 \end{bmatrix}\)
  by row-reducing \([A\,|\,I]\).
\end{activity}

\begin{activity}{5}
Is the matrix $\begin{bmatrix} 2 & 3 & 1 \\ -1 & -4 & 2 \\ 0 & -5 & 5 \end{bmatrix}$ invertible? Give a reason for your answer.
\end{activity}

\begin{observation}
    An \(n\times n\) matrix $A$ is invertible if and only if $\RREF(A) = I_n$.
\end{observation}

\begin{activity}{10}
  Let \(T:\IR^2\to\IR^2\) be the bijective linear map defined by
  \(T\left(\begin{bmatrix}x\\y\end{bmatrix}\right)=\begin{bmatrix} 2x -3y \\ -3x + 5y\end{bmatrix}\),
  with the inverse map
  \(T^{-1}\left(\begin{bmatrix}x\\y\end{bmatrix}\right)=\begin{bmatrix} 5x+ 3y \\ 3x + 2y\end{bmatrix}\).
  \begin{subactivity}
    Compute \((T^{-1}\circ T)\left(\begin{bmatrix}-2\\1\end{bmatrix}\right)\).
  \end{subactivity}
  \begin{subactivity}
    If \(A\) is the standard matrix for \(T\) and \(A^{-1}\) is the
    standard matrix for \(T^{-1}\), find the \(2\times 2\) matrix 
    \[A^{-1}A=\begin{bmatrix}\unknown&\unknown\\\unknown&\unknown\end{bmatrix}.\]
  \end{subactivity}
\end{activity}

\begin{observation}
  \(T^{-1}\circ T=T\circ T^{-1}\) is the identity map for any bijective
  linear transformation \(T\). Therefore
  \(A^{-1}A=AA^{-1}=I\) is the identity matrix for any invertible matrix
  \(A\).
\end{observation}


