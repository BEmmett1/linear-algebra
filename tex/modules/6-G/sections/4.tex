


\begin{applicationActivities}{4}{28}


\begin{observation}
Recall from last class:
\begin{itemize}
\item To find the eigenvalues of a matrix $A$, we need to find values of $\lambda$ such that $A-\lambda I$ has a nontrivial kernel. Equivalently,
we want values where $A-\lambda I$ is not invertible, so we want to know
the values of \(\lambda\) where $\det(A-\lambda I)=0$.
\item $\det(A-\lambda I)$ is a polynomial with variable \(\lambda\),
called the \term{characteristic polynomial} of $A$. Thus the roots of
the characteristic polynomial of \(A\) are exactly the eigenvalues of \(A\).
\item Once an eigenvalue \(\lambda\) is found, the \term{eigenspace}
containing all \term{eigenvectors} \(\vec x\) satisfying
\(A\vec x=\lambda\vec x\) is given by $\ker(A-\lambda I)$.
\end{itemize}
\end{observation}

\begin{activity}{5}
Let  $A=\begin{bmatrix}0 & -1 \\ 1 & 0 \end{bmatrix}$.
  \begin{subactivity}
    Compute the eigenvalues of $A$.
  \end{subactivity}
  \begin{subactivity}
     Sketch a picture of the transformation of the unit square. 
     What about this picture reveals that \(A\) has no real eigenvectors?
   \end{subactivity}
\end{activity}

\begin{activity}{5}
  If $A$ is a $4 \times 4$ matrix, what is the largest number of eigenvalues $A$ can have?
  \begin{enumerate}[(a)]
  \item $3$
  \item $4$
  \item $5$
  \item $6$
  \item It can have infinitely many
  \end{enumerate}
\end{activity}

\begin{observation}
  An \(n\times n\) matrix may have  between \(0\) and \(n\)
  real-valued eigenvalues. But the Fundamental Theorem of Algebra implies that if complex eigenvalues are included,
  then every \(n\times n\) matrix has exactly \(n\) eigenvalues (counting
  algebraic multiplicites).



\end{observation}

\begin{activity}{5}
  The matrix
  $A=\begin{bmatrix} 1 & -2 & 1 \\ -1 & 0 & 1 \\ -1 & -2 & 3\end{bmatrix}$
  has characteristic polynomial \(-\lambda(\lambda-2)^2\).

  Find the dimension of the eigenspace of $A$ associated to the
  eigenvalue $2$ (the dimension of the kernel of \(A-2I\)).
  % \[
  %   A-2I
  %     =
  %   \begin{bmatrix}
  %     1-2 & -2 & 1 \\
  %     -1 & 0-2 & 1 \\
  %     -1 & -2 & 3-2
  %   \end{bmatrix}
  %     =
  %   \begin{bmatrix}
  %     -1 & -2 & 1 \\
  %     -1 & -2 & 1 \\
  %     -1 & -2 & 1
  %   \end{bmatrix}
  % \]
\end{activity}

\begin{activity}{5}
  The matrix
  $B=\begin{bmatrix} -3 & -9 & 5 \\ -2 & -2 & 2 \\ -7 & -13 & 9 \end{bmatrix}$
  has characteristic polynomial \(-\lambda(\lambda-2)^2\).

  Find the dimension of the eigenspace of $B$ associated to the
  eigenvalue $2$ (the dimension of the kernel of \(B-2I\)).
\end{activity}

\begin{observation}
In the first example, the (2 dimensional) plane spanned by $\begin{bmatrix} 1 \\ 0 \\ 1\end{bmatrix}$ and $\begin{bmatrix} 4 \\ -2 \\ 0 \end{bmatrix}$ was preserved.  In the second example, only the (one dimensional) line spanned by $\begin{bmatrix} 1 \\ 0 \\ 1\end{bmatrix}$ is preserved. 
\end{observation}

\begin{definition}
While the \term{algebraic multiplicity} of an eigenvalue is its multiplicity as
a root of the characteristic polynomial,
the \term{geometric multiplicity} of an eigenvalue is the dimension of its
eigenspace.
\end{definition}

\begin{fact}
  As we've seen, the geometric multiplicity may be different than
  its algebraic multiplicity, but it cannot exceed it.  
  \ \\
  This fact is explored deeper and explained in Math 316, Linear Algebra II  
  \end{fact}

% \begin{activity}{5} How are the algebraic and geometric multiplicities related?
% \begin{enumerate}[(a)]
% \item The algebraic multiplicity is always at least as big as than the geometric multiplicity.
% \item The geometric multiplicity is always at least as big as the algebraic multiplicity.
% \item Sometimes the algebraic multiplicity is larger and sometimes the geometric multiplicity is larger.
% \end{enumerate}
% \end{activity}

\begin{activity}{20}
  Consider the \(4\times 4\) matrix
  \[
    \begin{bmatrix}
      -3 & 1 & 2 & 1 \\
      -9 & 5 & -2 & -1 \\
      31 & -17 & 6 & 3 \\
      -69 & 39 & -18 & -9
    \end{bmatrix}
  \]
  \begin{subactivity}
    Use technology (e.g. Wolfram Alpha) to find its characteristic polynomial.
  \end{subactivity}
  \begin{subactivity}
    Find the algebraic and geometric multiplicities for both eigenvalues.
  \end{subactivity}
\end{activity}



\end{applicationActivities}
