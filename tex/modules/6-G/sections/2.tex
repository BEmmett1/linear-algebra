\begin{applicationActivities}

\begin{remark}
  Recall that the column versions of the three row-reducing operations
  a matrix may be used to simplify a determinant:
  \begin{enumerate}[(a)]
  \item Multiplying columns by scalars:
        \[c\det([\cdots\hspace{0.5em}\vec{v}\hspace{0.5em} \cdots])=
        \det([\cdots\hspace{0.5em}c\vec{v}\hspace{0.5em} \cdots])\]
  \item Swapping two columns:
        \[\det([\cdots\hspace{0.5em}\vec{v}\hspace{0.5em}
        \cdots\hspace{1em}\vec{w}\hspace{0.5em} \cdots])=
        -\det([\cdots\hspace{0.5em}\vec{w}\hspace{0.5em}
        \cdots\hspace{1em}\vec{v}\hspace{0.5em} \cdots])\]
  \item Adding a multiple of a column to another column:
        \[\det([\cdots\hspace{0.5em}\vec{v}\hspace{0.5em}
        \cdots\hspace{1em}\vec{w}\hspace{0.5em} \cdots])=
        \det([\cdots\hspace{0.5em}\vec{v}+c\vec{w}\hspace{0.5em}
        \cdots\hspace{1em}\vec{w}\hspace{0.5em} \cdots])\]
  \end{enumerate}
\end{remark}
\begin{remark}
The determinants of row operation matrices may be computed
by manipulating columns to reduce each matrix to the identity:
\begin{itemize}
\item Scaling a row: \(\det
  \begin{bmatrix}
  c & 0 & 0 \\
  0 & 1 & 0 \\
  0 & 0 & 1
  \end{bmatrix}
    =
  c\det
  \begin{bmatrix}
  1 & 0 & 0 \\
  0 & 1 & 0 \\
  0 & 0 & 1
  \end{bmatrix}
    =
  c
\)
\item Swapping rows: \(\det
  \begin{bmatrix}
  0 & 1 & 0 \\
  1 & 0 & 0 \\
  0 & 0 & 1
  \end{bmatrix}
    =
  -1\det
  \begin{bmatrix}
  1 & 0 & 0 \\
  0 & 1 & 0 \\
  0 & 0 & 1
  \end{bmatrix}
    =
  -1
\)
\item Adding a row multiple to another row: \(\det
  \begin{bmatrix}
  1 & 0 & c \\
  0 & 1 & 0 \\
  0 & 0 & 1
  \end{bmatrix}
    =
  \det
  \begin{bmatrix}
  1 & 0 & c-1c \\
  0 & 1 & 0-0c \\
  0 & 0 & 1-0c
  \end{bmatrix}
    =
  \det(I)=1
\)
\end{itemize}
\end{remark}
\begin{fact}
Thus we can also use row operations to simplify determinants:
\begin{enumerate}
\item Multiplying rows by scalars:
  \(\det\begin{bmatrix}\vdots\\cR\\\vdots\end{bmatrix}=
  c\det\begin{bmatrix}\vdots\\R\\\vdots\end{bmatrix}\)
\item Swapping two rows:
  \(\det\begin{bmatrix}\vdots\\R\\\vdots\\S\\\vdots\end{bmatrix}=
  -\det\begin{bmatrix}\vdots\\S\\\vdots\\R\\\vdots\end{bmatrix}\)
\item Adding multiples of rows to other rows:
  \(\det\begin{bmatrix}\vdots\\R\\\vdots\\S\\\vdots\end{bmatrix}=
  \det\begin{bmatrix}\vdots\\R+cS\\\vdots\\S\\\vdots\end{bmatrix}\)
\end{enumerate}
\end{fact}

%\begin{definition}
%  The \term{transpose} of a matrix is given by rewriting its columns as
%  rows and vice versa:
%  \[
%    \begin{bmatrix}
%      1 & 2 & 3 \\
%      4 & 5 & 6
%    \end{bmatrix}^T
%  =
%    \begin{bmatrix}
%      1 & 4  \\
%      2 & 5  \\
%      3 & 6
%    \end{bmatrix}
%  \]
%\end{definition}
%
%\begin{fact}
%Since row and column operations both affect determinants the same way,
%the determinant of a matrix and its transpose are the same.
%For example, let \(A=\begin{bmatrix}3 & 3 \\ 0 & 2\end{bmatrix}\),
%so \(A^T=\begin{bmatrix}3 & 0 \\ 3 & 2\end{bmatrix}\); both matrices
%scale the unit square by \(6\), even though the parallelograms are not
%congruent.
%
%\begin{center}
%  \begin{tikzpicture}[scale=0.5]
%  \draw[thin,gray,<->] (-1,0)-- (4,0);
%  \draw[thin,gray,<->] (0,-1)-- (0,4);
%  \draw[blue,dashed] (0,0) -- (3,0) -- (6,2) -- (3,2) -- (0,0);
%  \draw[thick,blue,->] (0,0) -- node[below] {$A \vec{e}_1= \begin{bmatrix}3 \\ 0 \end{bmatrix}$}++ (3,0);
%  \draw[thick,blue,->] (0,0) -- ++(3,2) node[above] {$A \vec{e}_2 = \begin{bmatrix} 3 \\ 2 \end{bmatrix}$};
%  \end{tikzpicture}
%  \begin{tikzpicture}[scale=0.5]
%  \draw[thin,gray,<->] (-1,0)-- (4,0);
%  \draw[thin,gray,<->] (0,-1)-- (0,4);
%  \draw[blue,dashed] (0,0) -- (3,3) -- (3,5) -- (0,2) -- (0,0);
%  \draw[thick,blue,->] (0,0) -- node[right] {$A^T \vec{e}_1= \begin{bmatrix}3 \\ 3 \end{bmatrix}$}++ (3,3);
%  \draw[thick,blue,->] (0,0) -- ++(0,2) node[left] {$A^T \vec{e}_2 = \begin{bmatrix} 0 \\ 2 \end{bmatrix}$};
%  \end{tikzpicture}
%\end{center}
%\end{fact}

\begin{activity}{10}
  Compute the determinant of \(\begin{bmatrix} 2 & 4 \\ 2 & 3 \end{bmatrix}\) 
  by manipulating its rows and columns to simplify the matrix to \(I\):

  \begin{align*}
    \det\begin{bmatrix} 2 & 4 \\ 2 & 3 \end{bmatrix}
      &=
    \unknown \det \begin{bmatrix} 1 & 2 \\ 2 & 3 \end{bmatrix}\\
      &\vdots\\
      &=
    \unknown \det \begin{bmatrix} 1 & 0 \\ 0 & 1 \end{bmatrix}\\
      &=
    \unknown
  \end{align*}
\end{activity}

\begin{fact}
  Applying that same process to a generalized \(2\times 2\)
  matrix gives us the formula memorized in many math classes:
    \begin{align*}
    \det\begin{bmatrix} a & b \\ c & d \end{bmatrix}
  &=
    a
    \det\begin{bmatrix} 1 & b/a \\ c & d \end{bmatrix}
  \\ &=
    a
    \det\begin{bmatrix} 1 & b/a \\ 0 & d-bc/a \end{bmatrix}
  \\ &=
    a(d-bc/a)
    \det\begin{bmatrix} 1 & b/a \\ 0 & 1 \end{bmatrix}
  \\ &=
    (ad-bc)
    \det\begin{bmatrix} 1 & b/a \\ 0 & 1 \end{bmatrix}
  \\ &=
    (ad-bc)
    \det\begin{bmatrix} 1 & 0 \\ 0 & 1 \end{bmatrix}
  \\ 
    \det\begin{bmatrix} a & b \\ c & d \end{bmatrix}
  &=
    ad-bc
    \end{align*}
\end{fact}

\begin{remark}
But memorizing formulas quickly becomes unmanageable.
For example, part of the formula for \(4\times4\) matrices looks like this:
\[
  \det\begin{bmatrix}
    a_{11} & a_{12} & a_{13} & a_{14} \\
    a_{21} & a_{22} & a_{23} & a_{24} \\
    a_{31} & a_{32} & a_{33} & a_{34} \\
    a_{41} & a_{42} & a_{43} & a_{44}
  \end{bmatrix}
    =
  a_{11}(a_{22}(a_{33}a_{44}-a_{43}a_{34})-a_{23}(a_{32}a_{44}-a_{42}a_{34})+\dots)+\dots
\]
So instead, we will learn how to simplify determinants into
smaller determinants.
\end{remark}


\begin{activity}{5}
  The following image illustrates the transformation of the unit cube
  by the matrix
  $\begin{bmatrix} 3 & 1 & 0 \\  1 & 1 & 1 \\  0 & 0 & 1\end{bmatrix}$.

  \begin{center}
  \begin{tikzpicture}
  \fill[purple!50!white] (0,0,0) -- (1,0,1) -- (4,0,2) -- (3,0,1) -- (0,0,0);
  \draw[thin,gray,->] (0,0,0) -- (3,0,0);
  \draw[thin,gray,->] (0,0,0) -- (0,2,0);
  \draw[thin,gray,->] (0,0,0) -- (0,0,2);
  %(y,z,x)
  \draw[blue] (1,0,1) -- (4,0,2) -- (3,0,1);
  \draw[blue] (1,1,0) -- (2,1,1) -- (5,1,2) -- (4,1,1) -- (1,1,0);
  \draw[blue] (1,0,1) -- +(1,1,0);
  \draw[blue] (4,0,2) -- +(1,1,0);
  \draw[blue] (3,0,1) -- +(1,1,0);

  \draw[purple,thick,->] (0,0,0) -- (1,1,0)
    node[above left]{\tiny$\begin{bmatrix} 0 \\ 1 \\ 1\end{bmatrix}$};
  \draw[purple,thick,->] (0,0,0) -- (1,0,1)
    node[below]{\tiny$\begin{bmatrix} 1 \\ 1 \\ 0\end{bmatrix}$};
  \draw[purple,thick,->] (0,0,0) -- (3,0,1)
    node[above right]{\tiny$\begin{bmatrix} 3 \\ 1 \\ 0\end{bmatrix}$};
  \draw[purple,dashed,very thick] (0,0,0) -- node[left] {\tiny\(h=1\)} (0,1,0);
  \end{tikzpicture}
  \end{center}
  Recall that \(V=Bh\).
This volume is equal to which of the following areas?
\begin{multicols}{4}
\begin{enumerate}[(a)]
\item $\det \begin{bmatrix} 3 & 1 \\ 1 & 1 \end{bmatrix}$
\item $\det \begin{bmatrix} 3 & 1 \\ 1 & 0 \end{bmatrix}$
\item $\det \begin{bmatrix} 3 & 1 \\ 0 & 1 \end{bmatrix}$
\item $\det \begin{bmatrix} 1 & 1 \\ 0 & 1 \end{bmatrix}$
\end{enumerate}
\end{multicols}
\end{activity}

\begin{fact}
If row \(i\) contains all zeros except for a \(1\) on the diagonal, 
then both column and row \(i\)
may be removed without changing the value of the determinant.
\[
  \det \begin{bmatrix}
    3 & \textcolor{red}{2} & -1 & 3 \\
    \textcolor{red}{0} & \textcolor{red}{1} 
      & \textcolor{red}{0} & \textcolor{red}{0} \\
    -1 & \textcolor{red}{4} & 1 & 0 \\
    5 & \textcolor{red}{0} & 11 & 1
  \end{bmatrix} =
  \det \begin{bmatrix}
    3 & -1 & 3 \\
    -1 & 1 & 0 \\
    5 & 11 & 1
  \end{bmatrix}
\]
Since row and column operations affect the determinants in the same
way, the same technique works for a column of all zeros except for
a \(1\) on the diagonal.
\[
  \det \begin{bmatrix}
    3 & \textcolor{red}{0} & -1 & 5 \\
    \textcolor{red}{2} & \textcolor{red}{1} & \textcolor{red}{4} & 
       \textcolor{red}{0} \\
    -1 & \textcolor{red}{0} & 1 & 11 \\
    3 & \textcolor{red}{0} & 0 & 1
  \end{bmatrix} =
  \det \begin{bmatrix}
    3 & -1 & 5 \\
    -1 & 1 & 11 \\
    3 & 0 & 1
  \end{bmatrix}
\] 
\end{fact}

% \begin{activity}{5}
%   The following image illustrates the transformation of the unit cube
%   by the matrix
%   $\begin{bmatrix} 3 & 1 & 0 \\  1 & 1 & 1 \\  0 & 0 & 1\end{bmatrix}$.
%
%
%   \begin{center}
%   \begin{tikzpicture}
%   \draw[thin,gray,->] (0,0,0) -- (4,0,0);
%   \draw[thin,gray,->] (0,0,0) -- (0,2,0);
%   \draw[thin,gray,->] (0,0,0) -- (0,0,2);
%   %(y,z,x)
%
%   \draw[blue] (0,0,0) -- (1,1,0);
%   \draw[blue] (0,0,0) -- (1,0,1);
%   \draw[blue] (0,0,0) -- (3,0,1);
%   \draw[blue] (1,1,0) -- (2,1,1);
%   \draw[blue] (1,1,0) -- (4,1,1);
%   \draw[blue] (3,0,1) -- (4,0,2);
%   \draw[blue] (3,0,1) -- (4,1,1);
%   \draw[blue] (1,0,1) -- (4,0,2);
%   \draw[blue] (1,0,1) -- (2,1,1);
%   \draw[blue] (2,1,1) -- (5,1,2);
%   \draw[blue] (4,0,2) -- (5,1,2);
%   \draw[blue] (4,1,1) -- (5,1,2);
%   \end{tikzpicture}
%   \end{center}
%
%   This volume is equal to which of the following?
%
% \begin{multicols}{2}
% \begin{enumerate}[(a)]
% \item $\det \begin{bmatrix} 3 & 1 & 0 \\  1 & 1 & 0 \\  0 & 0 & 1\end{bmatrix}
%       = \det \begin{bmatrix} 3 & 1 \\ 1 & 1 \end{bmatrix}$
% \item $\det \begin{bmatrix} 3 & 1 & 0 \\  1 & 0 & 0 \\  0 & 0 & 1\end{bmatrix}
%       = \det \begin{bmatrix} 3 & 1 \\ 1 & 0 \end{bmatrix}$
% \item $\det \begin{bmatrix} 3 & 1 & 0 \\  0 & 1 & 0 \\  0 & 0 & 1\end{bmatrix}
%       = \det \begin{bmatrix} 3 & 1 \\ 0 & 1 \end{bmatrix}$
% \item $\det \begin{bmatrix} 1 & 1 & 0 \\  0 & 1 & 0 \\  0 & 0 & 1\end{bmatrix}
%       = \det \begin{bmatrix} 1 & 1 \\ 0 & 1 \end{bmatrix}$
% \end{enumerate}
%
% \end{multicols}
% \end{activity}
%
% \begin{activity}{5}
%   Which of the following is the same as
%   $\det \begin{bmatrix}
%     2 & 0 & 1 \\
%     1 & 3 & 1 \\
%     0 & 0 & 2
%   \end{bmatrix}$?
%
% \begin{multicols}{2}
% \begin{enumerate}[(a)]
% \item $3\begin{bmatrix} 2 & 1 \\ 0 & 2 \end{bmatrix}$
% \item $3\begin{bmatrix} 2 & 0 \\ 1 & 3 \end{bmatrix}$
% \item $3\begin{bmatrix} 2 & 1 \\ 1 & 1 \end{bmatrix}$
% \item $3\begin{bmatrix} 3 & 1 \\ 0 & 2 \end{bmatrix}$
% \end{enumerate}
%
%
% \begin{center}
% \begin{tikzpicture}
% \draw[thin,gray,->] (0,0,0) -- (3,0,0);
% \draw[thin,gray,->] (0,0,0) -- (0,3,0);
% \draw[thin,gray,->] (0,0,0) -- (0,0,2);
% %(y,z,x)
%
% \draw[blue] (0,0,0) -- (1,0,2);
% \draw[blue] (0,0,0) -- (1,2,1);
% \draw[blue] (0,0,0) -- (3,0,0);
% \draw[blue] (1,0,2) -- (2,2,3);
% \draw[blue] (1,0,2) -- (4,0,2);
% \draw[blue] (3,0,0) -- (4,0,2);
% \draw[blue] (3,0,0) -- (4,2,1);
% \draw[blue] (1,2,1) -- (2,2,3);
% \draw[blue] (1,2,1) -- (4,2,1);
% \draw[blue] (2,2,3) -- (5,2,3);
% \draw[blue] (4,0,2) -- (5,2,3);
% \draw[blue] (4,2,1) -- (5,2,3);
% \end{tikzpicture}
% \end{center}
% \end{multicols}
%
% \end{activity}


% \begin{activity}{5}
%   Compute $\det \begin{bmatrix} 0 & 3 & -2 \\ 1 & 5 & 12 \\ 0 & 2 & -1 \end{bmatrix}$.
%
%   {\em Hint: Swap rows or columns to reduce to an easier problem}.
% \end{activity}
%
% \begin{activity}{10}
%    Using the fact that $\begin{bmatrix} 2 \\ 1 \\ 0 \end{bmatrix} = \begin{bmatrix} 2 \\ 0 \\ 0 \end{bmatrix} + \begin{bmatrix} 0 \\ 1 \\ 0 \end{bmatrix}$, compute $\det \begin{bmatrix} 2 & 2 & 3 \\ 1 & -2 & -5 \\ 0 & 3 & 3 \end{bmatrix}$.
% \end{activity}
%
% \begin{activity}{10}
%    Compute $\det \begin{bmatrix} 2 & 3 & 5  \\ 1 & 1 & 0  \\ -1 & 2 & -1 \end{bmatrix}$.
% \end{activity}
%
% \begin{activity}{10}
%    Compute $\det \begin{bmatrix} 2 & 3 & 5 & 0 \\ 0 & 1 & -1 & 0 \\ 1 & 2 & 0 & 3 \\ -1 & -1 & 2 & 2 \end{bmatrix}$.
% \end{activity}


\begin{activity}{5}
  Compute  $\det \begin{bmatrix} 1 & 0 & 0 \\ 1 & 5 & 12 \\ 3 & 2 & -1 \end{bmatrix}$
  by doing the following:
  \begin{itemize}
  \item Reduce its dimension by eliminating a row and column.
  \item Evaluate the resulting \(2\times 2\) determinant.
  \end{itemize}
\end{activity}

\begin{activity}{5}
  Compute \(\det \begin{bmatrix} 0 & 3 & -2 \\ 2 & 5 & 12 \\ 0 & 2 & -1 \end{bmatrix}\)
  by first doing the following:
  \begin{itemize}
    \item Factor out a \(2\) from a column.
    \item Swap rows to put a \(1\) on the diagonal.
  \end{itemize}
\end{activity}

\begin{activity}{5}
  Compute \(\det \begin{bmatrix} 4 & -2 & 2 \\ 3 & 1 & 4 \\ 1 & -1 & 3\end{bmatrix}\)
  by first doing the following:
  \begin{itemize}
    \item Add the middle row to the bottom row.
    \item Add a multiple of the middle row to the top row.
  \end{itemize}
\end{activity}

\begin{activity}{10}
  Compute \(\det \begin{bmatrix} 1 & 2 & 3 \\ 1 & -2 & -5 \\ 0 & 3 & 3 \end{bmatrix}\)
  by first doing the following:
  \begin{itemize}
    \item Expand the first column into the sum of two determinants:
    \(
      \det \begin{bmatrix} 1 & 2 & 3 \\ 0 & -2 & -5 \\ 0 & 3 & 3 \end{bmatrix}   
      +\det \begin{bmatrix} 0 & 2 & 3 \\ 1 & -2 & -5 \\ 0 & 3 & 3 \end{bmatrix}   
    \) 
    \item Reduce each of these to a \(2\times 2\) determinant.
  \end{itemize}
\end{activity}

\begin{observation}
Row/column operations can be used to add zeroes and reduce dimension
in large matrices.

\begin{align*}
    \det\begin{bmatrix} 
      2 & 3 & 5 & 0 \\ 
      0 & 3 & 2 & 0 \\ 
      1 & 2 & 0 & 3 \\ 
      -1 & -1 & 2 & 2 
    \end{bmatrix}
  &=
    -\det\begin{bmatrix}
      1 & 2 & 0 & 3 \\ 
      0 & 3 & 2 & 0 \\ 
      2 & 3 & 5 & 0 \\ 
      -1 & -1 & 2 & 2 
    \end{bmatrix}
  \\&=
    -\det\begin{bmatrix}
      \textcolor{red}{1} & \textcolor{red}{2} & 
         \textcolor{red}{0} & \textcolor{red}{3} \\ 
      \textcolor{red}{0} & 3 & 2 & 0 \\ 
      \textcolor{red}{2-2} & 3-4 & 5-0 & 0-6 \\ 
      \textcolor{red}{-1+1} & -1+2 & 2+0 & 2+3 
    \end{bmatrix}
  \\&=
    -\det\begin{bmatrix}
      3 & 2 & 0 \\ 
      -1 & 5 & -6 \\ 
      1 & 2 & 5 
    \end{bmatrix}
   = \dots
\end{align*} 
\end{observation}

\begin{activity}{10}
  Compute 
  \(
    \det\begin{bmatrix} 
      2 & 3 & 5 & 0 \\ 
      0 & 3 & 2 & 0 \\ 
      1 & 2 & 0 & 3 \\ 
      -1 & -1 & 2 & 2 
    \end{bmatrix}
  \) by using any combination of row/column operations.
\end{activity}

\begin{observation}
Larger matrices can be whittled down by using a technique called
\term{Laplace expansion} or \term{cofactor expansion} on
\textit{any} row or column.

\newcommand{\tikzmark}[2]{\tikz[overlay,remember picture,baseline] \node [anchor=base] (#1) {$#2$};}

\newcommand{\DrawVLine}[3][]{%
  \begin{tikzpicture}[overlay,remember picture]
    \draw[#1] (#2.north) -- (#3.south);
  \end{tikzpicture}
}
\newcommand{\DrawHLine}[3][]{%
  \begin{tikzpicture}[overlay,remember picture]
    \draw[#1] (#2.west) -- (#3.east);
  \end{tikzpicture}
}

  \begin{align*}
\det \begin{bmatrix} 2 & 3 & 5 & 0 \\ 0 & 3 & 2 & 0 \\ 1 & 2 & 0 & 3 \\ -1 & -1 & 2 & 2 \end{bmatrix} &=
- 0 \det \begin{bmatrix} \tikzmark{topA}{2} & 3 & 5 & 0 \\ \tikzmark{leftA}{0} & 3 & 2 & \tikzmark{rightA}{0} \\ 1 & 2 & 0 & 3 \\ \tikzmark{bottomA}{-1} & -1 & 2 & 2 \end{bmatrix}
+ 3 \det \begin{bmatrix} 2 & \tikzmark{topB}{3} & 5 & 0 \\ \tikzmark{leftB}{0} & 3 & 2 & \tikzmark{rightB}{0} \\ 1 & 2 & 0 & 3 \\ -1 & \tikzmark{bottomB}{-1} & 2 & 2 \end{bmatrix}   \\
&\phantom{=} - 2 \det \begin{bmatrix} 2 & 3 & \tikzmark{topC}{5} & 0 \\ \tikzmark{leftC}{0} & 3 & 2 & \tikzmark{rightC}{0} \\ 1 & 2 & 0 & 3 \\ -1 & -1 & \tikzmark{bottomC}{2} & 2 \end{bmatrix}
+ 0 \det \begin{bmatrix} 2 & 3 & 5 & \tikzmark{topD}{0} \\ \tikzmark{leftD}{0} & 3 & 2 & \tikzmark{rightD}{0} \\ 1 & 2 & 0 & 3 \\ -1 & -1 & 2 & \tikzmark{bottomD}{2} \end{bmatrix}  \\
&= 3 \det \begin{bmatrix}2 & 5 & 0 \\ 1 & 0 & 3 \\ -1 & 2 & 2 \end{bmatrix} -2 \det \begin{bmatrix} 2 & 3 & 0 \\ 1 & 2 & 3 \\ -1 & -1 & 2 \end{bmatrix} \\
&= \dots
\end{align*}
\DrawHLine[red,thick]{leftA}{rightA}
\DrawVLine[red,thick]{topA}{bottomA}
\DrawHLine[red,thick]{leftB}{rightB}
\DrawVLine[red,thick]{topB}{bottomB}
\DrawHLine[red,thick]{leftC}{rightC}
\DrawVLine[red,thick]{topC}{bottomC}
\DrawHLine[red,thick]{leftD}{rightD}
\DrawVLine[red,thick]{topD}{bottomD}

Notice that the diagonal entry \(3\) got a \(+\), and that \(-\)
signs alternate outward due to the omitted column swaps.
  \end{observation}

\begin{activity}{10}
  Compute 
  \(
    \det\begin{bmatrix} 
      1 & 2 & 1 & 0 \\ 
      0 & 3 & 2 & -1 \\ 
      1 & 2 & 0 & 3 \\ 
      -1 & -3 & 2 & -2 
    \end{bmatrix}
  \).
\end{activity}


\end{applicationActivities}
