\begin{applicationActivities}

\begin{remark}
  Recall that the column versions of the three row-reducing operations
  a matrix may be used to simplify a determinant:
  \begin{enumerate}[(a)]
  \item Multiplying columns by scalars:
        \[c\det([\cdots\hspace{0.5em}\vec{v}\hspace{0.5em} \cdots])=
        \det([\cdots\hspace{0.5em}c\vec{v}\hspace{0.5em} \cdots])\]
  \item Swapping two columns:
        \[\det([\cdots\hspace{0.5em}\vec{v}\hspace{0.5em}
        \cdots\hspace{1em}\vec{w}\hspace{0.5em} \cdots])=
        -\det([\cdots\hspace{0.5em}\vec{w}\hspace{0.5em}
        \cdots\hspace{1em}\vec{v}\hspace{0.5em} \cdots])\]
  \item Adding a multiple of a column to another column:
        \[\det([\cdots\hspace{0.5em}\vec{v}\hspace{0.5em}
        \cdots\hspace{1em}\vec{w}\hspace{0.5em} \cdots])=
        \det([\cdots\hspace{0.5em}\vec{v}+c\vec{w}\hspace{0.5em}
        \cdots\hspace{1em}\vec{w}\hspace{0.5em} \cdots])\]
  \end{enumerate}
\end{remark}
\begin{remark}
The determinants of row operation matrices may be computed
by manipulating columns to reduce each matrix to the identity:
\begin{itemize}
\item Scaling a row: \(\det
  \begin{bmatrix}
  c & 0 & 0 \\
  0 & 1 & 0 \\
  0 & 0 & 1
  \end{bmatrix}
    =
  c\det
  \begin{bmatrix}
  1 & 0 & 0 \\
  0 & 1 & 0 \\
  0 & 0 & 1
  \end{bmatrix}
    =
  c
\)
\item Swapping rows: \(\det
  \begin{bmatrix}
  0 & 1 & 0 \\
  1 & 0 & 0 \\
  0 & 0 & 1
  \end{bmatrix}
    =
  -1\det
  \begin{bmatrix}
  1 & 0 & 0 \\
  0 & 1 & 0 \\
  0 & 0 & 1
  \end{bmatrix}
    =
  -1
\)
\item Adding a row multiple to another row: \(\det
  \begin{bmatrix}
  1 & 0 & c \\
  0 & 1 & 0 \\
  0 & 0 & 1
  \end{bmatrix}
    =
  \det
  \begin{bmatrix}
  1 & 0 & c-1c \\
  0 & 1 & 0-0c \\
  0 & 0 & 1-0c
  \end{bmatrix}
    =
  \det(I)=1
\)
\end{itemize}
\end{remark}
\begin{fact}
Thus we can also use row operations to simplify determinants:
\begin{enumerate}
\item Multiplying rows by scalars:
  \(\det\begin{bmatrix}\vdots\\cR\\\vdots\end{bmatrix}=
  c\det\begin{bmatrix}\vdots\\R\\\vdots\end{bmatrix}\)
\item Swapping two rows:
  \(\det\begin{bmatrix}\vdots\\R\\\vdots\\S\\\vdots\end{bmatrix}=
  -\det\begin{bmatrix}\vdots\\S\\\vdots\\R\\\vdots\end{bmatrix}\)
\item Adding multiples of rows to other rows:
  \(\det\begin{bmatrix}\vdots\\R\\\vdots\\S\\\vdots\end{bmatrix}=
  \det\begin{bmatrix}\vdots\\R+cS\\\vdots\\S\\\vdots\end{bmatrix}\)
\end{enumerate}
\end{fact}



\begin{activity}{10}
  Compute the determinant of \(\begin{bmatrix} 2 & 4 \\ 2 & 3 \end{bmatrix}\) 
  by manipulating its rows and columns to simplify the matrix to \(I\):

  \begin{align*}
    \det\begin{bmatrix} 2 & 4 \\ 2 & 3 \end{bmatrix}
      &=
    \unknown \det \begin{bmatrix} 1 & 2 \\ 2 & 3 \end{bmatrix}\\
      &\vdots\\
      &=
    \unknown \det \begin{bmatrix} 1 & 0 \\ 0 & 1 \end{bmatrix}\\
      &=
    \unknown
  \end{align*}
\end{activity}

\begin{observation}
This is manageable in the $2 \times 2$ case, but we saw in Module E that for larger matrices, row reducing all
the way to the identity can be a chore!

\vspace{1em}

So we will explore some more techniques in the next few slides.
\end{observation}



\begin{activity}{5}
  The following image illustrates the transformation of the unit cube
  by the matrix
  $\begin{bmatrix} 3 & 1 & 0 \\  1 & 1 & 1 \\  0 & 0 & 1\end{bmatrix}$.

  \begin{center}
  \begin{tikzpicture}
  \fill[purple!50!white] (0,0,0) -- (1,0,1) -- (4,0,2) -- (3,0,1) -- (0,0,0);
  \draw[thin,gray,->] (0,0,0) -- (3,0,0);
  \draw[thin,gray,->] (0,0,0) -- (0,2,0);
  \draw[thin,gray,->] (0,0,0) -- (0,0,2);
  %(y,z,x)
  \draw[blue] (1,0,1) -- (4,0,2) -- (3,0,1);
  \draw[blue] (1,1,0) -- (2,1,1) -- (5,1,2) -- (4,1,1) -- (1,1,0);
  \draw[blue] (1,0,1) -- +(1,1,0);
  \draw[blue] (4,0,2) -- +(1,1,0);
  \draw[blue] (3,0,1) -- +(1,1,0);

  \draw[purple,thick,->] (0,0,0) -- (1,1,0)
    node[above left]{\tiny$\begin{bmatrix} 0 \\ 1 \\ 1\end{bmatrix}$};
  \draw[purple,thick,->] (0,0,0) -- (1,0,1)
    node[below]{\tiny$\begin{bmatrix} 1 \\ 1 \\ 0\end{bmatrix}$};
  \draw[purple,thick,->] (0,0,0) -- (3,0,1)
    node[above right]{\tiny$\begin{bmatrix} 3 \\ 1 \\ 0\end{bmatrix}$};
  \draw[purple,dashed,very thick] (0,0,0) -- node[left] {\tiny\(h=1\)} (0,1,0);
  \end{tikzpicture}
  \end{center}
  Recall that \(V=Bh\).
This volume is equal to which of the following areas?
\begin{multicols}{4}
\begin{enumerate}[(a)]
\item $\det \begin{bmatrix} 3 & 1 \\ 1 & 1 \end{bmatrix}$
\item $\det \begin{bmatrix} 3 & 1 \\ 1 & 0 \end{bmatrix}$
\item $\det \begin{bmatrix} 3 & 1 \\ 0 & 1 \end{bmatrix}$
\item $\det \begin{bmatrix} 1 & 1 \\ 0 & 1 \end{bmatrix}$
\end{enumerate}
\end{multicols}
\end{activity}

\begin{fact}
If row \(i\) contains all zeros except for a \(1\) on the diagonal, 
then both column and row \(i\)
may be removed without changing the value of the determinant.
\[
  \det \begin{bmatrix}
    3 & \textcolor{red}{2} & -1 & 3 \\
    \textcolor{red}{0} & \textcolor{red}{1} 
      & \textcolor{red}{0} & \textcolor{red}{0} \\
    -1 & \textcolor{red}{4} & 1 & 0 \\
    5 & \textcolor{red}{0} & 11 & 1
  \end{bmatrix} =
  \det \begin{bmatrix}
    3 & -1 & 3 \\
    -1 & 1 & 0 \\
    5 & 11 & 1
  \end{bmatrix}
\]
Since row and column operations affect the determinants in the same
way, the same technique works for a column of all zeros except for
a \(1\) on the diagonal.
\[
  \det \begin{bmatrix}
    3 & \textcolor{red}{0} & -1 & 5 \\
    \textcolor{red}{2} & \textcolor{red}{1} & \textcolor{red}{4} & 
       \textcolor{red}{0} \\
    -1 & \textcolor{red}{0} & 1 & 11 \\
    3 & \textcolor{red}{0} & 0 & 1
  \end{bmatrix} =
  \det \begin{bmatrix}
    3 & -1 & 5 \\
    -1 & 1 & 11 \\
    3 & 0 & 1
  \end{bmatrix}
\] 
\end{fact}


\begin{activity}{5}
  Compute  $\det \begin{bmatrix} 1 & 0 & 0 \\ 1 & 5 & 12 \\ 3 & 2 & -1 \end{bmatrix}$
  by doing the following:
  \begin{itemize}
  \item Reduce its dimension by eliminating a row and column.
  \item Evaluate the resulting \(2\times 2\) determinant.
  \end{itemize}
\end{activity}

\begin{activity}{5}
  Compute \(\det \begin{bmatrix} 0 & 3 & -2 \\ 2 & 5 & 12 \\ 0 & 2 & -1 \end{bmatrix}\)
  by first doing the following:
  \begin{itemize}
    \item Factor out a \(2\) from a column.
    \item Swap rows or columns to put a \(1\) on the diagonal.
  \end{itemize}
\end{activity}

\begin{activity}{5}
  Compute \(\det \begin{bmatrix} 4 & -2 & 2 \\ 3 & 1 & 4 \\ 1 & -1 & 3\end{bmatrix}\)
  by first doing the following:
  \begin{itemize}
  	\item Use two row or column operations to create two zeroes in the same row or column.
    \item Factor a constant out if needed.
    \item Swap rows or columns if needed.
  \end{itemize}
\end{activity}


\begin{observation}
Row/column operations can be used to add zeroes and reduce dimension
in large matrices.

\begin{align*}
    \det\begin{bmatrix} 
      2 & 3 & 5 & 0 \\ 
      0 & 3 & 2 & 0 \\ 
      1 & 2 & 0 & 3 \\ 
      -1 & -1 & 2 & 2 
    \end{bmatrix}
  &=
    -\det\begin{bmatrix}
      1 & 2 & 0 & 3 \\ 
      0 & 3 & 2 & 0 \\ 
      2 & 3 & 5 & 0 \\ 
      -1 & -1 & 2 & 2 
    \end{bmatrix}
  \\&=
    -\det\begin{bmatrix}
      \textcolor{red}{1} & \textcolor{red}{2} & 
         \textcolor{red}{0} & \textcolor{red}{3} \\ 
      \textcolor{red}{0} & 3 & 2 & 0 \\ 
      \textcolor{red}{2-2} & 3-4 & 5-0 & 0-6 \\ 
      \textcolor{red}{-1+1} & -1+2 & 2+0 & 2+3 
    \end{bmatrix}
  \\&=
    -\det\begin{bmatrix}
      3 & 2 & 0 \\ 
      -1 & 5 & -6 \\ 
      1 & 2 & 5 
    \end{bmatrix}
   = \dots
\end{align*} 
\end{observation}

\begin{activity}{10}
  Compute 
  \(
    \det\begin{bmatrix} 
      2 & 3 & 5 & 0 \\ 
      0 & 3 & 2 & 0 \\ 
      1 & 2 & 0 & 3 \\ 
      -1 & -1 & 2 & 2 
    \end{bmatrix}
  \) by using any combination of row/column operations.
\end{activity}

\begin{observation}
Another option is to take advantage of the fact that the determinant is linear in each row or column.  This approach is called
\term{Laplace expansion} or \term{cofactor expansion}. 

For example, since $\begin{bmatrix} 1 & 2 & 4 \end{bmatrix} = \begin{bmatrix} 1 & 0 & 0 \end{bmatrix}+2\begin{bmatrix} 0 & 1 &  0 \end{bmatrix}+4\begin{bmatrix} 0  & 0 & 1 \end{bmatrix}$,

  \begin{align*}
\det \begin{bmatrix} 2 & 3 & 5  \\ -1 & 3 & 5 \\ 1 & 2 & 4 \end{bmatrix} &=
1\det \begin{bmatrix} 2 & 3 & 5  \\ -1 & 3 & 5 \\ 1 & 0 & 0 \end{bmatrix} +
2\det \begin{bmatrix} 2 & 3 & 5  \\ -1 & 3 & 5 \\ 0 & 1 & 0 \end{bmatrix} +
4\det \begin{bmatrix} 2 & 3 & 5  \\ -1 & 3 & 5 \\ 0 & 0 & 1 \end{bmatrix} \\
&= -1\det \begin{bmatrix}  5 & 3 & 2 \\ 5 & 3 & -1 \\ 0 & 0 & 1 \end{bmatrix} 
-2\det \begin{bmatrix} 2 & 5 & 3  \\ -1 & 5 & 3 \\ 0 & 0 & 1 \end{bmatrix} +
4\det \begin{bmatrix} 2 & 3 & 5  \\ -1 & 3 & 5 \\ 0 & 0 & 1 \end{bmatrix} \\
&= -\det \begin{bmatrix} 5 & 3 \\ 5 & 3 \end{bmatrix} 
-2 \det \begin{bmatrix} 2 & 5 \\ -1 & 5 \end{bmatrix}
+4 \det \begin{bmatrix} 2 & 3 \\ -1 & 3 \end{bmatrix}
\end{align*}

\end{observation}

\begin{observation}
Applying Laplace expansion to a $2 \times 2$ matrix yields the usual formula:
$$\det \begin{bmatrix} a & b \\ c & d \end{bmatrix}=a\det \begin{bmatrix} 1 & 0 \\ c & d \end{bmatrix} + b \det \begin{bmatrix} 0 & 1 \\ c & d \end{bmatrix} = ad-bc.$$

\vspace{1em}

One can similarly obtain a formula (with six terms) in the $3 \times 3$ case, but other methods are almost always more efficient.

\vspace{1em}

In the $4\times 4$ case, any formula must have 24 terms!
\end{observation}

\begin{activity}{10}
  Use Laplace expansion to compute 
  \(
    \det\begin{bmatrix} 
      2 & 2 & 1 & 0 \\ 
      0 & 3 & 2 & -1 \\ 
      3 & 2 & 0 & 3 \\ 
      0 & -3 & 2 & -2 
    \end{bmatrix}
  \).
\end{activity}

\begin{activity}{5}
Based on what we've done today, which technique is easier for computing determinants?
\begin{enumerate}[(a)]
\item Row reducing all the way to the identity
\item Doing row and/or column operations and then reducing the dimension
\item Laplace expansion
\item Some other technique (be prepared to describe it)
\end{enumerate}
\end{activity}

\begin{activity}{10}
  Use your preferred technique to compute 
  \(
    \det\begin{bmatrix} 
      4 & -3 & 0 & 0 \\ 
      1 & -3 & 2 & -1 \\ 
      3 & 2 & 0 & 3 \\ 
      0 & -3 & 2 & -2 
    \end{bmatrix}
  \).
\end{activity}

\end{applicationActivities}
