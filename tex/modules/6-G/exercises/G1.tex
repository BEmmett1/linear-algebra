\begin{problem}{G1}
Consider the row operation \(R_1+5R_3\to R_1\) applied as follows to show
\(A\sim B\):
\[
A=\begin{bmatrix}1&2&3\\4&5&6\\7&8&9\end{bmatrix}
  \sim
\begin{bmatrix}1+5(7)&2+5(8)&3+5(9)\\4&5&6\\7&8&9\end{bmatrix}=B
\]
\begin{enumerate}[(a)]
\item Find a matrix \(R\) such that \(B=RA\).
\item If \(C \in M_{3,3}\) is a matrix with \(\det C = 4\), how does applying this row operation to \(C\) change its determinant?
\end{enumerate}
\end{problem}
\begin{solution}
\begin{enumerate}
\item \(R= \begin{bmatrix} 1 & 0 & 5 \\ 0 & 1 & 0 \\ 0 & 0 & 1 \end{bmatrix}\).
\item \(\det(RC)= \det(R)\det(C)=(1)(4)=4\).
\end{enumerate}
\end{solution}

\begin{problem}{G1}
Consider the row operation \(R_2-4R_3\to R_2\) applied as follows to show
\(A\sim B\):
\[
A=\begin{bmatrix}1&2&3\\4&5&6\\7&8&9\end{bmatrix}
  \sim
\begin{bmatrix}1&2&3\\4-4(7)&5-4(8)&6-4(9)\\7&8&9\end{bmatrix}=B
\]
\begin{enumerate}[(a)]
\item Find a matrix \(R\) such that \(B=RA\).
\item If \(C \in M_{3,3}\) is a matrix with \(\det C = 7\), how does applying this row operation to \(C\) change its determinant?
\end{enumerate}
\end{problem}
\begin{solution}
\begin{enumerate}
\item \(R= \begin{bmatrix} 1 & 0 & 0 \\ 0 & 1 & -4 \\ 0 & 0 & 1 \end{bmatrix}\).
\item \(\det(RC)= \det(R)\det(C)=(1)(7)=7\).
\end{enumerate}
\end{solution}

\begin{problem}{G1}
Consider the row operation \(R_3-2R_1\to R_3\) applied as follows to show
\(A\sim B\):
\[
A=\begin{bmatrix}1&2&3\\4&5&6\\7&8&9\end{bmatrix}
  \sim
\begin{bmatrix}1&2&3\\4&5&6\\7-2(1)&8-2(1)&9-2(1)\end{bmatrix}=B
\]
\begin{enumerate}[(a)]
\item Find a matrix \(R\) such that \(B=RA\).
\item If \(C \in M_{3,3}\) is a matrix with \(\det C = -8\), how does applying this row operation to \(C\) change its determinant?
\end{enumerate}
\end{problem}
\begin{solution}
\begin{enumerate}
\item \(R= \begin{bmatrix} 1 & 0 & 0 \\ 0 & 1 & 0 \\ -2 & 0 & 1 \end{bmatrix}\).
\item \(\det(RC)= \det(R)\det(C)=(1)(-8)=-8\).
\end{enumerate}
\end{solution}

\begin{problem}{G1}
Consider the row operation \(4R_3\to R_3\) applied as follows to show
\(A\sim B\):
\[
A=\begin{bmatrix}1&2&3\\4&5&6\\7&8&9\end{bmatrix}
  \sim
\begin{bmatrix}1&2&3\\4&5&6\\(4)7&(4)8&(4)9)\end{bmatrix}=B
\]
\begin{enumerate}[(a)]
\item Find a matrix \(R\) such that \(B=RA\).
\item If \(C \in M_{3,3}\) is a matrix with \(\det C = -12\), how does applying this row operation to \(C\) change its determinant?
\end{enumerate}
\end{problem}
\begin{solution}
\begin{enumerate}
\item \(R= \begin{bmatrix} 1 & 0 & 0 \\ 0 & 1 & 0 \\ 0 & 0 & 4 \end{bmatrix}\).
\item \(\det(RC)= \det(R)\det(C)=(4)(-12)=-48\).
\end{enumerate}
\end{solution}

\begin{problem}{G1}
Consider the row operation \(-8R_1\to R_1\) applied as follows to show
\(A\sim B\):
\[
A=\begin{bmatrix}1&2&3\\4&5&6\\7&8&9\end{bmatrix}
  \sim
\begin{bmatrix}(-8)1&(-8)2&(-8)3\\4&5&6\\7&8&9)\end{bmatrix}=B
\]
\begin{enumerate}[(a)]
\item Find a matrix \(R\) such that \(B=RA\).
\item If \(C \in M_{3,3}\) is a matrix with \(\det C = -2\), how does applying this row operation to \(C\) change its determinant?
\end{enumerate}
\end{problem}
\begin{solution}
\begin{enumerate}
\item \(R= \begin{bmatrix} -8 & 0 & 0 \\ 0 & 1 & 0 \\ 0 & 0 & 1 \end{bmatrix}\).
\item \(\det(RC)= \det(R)\det(C)=(-8)(-2)=16\).
\end{enumerate}
\end{solution}

\begin{problem}{G1}
Consider the row operation \(5R_2\to R_2\) applied as follows to show
\(A\sim B\):
\[
A=\begin{bmatrix}1&2&3\\4&5&6\\7&8&9\end{bmatrix}
  \sim
\begin{bmatrix}1&2&3\\(5)4&(5)5&(5)6\\7&8&9)\end{bmatrix}=B
\]
\begin{enumerate}[(a)]
\item Find a matrix \(R\) such that \(B=RA\).
\item If \(C \in M_{3,3}\) is a matrix with \(\det C = 3\), how does applying this row operation to \(C\) change its determinant?
\end{enumerate}
\end{problem}
\begin{solution}
\begin{enumerate}
\item \(R= \begin{bmatrix} 1 & 0 & 0 \\ 0 & 5 & 0 \\ 0 & 0 & 1 \end{bmatrix}\).
\item \(\det(RC)= \det(R)\det(C)=(5)(3)=15\).
\end{enumerate}
\end{solution}

\begin{problem}{G1}
Consider the row operation that swaps \(R_1\) and \(R_2\) applied as follows to show
\(A\sim B\):
\[
A=\begin{bmatrix}1&2&3\\4&5&6\\7&8&9\end{bmatrix}
  \sim
\begin{bmatrix}4&5&6\\1&2&3\\7&8&9\end{bmatrix}=B
\]
\begin{enumerate}[(a)]
\item Find a matrix \(R\) such that \(B=RA\).
\item If \(C \in M_{3,3}\) is a matrix with \(\det C = 3\), how does applying this row operation to \(C\) change its determinant?
\end{enumerate}
\end{problem}
\begin{solution}
\begin{enumerate}
\item \(R= \begin{bmatrix} 0 & 1 & 0 \\ 1 & 0 & 0 \\ 0 & 0 & 1 \end{bmatrix}\).
\item \(\det(RC)= \det(R)\det(C)=(-1)(3)=-3\).
\end{enumerate}
\end{solution}

\begin{problem}{G1}
Consider the row operation that swaps \(R_3\) and \(R_2\) applied as follows to show
\(A\sim B\):
\[
A=\begin{bmatrix}1&2&3\\4&5&6\\7&8&9\end{bmatrix}
  \sim
\begin{bmatrix}1&2&3\\7&8&9\\4&5&6\end{bmatrix}=B
\]
\begin{enumerate}[(a)]
\item Find a matrix \(R\) such that \(B=RA\).
\item If \(C \in M_{3,3}\) is a matrix with \(\det C = -7\), how does applying this row operation to \(C\) change its determinant?
\end{enumerate}
\end{problem}
\begin{solution}
\begin{enumerate}
\item \(R= \begin{bmatrix} 1 & 0 & 0 \\ 0 & 0 & 1 \\ 0 & 1 & 0 \end{bmatrix}\).
\item \(\det(RC)= \det(R)\det(C)=(-1)(-7)=7\).
\end{enumerate}
\end{solution}

\begin{problem}{G1}
Consider the row operation that swaps \(R_3\) and \(R_1\) applied as follows to show
\(A\sim B\):
\[
A=\begin{bmatrix}1&2&3\\4&5&6\\7&8&9\end{bmatrix}
  \sim
\begin{bmatrix}7&8&9\\4&5&6\\1&2&3\end{bmatrix}=B
\]
\begin{enumerate}[(a)]
\item Find a matrix \(R\) such that \(B=RA\).
\item If \(C \in M_{3,3}\) is a matrix with \(\det C = -11\), how does applying this row operation to \(C\) change its determinant?
\end{enumerate}
\end{problem}
\begin{solution}
\begin{enumerate}
\item \(R= \begin{bmatrix} 0 & 0 & 1 \\ 0 & 1 & 0 \\ 1 & 0 & 0 \end{bmatrix}\).
\item \(\det(RC)= \det(R)\det(C)=(-1)(-11)=11\).
\end{enumerate}
\end{solution}
