


\begin{applicationActivities}{2}{13}

\begin{activity}{10}
  Last time we saw that \(\left\{
  x^3+1,x^2+2,4-7x,2x^3+x
  \right\}\) is linearly independent. Show that it spans \(\P^3\).
\end{activity}

\begin{definition}
  A \term{basis} is a linearly independent set that spans a vector space.
\end{definition}

\begin{observation}
  A basis may be thought of as a collection of building blocks for a vector
  space, since every vector in the space can be expressed as a unique linear
  combination of basis vectors.
\end{observation}



\begin{activity}{15}
  Which of the following sets are bases for \(\IR^4\)? \\
    \begin{tabular}{cc}
	\(\left\{
    \begin{bmatrix}1\\0\\0\\0\end{bmatrix},
    \begin{bmatrix}0\\1\\0\\0\end{bmatrix},
    \begin{bmatrix}0\\0\\1\\0\end{bmatrix},
    \begin{bmatrix}0\\0\\0\\1\end{bmatrix}
    \right\}
    \)   & 

  \(\left\{
    \begin{bmatrix}2\\3\\0\\-1\end{bmatrix},
    \begin{bmatrix}2\\0\\0\\3\end{bmatrix},
    \begin{bmatrix}4\\3\\0\\2\end{bmatrix},
    \begin{bmatrix}-3\\0\\1\\3\end{bmatrix}
    \right\}
    \)  \\ 


    \(\left\{
    \begin{bmatrix}2\\3\\0\\-1\end{bmatrix},
    \begin{bmatrix}2\\0\\0\\3\end{bmatrix},
    \begin{bmatrix}3\\13\\7\\16\end{bmatrix},
    \begin{bmatrix}-1\\10\\7\\14\end{bmatrix},
    \begin{bmatrix}4\\3\\0\\2\end{bmatrix}
    \right\}
    \) &

	\(\left\{
    \begin{bmatrix}2\\3\\0\\-1\end{bmatrix},
    \begin{bmatrix}4\\3\\0\\2\end{bmatrix},
    \begin{bmatrix}-3\\0\\1\\3\end{bmatrix},
    \begin{bmatrix}3\\6\\1\\5\end{bmatrix}
    \right\}
    \) \\

   \(\left\{
    \begin{bmatrix}5\\3\\0\\-1\end{bmatrix},
    \begin{bmatrix}-2\\1\\0\\3\end{bmatrix},
    \begin{bmatrix}4\\5\\1\\3\end{bmatrix}
    \right\}
    \)  &
  \end{tabular}
\end{activity}

\begin{activity}{10}
  If \(\{\vect v_1,\vect v_2,\vect v_3,\vect v_4\}\) is a basis for
  \(\IR^4\), that means \(\RREF[\vect v_1\,\vect v_2\,\vect v_3\,\vect v_4]\)
  doesn't have a column without a pivot position, and doesn't have a
  row of zeros. What is \(\RREF[\vect v_1\,\vect v_2\,\vect v_3\,\vect v_4]\)?
\end{activity}

\begin{fact}
  The set \(\{\vect v_1,\dots,\vect v_m\}\) is a basis for \(\IR^n\) if and
  only if \(m=n\) and
  \(\RREF[\vect v_1\,\dots\,\vect v_n]=
  \begin{bmatrix}
    1&0&\dots&0\\
    0&1&\dots&0\\
    \vdots&\vdots&\ddots&\vdots\\
    0&0&\dots&1
  \end{bmatrix}
  \).

  That is, a basis for \(\IR^n\) must have exactly \(n\) vectors and
  its square matrix must row-reduce to the \term{identity matrix}
  containing all zeros except for a downward diagonal of ones.
\end{fact}

\begin{activity}{10}
  Consider the set \(\left\{
  \begin{bmatrix}2\\3\\0\\1\end{bmatrix},
  \begin{bmatrix}2\\0\\1\\-1\end{bmatrix},
  \begin{bmatrix}2\\-3\\2\\-3\end{bmatrix},
  \begin{bmatrix}1\\5\\-1\\0\end{bmatrix}
  \right\}
  \).

  \begin{subactivity}
    Use \(\RREF\begin{bmatrix}
    2&2&2&1\\
    3&0&-3&5\\
    0&1&2&-1\\
    1&-1&-3&0
    \end{bmatrix}\) to identify which vector may be removed to make the set
    linearly independent.
  \end{subactivity}

  \begin{subactivity}
    Find a basis for \(\vspan\left\{
    \begin{bmatrix}2\\3\\0\\1\end{bmatrix},
    \begin{bmatrix}2\\0\\1\\-1\end{bmatrix},
    \begin{bmatrix}2\\-3\\2\\-3\end{bmatrix},
    \begin{bmatrix}1\\5\\-1\\0\end{bmatrix}
    \right\}
    \).
  \end{subactivity}
\end{activity}

\end{applicationActivities}
