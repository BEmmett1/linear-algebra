


\begin{applicationActivities}{3}{14}

\begin{fact}
  To compute a basis for the subspace \(\vspan\{\vect v_1,\dots,\vect v_m\}\),
  simply remove the vectors corresponding to the non-pivot columns of
  \(\RREF[\vect v_1\,\dots\,\vect v_m]\).
\end{fact}

\begin{activity}{10}
  Find all subsets of \(S=\left\{
  \begin{bmatrix}2\\3\\0\\1\end{bmatrix},
  \begin{bmatrix}2\\0\\1\\-1\end{bmatrix},
  \begin{bmatrix}2\\-3\\2\\-3\end{bmatrix},
  \begin{bmatrix}1\\5\\-1\\0\end{bmatrix}
  \right\}
  \) that are a basis for \(\vspan S\) by changing the order
  of the vectors in \(S\).
\end{activity}

\begin{activity}{10}
  Assume \(\vect w_1\not=\vect w_2\) are distinct vectors in \(V\),
  which has a basis containing a single vector: \(\{\vect v\}\).
  Could \(\{\vect w_1,\vect w_2\}\) be a basis?
\end{activity}

\begin{fact}
  All bases for a vector space are the same size.
\end{fact}

\begin{definition}
  The \term{dimension} of a vector space is given by the cardinality/size
  of any basis for the vector space.
\end{definition}

\begin{activity}{15}
  Find the dimension of each subspace of \(\IR^4\). \\
	\begin{tabular}{ll}
     \(\vspan\left\{
    \begin{bmatrix}1\\0\\0\\0\end{bmatrix},
    \begin{bmatrix}0\\1\\0\\0\end{bmatrix},
    \begin{bmatrix}0\\0\\1\\0\end{bmatrix},
    \begin{bmatrix}0\\0\\0\\1\end{bmatrix}
    \right\}
    \) & 

     \(\vspan\left\{
    \begin{bmatrix}2\\3\\0\\-1\end{bmatrix},
    \begin{bmatrix}2\\0\\0\\3\end{bmatrix},
    \begin{bmatrix}4\\3\\0\\2\end{bmatrix},
    \begin{bmatrix}-3\\0\\1\\3\end{bmatrix}
    \right\}
    \) \\ 

 & \\
     \(\vspan\left\{
    \begin{bmatrix}2\\3\\0\\-1\end{bmatrix},
    \begin{bmatrix}2\\0\\0\\3\end{bmatrix},
    \begin{bmatrix}3\\13\\7\\16\end{bmatrix},
    \begin{bmatrix}-1\\10\\7\\14\end{bmatrix},
    \begin{bmatrix}4\\3\\0\\2\end{bmatrix}
    \right\}
    \)
 &
     \(\vspan\left\{
    \begin{bmatrix}2\\3\\0\\-1\end{bmatrix},
    \begin{bmatrix}4\\3\\0\\2\end{bmatrix},
    \begin{bmatrix}-3\\0\\1\\3\end{bmatrix},
    \begin{bmatrix}3\\6\\1\\5\end{bmatrix}
    \right\}
    \) \\

 & \\
     \(\vspan\left\{
    \begin{bmatrix}5\\3\\0\\-1\end{bmatrix},
    \begin{bmatrix}-2\\1\\0\\3\end{bmatrix},
    \begin{bmatrix}4\\5\\1\\3\end{bmatrix}
    \right\}
	\) \\
\end{tabular}
\end{activity}

\begin{activity}{5}
  What is the dimension of the vector space of \(7\)th-degree (or less)
  polynomials \(\P^7\)?
  \begin{multicols}{4}
    \begin{enumerate}[a)]
      \item 6
      \item 7
      \item 8
      \item infinite
    \end{enumerate}
  \end{multicols}
\end{activity}

\begin{activity}{5}
  What is the dimension of the vector space of all polynomials
  \(\P\)?
  \begin{multicols}{4}
    \begin{enumerate}[a)]
      \item 6
      \item 7
      \item 8
      \item infinite
    \end{enumerate}
  \end{multicols}
\end{activity}

\begin{observation}
  Several interesting vector spaces are infinite-dimensional:
  \begin{itemize}
    \item The space of polynomials \(\P\) (consider the set
          \(\{1,x,x^2,x^3,\dots\}\)).
    \item The space of continuous functions \(C(\IR)\) (which contains
          all polynomials, in addition to other functions like
          \(e^x=1+x+x^2/2+x^3/3+\dots\)).
    \item The space of real number sequences \(\IR^\infty\) (consider
          the set \(\{(1,0,0,\dots),(0,1,0,\dots),(0,0,1,\dots),\dots\}\)).
  \end{itemize}
\end{observation}

\begin{fact}
  Every vector space with finite dimension, that is, every
  vector space with a basis of the form
  \(\{\vect v_1,\vect v_2,\dots,\vect v_n\}\) is isomorphic to a
  Euclidean space \(\IR^n\):

  \[
    c_1\vect v_1+c_2\vect v_2+\dots+c_n\vect v_n
    \leftrightarrow
    \begin{bmatrix}
      c_1\\c_2\\\vdots\\c_n
    \end{bmatrix}
  \]
\end{fact}

\end{applicationActivities}
