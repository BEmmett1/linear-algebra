\documentclass{article}
\usepackage{tbil-la}
\usepackage[top=1in,bottom=1in,right=1in,left=1in]{geometry}

\parindent=0pt

%Problem environment -- takes an argument which is the standard
\newenvironment{problem}[1]
%before
{
  \begin{flushleft}
  %{\bfseries \arabic{problem} .}
  %Problem numbering by standard
  \textbf{#1}.
  \ignorespaces
}
%after
{
  \end{flushleft}
}

\newenvironment{solution}
%before
{
  \ignorespaces
  \textbf{Solution:}
}
%after
{
  \ignorespacesafterend
  \begin{flushright}
  {\bfseries \qed}
  \end{flushright}
}



\begin{document}

\begin{center}
\Large \textbf{Sample Assessment Exercises}
\end{center}

This document contains one exercise and solution for each standard.
The goal is to give you an idea of what the exercises might look like,
and what the expectations for a complete solution are.

\begin{problem}{E1}
Write a system of linear equations corresponding to the following
augmented matrix.
\[
\begin{bmatrix}[cccc|c]
3 & 2 & 0 & 1 & 1 \\
-1 & -4 & 1 & -7 & 0 \\
0 & 1 & -1 & 0 & -2
\end{bmatrix}
\]
\end{problem}
\begin{solution}
\begin{alignat*}{5}
3x_1 &\,+\,& 2x_2 &\,\,& &\,+\,&x_4 &= 1 \\
-x_1 &\,-\,& 4x_2 &\,+\,&x_3&\,-\,&7x_4 &= 0 \\
&\,\,& x_2 &\,-\,&x_3 &\,\,&  &= -2
\end{alignat*}
\end{solution}

\begin{problem}{E2}
Put the following matrix in reduced row echelon form.
\[
  \begin{bmatrix}
  0 & 3 & 1 & 2 \\
  1 & 2 & -1 & -3 \\
  2 & 4 & -1 & -1
  \end{bmatrix}
\]
\end{problem}
\begin{solution}
\begin{alignat*}{4}
  \begin{bmatrix}
  0 & 3 & 1 & 2 \\
  1 & 2 & -1 & -3 \\
  2 & 4 & -1 & -1
  \end{bmatrix}
  &\sim&
  \begin{bmatrix}
  \circledNumber{1} & 2 & -1 & -3 \\
  0 & 3 & 1 & 2 \\
  2 & 4 & -1 & -1
  \end{bmatrix}
  &\sim&
  \begin{bmatrix}
  \circledNumber{1} & 2 & -1 & -3 \\
  0 & 3 & 1 & 2 \\
  0 & 0 & 1 & 5
  \end{bmatrix}
  &\sim&
  \begin{bmatrix}
  \circledNumber{1} & 2 & -1 & -3 \\
  0 & \circledNumber{1} & \frac{1}{3} & \frac{2}{3} \\
  0 & 0 & 1 & 5
  \end{bmatrix} \\
  &\sim&
  \begin{bmatrix}
  \circledNumber{1} & 0 & -\frac{5}{3} & -\frac{13}{3} \\
  0 & \circledNumber{1} & \frac{1}{3} & \frac{2}{3} \\
  0 & 0 & \circledNumber{1} & 5
  \end{bmatrix}
  &\sim&
  \begin{bmatrix}
  \circledNumber{1} & 0 & -\frac{5}{3} & -\frac{13}{3} \\
  0 & \circledNumber{1} & 0 & -1 \\
  0 & 0 & \circledNumber{1} & 5
  \end{bmatrix}
  &\sim&
  \begin{bmatrix}
  \circledNumber{1} & 0 & 0 & 4 \\
  0 & \circledNumber{1} & 0 & -1 \\
  0 & 0 & \circledNumber{1} & 5
  \end{bmatrix}
\end{alignat*}
\end{solution}

\begin{problem}{E3}
Find the solution set for the following system of linear equations.
\begin{alignat*}{4}
2x&\,+\,&4y&\,+\,&z &= 5 \\
x&\,+\,&2y &\,\,& &= 3
\end{alignat*}
\end{problem}
\begin{solution}
\[
  \RREF \left(
    \begin{bmatrix}[ccc|c] 2 & 4 & 1 & 5 \\ 1 & 2 & 0 & 3 \end{bmatrix}
  \right)
    =
  \begin{bmatrix}[ccc|c] 1 & 2 & 0 & 3 \\ 0 & 0 & 1 & -1\end{bmatrix}
\]
This corresponds to the system
\begin{alignat*}{2}
x\,+\,2y&\,\,&  &= 3 \\
  &\,\,& z&= -1
\end{alignat*}
Since the $y$-column is a non-pivot column, it is a free variable, so we let $y=a$; then we have
\begin{alignat*}{3}
x&\,+\,&2y&\,\,&  &= 3 \\
 &\,\,&y &\,\,& &=a \\
 &\,\,& &\,\,& z&= -1
\end{alignat*}
and thus
\begin{align*}
x&= 3-2a \\
y&= a \\
z&= -1
\end{align*}
So the solution set is
\[
  \setBuilder{
    \begin{bmatrix} 3-2a \\ a \\ -1 \end{bmatrix}
  }{
    a \in \IR
  }
\]
\end{solution}


\begin{problem}{V1}
Let \(V\) be the set of all polynomials, together with the operations
\(\oplus\) and \(\odot\) defined by the following for all polynomials
\(f(x),g(x)\) and scalars \(c \in \IR\):
\begin{align*}
f(x) \oplus g(x) &= xf(x)+xg(x) \\
c\odot f(x) &= cf(x)
\end{align*}
\begin{enumerate}[(a)]
\item Show that distribution property
\[c\odot \left(f(x)\oplus g(x) \right) = c \odot f(x) \oplus c \odot g(x)\]
holds.
\item Show why \(V\) is not a vector space.
\end{enumerate}
\end{problem}
\begin{solution}
\begin{enumerate}[(a)]
\item
Compute
\begin{align*}
  c \odot \left(f(x) \oplus g(x) \right)
  &= c \odot \left(xf(x)+xg(x)\right) \\
  &= c\left(xf(x)+xg(x) \right) \\
  &= cxf(x)+cxg(x)
\end{align*}
and
\begin{align*}
  c\odot f(x) \oplus c \odot g(x)
  &= (cf(x)) \oplus (cg(x))\\
  &= xcf(x)+xcg(x)
\end{align*}
Since these are the same, we have shown that
\(c\odot \left(f(x)\oplus g(x) \right) = c \odot f(x) \oplus c \odot g(x)\)
holds.
\item
Suppose \(f(x)=1\), \(g(x)=2\), and \(h(x)=3\). Then
\begin{align*}
  \left( f(x) \oplus g(x) \right) \oplus h(x)
  &=\left(x+2x \right) \oplus 3\\
  &=3x \oplus 3\\
  &= 3x^2+3x
\end{align*}
and
\begin{align*}
  f(x) \oplus \left( g(x) \oplus h(x) \right)
  &=1 \oplus \left(2x+3x\right) \\
  &=1 \oplus 5x \\
  &=x + 5x^2
\end{align*}


Since \(3x^2+3x\not=x+5x^2\), we have shown
that the vector property
\(
  \left( f(x) \oplus g(x) \right) \oplus h(x)
=
  f(x) \oplus \left( g(x) \oplus h(x) \right)
\)
fails.
\end{enumerate}
\end{solution}


\begin{problem}{V2}
Determine if \(\begin{bmatrix} 3 \\ -1 \\ 2 \end{bmatrix} \) is a linear combination of the vectors
\(\begin{bmatrix} 1 \\ 0 \\ 1 \end{bmatrix} \),
\(\begin{bmatrix} 3 \\ 2 \\ -1 \end{bmatrix} \), and
\(\begin{bmatrix} 1 \\ 1 \\ -1 \end{bmatrix} \).
\end{problem}
\begin{solution}

We compute
\[ \RREF
  \begin{bmatrix}[ccc|c]
  1 & 3 & 1 & 3 \\
  0 & 2 & 1 & -1 \\
  1 & -1 & -1 & 2
  \end{bmatrix}
  =
  \begin{bmatrix}[ccc|c]
  1 & 0 & -\frac{1}{2} & 0 \\
  0 & 1 & \frac{1}{2} & 0 \\
  0 & 0 & 0 & 1
  \end{bmatrix}
\]
Since this corresponds to an inconsistent system of equations, \(\begin{bmatrix} 3 \\ -1 \\ 2 \end{bmatrix} \) is 
not a linear combination of \(\begin{bmatrix} 1 \\ 0 \\ 1 \end{bmatrix} \),
\(\begin{bmatrix} 3 \\ 2 \\ -1 \end{bmatrix} \), and
\(\begin{bmatrix} 1 \\ 1 \\ -1 \end{bmatrix} \).

\end{solution}

\begin{problem}{V3}
Determine if  the vectors
\(\begin{bmatrix} 1 \\ 0 \\ 1 \end{bmatrix} \),
\(\begin{bmatrix} 3 \\ 2 \\ -1 \end{bmatrix} \), and
\(\begin{bmatrix} 1 \\ 1 \\ -1 \end{bmatrix} \) span \(\IR^3\).
\end{problem}
\begin{solution}

We compute
\[ \RREF
  \begin{bmatrix}[ccc]
  1 & 3 & 1  \\
  0 & 2 & 1  \\
  1 & -1 & -1
  \end{bmatrix}
  =
  \begin{bmatrix}[ccc]
  1 & 0 & -\frac{1}{2} \\
  0 & 1 & \frac{1}{2} \\
  0 & 0 & 0
  \end{bmatrix}
\]
Since the last row lacks a pivot, the vectors do not span \(\IR^3\).
\end{solution}

\begin{problem}{V4}
Consider the following two sets of Euclidean vectors.
\[
  W = \setBuilder{\begin{bmatrix} x \\ y \\ z \end{bmatrix}}{x+y=3z}
\hspace{3em}
  U = \setBuilder{\begin{bmatrix} x \\ y \\ z \end{bmatrix}}{x+y=3z+2}
\]
Show that one of these sets is a subspace of \(\IR^3\), and
that one of the sets is not.
\end{problem}
\begin{solution}
First consider \(\vec 0\). Since \(0+0=3(0)\), we see that
\(\vec 0\in W\). But since \(0+0\not=3(0)+2\),
we see that \(\vec 0\not\in U\). Therefore \(U\) is not a subspace.

To show that \(W\) is a subspace, let
\(
  \vec v=\begin{bmatrix} x_1 \\y_1 \\ z_1 \end{bmatrix}\in W
\) and
\(
  \vec w=\begin{bmatrix} x_2 \\y_2 \\ z_2 \end{bmatrix} \in W
\),
so we know \(x_1+y_1=3z_1\) and \(x_2+y_2=3z_2\).
Consider
\[
\begin{bmatrix} x_1 \\y_1 \\ z_1 \end{bmatrix}
+\begin{bmatrix} x_2 \\y_2 \\ z_2 \end{bmatrix}
=\begin{bmatrix} x_1+x_2 \\y_1+y_2 \\ z_1+z_2 \end{bmatrix}
.\]
Since
\[
  (x_1+x_2)+(y_1+y_2) = (x_1+y_1)+(x_2+y_2) = 3z_1+3z_2=3(z_1+z_2)
\]
we see that \(\vec v+\vec w\in W\), so \(W\) is closed under vector addition.

Now consider
\[
c\vec v
=\begin{bmatrix} cx_1 \\cy_1 \\ cz_1 \end{bmatrix}
.\]
Since
\[cx_1+cx_2 = c(x_1+x_2)=c(3z_1)=3(cz_1)\]
we see that \(c\vec v\in W\), so \(W\) is closed under scalar 
multiplication. Therefore \(W\) is a subspace of \(\IR^3\).
\end{solution}


\begin{problem}{V5}
Determine if the vectors
\( \begin{bmatrix} 3 \\ 2 \\ 1 \\ 0 \end{bmatrix} \) ,
\( \begin{bmatrix} -1 \\ 1 \\ 2 \\ 3 \end{bmatrix} \) ,
\( \begin{bmatrix} 0 \\ 1 \\ -1 \\ 1 \end{bmatrix} \) , and
\( \begin{bmatrix} 2 \\ 5 \\ 1 \\ 5 \end{bmatrix} \)
are linearly dependent or linearly independent.
\end{problem}
\begin{solution}
Compute
\[\RREF \begin{bmatrix} 3 & -1 & 0 & 2 \\ 2 & 1 & 1 & 5 \\ 1 & 2 & -1 & 1 \\ 0 & 3 & 1 & 5 \end{bmatrix} =
\begin{bmatrix} 1 & 0 & 0 & 1 \\ 0 & 1 & 0 & 1 \\ 0 & 0 & 1 & 2 \\ 0 & 0 & 0 & 0 \end{bmatrix}.\] 
Since the fourth column is not a pivot column, the vectors are linearly dependent.
\end{solution}

\begin{problem}{V6}
Determine if the set \[ \left\{
 \begin{bmatrix} 3 \\ 2 \\ 1 \\ 0 \end{bmatrix} ,
 \begin{bmatrix} -1 \\ 1 \\ 2 \\ 3 \end{bmatrix} ,
 \begin{bmatrix} 0 \\ 1 \\ -1 \\ 1 \end{bmatrix} ,
 \begin{bmatrix} 2 \\ 5 \\ 1 \\ 5 \end{bmatrix} \right\} \]
is a basis of \(\IR^4\) or not.
\end{problem}
\begin{solution}
Compute
\[\RREF \begin{bmatrix} 3 & -1 & 0 & 2 \\ 2 & 1 & 1 & 5 \\ 1 & 2 & -1 & 1 \\ 0 & 3 & 1 & 5 \end{bmatrix} =
\begin{bmatrix} 1 & 0 & 0 & 1 \\ 0 & 1 & 0 & 1 \\ 0 & 0 & 1 & 2 \\ 0 & 0 & 0 & 0 \end{bmatrix}.\]
Since the fourth column is not a pivot column, the vectors are linearly
dependent and thus not a basis.

(Alternate solutions:
Since the fourth row not a pivot row, the vectors do not span
\(\IR^4\) and thus are not a basis. Or since the resulting matrix is not
the identity matrix, the vectors do not form a basis.)
\end{solution}


\begin{problem}{V7}
Find a basis for \(W\), the subspace of \(\IR^4\) given by
\[
  W = \vspan \left\{
  \begin{bmatrix} 1 \\ -3 \\ -1 \\ 2 \end{bmatrix} ,
  \begin{bmatrix} 1 \\ 0 \\ 1 \\ -2 \end{bmatrix} ,
  \begin{bmatrix} 3 \\ -6 \\ -1 \\ 2 \end{bmatrix} ,
  \begin{bmatrix} 1 \\ 6 \\ 1 \\ -1 \end{bmatrix} ,
  \begin{bmatrix} 2 \\ 3 \\ 0 \\ 1 \end{bmatrix}
  \right\} .
\]
\end{problem}
\begin{solution}
Observe that
\[ \RREF
    \begin{bmatrix}
    1 & 1 & 3 & 1 & 2 \\
    -3 & 0 & -6 & 6 & 3 \\
    -1 & 1 & -1 & 1 & 0 \\
    2 & -2 & 2 & -1 & 1
    \end{bmatrix} =
    \begin{bmatrix}
    1 & 0 & 2 & 0 & 1 \\
    0 & 1 & 1 & 0 & 0 \\
    0 & 0 & 0 & 1 & 1 \\
    0 & 0 & 0 & 0 & 0
    \end{bmatrix}
\]
has pivot columns in the first, second, and fourth columns, and therefore
\[\left\{
  \begin{bmatrix} 1 \\ -3 \\ -1 \\ 2 \end{bmatrix} ,
  \begin{bmatrix} 1 \\ 0 \\ 1 \\ -2 \end{bmatrix} ,
  \begin{bmatrix} 1 \\ 6 \\ 1 \\ -1 \end{bmatrix}
  \right\}
\]
is a basis of \(W\).
\end{solution}

\begin{problem}{V8}
Find the dimension of \(W\), the subspace of \(\IR^4\) given by
\[
  W = \vspan \left\{
  \begin{bmatrix} 1 \\ -3 \\ -1 \\ 2 \end{bmatrix} ,
  \begin{bmatrix} 1 \\ 0 \\ 1 \\ -2 \end{bmatrix} ,
  \begin{bmatrix} 3 \\ -6 \\ -1 \\ 2 \end{bmatrix} ,
  \begin{bmatrix} 1 \\ 6 \\ 1 \\ -1 \end{bmatrix} ,
  \begin{bmatrix} 2 \\ 3 \\ 0 \\ 1 \end{bmatrix}
  \right\} .
\]
\end{problem}
\begin{solution}
Observe that
\[ \RREF
    \begin{bmatrix}
    1 & 1 & 3 & 1 & 2 \\
    -3 & 0 & -6 & 6 & 3 \\
    -1 & 1 & -1 & 1 & 0 \\
    2 & -2 & 2 & -1 & 1
    \end{bmatrix} =
    \begin{bmatrix}
    1 & 0 & 2 & 0 & 1 \\
    0 & 1 & 1 & 0 & 0 \\
    0 & 0 & 0 & 1 & 1 \\
    0 & 0 & 0 & 0 & 0
    \end{bmatrix}
\]
has three pivot columns, and therefore \(\dim W = 3\).
\end{solution}

\begin{problem}{V9}
Find a basis for the subspace
\[W=\vspan\setList{3x^3+2x^2+x,-x^3+x^2+2x+3,x^2-x+1,2x^3+5x^2+x+5}\]
of \(\P^3\).
\end{problem}
\begin{solution}
This question is equivalent to finding a basis for the subspace
\[
  W' = \vspan \setList{
  \begin{bmatrix} 3\\2\\1\\0\end{bmatrix},
  \begin{bmatrix} -1\\1\\2\\3\end{bmatrix},
  \begin{bmatrix} 0\\1\\-1\\1\end{bmatrix},
  \begin{bmatrix} 2\\5\\1\\5\end{bmatrix}
  }
\]
of Euclidean vectors.

Compute
\[\RREF \begin{bmatrix} 3 & -1 & 0 & 2 \\ 2 & 1 & 1 & 5 \\ 1 & 2 & -1 & 1 \\ 0 & 3 & 1 & 5 \end{bmatrix} =
\begin{bmatrix} 1 & 0 & 0 & 1 \\ 0 & 1 & 0 & 1 \\ 0 & 0 & 1 & 2 \\ 0 & 0 & 0 & 0 \end{bmatrix}.\]
Since the fourth column is not a pivot column, a basis for \(W'\) is given by
\[
  \setList{
  \begin{bmatrix} 3\\2\\1\\0\end{bmatrix},
  \begin{bmatrix} -1\\1\\2\\3\end{bmatrix},
  \begin{bmatrix} 0\\1\\-1\\1\end{bmatrix},
  }
\]

Thus a basis for \(W\) is given by
\[
  \setList{3x^3+2x^2+x,-x^3+x^2+2x+3,x^2-x+1}
\]
\end{solution}

\begin{problem}{V10}
Find a basis for the solution set of the homogeneous system of equations
\begin{alignat*}{6}
x_1 &\,+\,& x_2 &\,+\,& 3x_3 &\,+\,& x_4 &\,+\,& 2x_5 &=& 0 \\
-3x_1 &\,\,&  &\,-\,& 6x_3 &\,+\,&6 x_4 &\,+\,& 3x_5 &=& 0 \\
-x_1 &\,+\,& x_2 &\,-\,& x_3 &\,+\,& x_4 &\,\,&  &=& 0 \\
2x_1 &\,-\,& 2x_2 &\,+\,& 2x_3 &\,-\,& x_4 &\,+\,& x_5 &=& 0 .
\end{alignat*}
\end{problem}
\begin{solution}
Observe that
\[ \RREF
    \begin{bmatrix}[ccccc|c]
    1 & 1 & 3 & 1 & 2 & 0\\
    -3 & 0 & -6 & 6 & 3 & 0\\
    -1 & 1 & -1 & 1 & 0 & 0\\
    2 & -2 & 2 & -1 & 1& 0
    \end{bmatrix} =
    \begin{bmatrix}[ccccc|c]
    1 & 0 & 2 & 0 & 1 &0\\
    0 & 1 & 1 & 0 & 0 &0\\
    0 & 0 & 0 & 1 & 1 &0\\
    0 & 0 & 0 & 0 & 0&0
    \end{bmatrix}
\]

Letting \(x_3=a\) and \(x_5=b\)
(since those correspond to the non-pivot columns),
this is equivalent to the system

\begin{alignat*}{6}
x_1 &\,\,&  &\,+\,& 2x_3 &\,\,&  &\,+\,& x_5 &=& 0 \\
 &\,\,& x_2 &\,+\,& x_3 &\,\,& &\,\,&  &=& 0 \\
 &\,\,&  &\,\,& x_3 &\,\,&  &\,\,&  &=& a \\
 &\,\,&  &\,\,&  &\,\,& x_4 &\,+\,& x_5 &=& 0 \\
 &\,\,&  &\,\,&  &\,\,&  &\,\,& x_5 &=& b \\
\end{alignat*}

Thus, the solution set is
\[ \setBuilder{\begin{bmatrix} -2a-b \\ -a \\ a \\ -b \\ b \end{bmatrix}}{a,b \in \IR} .\]

Since we can write \[\begin{bmatrix} -2a-b \\ -a \\ a \\ -b \\ b \end{bmatrix} = a \begin{bmatrix} -2 \\ -1 \\ 1 \\ 0 \\ 0 \end{bmatrix} + b \begin{bmatrix} -1 \\ 0 \\ 0 \\ -1 \\ 1 \end{bmatrix}, \]
a basis for the solution space is
\[ \left \{ \begin{bmatrix} -2 \\ -1 \\ 1 \\ 0 \\ 0 \end{bmatrix} , \begin{bmatrix} -1 \\ 0 \\ 0 \\ -1 \\ 1 \end{bmatrix} \right\}.\]
\end{solution}

\begin{problem}{A1}
Consider the following maps of polynomials \(S: \P \rightarrow \P\)
and \(T:\P\rightarrow\P\) defined by
\[S(f(x))= f(x)-3x \text{ and }T(f(x)) = f(x)-3f'(x).\]
Show that one of these maps is a linear transformation, and that the other
map is not.
\end{problem}
\begin{solution}
\(S\) is not a linear transformation because \(S(0)=-3x\not=0\).
(Alternate reason: \(S(x+1)=1-2x\) but \(S(x)+S(1)=1-5x\).)

As for \(T\),
\[
  T(f(x)+g(x))=(f(x)+g(x))-3(f(x)+g(x))'=f(x)-3f'(x)+g(x)-3g'(x)
\]
\[
  T(f(x))+T(g(x))=(f(x)-3f'(x))+(g(x)-3g'(x))=f(x)-3f'(x)+g(x)-3g'(x)
\]
\[
	T(cf(x))=(cf(x))-3(cf(x))'=cf(x)-3cf'(x)
\]
\[
	cT(f(x))=c(f(x)-3f'(x))=cf(x)-3cf'(x)
\]
Since \(T\) preserves both addition and scalar multiplication,
\(T\) is a linear transformation.
\end{solution}

\begin{problem}{A2}
Let $T: \IR^3\rightarrow \IR^4$ be the linear transformation given by $$T\left(\begin{bmatrix} x \\ y \\ z \\  \end{bmatrix} \right) = \begin{bmatrix} -x+y \\ -x+3y-z \\ 7x+y+3z \\ 0 \end{bmatrix}.$$
\begin{enumerate}[(a)]
\item Write the standard matrix for \(T\).
\item Compute \( T\left( \begin{bmatrix} -2 \\ 0 \\ 3 \end{bmatrix} \right) \)
\end{enumerate}
\end{problem}

\begin{solution}
\begin{enumerate}[(a)]
\item Since
\begin{align*}
T\left(\begin{bmatrix} 1 \\ 0 \\ 0 \end{bmatrix}\right) &= \begin{bmatrix} -1 \\ -1 \\ 7 \\0\end{bmatrix} &
T\left(\begin{bmatrix} 0 \\ 1 \\ 0 \end{bmatrix}\right) &= \begin{bmatrix} 1 \\ 3 \\ 1 \\0 \end{bmatrix}  &
T\left(\begin{bmatrix} 0 \\ 0 \\ 1 \end{bmatrix}\right) &= \begin{bmatrix} 0 \\ -1 \\ 3 \\ 0  \end{bmatrix}
\end{align*}
The standard matrix is \( \begin{bmatrix} -1 & 1 & 0 \\ -1 & 3 & -1 \\ 7 & 1 & 3  \\ 0 & 0 & 0 \end{bmatrix} \).
\item \(
T\left(\begin{bmatrix} -2 \\ 0 \\ 3 \\  \end{bmatrix} \right) =
\begin{bmatrix} -(-2)+(0) \\ -(-2)+3(0)-(3) \\ 7(-2)+(0)+3(3) \\ 0\end{bmatrix}
  =
\begin{bmatrix} 2 \\ -1 \\ -5 \\ 0 \end{bmatrix}
\)

Alternatively,
\(
\begin{bmatrix} -1 & 1 & 0 \\ -1 & 3 & -1 \\ 7 & 1 & 3  \\ 0 & 0 & 0 \end{bmatrix}
\begin{bmatrix} -2 \\ 0 \\ 3 \end{bmatrix} =
\begin{bmatrix} -1(-2)+1(0)+0(3) \\ -1(-2)+3(0)-1(3) \\ 7(-2)+1(0)+3(3) \\ 0(-2)+0(0)+0(3)\end{bmatrix}
  =
\begin{bmatrix} 2 \\ -1 \\ -5 \\ 0 \end{bmatrix}
\).
\end{enumerate}
\end{solution}

\begin{problem}{A3}
Let \(T: \IR^4 \rightarrow \IR^3\) be the linear transformation given by
\[
  T\left(\begin{bmatrix}x\\y\\z\\w\end{bmatrix} \right) =
  \begin{bmatrix}
    x+3y+2z-3w \\
    2x+4y+6z-10w \\
    x+6y-z+3w
  \end{bmatrix}
\]
Compute a basis for the kernel and a basis for the image of \(T\).
\end{problem}
\begin{solution}
First, we note the standard matrix
\[A=\begin{bmatrix}1 & 3 & 2 & -3 \\ 2 & 4 & 6 & -10 \\ 1 & 6 & -1 & 3 \end{bmatrix}\]
and compute
\[\RREF\left(A\right) = \begin{bmatrix} 1 & 0 & 5 & -9 \\ 0 & 1 & -1 & 2 \\ 0 & 0 & 0 & 0\end{bmatrix}.\]

The kernel is given by solution set of the corresponding homogeneous system of equations
\[ \ker T = \setBuilder{\begin{bmatrix}-5a+9b \\ a-2b \\ a \\ b \end{bmatrix}}{a,b \in \IR} \]
so a basis for the kernel is
\[\setList{\begin{bmatrix}-5 \\ 1 \\ 1 \\ 0 \end{bmatrix},
\begin{bmatrix}9 \\ -2 \\ 0 \\ 1 \end{bmatrix}}.\]
A basis for the image is given by the pivot columns, namely
\[\setList{\begin{bmatrix}1\\2\\1 \end{bmatrix},
\begin{bmatrix}3\\4\\6 \end{bmatrix}}.\]
\end{solution}

\begin{problem}{A4}
Determine if each of the following linear transformations is injective (one-to-one) and/or surjective (onto).
\begin{enumerate}[(a)]
\item $S: \IR^2 \rightarrow \IR^2$ given by the standard matrix $\begin{bmatrix} 1 & 1 \\ -1 & 0 \end{bmatrix}$.
\item $T: \IR^4 \rightarrow \IR^3$ given by the standard matrix $\begin{bmatrix} 1 & 3 & 2 & -3 \\ 2 & 4 & 6 & -10 \\ 1 & 6 & -1 & 3 \end{bmatrix}$
\end{enumerate}
\end{problem}
\begin{solution}
\begin{enumerate}[(a)]
\item $ \RREF\begin{bmatrix} 1 & 1 \\ -1 & 0 \end{bmatrix}=\begin{bmatrix}1 & 0 \\ 0 & 1 \end{bmatrix}$.  Since each column is a pivot column, $S$ is injective.  Since there is no zero row, $S$ is surjective.
(Alternatively, since the result is the identity matrix, \(S\) is bijective.)
\item $$\RREF\begin{bmatrix}1 & 3 & 2 & -3 \\ 2 & 4 & 6 & -10 \\ 1 & 6 & -1 & 3 \end{bmatrix} = \begin{bmatrix} 1 & 0 & 5 & -9 \\ 0 & 1 & -1 & 2 \\ 0 & 0 & 0 & 0\end{bmatrix}$$
Since there is a row of zeroes, the span of the columns does not equal \(\IR^3\), so \(T\) is not surjective.  Since there are non-pivot columns, \(T\) is not injective either.
(Alternatively, since $\dim \IR^4 > \dim \IR^3$, $T$ is not injective.)
\end{enumerate}
\end{solution}



\begin{problem}{M1}
Let
\begin{align*}
A &= \begin{bmatrix} 1 & -3 \\ 0 & 1 \end{bmatrix} & B&= \begin{bmatrix} 4 & 1 & 2 \end{bmatrix} & C&= \begin{bmatrix} 0 & 1 & 3 \\ 1 & -2 & 5 \end{bmatrix}
\end{align*}

Exactly one of the six products $AB$, $AC$, $BA$, $BC$, $CA$, $CB$ can be computed.  Determine which one, and show how to compute it.
\end{problem}
\begin{solution}
\(AC\) is the only one that can be computed, since \(A\) is \(2\times 2\)
and \(C\) is \(2\times 3\). Thus \(AC\) will be the \(2\times 3\) matrix
given by
\begin{align*}
AC\left( \vec{e}_1 \right) &= A \left( \begin{bmatrix} 0 \\ 1 \end{bmatrix} \right) = 0 \begin{bmatrix} 1 \\ 0 \end{bmatrix} + 1\begin{bmatrix} -3 \\ 1 \end{bmatrix} = \begin{bmatrix} -3 \\ 1 \end{bmatrix} \\
AC\left( \vec{e}_2 \right) &= A \left( \begin{bmatrix} 1 \\ -2 \end{bmatrix} \right) = 1 \begin{bmatrix} 1 \\ 0 \end{bmatrix} -2\begin{bmatrix} -3 \\ 1 \end{bmatrix} = \begin{bmatrix} 7 \\ -2 \end{bmatrix} \\
AC\left( \vec{e}_3 \right) &= A \left( \begin{bmatrix} 1 \\ -2 \end{bmatrix} \right) = 3 \begin{bmatrix} 1 \\ 0 \end{bmatrix} + 5\begin{bmatrix} -3 \\ 1 \end{bmatrix} = \begin{bmatrix} -12 \\ 5 \end{bmatrix} \\
\end{align*}
Thus
$$AC = \begin{bmatrix} -3 & 7 & -12 \\ 1 & -2 & 5 \end{bmatrix}.$$
\end{solution}

\begin{problem}{M2}
Consider the two row operations 
\(R_2-3R_1\to R_2\) and \(2R_2\to R_2\)
applied as follows to show \(A\sim B\):
\begin{align*}
A
  =
\begin{bmatrix}
1&2&3\\
4&5&6\\
7&8&9
\end{bmatrix}
  &\sim
\begin{bmatrix}
1&2&3\\
4-3(1)&5-3(2)&6-3(3)\\
7&8&9
\end{bmatrix}
  =
\begin{bmatrix}
1&2&3\\
1&-1&-3\\
7&8&9
\end{bmatrix}
  \\&\sim
\begin{bmatrix}
1&2&3\\
2(1)&2(-1)&2(-3)\\
7&8&9
\end{bmatrix}
  =
\begin{bmatrix}
1&2&3\\
2&-2&-6\\
7&8&9
\end{bmatrix}
  = 
B
\end{align*}
Express these row operations as matrix multiplication
by expressing \(B\) as the product of two matrices and \(A\).
\end{problem}
\begin{solution}
Each row operation may be applied to the identity matrix \(I\):
\[
I
  =
\begin{bmatrix}
1&0&0\\
0&1&0\\
0&0&1
\end{bmatrix}
\sim
\begin{bmatrix}
1&0&0\\
-3&1&0\\
0&0&1
\end{bmatrix}
\]
\[
I
  =
\begin{bmatrix}
1&0&0\\
0&1&0\\
0&0&1
\end{bmatrix}
\sim
\begin{bmatrix}
1&0&0\\
0&2&0\\
0&0&1
\end{bmatrix}
\]
and then left-multiplied with \(A\) to obtain the solution:
\[
B
  =
\begin{bmatrix}
  1&0&0\\
  0&2&0\\
  0&0&1
\end{bmatrix}
\begin{bmatrix}
  1&0&0\\
  -3&1&0\\
  0&0&1
\end{bmatrix}
A
\]
\end{solution}

\begin{problem}{M3}
Determine if the matrix \(\begin{bmatrix}1 & 3 & 2  \\ 2 & 4 & 6  \\ 1 & 6 & -1 \end{bmatrix}\) is invertible or not.
\end{problem}
\begin{solution}
We compute
\[\RREF\left(\begin{bmatrix}1 & 3 & 2  \\ 2 & 4 & 6  \\ 1 & 6 & -1 \end{bmatrix}\right) = \begin{bmatrix} 1 & 0 & 5 \\ 0 & 1 & -1 \\ 0 & 0 & 0 \end{bmatrix}.\]
Since its \(\RREF\) is not the identity matrix, the matrix is not invertible.
\end{solution}

\begin{problem}{M4}
Show how to compute the inverse of the matrix
\(\begin{bmatrix}
1 & 2 & 3 & 5 \\ 0 & -1 & 4 & -2 \\ 0 & 0 & 1 & 3 \\ 0 & 0 & 0 & 1
\end{bmatrix}\).
\end{problem}
\begin{solution}
\[\RREF\left(\begin{bmatrix}[cccc|cccc] 1 & 2 & 3 & 5 & 1 & 0 & 0 & 0  \\ 0 & -1 & 4 & -2 & 0 & 1 & 0 & 0 \\ 0 & 0 & 1 & 3 & 0 & 0 & 1 & 0 \\ 0 & 0 & 0 & 1 & 0 & 0 & 0 & 1 \end{bmatrix}\right) = \begin{bmatrix}[cccc|cccc] 1 & 0 & 0 &0 & 1 & 2 & -11 & 32 \\ 0 & 1 & 0 & 0 & 0 & -1 & 4 & -14 \\ 0 & 0 & 1 & 0 & 0 & 0 & 1 & -3 \\ 0 & 0 & 0 & 1 & 0 & 0 & 0 & 1 \end{bmatrix}\]
So the inverse is \(\begin{bmatrix}  1 & 2 & -11 & 32 \\ 0 & -1 & 4 & -14 \\  0 & 0 & 1 & -3 \\ 0 & 0 & 0 & 1 \end{bmatrix}\).
\end{solution}




\begin{problem}{G1}
\begin{enumerate}[(a)]
\item Find \(3 \times 3\) matrices \(S\) and \(T\) whose left multiplication represents the row operations \(R_2-4R_1 \to R_2\) and \(R_3 \leftrightarrow R_2\), respectively.
\item If \(A \in M_{3,3}\) is a matrix with \(\det A = 12\), find the determinant of \(STA\). 
\end{enumerate}
\end{problem}
\begin{solution}
\begin{enumerate}
\item 
\(S=\begin{bmatrix} 1 & 0 & 0 \\ -4 & 1 & 0 \\ 0 & 0 & 1 \end{bmatrix}\) and \(T=\begin{bmatrix} 1 & 0 & 0 \\ 0 & 0 & 1 \\ 0 & 1 & 0 \end{bmatrix}\)
\item \(\det (STA) = \det(S) \det(T) \det(A) = (1) (-1) (12) = -12\).
\end{enumerate}
\end{solution}

\begin{problem}{G2}
Find the determinant of the matrix
\[
  A
    =
  \begin{bmatrix}
    1 & 3 & 0 & -1 \\
    1 & 1 & 2 & 4 \\
    1 & 1 & 1 & 3 \\
    -3 & 1 & 2 & -5
  \end{bmatrix}
\]
\end{problem}
\begin{solution}
Here is one possible solution, first applying a single row operation,
 and then performing Laplace/cofactor expansions to reduce the determinant
 to a linear combination of \(2\times 2\) determinants:

 \begin{align*}
\det
   \begin{bmatrix}
     1 & 3 & 0 & -1 \\
     1 & 1 & 2 & 4 \\
     1 & 1 & 1 & 3 \\
     -3 & 1 & 2 & -5
   \end{bmatrix}
 &= \det \begin{bmatrix} 1 & 3 & 0 & -1 \\ 0 & 0 & 1 & 1 \\ 1 & 1 & 1 & 3 \\ -3 & 1 & 2 & -5 \end{bmatrix} =
 (-1) \det \begin{bmatrix} 1 & 3 & -1 \\ 1 & 1 & 3 \\ -3 & 1 & -5 \end{bmatrix} + (1) \det \begin{bmatrix} 1 & 3 & 0 \\ 1 & 1 & 1 \\ -3 & 1 & 2 \end{bmatrix} \\
 &= (-1) \left( (1) \det \begin{bmatrix} 1 & 3 \\ 1 & -5 \end{bmatrix} - (1) \det \begin{bmatrix} 3 & -1 \\ 1 & -5 \end{bmatrix} + (-3) \det \begin{bmatrix} 3 & -1 \\ 1 & 3 \end{bmatrix} \right) + \\
 &\phantom{==} (1) \left( (1) \det \begin{bmatrix} 1 & 1 \\ 1 & 2 \end{bmatrix} - (3) \det \begin{bmatrix} 1 & 1 \\ -3 & 2 \end{bmatrix} \right) \\
 % &= (-1)\left( (1)(-8)-(1)(-14)+(-3)(10) \right) + (1) \left( (1)(1)-(3)(5) \right) \\
 &= (-1) \left( -8+14-30 \right) + (1) \left(1-15 \right) \\
 &=10
 \end{align*}

Here is another possible solution, using row and column operations to first reduce
the determinant to a \(3\times 3\) matrix and then applying a formula:
\begin{align*}
\det
  \begin{bmatrix}
    1 & 3 & 0 & -1 \\
    1 & 1 & 2 & 4 \\
    1 & 1 & 1 & 3 \\
    -3 & 1 & 2 & -5
  \end{bmatrix}
&=
\det \begin{bmatrix} 1 & 3 & 0 & -1 \\ 0 & 0 & 1 & 1 \\ 1 & 1 & 1 & 3 \\ -3 & 1 & 2 & -5 \end{bmatrix} =
\det \begin{bmatrix} 1 & 3 & 0 & -1 \\ 0 & 0 & 1 & 0 \\ 1 & 1 & 1 & 2 \\ -3 & 1 & 2 & -7 \end{bmatrix} \\ &=
-\det \begin{bmatrix} 1 & 3 & 0 & -1 \\ 1 & 1 & 1 & 2 \\ 0 & 0 & 1 & 0 \\ -3 & 1 & 2 & -7 \end{bmatrix} =
-\det \begin{bmatrix} 1 & 3 & -1 \\ 1 & 1 & 2 \\ -3 & 1  & -7 \end{bmatrix} \\  &=
-((-7-18-1)-(3+2-21)) \\
 &=10
 \end{align*}



\end{solution}

\begin{problem}{G3}
Find the eigenvalues of the matrix \(\begin{bmatrix} -2 & -2 \\ 10 & 7 \end{bmatrix} \).
\end{problem}
\begin{solution}
Compute
\[\det(A-\lambda I) = \det \begin{bmatrix} -2 - \lambda & -2 \\ 10 & 7-\lambda \end{bmatrix} = (-2-\lambda)(7-\lambda)+20 = \lambda ^2 -5\lambda +6 = (\lambda -2)(\lambda -3)\]
The eigenvalues are the roots of the characteristic polynomial, $2$ and $3$.
\end{solution}

\begin{problem}{G4}
Find a basis for the eigenspace associated to the eigenvalue $3$ in the matrix \[\begin{bmatrix} -7 & -8 & 2 \\ 8 & 9 & -1 \\ \frac{13}{2} & 5 & 2 \end{bmatrix}.\]
\end{problem}
\begin{solution}
\[\RREF(A-3I) = \RREF \begin{bmatrix} -7-3 & -8 & 2 \\ 8 & 9-3 & -1 \\ \frac{13}{2} & 5 & 2-3 \end{bmatrix} = \RREF \begin{bmatrix} -10 & -8 & 2 \\ 8 & 6 & -1 \\ \frac{13}{2} & 5 & -1 \end{bmatrix} = \begin{bmatrix} 1 & 0 & 1 \\ 0 & 1 & -\frac{3}{2} \\ 0 & 0 & 0 \end{bmatrix}\]
The eigenspace associated to $3$ is the kernel of $A-3I$, namely  \[\setBuilder{\begin{bmatrix} -a \\ \frac{3}{2} a \\ a \end{bmatrix}}{a \in \IR}\]
which has a basis of \(\left\{ \begin{bmatrix} -1 \\ \frac{3}{2} \\ 1 \end{bmatrix} \right\}\).
\end{solution}

\end{document}
