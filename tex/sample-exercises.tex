\documentclass{article}
\usepackage{tbil-la}
\usepackage[top=1in,bottom=1in,right=1in,left=1in]{geometry}

\parindent=0pt
\setlength{\parskip}{1em}

%Problem environment -- takes an argument which is the standard
\newenvironment{problem}[1]
%before
{
  \begin{flushleft}
  %{\bfseries \arabic{problem} .}
  %Problem numbering by standard
  \textbf{#1}.
  \ignorespaces
}
%after
{
  \end{flushleft}
}

\newenvironment{solution}
%before
{
  \ignorespaces
  \textbf{Solution:}
}
%after
{
  \ignorespacesafterend
  \begin{flushright}
  {\bfseries \qed}
  \end{flushright}
}



\begin{document}

\begin{center}
\Large \textbf{Sample Assessment Exercises}
\end{center}

This document contains one exercise and solution for each standard.
The goal is to give you an idea of what the exercises might look like,
and what the expectations for a complete solution are.

\begin{problem}{E1}
Consider the scalar system of equations 
\begin{alignat*}{5}
3x_1 &\,+\,& 2x_2 &\,\,& &\,+\,&x_4 &= 1 \\
-x_1 &\,-\,& 4x_2 &\,+\,&x_3&\,-\,&7x_4 &= 0 \\
&\,\,& x_2 &\,-\,&x_3 &\,\,&  &= -2
\end{alignat*}
\begin{enumerate}[(a)]
\item Rewrite this system as a vector equation.
\item Write an augmented matrix corresponding to this system.
\end{enumerate}
\end{problem}
\begin{solution}
\begin{enumerate}[(a)]
\item
\[ x_1\begin{bmatrix} 3 \\ -1 \\ 0 \end{bmatrix} + x_2 \begin{bmatrix}2 \\ -4 \\ 1 \end{bmatrix}+x_3 \begin{bmatrix} 1 \\ 1 \\ -1 \end{bmatrix} + x_4 \begin{bmatrix} 1 \\ -7 \\ 0 \end{bmatrix} = \begin{bmatrix} 1 \\ 0 \\ -2 \end{bmatrix} \]
\item \[
\begin{bmatrix}[cccc|c]
3 & 2 & 0 & 1 & 1 \\
-1 & -4 & 1 & -7 & 0 \\
0 & 1 & -1 & 0 & -2
\end{bmatrix}
\]
\end{enumerate}
\end{solution}

\begin{problem}{E2}
Put the following matrix in reduced row echelon form.
\[
  \begin{bmatrix}
  0 & 3 & 1 & 2 \\
  1 & 2 & -1 & -3 \\
  2 & 4 & -1 & -1
  \end{bmatrix}
\]
\end{problem}
\begin{solution}
\begin{alignat*}{4}
  \begin{bmatrix}
  0 & 3 & 1 & 2 \\
  1 & 2 & -1 & -3 \\
  2 & 4 & -1 & -1
  \end{bmatrix}
  &\sim&
  \begin{bmatrix}
  \circledNumber{1} & 2 & -1 & -3 \\
  0 & 3 & 1 & 2 \\
  2 & 4 & -1 & -1
  \end{bmatrix} 
  &\hspace{1in} \text{Swap Rows 1 and 2}& \\
  &\sim&
  \begin{bmatrix}
  \circledNumber{1} & 2 & -1 & -3 \\
  0 & 3 & 1 & 2 \\
  0 & 0 & 1 & 5
  \end{bmatrix}
  &\hspace{1in} \text{Add \(-2\) Row 1 to Row 3}& \\
  &\sim&
  \begin{bmatrix}
  \circledNumber{1} & 2 & -1 & -3 \\
  0 & \circledNumber{1} & \frac{1}{3} & \frac{2}{3} \\
  0 & 0 & 1 & 5
  \end{bmatrix} 
  &\hspace{1in} \text{Multiply Row 3 by \(\frac{1}{3}\)}& \\
  &\sim&
  \begin{bmatrix}
  \circledNumber{1} & 0 & -\frac{5}{3} & -\frac{13}{3} \\
  0 & \circledNumber{1} & \frac{1}{3} & \frac{2}{3} \\
  0 & 0 & \circledNumber{1} & 5
  \end{bmatrix}
  &\hspace{1in} \text{Add \(-2\) Row 2 to Row 1}& \\
  &\sim&
  \begin{bmatrix}
  \circledNumber{1} & 0 & -\frac{5}{3} & -\frac{13}{3} \\
  0 & \circledNumber{1} & 0 & -1 \\
  0 & 0 & \circledNumber{1} & 5
  \end{bmatrix}
  &\hspace{1in} \text{Add \(-\frac{1}{3}\) Row 3 to Row 2}& \\
  &\sim&
  \begin{bmatrix}
  \circledNumber{1} & 0 & 0 & 4 \\
  0 & \circledNumber{1} & 0 & -1 \\
  0 & 0 & \circledNumber{1} & 5
  \end{bmatrix}
  &\hspace{1in} \text{Add \(\frac{5}{3}\) Row 3 to Row 1}& \\
\end{alignat*}
\end{solution}

\begin{problem}{E3}
Show how to find the solution set for the following system of linear equations.
\begin{alignat*}{4}
2x&\,+\,&4y&\,+\,&z &= 5 \\
x&\,+\,&2y &\,\,& &= 3
\end{alignat*}
\end{problem}
\begin{solution}
First, note that this system corresponds to the matrix 
\( \left( \begin{bmatrix}[ccc|c] 2 & 4 & 1 & 5 \\ 1 & 2 & 0 & 3 \end{bmatrix} \right)\).
Then we compute (using technology) 
\[
  \RREF \left(
    \begin{bmatrix}[ccc|c] 2 & 4 & 1 & 5 \\ 1 & 2 & 0 & 3 \end{bmatrix}
  \right)
    =
  \begin{bmatrix}[ccc|c] 1 & 2 & 0 & 3 \\ 0 & 0 & 1 & -1\end{bmatrix}.
\]
This corresponds to the system
\begin{alignat*}{2}
x\,+\,2y&\,\,&  &= 3 \\
  &\,\,& z&= -1
\end{alignat*}
Since the $y$-column is a non-pivot column, it is a free variable, so we let $y=a$; then we have
\begin{alignat*}{3}
x&\,+\,&2y&\,\,&  &= 3 \\
 &\,\,&y &\,\,& &=a \\
 &\,\,& &\,\,& z&= -1
\end{alignat*}
and thus
\begin{align*}
x&= 3-2a \\
y&= a \\
z&= -1
\end{align*}
So the solution set is
\[
  \setBuilder{
    \begin{bmatrix} 3-2a \\ a \\ -1 \end{bmatrix}
  }{
    a \in \IR
  }
\]
\end{solution}


\begin{problem}{V1}
Let \(V\) be the set of all pairs of numbers \((x,y)\) of real numbers together with the following operations:
\begin{align*}
(x_1,y_2) \oplus (x_2,y_2) &= (2x_1+2x_2,2y_1+2y_2) \\
c\odot (x,y) &= (cx,c^2y)
\end{align*}
\begin{enumerate}[(a)]
\item Show that scalar multiplication distributes over vector addition:
\[c\odot \left((x_1,y_1) \oplus (x_2,y_2) \right) = c \odot (x_1,y_1) \oplus c \odot (x_2,y_2)\]
\item Explain why \(V\) nonetheless is not a vector space.
\end{enumerate}
\end{problem}
\begin{solution}
\begin{enumerate}[(a)]
\item
We compute both sides:
\begin{align*}
  c \odot \left((x_1,y_1) \oplus (x_2,y_2) \right)
  &= c \odot (2x_1+2x_2,2y_1+2y_2) \\
  &= (c(2x_1+2x_2),c^2(2y_1+2y_2)) \\
  &= (2cx_1+2cx_2,2c^2y_1+2c^2y_2)
\end{align*}
and
\begin{align*}
  c\odot (x_1,y_1) \oplus c \odot (x_2,y_2) 
  &= (cx_1,c^2y_1) \oplus (cx_2,c^2y_2)\\
  &= (2cx_1+2cx_2,2c^2y_1+2c^2y_2)
\end{align*}
Since these are the same, we have shown that the property holds.
\item

To show \(V\) is not a vector space, we must show that it fails one of the 8 defining properties of vector spaces.
We will show that scalar multiplication does not distribute over scalar addition, i.e., there are values
such that
\[ (c+d)\odot(x,y) \neq c \odot(x,y) \oplus d\odot(x,y) \]

\textit{(Solution method 1)}
First, we compute
\begin{align*}
(c+d)\odot(x,y) &= ((c+d)x,(c+d)^2y) \\
&= ( (c+d)x, (c^2+2cd+d^2)y).
\end{align*}
Then we compute
\begin{align*}
c\odot (x,y) \oplus d\odot(x,y) &= (cx,c^2y) \oplus (dx,d^2y) \\
&= ( 2cx+2dx, 2c^2y+2d^2y).
\end{align*}
Since \((c+d)x\not=2cx+2dy\) when \(c,d,x,y=1\), the property fails to hold.

\textit{(Solution method 2)} When we let \(c,d,x,y=1\), we may simplify both sides as follows.

\begin{align*}
(c+d)\odot(x,y) &= 2\odot(1,1) \\
&= (2\cdot1,2^2\cdot1)\\
&=(2,4)
\end{align*}
\begin{align*}
c\odot (x,y) \oplus d\odot(x,y) &= 1\odot(1,1)\oplus 1\odot(1,1) \\
&= (1\cdot1,1^2\cdot1)\oplus(1\cdot1,1^2\cdot1)\\
&= (1,1)\oplus(1,1)\\
&= (2\cdot1+2\cdot1,2\cdot1+2\cdot1)\\
&= (4,4)\\
\end{align*}
Since these ordered pairs are different, the property fails to hold.
\end{enumerate}
\end{solution}


\begin{problem}{V2}
Consider the statement
\begin{center}\begin{minipage}{0.8\textwidth}
The vector \(\begin{bmatrix} 3 \\ -1 \\ 2 \end{bmatrix} \) is a linear combination of the vectors
\(\begin{bmatrix} 1 \\ 0 \\ 1 \end{bmatrix} \),
\(\begin{bmatrix} 3 \\ 2 \\ -1 \end{bmatrix} \), and
\(\begin{bmatrix} 1 \\ 1 \\ -1 \end{bmatrix} \).
\end{minipage}\end{center}
\begin{enumerate}[(a)]
\item Write an equivalent statement using a vector equation.
\item Explain why your statement is true or false.
\end{enumerate}
\end{problem}
\begin{solution}
\begin{enumerate}[(a)]
\item By definition, the statement 
\begin{center}\begin{minipage}{0.8\textwidth}
The vector \(\begin{bmatrix} 3 \\ -1 \\ 2 \end{bmatrix} \) is a linear combination of the vectors
\(\begin{bmatrix} 1 \\ 0 \\ 1 \end{bmatrix} \),
\(\begin{bmatrix} 3 \\ 2 \\ -1 \end{bmatrix} \), and
\(\begin{bmatrix} 1 \\ 1 \\ -1 \end{bmatrix} \).
\end{minipage}\end{center}
is equivalent to the statement
\begin{center}\begin{minipage}{0.8\textwidth}
There exists a solution to the vector equation
\(x_1\begin{bmatrix} 1 \\ 0 \\ 1 \end{bmatrix} +
x_2\begin{bmatrix} 3 \\ 2 \\ -1 \end{bmatrix} +
x_3\begin{bmatrix} 1 \\ 1 \\ -1 \end{bmatrix} =
\begin{bmatrix} 3 \\ -1 \\ 2 \end{bmatrix}.\)
\end{minipage}\end{center}

\item This vector equation corresponds to the augmented matrix 
  \(\begin{bmatrix}[ccc|c]
  1 & 3 & 1 & 3 \\
  0 & 2 & 1 & -1 \\
  1 & -1 & -1 & 2
  \end{bmatrix}\).  Therefore, we compute
\[ \RREF
  \begin{bmatrix}[ccc|c]
  1 & 3 & 1 & 3 \\
  0 & 2 & 1 & -1 \\
  1 & -1 & -1 & 2
  \end{bmatrix}
  =
  \begin{bmatrix}[ccc|c]
  1 & 0 & -\frac{1}{2} & 0 \\
  0 & 1 & \frac{1}{2} & 0 \\
  0 & 0 & 0 & 1
  \end{bmatrix}.
\]
Since this corresponds to an inconsistent system of equations, the vector equation
\[
x_1\begin{bmatrix} 1 \\ 0 \\ 1 \end{bmatrix} +
x_2\begin{bmatrix} 3 \\ 2 \\ -1 \end{bmatrix} +
x_3\begin{bmatrix} 1 \\ 1 \\ -1 \end{bmatrix} =
\begin{bmatrix} 3 \\ -1 \\ 2 \end{bmatrix} \]
has no solution, and therefore 
\(\begin{bmatrix} 3 \\ -1 \\ 2 \end{bmatrix} \) is 
not a linear combination of \(\begin{bmatrix} 1 \\ 0 \\ 1 \end{bmatrix} \),
\(\begin{bmatrix} 3 \\ 2 \\ -1 \end{bmatrix} \), and
\(\begin{bmatrix} 1 \\ 1 \\ -1 \end{bmatrix} \).
\end{enumerate}
\end{solution}

\begin{problem}{V3}
Explain why the vectors
\(\begin{bmatrix} 1 \\ 0 \\ 1 \end{bmatrix} \),
\(\begin{bmatrix} 3 \\ 2 \\ -1 \end{bmatrix} \), and
\(\begin{bmatrix} 1 \\ 1 \\ -1 \end{bmatrix} \) span or don't span \(\IR^3\).
\end{problem}
\begin{solution}
By definition, the statement
\[\text{The vectors }\begin{bmatrix} 1 \\ 0 \\ 1 \end{bmatrix} ,\ 
\begin{bmatrix} 3 \\ 2 \\ -1 \end{bmatrix}, \text{ and }
\begin{bmatrix} 1 \\ 1 \\ -1 \end{bmatrix}  \text{ span } \IR^3\]
is equivalent to the statement
\[\text{For every }\vec{v} \in \IR^3,\text{ the system }
x_1\begin{bmatrix} 1 \\ 0 \\ 1 \end{bmatrix}+  
x_2 \begin{bmatrix} 3 \\ 2 \\ -1 \end{bmatrix}+
x_3\begin{bmatrix} 1 \\ 1 \\ -1 \end{bmatrix}=\vec{v}\text{ has a solution.}\]


We compute
\[ \RREF
  \begin{bmatrix}[ccc]
  1 & 3 & 1  \\
  0 & 2 & 1  \\
  1 & -1 & -1
  \end{bmatrix}
  =
  \begin{bmatrix}[ccc]
  1 & 0 & -\frac{1}{2} \\
  0 & 1 & \frac{1}{2} \\
  0 & 0 & 0
  \end{bmatrix}
\]
Since the last row lacks a pivot, there is some vector \(\vec{v} \in \IR^3\) that upon augmenting this matrix will produce
an inconsistent system. That vector will not be in the span of these three vectors, so the vectors do not span \(\IR^3\).
\end{solution}

\begin{problem}{V4}
Consider the following two sets of Euclidean vectors.
\[
  W = \setBuilder{\begin{bmatrix} x \\ y \\ z \\ w \end{bmatrix} }{x+y=3z+2w}
\hspace{3em}
  U = \setBuilder{\begin{bmatrix} x \\ y \\ z \\ w \end{bmatrix}}{x+y=3z+w^2}
\]
Explain why one of these sets is a subspace of \(\IR^3\), and
why the other is not.
\end{problem}
\begin{solution}

To show that \(W\) is a subspace, let
\(
  \vec v=\begin{bmatrix} x_1 \\y_1 \\ z_1 \\ w_1 \end{bmatrix}\in W
\) and
\(
  \vec w=\begin{bmatrix} x_2 \\y_2 \\ z_2 \\ w_2 \end{bmatrix} \in W
\),
so we know that \(x_1+y_1=3z_1+2w_1\) and \(x_2+y_2=3z_2+2w_2\).
Consider
\[
\begin{bmatrix} x_1 \\y_1 \\ z_1 \\ w_1\end{bmatrix}
+\begin{bmatrix} x_2 \\y_2 \\ z_2 \\ w_2 \end{bmatrix}
=\begin{bmatrix} x_1+x_2 \\y_1+y_2 \\ z_1+z_2 \\w_1+w_2 \end{bmatrix}
.\]
To see if \(\vec{v}+\vec{w} \in W\), we need to check if \( (x_1+x_2)+(y_1+y_2) = 3(z_1+z_2)+2(w_1+w_2)\).
We compute
  \begin{align*}
  (x_1+x_2)+(y_1+y_2) &= (x_1+y_1)+(x_2+y_2) &\text{by regrouping} \\
  &= (3z_1+2w_1)+(3z_2+2w_2) & \text{since \(\vec{v},\vec{w} \in W\)} \\
  &=3(z_1+z_2)+2(w_1+w_2) & \text{by regrouping.}
  \end{align*}
Thus \(\vec v+\vec w\in W\), so \(W\) is closed under vector addition.

Now consider
\[
c\vec v
=\begin{bmatrix} cx_1 \\cy_1 \\ cz_1 \\ cw_1 \end{bmatrix}
.\]
Similarly, to check that \(c\vec{v} \in W\), we need to check if \(cx_1+cy_1=3(cz_1)+2(cw_1)\), so we compute
\begin{align*}
cx_1+cy_1 & = c(x_1+y_1) &\text{by factoring} \\
&=c(3z_1+2w_1) &\text{since \(\vec{v} \in W\)}\\
&=3(cz_1)+2(cw_1) &\text{by regrouping}
\end{align*}
and we see that \(c\vec v\in W\), so \(W\) is closed under scalar 
multiplication. Therefore \(W\) is a subspace of \(\IR^3\).

Now, to show \(U\) is not a subspace, we will show that it is not closed under vector addition.


\textit{(Solution Method 1)} Now let 
\(
  \vec v=\begin{bmatrix} x_1 \\y_1 \\ z_1 \\ w_1 \end{bmatrix}\in U
\) and
\(
  \vec w=\begin{bmatrix} x_2 \\y_2 \\ z_2 \\ w_2 \end{bmatrix} \in U
\),
so we know that \(x_1+y_1=3z_1+w_1^2\) and \(x_2+y_2=3z_2+w_2^2\).
Consider
\[\vec{v}+\vec{w}=
\begin{bmatrix} x_1 \\y_1 \\ z_1 \\ w_1\end{bmatrix}
+\begin{bmatrix} x_2 \\y_2 \\ z_2 \\ w_2 \end{bmatrix}
=\begin{bmatrix} x_1+x_2 \\y_1+y_2 \\ z_1+z_2 \\w_1+w_2 \end{bmatrix}
.\]
To see if \(\vec{v}+\vec{w} \in U\), we need to check if \( (x_1+x_2)+(y_1+y_2) = 3(z_1+z_2)+(w_1+w_2)^2\).
We compute
\begin{align*}
  (x_1+x_2)+(y_1+y_2) &= (x_1+y_1)+(x_2+y_2) &\text{by regrouping} \\
  &= (3z_1+w_1^2)+(3z_2+w_2^2) &\text{since \(\vec{v},\vec{w} \in W\)}\\
  &=3(z_1+z_2)+(w_1^2+w_2^2) &\text{by regrouping}
\end{align*}
and thus \(\vec v+\vec w\in U\) \textbf{only when} \(w_1^2+w_2^2=(w_1+w_2)^2\).
Since this is not true in general, \(U\) is not closed under vector addition, and thus cannot be a subspace.

\textit{(Solution Method 2)}
Note that the vector
\(
  \vec v=\begin{bmatrix} 0\\1\\0\\1\end{bmatrix} 
\)
belongs to \(U\) since \(0+1=3(0)+1^2\).
However, the vector
\(
  2\vec v=\begin{bmatrix} 0\\2\\0\\2\end{bmatrix} 
\)
does not belong to \(U\) since \(0+2\not=3(0)+2^2\).
Therefore \(U\) is not closed under scalar multiplication,
and thus is not a subspace.



\end{solution}


\begin{problem}{V5}
Explain why the set of vectors
\[\left\{
 \begin{bmatrix} 3 \\ 2 \\ 1 \\ 0 \end{bmatrix}  ,
 \begin{bmatrix} -1 \\ 1 \\ 2 \\ 3 \end{bmatrix}  ,
 \begin{bmatrix} 0 \\ 1 \\ -1 \\ 1 \end{bmatrix}  , 
 \begin{bmatrix} 2 \\ 5 \\ 1 \\ 5 \end{bmatrix} \right\}\]
is linearly dependent or linearly independent.
\end{problem}
\begin{solution}
The vectors 
\( \begin{bmatrix} 3 \\ 2 \\ 1 \\ 0 \end{bmatrix} \) ,
\( \begin{bmatrix} -1 \\ 1 \\ 2 \\ 3 \end{bmatrix} \) ,
\( \begin{bmatrix} 0 \\ 1 \\ -1 \\ 1 \end{bmatrix} \) , and
\( \begin{bmatrix} 2 \\ 5 \\ 1 \\ 5 \end{bmatrix} \) are linearly independent precisely when the system of equations 
\[ x_1\begin{bmatrix} 3 \\ 2 \\ 1 \\ 0 \end{bmatrix} +
 x_2\begin{bmatrix} -1 \\ 1 \\ 2 \\ 3 \end{bmatrix} +
 x_3\begin{bmatrix} 0 \\ 1 \\ -1 \\ 1 \end{bmatrix}+ 
 x_4\begin{bmatrix} 2 \\ 5 \\ 1 \\ 5 \end{bmatrix} =\vec{0}\]
has a unique solution (namely, \(\vec{0}\)). 

Converting the left side of this system to the corresponding matrix and row reducing, we have
\[\RREF \left[\begin{array}{cccc} 3 & -1 & 0 & 2  \\ 2 & 1 & 1 & 5  \\ 1 & 2 & -1 & 1  \\ 0 & 3 & 1 & 5 \end{array}\right] =
\left[\begin{array}{cccc} 1 & 0 & 0 & 1  \\ 0 & 1 & 0 & 1  \\ 0 & 0 & 1 & 2  \\ 0 & 0 & 0 & 0  \end{array}\right].\] 
Since the fourth column is not a pivot column, the system has (infinitely many) nontrivial solutions.
Thus the set of vectors is linearly dependent.
\end{solution}

\begin{problem}{V6}
Explain why the set of vectors \[ \left\{
 \begin{bmatrix} 3 \\ 2 \\ 1 \\ 0 \end{bmatrix} ,
 \begin{bmatrix} -1 \\ 1 \\ 2 \\ 3 \end{bmatrix} ,
 \begin{bmatrix} 0 \\ 1 \\ -1 \\ 1 \end{bmatrix} ,
 \begin{bmatrix} 2 \\ 5 \\ 1 \\ 5 \end{bmatrix} \right\} \]
is or is not a basis of \(\IR^4\).
\end{problem}
\begin{solution}
Compute
\[\RREF \begin{bmatrix} 3 & -1 & 0 & 2 \\ 2 & 1 & 1 & 5 \\ 1 & 2 & -1 & 1 \\ 0 & 3 & 1 & 5 \end{bmatrix} =
\begin{bmatrix} 1 & 0 & 0 & 1 \\ 0 & 1 & 0 & 1 \\ 0 & 0 & 1 & 2 \\ 0 & 0 & 0 & 0 \end{bmatrix}.\]
Since the fourth column is not a pivot column, the vectors are linearly
dependent and thus not a basis of \(\IR^4\).

(Alternate solution:)
Since the fourth row is not a pivot row, the vectors do not span
\(\IR^4\) and thus are not a basis of \(\IR^4\).
\end{solution}


\begin{problem}{V7}
Find a basis for the subspace 
\[
  W = \vspan \left\{
  \begin{bmatrix} 1 \\ -3 \\ -1 \\ 2 \end{bmatrix} ,
  \begin{bmatrix} 1 \\ 0 \\ 1 \\ -2 \end{bmatrix} ,
  \begin{bmatrix} 3 \\ -6 \\ -1 \\ 2 \end{bmatrix} ,
  \begin{bmatrix} 1 \\ 6 \\ 1 \\ -1 \end{bmatrix} ,
  \begin{bmatrix} 2 \\ 3 \\ 0 \\ 1 \end{bmatrix}
  \right\} .
\]
Be sure to explain why your subspace is a basis.
\end{problem}
\begin{solution}
Observe that
\[ \RREF
    \begin{bmatrix}
    1 & 1 & 3 & 1 & 2 \\
    -3 & 0 & -6 & 6 & 3 \\
    -1 & 1 & -1 & 1 & 0 \\
    2 & -2 & 2 & -1 & 1
    \end{bmatrix} =
    \begin{bmatrix}
    1 & 0 & 2 & 0 & 1 \\
    0 & 1 & 1 & 0 & 0 \\
    0 & 0 & 0 & 1 & 1 \\
    0 & 0 & 0 & 0 & 0
    \end{bmatrix}
\]
If we remove the vectors yielding non-pivot columns, the resulting
set will span the same vectors while being linearly independent.
Therefore
\[\left\{
  \begin{bmatrix} 1 \\ -3 \\ -1 \\ 2 \end{bmatrix} ,
  \begin{bmatrix} 1 \\ 0 \\ 1 \\ -2 \end{bmatrix} ,
  \begin{bmatrix} 1 \\ 6 \\ 1 \\ -1 \end{bmatrix}
  \right\}
\]
is a basis of \(W\).
\end{solution}

\begin{problem}{V8}
Explain how to find the dimension of 
\[
  W = \vspan \left\{
  \begin{bmatrix} 1 \\ -3 \\ -1 \\ 2 \end{bmatrix} ,
  \begin{bmatrix} 1 \\ 0 \\ 1 \\ -2 \end{bmatrix} ,
  \begin{bmatrix} 3 \\ -6 \\ -1 \\ 2 \end{bmatrix} ,
  \begin{bmatrix} 1 \\ 6 \\ 1 \\ -1 \end{bmatrix} ,
  \begin{bmatrix} 2 \\ 3 \\ 0 \\ 1 \end{bmatrix}
  \right\} .
\]
\end{problem}
\begin{solution}
The dimension of a space is equal to how many vectors belong
to any basis for the space.
So since
\[ \RREF
    \begin{bmatrix}
    1 & 1 & 3 & 1 & 2 \\
    -3 & 0 & -6 & 6 & 3 \\
    -1 & 1 & -1 & 1 & 0 \\
    2 & -2 & 2 & -1 & 1
    \end{bmatrix} =
    \begin{bmatrix}
    1 & 0 & 2 & 0 & 1 \\
    0 & 1 & 1 & 0 & 0 \\
    0 & 0 & 0 & 1 & 1 \\
    0 & 0 & 0 & 0 & 0
    \end{bmatrix}
\]
has three pivot columns, any basis of \(W\) has three elements, and therefore \(\dim W = 3\).
\end{solution}

\begin{problem}{V9}
Find a basis for the subspace of \(\P^3\)
\[W=\vspan\setList{3x^3+2x^2+x,-x^3+x^2+2x+3,x^2-x+1,2x^3+5x^2+x+5}.\]
Be sure to explain why your result is a basis.
\end{problem}
\begin{solution}
This question is equivalent to finding a basis for the subspace
\[
  W' = \vspan \setList{
  \begin{bmatrix} 3\\2\\1\\0\end{bmatrix},
  \begin{bmatrix} -1\\1\\2\\3\end{bmatrix},
  \begin{bmatrix} 0\\1\\-1\\1\end{bmatrix},
  \begin{bmatrix} 2\\5\\1\\5\end{bmatrix}
  }
\]
of Euclidean vectors.

Compute
\[\RREF \begin{bmatrix} 3 & -1 & 0 & 2 \\ 2 & 1 & 1 & 5 \\ 1 & 2 & -1 & 1 \\ 0 & 3 & 1 & 5 \end{bmatrix} =
\begin{bmatrix} 1 & 0 & 0 & 1 \\ 0 & 1 & 0 & 1 \\ 0 & 0 & 1 & 2 \\ 0 & 0 & 0 & 0 \end{bmatrix}.\]
Since the fourth column is not a pivot column, a basis for \(W'\) is given by
\[
  \setList{
  \begin{bmatrix} 3\\2\\1\\0\end{bmatrix},
  \begin{bmatrix} -1\\1\\2\\3\end{bmatrix},
  \begin{bmatrix} 0\\1\\-1\\1\end{bmatrix}
  }
\]

Thus a basis for \(W\) is given by
\[
  \setList{3x^3+2x^2+x,-x^3+x^2+2x+3,x^2-x+1}
\]
\end{solution}

\begin{problem}{V10}
Explain how to find a basis for the solution set of the homogeneous system of equations
\begin{alignat*}{6}
x_1 &\,+\,& x_2 &\,+\,& 3x_3 &\,+\,& x_4 &\,+\,& 2x_5 &=& 0 \\
-3x_1 &\,\,&  &\,-\,& 6x_3 &\,+\,&6 x_4 &\,+\,& 3x_5 &=& 0 \\
-x_1 &\,+\,& x_2 &\,-\,& x_3 &\,+\,& x_4 &\,\,&  &=& 0 \\
2x_1 &\,-\,& 2x_2 &\,+\,& 2x_3 &\,-\,& x_4 &\,+\,& x_5 &=& 0 .
\end{alignat*}
\end{problem}
\begin{solution}
Observe that
\[ \RREF
    \begin{bmatrix}[ccccc|c]
    1 & 1 & 3 & 1 & 2 & 0\\
    -3 & 0 & -6 & 6 & 3 & 0\\
    -1 & 1 & -1 & 1 & 0 & 0\\
    2 & -2 & 2 & -1 & 1& 0
    \end{bmatrix} =
    \begin{bmatrix}[ccccc|c]
    1 & 0 & 2 & 0 & 1 &0\\
    0 & 1 & 1 & 0 & 0 &0\\
    0 & 0 & 0 & 1 & 1 &0\\
    0 & 0 & 0 & 0 & 0&0
    \end{bmatrix}
\]

Letting \(x_3=a\) and \(x_5=b\)
(since those correspond to the non-pivot columns),
this is equivalent to the system

\begin{alignat*}{6}
x_1 &\,\,&  &\,+\,& 2x_3 &\,\,&  &\,+\,& x_5 &=& 0 \\
 &\,\,& x_2 &\,+\,& x_3 &\,\,& &\,\,&  &=& 0 \\
 &\,\,&  &\,\,& x_3 &\,\,&  &\,\,&  &=& a \\
 &\,\,&  &\,\,&  &\,\,& x_4 &\,+\,& x_5 &=& 0 \\
 &\,\,&  &\,\,&  &\,\,&  &\,\,& x_5 &=& b \\
\end{alignat*}

Thus, the solution set is
\[ \setBuilder{\begin{bmatrix} -2a-b \\ -a \\ a \\ -b \\ b \end{bmatrix}}{a,b \in \IR} .\]

Since we can write \[\begin{bmatrix} -2a-b \\ -a \\ a \\ -b \\ b \end{bmatrix} = a \begin{bmatrix} -2 \\ -1 \\ 1 \\ 0 \\ 0 \end{bmatrix} + b \begin{bmatrix} -1 \\ 0 \\ 0 \\ -1 \\ 1 \end{bmatrix}, \]
a basis for the solution space is
\[ \left \{ \begin{bmatrix} -2 \\ -1 \\ 1 \\ 0 \\ 0 \end{bmatrix} , \begin{bmatrix} -1 \\ 0 \\ 0 \\ -1 \\ 1 \end{bmatrix} \right\}.\]
\end{solution}

\begin{problem}{A1}
Consider the following maps of polynomials \(S: \P \rightarrow \P\)
and \(T:\P\rightarrow\P\) defined by
\[S(f(x))= 3xf(x) \text{ and }T(f(x)) = 3f'(x)f(x).\]
Explain why one of these maps is a linear transformation, and why the other
map is not.
\end{problem}
\begin{solution}
To show \(S\) is a linear transformation, we must show two things:
\begin{align*}
&S\left(f(x)+g(x)\right)=S(f(x))+s(g(x)) \\
&S(cf(x)) = cS(f(x))
\end{align*}

To show \(S\) respects addition, we compute
\begin{align*}
S\left(f(x)+g(x)\right) &= 3x\left(f(x)+g(x)\right) & \text{by definition of \(S\)} \\
&= 3xf(x)+3xg(x) & \text{by distributing}
\end{align*}
But note that \(S(f(x))=3xf(x)\) and \(S(g(x))=3xg(x)\), so we have \(S(f(x)+g(x))=S(f(x))+S(g(x))\).

For the second part, we compute
\begin{align*}
S\left(cf(x)\right) &= 3x\left(cf(x)\right) & \text{by definition of \(S\)} \\
&= 3cxf(x) & \text{rewriting the multiplication.}
\end{align*}
But note that \(cS(f(x))=c(3xf(x))=3cxf(x)\) as well, so we have \(S(cf(x))=cS(f(x))\).  Now, since \(S\) respects both addition and scalar multiplication, we can conclude \(S\) is a linear transformation.

\textit{(Solution method 1)}
As for \(T\), we compute
  \begin{align*}
  T(f(x)+g(x))& =3 (f(x)+g(x))'(f(x)+g(x)) &\text{by definition of \(T\)} \\
  &= 3(f'(x)+g'(x))(f(x)+g(x)) & \text{since the derivative is linear} \\
  &= 3f(x)f'(x)+3f(x)g'(x)+3f'(x)g(x)+3g(x)g'(x) &\text{by distributing}
  \end{align*}
However, note that \(T(f(x))+T(g(x))=3f'(x)f(x)+3g'(x)g(x)\), which is not
always the same polynomial (for example, when \(f(x)=g(x)=x\)). 
So we see that \(T(f(x)+g(x)) \neq T(f(x))+T(g(x))\),
so \(T\) does not respect addition and is therefore not a linear transformation.

\textit{(Solution method 2)}
As for \(T\), we may choose the polynomial \(f(x)=x\) and scalar \(c=2\). Then
\[T(cf(x))=T(2x)=3(2x)'(2x)=3(2)(2x)=12x.\]

But on the other hand, \[cT(f(x))=2T(x)=2(3)(x)'(x)=2(3)(1)(x)=6x.\]
Since this isn't the same polynomial, \(T\) does not preserve multiplication
and is therefore not a linear transformation. 
\end{solution}

\begin{problem}{A2}
Let $T: \IR^3\rightarrow \IR^4$ be the linear transformation given by $$T\left(\begin{bmatrix} x \\ y \\ z \\  \end{bmatrix} \right) = \begin{bmatrix} -x+y \\ -x+3y-z \\ 7x+y+3z \\ 0 \end{bmatrix}.$$
\begin{enumerate}[(a)]
\item Explain how to find the standard matrix for \(T\).
\item Explain how to find \( T\left( \begin{bmatrix} -2 \\ 0 \\ 3 \end{bmatrix} \right) \)
\end{enumerate}
\end{problem}

\begin{solution}
\begin{enumerate}[(a)]
\item Since
\begin{align*}
T\left(\begin{bmatrix} 1 \\ 0 \\ 0 \end{bmatrix}\right) &= \begin{bmatrix} -1 \\ -1 \\ 7 \\0\end{bmatrix} &
T\left(\begin{bmatrix} 0 \\ 1 \\ 0 \end{bmatrix}\right) &= \begin{bmatrix} 1 \\ 3 \\ 1 \\0 \end{bmatrix}  &
T\left(\begin{bmatrix} 0 \\ 0 \\ 1 \end{bmatrix}\right) &= \begin{bmatrix} 0 \\ -1 \\ 3 \\ 0  \end{bmatrix}
\end{align*}
The standard matrix is \( \begin{bmatrix} -1 & 1 & 0 \\ -1 & 3 & -1 \\ 7 & 1 & 3  \\ 0 & 0 & 0 \end{bmatrix} \).
\item \(
T\left(\begin{bmatrix} -2 \\ 0 \\ 3 \\  \end{bmatrix} \right) =
\begin{bmatrix} -(-2)+(0) \\ -(-2)+3(0)-(3) \\ 7(-2)+(0)+3(3) \\ 0\end{bmatrix}
  =
\begin{bmatrix} 2 \\ -1 \\ -5 \\ 0 \end{bmatrix}
\)

Alternatively,
\(
\begin{bmatrix} -1 & 1 & 0 \\ -1 & 3 & -1 \\ 7 & 1 & 3  \\ 0 & 0 & 0 \end{bmatrix}
\begin{bmatrix} -2 \\ 0 \\ 3 \end{bmatrix} =
\begin{bmatrix} -1(-2)+1(0)+0(3) \\ -1(-2)+3(0)-1(3) \\ 7(-2)+1(0)+3(3) \\ 0(-2)+0(0)+0(3)\end{bmatrix}
  =
\begin{bmatrix} 2 \\ -1 \\ -5 \\ 0 \end{bmatrix}
\).
\end{enumerate}
\end{solution}

\begin{problem}{A3}
Let \(T: \IR^4 \rightarrow \IR^3\) be the linear transformation given by
\[
  T\left(\begin{bmatrix}x\\y\\z\\w\end{bmatrix} \right) =
  \begin{bmatrix}
    x+3y+2z-3w \\
    2x+4y+6z-10w \\
    x+6y-z+3w
  \end{bmatrix}
\]
Show how to compute a basis for the image and a basis for the kernel of \(T\).
\end{problem}
\begin{solution}
First, we note the standard matrix
\[A=\begin{bmatrix}1 & 3 & 2 & -3 \\ 2 & 4 & 6 & -10 \\ 1 & 6 & -1 & 3 \end{bmatrix}\]
and compute
\[\RREF\left(A\right) = \begin{bmatrix} 1 & 0 & 5 & -9 \\ 0 & 1 & -1 & 2 \\ 0 & 0 & 0 & 0\end{bmatrix}.\]

A basis for the image is given by the pivot columns, namely
\[\setList{\begin{bmatrix}1\\2\\1 \end{bmatrix},
\begin{bmatrix}3\\4\\6 \end{bmatrix}}.\]

The kernel is the solution set of the corresponding homogeneous system of equations
\[ \ker T = \setBuilder{\begin{bmatrix}-5a+9b \\ a-2b \\ a \\ b \end{bmatrix}}{a,b \in \IR} \]
so a basis for the kernel is
\[\setList{\begin{bmatrix}-5 \\ 1 \\ 1 \\ 0 \end{bmatrix},
\begin{bmatrix}9 \\ -2 \\ 0 \\ 1 \end{bmatrix}}.\]
\end{solution}

\begin{problem}{A4}
Let $T: \IR^4 \rightarrow \IR^3$ be the linear transformation given by the standard matrix $\begin{bmatrix} 1 & 3 & 2 & -3 \\ 2 & 4 & 6 & -10 \\ 1 & 6 & -1 & 3 \end{bmatrix}$.
\begin{enumerate}[(a)]
\item Explain why \(T\) is or is not injective.
\item Explain why \(T\) is or is not surjective.
\end{enumerate}
\end{problem}
\begin{solution}
Compute \[\RREF\begin{bmatrix}1 & 3 & 2 & -3 \\ 2 & 4 & 6 & -10 \\ 1 & 6 & -1 & 3 \end{bmatrix} = \begin{bmatrix} 1 & 0 & 5 & -9 \\ 0 & 1 & -1 & 2 \\ 0 & 0 & 0 & 0\end{bmatrix}.\]
\begin{enumerate}[(a)]
\item Note that the third and fourth columns are non-pivot columns, which means \(\ker T\) contains infinitely many vectors, so \(T\) is not injective.
\item Since the third row lacks a pivot, the image (i.e. the span of the columns) is a 2-dimensional subspace (and thus does not equal \(\IR^3\)), so \(T\) is not surjective.
\end{enumerate}
\end{solution}



\begin{problem}{M1}
Of the following three matrices, only two may be multiplied.
\begin{align*}
A &= \begin{bmatrix} 1 & -3 \\ 0 & 1 \end{bmatrix} & B&= \begin{bmatrix} 4 & 1 & 2 \end{bmatrix} & C&= \begin{bmatrix} 0 & 1 & 3 \\ 1 & -2 & 5 \end{bmatrix}
\end{align*}
Explain which two may be multiplied and why. Then show how to find their product.
\end{problem}
\begin{solution}
\(AC\) is the only one that can be computed, since \(A\) is \(2\times 2\)
and \(C\) is \(2\times 3\). Thus \(AC\) will be the \(2\times 3\) matrix
given by
\begin{align*}
AC\left( \vec{e}_1 \right) &= A \left( \begin{bmatrix} 0 \\ 1 \end{bmatrix} \right) = 0 \begin{bmatrix} 1 \\ 0 \end{bmatrix} + 1\begin{bmatrix} -3 \\ 1 \end{bmatrix} = \begin{bmatrix} -3 \\ 1 \end{bmatrix} \\
AC\left( \vec{e}_2 \right) &= A \left( \begin{bmatrix} 1 \\ -2 \end{bmatrix} \right) = 1 \begin{bmatrix} 1 \\ 0 \end{bmatrix} -2\begin{bmatrix} -3 \\ 1 \end{bmatrix} = \begin{bmatrix} 7 \\ -2 \end{bmatrix} \\
AC\left( \vec{e}_3 \right) &= A \left( \begin{bmatrix} 3 \\ 5 \end{bmatrix} \right) = 3 \begin{bmatrix} 1 \\ 0 \end{bmatrix} + 5\begin{bmatrix} -3 \\ 1 \end{bmatrix} = \begin{bmatrix} -12 \\ 5 \end{bmatrix} \\
\end{align*}
Thus
$$AC = \begin{bmatrix} -3 & 7 & -12 \\ 1 & -2 & 5 \end{bmatrix}.$$
\end{solution}

\begin{problem}{M2}
Explain why the matrix \(\begin{bmatrix}1 & 3 & 2  \\ 2 & 4 & 6  \\ 1 & 6 & -1 \end{bmatrix}\) is or is not invertible.
\end{problem}
\begin{solution}
We compute
\[\RREF\left(\begin{bmatrix}1 & 3 & 2  \\ 2 & 4 & 6  \\ 1 & 6 & -1 \end{bmatrix}\right) = \begin{bmatrix} 1 & 0 & 5 \\ 0 & 1 & -1 \\ 0 & 0 & 0 \end{bmatrix}.\]
Since its \(\RREF\) is not the identity matrix, the linear map is not bijective and thus the matrix is not invertible.
\end{solution}

\begin{problem}{M3}
Show how to compute the inverse of the matrix
\(A=\begin{bmatrix}
1 & 2 & 3 & 5 \\ 0 & -1 & 4 & -2 \\ 0 & 0 & 1 & 3 \\ 0 & 0 & 0 & 1
\end{bmatrix}\).
\end{problem}
\begin{solution}
To find the matrix \(A^{-1}\) where \(AA^{-1}=I\), we need to solve the augmented matrix \([A|I]\).
\[\RREF\left(\begin{bmatrix}[cccc|cccc] 1 & 2 & 3 & 5 & 1 & 0 & 0 & 0  \\ 0 & -1 & 4 & -2 & 0 & 1 & 0 & 0 \\ 0 & 0 & 1 & 3 & 0 & 0 & 1 & 0 \\ 0 & 0 & 0 & 1 & 0 & 0 & 0 & 1 \end{bmatrix}\right) = \begin{bmatrix}[cccc|cccc] 1 & 0 & 0 &0 & 1 & 2 & -11 & 32 \\ 0 & 1 & 0 & 0 & 0 & -1 & 4 & -14 \\ 0 & 0 & 1 & 0 & 0 & 0 & 1 & -3 \\ 0 & 0 & 0 & 1 & 0 & 0 & 0 & 1 \end{bmatrix}\]
So the inverse is \(\begin{bmatrix}  1 & 2 & -11 & 32 \\ 0 & -1 & 4 & -14 \\  0 & 0 & 1 & -3 \\ 0 & 0 & 0 & 1 \end{bmatrix}\).
\end{solution}




\begin{problem}{G1}
\begin{enumerate}[(a)]
\item Give a \(3\times 3\) matrix \(P\) that may be used to perform the row
      operation \({R_3} \to R_3+4 \, {R_1} \).
\item Give a \(3\times 3\) matrix \(Q\) that may be used to perform the row
      operation \({R_1} \to -4 \, {R_1}\).
\item Suppose \(A\) is a \(3\times 3\) matrix with determinant \(-5\).
      Use matrix multiplication to describe the matrix obtained by applying 
        \({R_3} \to 4 \, {R_1} + {R_3}\) and then \({R_1} \to -4 \, {R_1}\)
      to \(A\) (note the order).
\item Finally, explain how to find the determinant of the matrix described in (c).

\end{enumerate}
\end{problem}
\begin{solution}
\begin{enumerate}
\item 
\(P=\begin{bmatrix} 1 & 0 & 0 \\ 0 & 1 & 0 \\ 4 & 0 & 1 \end{bmatrix}\)
\item
\(Q=\begin{bmatrix} -4 & 0 & 0 \\ 0 & 1 & 0 \\ 0 & 0 & 1 \end{bmatrix}\)
\item \( QPA\)
\item \(\det (QPA) = \det(Q) \det(P) \det(A) = (-4) (1) (-5) = 20\).
\end{enumerate}
\end{solution}

\begin{problem}{G2}
Show how to compute the determinant of the matrix
\[
  A
    =
  \begin{bmatrix}
    1 & 3 & 0 & -1 \\
    1 & 1 & 2 & 4 \\
    1 & 1 & 1 & 3 \\
    -3 & 1 & 2 & -5
  \end{bmatrix}
\]
\end{problem}
\begin{solution}
Here is one possible solution, first applying a single row operation,
 and then performing Laplace/cofactor expansions to reduce the determinant
 to a linear combination of \(2\times 2\) determinants:

 \begin{align*}
\det
   \begin{bmatrix}
     1 & 3 & 0 & -1 \\
     1 & 1 & 2 & 4 \\
     1 & 1 & 1 & 3 \\
     -3 & 1 & 2 & -5
   \end{bmatrix}
 &= \det \begin{bmatrix} 1 & 3 & 0 & -1 \\ 0 & 0 & 1 & 1 \\ 1 & 1 & 1 & 3 \\ -3 & 1 & 2 & -5 \end{bmatrix} =
 (-1) \det \begin{bmatrix} 1 & 3 & -1 \\ 1 & 1 & 3 \\ -3 & 1 & -5 \end{bmatrix} + (1) \det \begin{bmatrix} 1 & 3 & 0 \\ 1 & 1 & 1 \\ -3 & 1 & 2 \end{bmatrix} \\
 &= (-1) \left( (1) \det \begin{bmatrix} 1 & 3 \\ 1 & -5 \end{bmatrix} - (1) \det \begin{bmatrix} 3 & -1 \\ 1 & -5 \end{bmatrix} + (-3) \det \begin{bmatrix} 3 & -1 \\ 1 & 3 \end{bmatrix} \right) + \\
 &\phantom{==} (1) \left( (1) \det \begin{bmatrix} 1 & 1 \\ 1 & 2 \end{bmatrix} - (3) \det \begin{bmatrix} 1 & 1 \\ -3 & 2 \end{bmatrix} \right) \\
 % &= (-1)\left( (1)(-8)-(1)(-14)+(-3)(10) \right) + (1) \left( (1)(1)-(3)(5) \right) \\
 &= (-1) \left( -8+14-30 \right) + (1) \left(1-15 \right) \\
 &=10
 \end{align*}

Here is another possible solution, using row and column operations to first reduce
the determinant to a \(3\times 3\) matrix and then applying a formula:
\begin{align*}
\det
  \begin{bmatrix}
    1 & 3 & 0 & -1 \\
    1 & 1 & 2 & 4 \\
    1 & 1 & 1 & 3 \\
    -3 & 1 & 2 & -5
  \end{bmatrix}
&=
\det \begin{bmatrix} 1 & 3 & 0 & -1 \\ 0 & 0 & 1 & 1 \\ 1 & 1 & 1 & 3 \\ -3 & 1 & 2 & -5 \end{bmatrix} =
\det \begin{bmatrix} 1 & 3 & 0 & -1 \\ 0 & 0 & 1 & 0 \\ 1 & 1 & 1 & 2 \\ -3 & 1 & 2 & -7 \end{bmatrix} \\ &=
-\det \begin{bmatrix} 1 & 3 & 0 & -1 \\ 1 & 1 & 1 & 2 \\ 0 & 0 & 1 & 0 \\ -3 & 1 & 2 & -7 \end{bmatrix} =
-\det \begin{bmatrix} 1 & 3 & -1 \\ 1 & 1 & 2 \\ -3 & 1  & -7 \end{bmatrix} \\  &=
-((-7-18-1)-(3+2-21)) \\
 &=10
 \end{align*}



\end{solution}

\begin{problem}{G3}
Explain how to find the eigenvalues of the matrix \(\begin{bmatrix} -2 & -2 \\ 10 & 7 \end{bmatrix} \).
\end{problem}
\begin{solution}
Compute the characteristic polynomial: 
\[\det(A-\lambda I) = \det \begin{bmatrix} -2 - \lambda & -2 \\ 10 & 7-\lambda \end{bmatrix} = (-2-\lambda)(7-\lambda)+20 = \lambda ^2 -5\lambda +6 = (\lambda -2)(\lambda -3)\]
The eigenvalues are the roots of the characteristic polynomial, namely $2$ and $3$.
\end{solution}

\begin{problem}{G4}
Explain how to find a basis for the eigenspace associated to the eigenvalue $3$ in the matrix \[\begin{bmatrix} -7 & -8 & 2 \\ 8 & 9 & -1 \\ \frac{13}{2} & 5 & 2 \end{bmatrix}.\]
\end{problem}
\begin{solution}
The eigenspace associated to $3$ is the kernel of $A-3I$, so we compute
\[\RREF(A-3I) = \RREF \begin{bmatrix} -7-3 & -8 & 2 \\ 8 & 9-3 & -1 \\ \frac{13}{2} & 5 & 2-3 \end{bmatrix} = \RREF \begin{bmatrix} -10 & -8 & 2 \\ 8 & 6 & -1 \\ \frac{13}{2} & 5 & -1 \end{bmatrix} = \begin{bmatrix} 1 & 0 & 1 \\ 0 & 1 & -\frac{3}{2} \\ 0 & 0 & 0 \end{bmatrix}.\]
Thus we see the kernel is \[\setBuilder{\begin{bmatrix} -a \\ \frac{3}{2} a \\ a \end{bmatrix}}{a \in \IR}\]
which has a basis of \(\left\{ \begin{bmatrix} -1 \\ \frac{3}{2} \\ 1 \end{bmatrix} \right\}\).
\end{solution}

\end{document}
