\documentclass{article}
\usepackage{tbil-la}
\usepackage[top=1in,bottom=1in,right=1in,left=1in]{geometry}

%Problem environment -- takes an argument which is the standard
\newenvironment{problem}[1]
%before
{
	\begin{flushleft}
	%{\bfseries \arabic{problem} .}
	%Problem numbering by standard
	\textbf{#1}.
	\ignorespaces
}
%after
{
	\end{flushleft}
}

\newenvironment{solution}
%before
{
	\ignorespaces
	\textbf{Solution:}
}
%after
{
	\ignorespacesafterend
	\begin{flushright}
	{\bfseries \qed}
	\end{flushright}
}



\begin{document}

\begin{center}
\Large \textbf{Sample Assessment Exercises}
\end{center}

This document contains one exercise and solution for each standard.
The goal is to give you an idea of what the exercises might look like,
and what the expectations for a complete solution are.

\begin{problem}{E1}
Write a system of linear equations corresponding to the following
augmented matrix.
\[
\begin{bmatrix}[cccc|c]
3 & 2 & 0 & 1 & 1 \\
-1 & -4 & 1 & -7 & 0 \\
0 & 1 & -1 & 0 & -2
\end{bmatrix}
\]
\end{problem}
\begin{solution}
\begin{alignat*}{5}
3x_1 &\,+\,& 2x_2 &\,\,& &\,+\,&x_4 &= 1 \\
-x_1 &\,-\,& 4x_2 &\,+\,&x_3&\,-\,&7x_4 &= 0 \\
&\,\,& x_2 &\,-\,&x_3 &\,\,&  &= -2
\end{alignat*}
\end{solution}

\begin{problem}{E2}
Put the following matrix in reduced row echelon form.
\[
	\begin{bmatrix}
	0 & 3 & 1 & 2 \\
	1 & 2 & -1 & -3 \\
	2 & 4 & -1 & -1
	\end{bmatrix}
\]
\end{problem}
\begin{solution}
\begin{alignat*}{4}
	\begin{bmatrix}
	0 & 3 & 1 & 2 \\
	1 & 2 & -1 & -3 \\
	2 & 4 & -1 & -1
	\end{bmatrix}
	&\sim&
	\begin{bmatrix}
	\circledNumber{1} & 2 & -1 & -3 \\
	0 & 3 & 1 & 2 \\
	2 & 4 & -1 & -1
	\end{bmatrix}
	&\sim&
	\begin{bmatrix}
	\circledNumber{1} & 2 & -1 & -3 \\
	0 & 3 & 1 & 2 \\
	0 & 0 & 1 & 5
	\end{bmatrix}
	&\sim&
	\begin{bmatrix}
	\circledNumber{1} & 2 & -1 & -3 \\
	0 & \circledNumber{1} & \frac{1}{3} & \frac{2}{3} \\
	0 & 0 & 1 & 5
	\end{bmatrix} \\
	&\sim&
	\begin{bmatrix}
	\circledNumber{1} & 0 & -\frac{5}{3} & -\frac{13}{3} \\
	0 & \circledNumber{1} & \frac{1}{3} & \frac{2}{3} \\
	0 & 0 & \circledNumber{1} & 5
	\end{bmatrix}
	&\sim&
	\begin{bmatrix}
	\circledNumber{1} & 0 & -\frac{5}{3} & -\frac{13}{3} \\
	0 & \circledNumber{1} & 0 & -1 \\
	0 & 0 & \circledNumber{1} & 5
	\end{bmatrix}
	&\sim&
	\begin{bmatrix}
	\circledNumber{1} & 0 & 0 & 4 \\
	0 & \circledNumber{1} & 0 & -1 \\
	0 & 0 & \circledNumber{1} & 5
	\end{bmatrix}
\end{alignat*}
\end{solution}

\begin{problem}{E3}
Find the solution set for the following system of linear equations.
\begin{alignat*}{4}
2x&\,+\,&4y&\,+\,&z &= 5 \\
x&\,+\,&2y &\,\,& &= 3
\end{alignat*}
\end{problem}
\begin{solution}
\[
  \RREF \left(
		\begin{bmatrix}[ccc|c] 2 & 4 & 1 & 5 \\ 1 & 2 & 0 & 3 \end{bmatrix}
	\right)
		=
  \begin{bmatrix}[ccc|c] 1 & 2 & 0 & 3 \\ 0 & 0 & 1 & -1\end{bmatrix}
\]
This corresponds to the system
\begin{alignat*}{2}
x\,+\,2y&\,\,&  &= 3 \\
  &\,\,& z&= -1
\end{alignat*}
Since the $y$-column is a non-pivot column, it is a free variable, so we let $y=a$; then we have
\begin{alignat*}{3}
x&\,+\,&2y&\,\,&  &= 3 \\
 &\,\,&y &\,\,& &=a \\
 &\,\,& &\,\,& z&= -1
\end{alignat*}
and thus
\begin{align*}
x&= 3-2a \\
y&= a \\
z&= -1
\end{align*}
So the solution set is
\[
  \setBuilder{
		\begin{bmatrix} 3-2a \\ a \\ -1 \end{bmatrix}
	}{
		a \in \IR
	}
\]
\end{solution}


\begin{problem}{V1}
Let \(V\) be the set of all polynomials, together with the operations \(\oplus\) and \(\odot\) defined by, for all polynomials \(f(x),g(x)\) and scalars \(c \in \IR\):
\begin{align*}
f(x) \oplus g(x) &= xf(x)+xg(x) \\
c\odot f(x) &= cf(x)
\end{align*}
\begin{enumerate}[(a)]
\item Show that scalar multiplication distributes over vector addition, i.e. \[c\odot \left(f(x)\oplus g(x) \right) = c \odot f(x) \oplus c \odot g(x).\]
\item Show that addition is not associative, i.e. for all polynomials $f(x),g(x),h(x)$, \[\left( f(x) \oplus g(x) \right) \oplus h(x) \neq f(x) \oplus \left( g(x) \oplus h(x) \right) .\]
\end{enumerate}
\end{problem}
\begin{solution}
\begin{enumerate}[(a)]
\item
Compute
\begin{align*}
	c \odot \left(f(x) \oplus g(x) \right) 
	&= c \odot \left(xf(x)+xg(x)\right) \\
	&= c\left(xf(x)+xg(x) \right) \\
	&= cxf(x)+cxg(x) 
\end{align*}
and
\begin{align*}
	c\odot f(x) \oplus c \odot g(x) 
	&= (cf(x)) \oplus (cg(x))\\
	&= xcf(x)+xcg(x)
\end{align*}
Since these are the same, we have shown \(c\odot \left(f(x)\oplus g(x) \right) = c \odot f(x) \oplus c \odot g(x).\)
\item
Compute
\begin{align*}
	\left( f(x) \oplus g(x) \right) \oplus h(x) 
	&=\left(xf(x)+xg(x) \right) \oplus h(x)\\
	&= x\left(xf(x)+xg(x)\right)+xh(x) \\
	&= x^2f(x)+x^2g(x)+xh(x)
\end{align*}
and 
\begin{align*}
	f(x) \oplus \left( g(x) \oplus h(x) \right) 
	&=f(x) \oplus \left(xg(x)+xh(x)\right) \\
	&=xf(x)+x\left(xg(x)+xh(x)\right) \\
	&=xf(x)+x^2g(x)+x^2h(x)
\end{align*}


Since $x^2f(x)+x^2g(x)+xh(x)\neq xf(x)+x^2g(x)+x^2h(x)$, we have shown \(\left( f(x) \oplus g(x) \right) \oplus h(x) \neq f(x) \oplus \left( g(x) \oplus h(x) \right) .\)
\end{enumerate}
\end{solution}

\begin{problem}
Let \(V\) be the set of all non-negative real numbers with the operations \(\oplus\) and \(\odot\) given by, for all $x,y \in V$ and $c \in \IR$,
\begin{align*}
x \oplus y &= x+y \\
c \odot x &= |c|x
\end{align*}

\begin{enumerate}[(a)]
\item List the 8 defining properties of a vector space.
\item Determine which of the 8 hold for $V$ with these operations, and conclude whether $V$ is a vector space or not.
\end{enumerate}
\end{problem}
\begin{solution}
\begin{enumerate}[(a)]
\item The eight properties are
	\begin{enumerate}[1)]
		\item Addition is associative, i.e. \((x\oplus y)\oplus z = x\oplus (y \oplus z)\) for all \(x,y,z \in V\).
		\item Addition is commutative, i.e. \(x\oplus y = y \oplus x\) for all \(x,y \in V\).
		\item There exists a zero element, i.e. an element \(0 \in V\) such that for all \(x \in V\), \(x\oplus 0 = x\).
		\item Additive inverses exist, i.e. for every \(x \in V\) there is an element \(-x \in V\) such that \(x \oplus (-x) = 0 \)
		\item Scalar multiplication is associative, i.e. for each \(c,d \in \IR\) and \(x \in V\), \(c\odot (d\odot x) = (cd) \odot x\).
		\item \(1\) is the multiplicative identity, i.e. for all \(x \in V\), \(1 \odot x = x \).
		\item Scalar multiplication distributes over vector addition, i.e. for all \(x,y \in V\) and \(c \in \IR\), \(c\odot(x\oplus y) = c\odot x \oplus c \odot y\)
		\item Scalar addition distributes over scalar multiplication, i.e. for all \(x \in V\) and \(c,d \in \IR\), \( (c+d)\odot x = c\odot x \oplus d \odot x \)
	\end{enumerate}
\item \(V\) is not a vector space, as only six properties hold: it does not have additive inverses, and scalar addition does not distribute over scalar multiplication.
\end{enumerate}
\end{solution}

\begin{problem}{V3}
Determine if \(\begin{bmatrix} 3 \\ -1 \\ 2 \end{bmatrix} \) is a linear combination of the vectors 
\(\begin{bmatrix} 1 \\ 0 \\ 1 \end{bmatrix} \),
\(\begin{bmatrix} 3 \\ 2 \\ -1 \end{bmatrix} \), and
\(\begin{bmatrix} 1 \\ 1 \\ -1 \end{bmatrix} \).
\end{problem}
\begin{solution}

We compute 
\[ \RREF
	\begin{bmatrix}[ccc|c] 
	1 & 3 & 1 & 3 \\ 
	0 & 2 & 1 & -1 \\ 
	1 & -1 & -1 & 2 
	\end{bmatrix}
	=
	\begin{bmatrix}[ccc|c]
	1 & 0 & -\frac{1}{2} & 0 \\
	0 & 1 & \frac{1}{2} & 0 \\
	0 & 0 & 0 & 1
	\end{bmatrix}
\]
Since this corresponds to an inconsistent system of equations, \(\begin{bmatrix} 3 \\ -1 \\ 2 \end{bmatrix} \) is \textbf{not} a linear combination of \(\begin{bmatrix} 1 \\ 0 \\ 1 \end{bmatrix} \),
\(\begin{bmatrix} 3 \\ 2 \\ -1 \end{bmatrix} \), and
\(\begin{bmatrix} 1 \\ 1 \\ -1 \end{bmatrix} \).

\end{solution}

\begin{problem}{V4}
Determine if  the vectors 
\(\begin{bmatrix} 1 \\ 0 \\ 1 \end{bmatrix} \),
\(\begin{bmatrix} 3 \\ 2 \\ -1 \end{bmatrix} \), and
\(\begin{bmatrix} 1 \\ 1 \\ -1 \end{bmatrix} \) span \(\IR^3\).
\end{problem}
\begin{solution}

We compute 
\[ \RREF
	\begin{bmatrix}[ccc] 
	1 & 3 & 1  \\ 
	0 & 2 & 1  \\ 
	1 & -1 & -1 
	\end{bmatrix}
	=
	\begin{bmatrix}[ccc]
	1 & 0 & -\frac{1}{2} \\
	0 & 1 & \frac{1}{2} \\
	0 & 0 & 0
	\end{bmatrix}
\]
Since the last row lacks a pivot, the vectors \textbf{do not span} \(\IR^3\).
\end{solution}

\begin{problem}{V5}
Determine if the set \[W = \setBuilder{\begin{bmatrix} x \\ y \\ z \end{bmatrix}}{ x+y=3z+2}\]
is a subspace of \(\IR^3\).
\end{problem}
\begin{solution}
Let \(\begin{bmatrix} x_1 \\y_1 \\ z_1 \end{bmatrix}, \begin{bmatrix} x_2 \\y_2 \\ z_2 \end{bmatrix} \in W\) (so we know \(x_1+y_1=3z_1+2\) and \(x_2+y_2=3z_2+2\).  We compute 
\[  
\begin{bmatrix} x_1 \\y_1 \\ z_1 \end{bmatrix} 
+\begin{bmatrix} x_2 \\y_2 \\ z_2 \end{bmatrix}
=\begin{bmatrix} x_1+x_2 \\y_1+y_2 \\ z_1+z_2 \end{bmatrix}
\]

However, \[ (x_1+x_2)+(y_1+y_2) = (x_1+y_1)+(x_2+y_2) = (3z_1+2)+(3z_2+2)=3z_1+3z_2+4 \neq 3(z_1+z_2)+2.\]
Thus, \(W\) is not closed under addition, so it is not a subspace.
\end{solution}


\end{document}
