\documentclass{article}
\usepackage{tbil-la}
\usepackage[top=1in,bottom=1in,right=1in,left=1in]{geometry}

\parindent=0pt

%Problem environment -- takes an argument which is the standard
\newenvironment{problem}[1]
%before
{
  \begin{flushleft}
  %{\bfseries \arabic{problem} .}
  %Problem numbering by standard
  \textbf{#1}.
  \ignorespaces
}
%after
{
  \end{flushleft}
}

\newenvironment{solution}
%before
{
  \ignorespaces
  \textbf{Solution:}
}
%after
{
  \ignorespacesafterend
  \begin{flushright}
  {\bfseries \qed}
  \end{flushright}
}



\begin{document}

\begin{center}
\Large \textbf{Sample Assessment Exercises}
\end{center}

This document contains one exercise and solution for each standard.
The goal is to give you an idea of what the exercises might look like,
and what the expectations for a complete solution are.

\begin{problem}{E1}
Write a system of linear equations corresponding to the following
augmented matrix.
\[
\begin{bmatrix}[cccc|c]
3 & 2 & 0 & 1 & 1 \\
-1 & -4 & 1 & -7 & 0 \\
0 & 1 & -1 & 0 & -2
\end{bmatrix}
\]
\end{problem}
\begin{solution}
\begin{alignat*}{5}
3x_1 &\,+\,& 2x_2 &\,\,& &\,+\,&x_4 &= 1 \\
-x_1 &\,-\,& 4x_2 &\,+\,&x_3&\,-\,&7x_4 &= 0 \\
&\,\,& x_2 &\,-\,&x_3 &\,\,&  &= -2
\end{alignat*}
\end{solution}

\begin{problem}{E2}
Put the following matrix in reduced row echelon form.
\[
  \begin{bmatrix}
  0 & 3 & 1 & 2 \\
  1 & 2 & -1 & -3 \\
  2 & 4 & -1 & -1
  \end{bmatrix}
\]
\end{problem}
\begin{solution}
\begin{alignat*}{4}
  \begin{bmatrix}
  0 & 3 & 1 & 2 \\
  1 & 2 & -1 & -3 \\
  2 & 4 & -1 & -1
  \end{bmatrix}
  &\sim&
  \begin{bmatrix}
  \circledNumber{1} & 2 & -1 & -3 \\
  0 & 3 & 1 & 2 \\
  2 & 4 & -1 & -1
  \end{bmatrix}
  &\sim&
  \begin{bmatrix}
  \circledNumber{1} & 2 & -1 & -3 \\
  0 & 3 & 1 & 2 \\
  0 & 0 & 1 & 5
  \end{bmatrix}
  &\sim&
  \begin{bmatrix}
  \circledNumber{1} & 2 & -1 & -3 \\
  0 & \circledNumber{1} & \frac{1}{3} & \frac{2}{3} \\
  0 & 0 & 1 & 5
  \end{bmatrix} \\
  &\sim&
  \begin{bmatrix}
  \circledNumber{1} & 0 & -\frac{5}{3} & -\frac{13}{3} \\
  0 & \circledNumber{1} & \frac{1}{3} & \frac{2}{3} \\
  0 & 0 & \circledNumber{1} & 5
  \end{bmatrix}
  &\sim&
  \begin{bmatrix}
  \circledNumber{1} & 0 & -\frac{5}{3} & -\frac{13}{3} \\
  0 & \circledNumber{1} & 0 & -1 \\
  0 & 0 & \circledNumber{1} & 5
  \end{bmatrix}
  &\sim&
  \begin{bmatrix}
  \circledNumber{1} & 0 & 0 & 4 \\
  0 & \circledNumber{1} & 0 & -1 \\
  0 & 0 & \circledNumber{1} & 5
  \end{bmatrix}
\end{alignat*}
\end{solution}

\begin{problem}{E3}
Find the solution set for the following system of linear equations.
\begin{alignat*}{4}
2x&\,+\,&4y&\,+\,&z &= 5 \\
x&\,+\,&2y &\,\,& &= 3
\end{alignat*}
\end{problem}
\begin{solution}
\[
  \RREF \left(
    \begin{bmatrix}[ccc|c] 2 & 4 & 1 & 5 \\ 1 & 2 & 0 & 3 \end{bmatrix}
  \right)
    =
  \begin{bmatrix}[ccc|c] 1 & 2 & 0 & 3 \\ 0 & 0 & 1 & -1\end{bmatrix}
\]
This corresponds to the system
\begin{alignat*}{2}
x\,+\,2y&\,\,&  &= 3 \\
  &\,\,& z&= -1
\end{alignat*}
Since the $y$-column is a non-pivot column, it is a free variable, so we let $y=a$; then we have
\begin{alignat*}{3}
x&\,+\,&2y&\,\,&  &= 3 \\
 &\,\,&y &\,\,& &=a \\
 &\,\,& &\,\,& z&= -1
\end{alignat*}
and thus
\begin{align*}
x&= 3-2a \\
y&= a \\
z&= -1
\end{align*}
So the solution set is
\[
  \setBuilder{
    \begin{bmatrix} 3-2a \\ a \\ -1 \end{bmatrix}
  }{
    a \in \IR
  }
\]
\end{solution}


\begin{problem}{V1}
Let \(V\) be the set of all polynomials, together with the operations
\(\oplus\) and \(\odot\) defined by the following for all polynomials
\(f(x),g(x)\) and scalars \(c \in \IR\):
\begin{align*}
f(x) \oplus g(x) &= xf(x)+xg(x) \\
c\odot f(x) &= cf(x)
\end{align*}
\begin{enumerate}[(a)]
\item Show that scalar distribution
\[c\odot \left(f(x)\oplus g(x) \right) = c \odot f(x) \oplus c \odot g(x)\]
holds.
\item Show that addition associativity
\[
  \left( f(x) \oplus g(x) \right) \oplus h(x)
=
  f(x) \oplus \left( g(x) \oplus h(x) \right)
\]
fails.
\end{enumerate}
\end{problem}
\begin{solution}
\begin{enumerate}[(a)]
\item
Compute
\begin{align*}
  c \odot \left(f(x) \oplus g(x) \right)
  &= c \odot \left(xf(x)+xg(x)\right) \\
  &= c\left(xf(x)+xg(x) \right) \\
  &= cxf(x)+cxg(x)
\end{align*}
and
\begin{align*}
  c\odot f(x) \oplus c \odot g(x)
  &= (cf(x)) \oplus (cg(x))\\
  &= xcf(x)+xcg(x)
\end{align*}
Since these are the same, we have shown that
\(c\odot \left(f(x)\oplus g(x) \right) = c \odot f(x) \oplus c \odot g(x)\)
holds.
\item
Suppose \(f(x)=1\), \(g(x)=2\), and \(h(x)=3\). Then
\begin{align*}
  \left( f(x) \oplus g(x) \right) \oplus h(x)
  &=\left(x+2x \right) \oplus 3\\
  &=3x \oplus 3\\
  &= 3x^2+3x
\end{align*}
and
\begin{align*}
  f(x) \oplus \left( g(x) \oplus h(x) \right)
  &=1 \oplus \left(2x+3x\right) \\
  &=1 \oplus 5x \\
  &=x + 5x^2
\end{align*}


Since \(3x^2+3x\not=x+5x^2\), we have shown
\(
  \left( f(x) \oplus g(x) \right) \oplus h(x)
=
  f(x) \oplus \left( g(x) \oplus h(x) \right)
\)
fails.
\end{enumerate}
\end{solution}

\begin{problem}{V2}
  Let \(V\) be the set of all non-negative real numbers with the operations
  \(\oplus\) and \(\odot\) given by, for all \(x,y \in V\) and \(c \in \IR\),
  \begin{align*}
  x \oplus y &= x+y \\
  c \odot x &= |c|x
  \end{align*}

  List the 8 defining properties of a vector space, and label each as
  ``TRUE'' or ``FALSE'' as they apply to \(V\). Based on these, conclude whether
  \(V\) is a vector space or not.
  \end{problem}
\begin{solution}
  \begin{enumerate}[1)]
    \item Addition associativity:
          \((x\oplus y)\oplus z = x\oplus (y \oplus z)\)
          for all \(x,y,z \in V\).
          \textbf{TRUE}
    \item Addition commutivity:
          \(x\oplus y = y \oplus x\)
          for all \(x,y \in V\).
          \textbf{TRUE}
    \item Addition identity:
           there exists an element \(z \in V\) such that for all \(x \in V\),
          \(x\oplus z = x\).
          \textbf{TRUE}
    \item Addition inverses:
          for every \(x \in V\) there is an element \(-x \in V\) such that
          \(x \oplus (-x) = z \).
          \textbf{FALSE}
    \item Scalar multiplication associativity:
          for each \(c,d \in \IR\) and \(x \in V\),
          \(c\odot (d\odot x) = (cd) \odot x\).
          \textbf{TRUE}
    \item Scalar multiplication identity:
          for all \(x \in V\), \(1 \odot x = x \).
          \textbf{TRUE}
    \item Scalar distribution:
          for all \(x,y \in V\) and \(c \in \IR\),
          \(c\odot(x\oplus y) = c\odot x \oplus c \odot y\).
          \textbf{TRUE}
    \item Vector distribution:
          for all \(x \in V\) and \(c,d \in \IR\),
          \( (c+d)\odot x = c\odot x \oplus d \odot x \)
          \textbf{FALSE}
  \end{enumerate}
Since at least one property fails, \(V\) is not a vector space.
\end{solution}

\begin{problem}{V3}
Determine if \(\begin{bmatrix} 3 \\ -1 \\ 2 \end{bmatrix} \) is a linear combination of the vectors
\(\begin{bmatrix} 1 \\ 0 \\ 1 \end{bmatrix} \),
\(\begin{bmatrix} 3 \\ 2 \\ -1 \end{bmatrix} \), and
\(\begin{bmatrix} 1 \\ 1 \\ -1 \end{bmatrix} \).
\end{problem}
\begin{solution}

We compute
\[ \RREF
  \begin{bmatrix}[ccc|c]
  1 & 3 & 1 & 3 \\
  0 & 2 & 1 & -1 \\
  1 & -1 & -1 & 2
  \end{bmatrix}
  =
  \begin{bmatrix}[ccc|c]
  1 & 0 & -\frac{1}{2} & 0 \\
  0 & 1 & \frac{1}{2} & 0 \\
  0 & 0 & 0 & 1
  \end{bmatrix}
\]
Since this corresponds to an inconsistent system of equations, \(\begin{bmatrix} 3 \\ -1 \\ 2 \end{bmatrix} \) is \textbf{not} a linear combination of \(\begin{bmatrix} 1 \\ 0 \\ 1 \end{bmatrix} \),
\(\begin{bmatrix} 3 \\ 2 \\ -1 \end{bmatrix} \), and
\(\begin{bmatrix} 1 \\ 1 \\ -1 \end{bmatrix} \).

\end{solution}

\begin{problem}{V4}
Determine if  the vectors
\(\begin{bmatrix} 1 \\ 0 \\ 1 \end{bmatrix} \),
\(\begin{bmatrix} 3 \\ 2 \\ -1 \end{bmatrix} \), and
\(\begin{bmatrix} 1 \\ 1 \\ -1 \end{bmatrix} \) span \(\IR^3\).
\end{problem}
\begin{solution}

We compute
\[ \RREF
  \begin{bmatrix}[ccc]
  1 & 3 & 1  \\
  0 & 2 & 1  \\
  1 & -1 & -1
  \end{bmatrix}
  =
  \begin{bmatrix}[ccc]
  1 & 0 & -\frac{1}{2} \\
  0 & 1 & \frac{1}{2} \\
  0 & 0 & 0
  \end{bmatrix}
\]
Since the last row lacks a pivot, the vectors \textbf{do not span} \(\IR^3\).
\end{solution}

\begin{problem}{V5}
Consider the following two sets of Euclidean vectors.
\[
  W = \setBuilder{\begin{bmatrix} x \\ y \\ z \end{bmatrix}}{x+y=3z}
\hspace{3em}
  U = \setBuilder{\begin{bmatrix} x \\ y \\ z \end{bmatrix}}{x+y=3z+2}
\]
Show that one of these sets is a subspace of \(\IR^3\), and
that one of the sets is not.
\end{problem}
\begin{solution}
Let
\(
  \begin{bmatrix} x_1 \\y_1 \\ z_1 \end{bmatrix},
  \begin{bmatrix} x_2 \\y_2 \\ z_2 \end{bmatrix} \in W
\),
so we know \(x_1+y_1=3z_1\) and \(x_2+y_2=3z_2\).
Consider
\[
\begin{bmatrix} x_1 \\y_1 \\ z_1 \end{bmatrix}
+\begin{bmatrix} x_2 \\y_2 \\ z_2 \end{bmatrix}
=\begin{bmatrix} x_1+x_2 \\y_1+y_2 \\ z_1+z_2 \end{bmatrix}
.\]
Since
\[
  (x_1+x_2)+(y_1+y_2) = (x_1+y_1)+(x_2+y_2) = 3z_1+3z_2=3(z_1+z_2)
\]
we see that \(W\) is closed under vector addition.
Now consider
\[
c\begin{bmatrix} x_1 \\y_1 \\ z_1 \end{bmatrix}
=\begin{bmatrix} cx_1 \\cy_1 \\ cz_1 \end{bmatrix}
.\]
Since
\[cx_1+cx_2 = c(x_1+x_2)=c(3z_1)=3(cz_1)\]
we see that \(W\) is closed under scalar multiplication. Therefore \(W\)
is a subspace of \(\IR^3\).

However, note that
\(
  \begin{bmatrix} 0 \\ 5 \\ 1 \end{bmatrix},
  \begin{bmatrix} 1 \\ 4 \\ 1 \end{bmatrix}
\)
are vectors in \(U\) since \(0+5=3(1)+2\) and \(1+4=3(1)+2\).
But
\[
  \begin{bmatrix} 0 \\ 5 \\ 1 \end{bmatrix}
    +
  \begin{bmatrix} 1 \\ 4 \\ 1 \end{bmatrix}
    =
  \begin{bmatrix} 1 \\ 9 \\ 2 \end{bmatrix}
\]
does not belong to \(U\) since \(1+9\not=3(2)+2\).
Since \(U\) is not closed under vector addition, \(U\) is not a subspace
of \(\IR^3\).
\end{solution}


\begin{problem}{S1}
Determine if the vectors
\( \begin{bmatrix} 3 \\ 2 \\ 1 \\ 0 \end{bmatrix} \) ,
\( \begin{bmatrix} -1 \\ 1 \\ 2 \\ 3 \end{bmatrix} \) ,
\( \begin{bmatrix} 0 \\ 1 \\ -1 \\ 1 \end{bmatrix} \) , and
\( \begin{bmatrix} 2 \\ 5 \\ 1 \\ 5 \end{bmatrix} \)
are linearly dependent or linearly independent.
\end{problem}
\begin{solution}
Compute
\[\RREF \begin{bmatrix} 3 & -1 & 0 & 2 \\ 2 & 1 & 1 & 5 \\ 1 & 2 & -1 & 1 \\ 0 & 3 & 1 & 5 \end{bmatrix} =
\begin{bmatrix} 1 & 0 & 0 & 1 \\ 0 & 1 & 0 & 1 \\ 0 & 0 & 1 & 2 \\ 0 & 0 & 0 & 0 \end{bmatrix}.\]
Since the fourth column is not a pivot column, the vectors are linearly dependent.
\end{solution}

\begin{problem}{S2}
Determine if the set \[ \left\{
 \begin{bmatrix} 3 \\ 2 \\ 1 \\ 0 \end{bmatrix} ,
 \begin{bmatrix} -1 \\ 1 \\ 2 \\ 3 \end{bmatrix} ,
 \begin{bmatrix} 0 \\ 1 \\ -1 \\ 1 \end{bmatrix} ,
 \begin{bmatrix} 2 \\ 5 \\ 1 \\ 5 \end{bmatrix} \right\} \]
is a basis of $\IR^4$ or not.
\end{problem}
\begin{solution}
Compute
\[\RREF \begin{bmatrix} 3 & -1 & 0 & 2 \\ 2 & 1 & 1 & 5 \\ 1 & 2 & -1 & 1 \\ 0 & 3 & 1 & 5 \end{bmatrix} =
\begin{bmatrix} 1 & 0 & 0 & 1 \\ 0 & 1 & 0 & 1 \\ 0 & 0 & 1 & 2 \\ 0 & 0 & 0 & 0 \end{bmatrix}.\]
Since the fourth column is not a pivot column, the vectors are linearly
dependent and thus not a basis.

(Alternate solution:
since the fourth row not a pivot row, the vectors do not span
\(\IR^4\) and thus are not a basis.)
\end{solution}


\begin{problem}{S3}
Find a basis for $W$, the subspace of $\IR^4$ given by
\[
  W = \vspan \left\{
  \begin{bmatrix} 1 \\ -3 \\ -1 \\ 2 \end{bmatrix} ,
  \begin{bmatrix} 1 \\ 0 \\ 1 \\ -2 \end{bmatrix} ,
  \begin{bmatrix} 3 \\ -6 \\ -1 \\ 2 \end{bmatrix} ,
  \begin{bmatrix} 1 \\ 6 \\ 1 \\ -1 \end{bmatrix} ,
  \begin{bmatrix} 2 \\ 3 \\ 0 \\ 1 \end{bmatrix}
  \right\} .
\]
\end{problem}
\begin{solution}
Observe that
\[ \RREF
    \begin{bmatrix}
    1 & 1 & 3 & 1 & 2 \\
    -3 & 0 & -6 & 6 & 3 \\
    -1 & 1 & -1 & 1 & 0 \\
    2 & -2 & 2 & -1 & 1
    \end{bmatrix} =
    \begin{bmatrix}
    1 & 0 & 2 & 0 & 1 \\
    0 & 1 & 1 & 0 & 0 \\
    0 & 0 & 0 & 1 & 1 \\
    0 & 0 & 0 & 0 & 0
    \end{bmatrix}
\]
has pivot columns in the first, second, and fourth columns, and therefore
\[\left\{
  \begin{bmatrix} 1 \\ -3 \\ -1 \\ 2 \end{bmatrix} ,
  \begin{bmatrix} 1 \\ 0 \\ 1 \\ -2 \end{bmatrix} ,
  \begin{bmatrix} 1 \\ 6 \\ 1 \\ -1 \end{bmatrix}
  \right\}
\]
is a basis of $W$.
\end{solution}

\begin{problem}{S4}
Find the dimension of $W$, the subspace of $\IR^4$ given by
\[
  W = \vspan \left\{
  \begin{bmatrix} 1 \\ -3 \\ -1 \\ 2 \end{bmatrix} ,
  \begin{bmatrix} 1 \\ 0 \\ 1 \\ -2 \end{bmatrix} ,
  \begin{bmatrix} 3 \\ -6 \\ -1 \\ 2 \end{bmatrix} ,
  \begin{bmatrix} 1 \\ 6 \\ 1 \\ -1 \end{bmatrix} ,
  \begin{bmatrix} 2 \\ 3 \\ 0 \\ 1 \end{bmatrix}
  \right\} .
\]
\end{problem}
\begin{solution}
Observe that
\[ \RREF
    \begin{bmatrix}
    1 & 1 & 3 & 1 & 2 \\
    -3 & 0 & -6 & 6 & 3 \\
    -1 & 1 & -1 & 1 & 0 \\
    2 & -2 & 2 & -1 & 1
    \end{bmatrix} =
    \begin{bmatrix}
    1 & 0 & 2 & 0 & 1 \\
    0 & 1 & 1 & 0 & 0 \\
    0 & 0 & 0 & 1 & 1 \\
    0 & 0 & 0 & 0 & 0
    \end{bmatrix}
\]
has three pivot columns, and therefore \(\dim W = 3\).
\end{solution}

\begin{problem}{S5}
Determine if the polynomials \(3x^3+2x^2+x\), \(-x^3+x^2+2x+3\), \(x^2-x+1\), and \(2x^3+5x^2+x+5\)
are linearly dependent or linearly independent.
\end{problem}
\begin{solution}
This question is equivalent to asking if the Euclidean vectors
\[
  \begin{bmatrix} 3\\2\\1\\0\end{bmatrix},
  \begin{bmatrix} -1\\1\\2\\3\end{bmatrix},
  \begin{bmatrix} 0\\1\\-1\\1\end{bmatrix},
  \begin{bmatrix} 2\\5\\1\\5\end{bmatrix}
\]
are linearly dependent or linearly independent.

Compute
\[\RREF \begin{bmatrix} 3 & -1 & 0 & 2 \\ 2 & 1 & 1 & 5 \\ 1 & 2 & -1 & 1 \\ 0 & 3 & 1 & 5 \end{bmatrix} =
\begin{bmatrix} 1 & 0 & 0 & 1 \\ 0 & 1 & 0 & 1 \\ 0 & 0 & 1 & 2 \\ 0 & 0 & 0 & 0 \end{bmatrix}.\]
Since the fourth column is not a pivot column, the Euclidean vectors
(and therefore the polynomials) are linearly dependent.
\end{solution}

\begin{problem}{S6}
Find a basis for the solution set of the homogeneous system of equations
\begin{alignat*}{6}
x_1 &\,+\,& x_2 &\,+\,& 3x_3 &\,+\,& x_4 &\,+\,& 2x_5 &=& 0 \\
-3x_1 &\,\,&  &\,-\,& 6x_3 &\,+\,&6 x_4 &\,+\,& 3x_5 &=& 0 \\
-x_1 &\,+\,& x_2 &\,-\,& x_3 &\,+\,& x_4 &\,\,&  &=& 0 \\
2x_1 &\,-\,& 2x_2 &\,+\,& 2x_3 &\,-\,& x_4 &\,+\,& x_5 &=& 0 .
\end{alignat*}
\end{problem}
\begin{solution}
Observe that
\[ \RREF
    \begin{bmatrix}[ccccc|c]
    1 & 1 & 3 & 1 & 2 & 0\\
    -3 & 0 & -6 & 6 & 3 & 0\\
    -1 & 1 & -1 & 1 & 0 & 0\\
    2 & -2 & 2 & -1 & 1& 0
    \end{bmatrix} =
    \begin{bmatrix}[ccccc|c]
    1 & 0 & 2 & 0 & 1 &0\\
    0 & 1 & 1 & 0 & 0 &0\\
    0 & 0 & 0 & 1 & 1 &0\\
    0 & 0 & 0 & 0 & 0&0
    \end{bmatrix}
\]

Letting \(x_3=a\) and \(x_5=b\)
(since those correspond to the non-pivot columns),
this is equivalent to the system

\begin{alignat*}{6}
x_1 &\,\,&  &\,+\,& 2x_3 &\,\,&  &\,+\,& x_5 &=& 0 \\
 &\,\,& x_2 &\,+\,& x_3 &\,\,& &\,\,&  &=& 0 \\
 &\,\,&  &\,\,& x_3 &\,\,&  &\,\,&  &=& a \\
 &\,\,&  &\,\,&  &\,\,& x_4 &\,+\,& x_5 &=& 0 \\
 &\,\,&  &\,\,&  &\,\,&  &\,\,& x_5 &=& b \\
\end{alignat*}

Thus, the solution set is
\[ \setBuilder{\begin{bmatrix} -2a-b \\ -a \\ a \\ -b \\ b \end{bmatrix}}{a,b \in \IR} .\]

Since we can write \[\begin{bmatrix} -2a-b \\ -a \\ a \\ -b \\ b \end{bmatrix} = a \begin{bmatrix} -2 \\ -1 \\ 1 \\ 0 \\ 0 \end{bmatrix} + b \begin{bmatrix} -1 \\ 0 \\ 0 \\ -1 \\ 1 \end{bmatrix}, \]
a basis for the solution space is
\[ \left \{ \begin{bmatrix} -2 \\ -1 \\ 1 \\ 0 \\ 0 \end{bmatrix} , \begin{bmatrix} -1 \\ 0 \\ 0 \\ -1 \\ 1 \end{bmatrix} \right\}.\]
\end{solution}



\end{document}
