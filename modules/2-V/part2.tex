%!TEX root =../../course-notes.tex
% ^ leave for LaTeXTools build functionality

\begin{applicationActivities}{2}{8}

\begin{remark}
  The following sets are examples of vector spaces, with the usual/natural
  operations for addition and scalar multiplication.
  \begin{itemize}
    \item \(\IR^n\): Euclidean vectors with \(n\) components.
    \item \(\IR^\infty\): Sequences of real numbers \((v_1,v_2,\dots)\).
    \item \(\IR^{m\times n}\): Matrices of real numbers with \(m\) rows and
          \(n\) columns.
    \item \(\IC\): Complex numbers.
    \item \(\P^n\): Polynomials of degree \(n\) or less.
    \item \(\P\): Polynomials of any degree.
    \item \(C(\IR)\): Real-valued continuous functions.
  \end{itemize}
\end{remark}

\begin{activity}{10}
  Let \(V=\{(a,b):a,b\text{ are real numbers}\}\), where
  \((a_1,b_1)\oplus(a_2,b_2)=
  (a_1+b_1+a_2+b_2,b_1^2+b_2^2)\) and \(c\odot(a,b)=(a^c,b+c)\). Show that
  this is not a vector space by finding a counterexample
  that does not satisfy one of the vector space properties.

  \vectorSpacePropertiesO
\end{activity}

\begin{definition}
  A \term{linear combination} of a set of vectors
  \(\{\vect v_1,\vect v_2,\dots,\vect v_m\}\) is given by
  \(c_1\vect v_1+c_2\vect v_2+\dots+c_m\vect v_m\) for any choice of
  scalar multiples \(c_1,c_2,\dots,c_m\).

	\ \\
	\ \\

  For example, we say $\begin{bmatrix}3 \\0 \\ 5\end{bmatrix}$ is a linear combination of the vectors $\begin{bmatrix} 1 \\ -1 \\ 2 \end{bmatrix}$ and $\begin{bmatrix} 1 \\ 2 \\ 1 \end{bmatrix}$ since $$\begin{bmatrix} 3 \\ 0 \\ 5 \end{bmatrix} = 2 \begin{bmatrix} 1 \\ -1 \\ 2 \end{bmatrix} + 1\begin{bmatrix} 1 \\ 2 \\ 1 \end{bmatrix}$$
\end{definition}

\begin{definition}
  The \term{span} of a set of vectors is the collection of all linear
  combinations of that set:
  \[
    \vspan\{\vect v_1,\vect v_2,\dots,\vect v_m\} =
    \{c_1\vect v_1+c_2\vect v_2+\dots+c_m\vect v_m :
    c_i\text{ is a real number}\}
  \]
\end{definition}

\begin{activity}{10}
  Consider \(\vspan\left\{\begin{bmatrix}1\\2\end{bmatrix}\right\}\).
  \begin{subactivity}
    Sketch
    \(c\begin{bmatrix}1\\2\end{bmatrix}\) in the \(xy\) plane
    for \(c=1,3,0,-2\).
  \end{subactivity}
  \begin{subactivity}
    Sketch a representation of all the vectors given by
    \(\vspan\left\{\begin{bmatrix}1\\2\end{bmatrix}\right\}\)
    in the \(xy\) plane.
  \end{subactivity}
\end{activity}

\begin{activity}{10}
  Consider \(\vspan\left\{\begin{bmatrix}1\\2\end{bmatrix},
  \begin{bmatrix}-1\\1\end{bmatrix}\right\}\).
  \begin{subactivity}
    Sketch the following linear combinations in the \(xy\) plane:
    \(1\begin{bmatrix}1\\2\end{bmatrix}+
    0\begin{bmatrix}-1\\1\end{bmatrix}\),
    \(0\begin{bmatrix}1\\2\end{bmatrix}+
    1\begin{bmatrix}-1\\1\end{bmatrix}\),
    \(2\begin{bmatrix}1\\2\end{bmatrix}+
    0\begin{bmatrix}-1\\1\end{bmatrix}\),
    \(2\begin{bmatrix}1\\2\end{bmatrix}+
    1\begin{bmatrix}-1\\1\end{bmatrix}\).
  \end{subactivity}
  \begin{subactivity}
    Sketch a representation of all the vectors given by
    \(\vspan\left\{\begin{bmatrix}1\\2\end{bmatrix},
     \begin{bmatrix}-1\\1\end{bmatrix}\right\}\)
    in the \(xy\) plane.
  \end{subactivity}
\end{activity}

\begin{activity}{5}
    Sketch a representation of all the vectors given by
    \(\vspan\left\{\begin{bmatrix}6\\-4\end{bmatrix},
     \begin{bmatrix}-2\\3\end{bmatrix}\right\}\)
    in the \(xy\) plane.
\end{activity}

% \begin{activity}{10} % Motivate with vectors first.
%   Consider the following linear system.
%     \begin{alignat*}{3}
%        x_1 &\,-\,&  x_2 &\,=\,& -1 \\
%            &\,-\,& 3x_2 &\,=\,& -6 \\
%      -3x_1 &\,+\,& 2x_2 &\,=\,&  1
%     \end{alignat*}
%   \begin{subactivity}
%     Solve this system by using a calculator to find
%     \[\RREF
%       \begin{bmatrix}[cc|c]
%         1 & -1 & -1 \\
%         0 & -3 & -6 \\
%         -3 & 2 & 1
%       \end{bmatrix}
%     \]
%   \end{subactivity}
%   \begin{subactivity}
%     Given this solution, does
%     \(\begin{bmatrix}-1\\-6\\1\end{bmatrix}\) belong to
%     \(\vspan\left\{\begin{bmatrix}1\\0\\-3\end{bmatrix},
%     \begin{bmatrix}-1\\-3\\2\end{bmatrix}\right\}\)?
%   \end{subactivity}
% \end{activity}

\begin{activity}{15}
  The vector
  \(\begin{bmatrix}-1\\-6\\1\end{bmatrix}\) belongs to
  \(\vspan\left\{\begin{bmatrix}1\\0\\-3\end{bmatrix},
  \begin{bmatrix}-1\\-3\\2\end{bmatrix}\right\}\) exactly when
  the vector equation
  \(x_1\begin{bmatrix}1\\0\\-3\end{bmatrix}+
  x_2\begin{bmatrix}-1\\-3\\2\end{bmatrix}
  =\begin{bmatrix}-1\\-6\\1\end{bmatrix}\) holds for some scalars
  \(x_1,x_2\).
  \begin{subactivity}
    Reinterpret this vector equation as a system of linear equations.
  \end{subactivity}
    % \begin{alignat*}{3}
    %    x_1 &\,-\,&  x_2 &\,=\,& -1 \\
    %        &\,-\,& 3x_2 &\,=\,& -6 \\
    %  -3x_1 &\,+\,& 2x_2 &\,=\,&  1
    % \end{alignat*}
  \begin{subactivity}
    Solve this system. (Remember, you should use a calculator to help
    find \(\RREF\).)
    % \[\RREF
    %   \begin{bmatrix}[cc|c]
    %     1 & -1 & -1 \\
    %     0 & -3 & -6 \\
    %     -3 & 2 & 1
    %   \end{bmatrix}
    % \]
  \end{subactivity}
  \begin{subactivity}
    Given this solution, does
    \(\begin{bmatrix}-1\\-6\\1\end{bmatrix}\) belong to
    \(\vspan\left\{\begin{bmatrix}1\\0\\-3\end{bmatrix},
    \begin{bmatrix}-1\\-3\\2\end{bmatrix}\right\}\)?
  \end{subactivity}
\end{activity}


\end{applicationActivities}
