%!TEX root =../../course-notes.tex
% ^ leave for LaTeXTools build functionality

\begin{applicationActivities}{Day 1}

\begin{activity}{15}
Let \(\vect u=\begin{bmatrix} u_1 \\ u_2 \end{bmatrix}\),
\(\vect v=\begin{bmatrix} v_1 \\ v_2 \end{bmatrix}\), and
\(\vect w=\begin{bmatrix} w_1 \\ w_2 \end{bmatrix}\).
Rewrite the following proofs about two-dimensional Euclidean vector addition,
filling in the missing information.
\begin{enumerate}
  \item Euclidean vector addition is \textbf{well-defined}.

        \textbf{Proof:} \(\vect u+\vect v=?\). Since this is a Euclidean
        vector, addition is well-defined. \(\square\)
  \item Euclidean vector addition is \textbf{associative}.

        \textbf{Proof:} \(\vect u+(\vect v+\vect w)=\,?\) and
        \((\vect u+\vect v)+\vect w=\,?\). Since both expressions are equal,
        addition is associative. \(\square\)
  \item Euclidean vector addition is \textbf{commutative}.

        \textbf{Proof:} \(\vect u+\vect v=\,?\) and
        \(\vect v+\vect u=\,?\). Since both expressions are equal,
        addition is commutative. \(\square\)
  \item There exists an \textbf{additive identity} for Euclidean vector
        addition.

        \textbf{Proof:} Let \(\vect e=\,?\). Since
        \(\vect v+\vect e=\,?=\vect v\), \(\vect e\) is an identity. \(\square\)
  \item Every Euclidean vector has an \textbf{additive inverse}.

        \textbf{Proof:} Let \(-\vect v=\,?\). Since
        \(\vect v+(-\vect v)=\,?=\vect e\), \(-\vect v\) is an inverse
        for \(\vect v\). \(\square\)
\end{enumerate}
\end{activity}

\begin{activity}{15}
Let \(\vect u=\begin{bmatrix} u_1 \\ u_2 \end{bmatrix}\) and
\(\vect v=\begin{bmatrix} v_1 \\ v_2 \end{bmatrix}\), and let \(a,b\) be
scalar numbers.
Rewrite the following proofs about scalar multiplication of
two-dimensional Euclidean vectors,
filling in the missing information.
\begin{enumerate}
  \item Scalar multiplication of Euclidean vectors is \textbf{well-defined}.

        \textbf{Proof:} \(a\vect v=?\). Since this is a Euclidean
        vector, scalar multiplication is well-defined. \(\square\)
  \item Scalar multiplication of Euclidean vectors is \textbf{compatible}.

        \textbf{Proof:} \(a(b\vect v)=\,?\) and
        \((ab)\vect v=\,?\). Since both expressions are equal,
        scalar multiplication is compatible. \(\square\)
  \item Scalar multiplication of Euclidean vectors
        \textbf{distributes scalars}.

        \textbf{Proof:} \(a(\vect u+\vect v)=\,?\) and
        \(a\vect u+a\vect v=\,?\). Since both expressions are equal,
        scalars distribute. \(\square\)
  \item Scalar multiplication of Euclidean vectors
        \textbf{distributes vectors}.

        \textbf{Proof:} \((a+b)\vect v=\,?\) and
        \(a\vect v+b\vect v=\,?\). Since both expressions are equal,
        vectors distribute. \(\square\)
  \item The scalar \(1\) is a \textbf{multiplicative identity} for the
        scalar multiplication of Euclidean vectors.

        \textbf{Proof:} Since
        \(1\vect v=\,?=\vect v\), \(1\) is an identity. \(\square\)
\end{enumerate}
\end{activity}

\begin{definition}
  A \textbf{vector space} \(V\) is any collection of mathematical objects with
  associated addition and scalar multiplication operations that satisfy
  the following properties. Let \(\vect u,\vect v,\vect w\) belong to \(V\),
  and let \(a,b\) be scalar numbers.
  \begin{multicols}{2}
  \begin{itemize}
    \item \textbf{Well-defined addition.}

          \(\vect v+\vect w\) belongs to \(V\).
    \item \textbf{Associative addition.}

          \(\vect u+(\vect v+\vect w)=
          (\vect u+\vect v)+\vect w\).
    \item \textbf{Commutative addition.}

          \(\vect u+\vect v=
          \vect v+\vect u\).
    \item \textbf{Additive identity.}

          There exists some \(\vect e\) in \(V\)
          where \(\vect v+\vect e=\vect v\).
    \item \textbf{Additive identity.}

          There exists some \(-\vect v\) in \(V\)
          where \(\vect v+(-\vect v)=\vect e\).
    \item \textbf{Well-defined scalar multipication.}

          \(a\vect v\) belongs to
          \(V\).
    \item \textbf{Compatible scalar multipication.}

          \(a(b\vect v)=(ab)\vect v\).
    \item \textbf{Scalar distribution.}

          \(a(\vect u+\vect v)=a\vect u+a\vect v\).
    \item \textbf{Vector distribution.}

          \((a+b)\vect v=a\vect v+b\vect v\).
    \item \textbf{Scalar multiplication identity.}

          \(1\vect v=\vect v\).
  \end{itemize}
  \end{multicols}
\end{definition}

\begin{activity}{15}
  Assuming the usual definitions of addition and scalar multiplication,
  label each of the following sets as a \textbf{valid} or \textbf{non-valid}
  vector space. If the set is not a valid vector space, give an example of
  a vector space property that fails to hold.
  \begin{multicols}{2}
  \begin{enumerate}
    \item \(\left\{
    \begin{bmatrix}
      v_1\\v_2\\v_3\\v_4
    \end{bmatrix}:v_i\text{ is an integer}\right\}\)
    \item \(\left\{
      a+bi
      :a,b\text{ are real numbers}\right\}\)
    \item \(\left\{
      x^2+bx+c
      :b,c\text{ are positive}\right\}\)
  \end{enumerate}
  \end{multicols}
\end{activity}



\end{applicationActivities}
