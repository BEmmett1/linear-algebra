%!TEX root =../../course-notes.tex
% ^ leave for LaTeXTools build functionality

\begin{applicationActivities}{1}{7}

% have students capture 8 properties of R2 from list of 11
% include ca=b, unique equidistant, nonzero orthogonal

\begin{activity}{25}
Consider each of the following vector properties. Label each property
with \(\IR^1\), \(\IR^2\), and/or \(\IR^3\) if that property holds for
Euclidean vectors/scalars \(\vect u,\vect v,\vect w\) of that dimension.
\begin{multicols}{2}
\begin{enumerate}
  \item \textbf{Addition associativity.}

        \(\vect u+(\vect v+\vect w)=
        (\vect u+\vect v)+\vect w\).
  \item \textbf{Addition commutivity.}

        \(\vect u+\vect v=
        \vect v+\vect u\).
  \item \textbf{Addition identity.}

        There exists some \(\vect 0\)
        where \(\vect v+\vect 0=\vect v\).
  \item \textbf{Addition inverse.}

        There exists some \(-\vect v\)
        where \(\vect v+(-\vect v)=\vect 0\).
  \item \textbf{Addition midpoint uniqueness.}

        There exists a unique \(\vect m\) where the distance from
        \(\vect u\) to \(\vect m\) equals the distance from \(\vect m\)
        to \(\vect v\).
  \item \textbf{Scalar multiplication associativity.}

        \(a(b\vect v)=(ab)\vect v\).
  \item \textbf{Scalar multiplication identity.}

        \(1\vect v=\vect v\).
  \item \textbf{Scalar multiplication relativity.}

        There exists some scalar \(c\) where either \(c\vect v=\vect w\)
        or \(c\vect w=\vect v\).
  \item \textbf{Scalar distribution.}

        \(a(\vect u+\vect v)=a\vect u+a\vect v\).
  \item \textbf{Vector distribution.}

        \((a+b)\vect v=a\vect v+b\vect v\).
  \item \textbf{Orthogonality.}

        There exists a non-zero vector \(\vect n\) such that
        \(\vect n\) is orthogonal to both \(\vect u\) and \(\vect v\).
  \item \textbf{Bidimensionality.}

        \(\vect v=a\vect i+b\vect j\) for some value of \(a,b\).
\end{enumerate}
\end{multicols}
\end{activity}

\begin{definition}
  A \textbf{vector space} \(V\) is any collection of mathematical objects with
  associated addition and scalar multiplication operations that satisfy
  the following properties. Let \(\vect u,\vect v,\vect w\) belong to \(V\),
  and let \(a,b\) be scalar numbers.

  \vectorSpaceProperties
\end{definition}

\begin{definition}
  The most important examples of vector spaces are the \term{Euclidean
  vector spaces} \(\IR^n\), but there are other examples as well.
\end{definition}

\begin{activity}{25}
  Consider the following vector space that models motion along the curve
  \(y=e^x\). Let \(V=\{(x,y):y=e^x\}\), where
  \((a_1,b_1)+(a_2,b_2)=(a_1+a_2,b_1b_2)\), and \(c(a,b)=(ca,b^c)\).

  \begin{subactivity}
    Verify that \(3((1,e)+(-2,\frac{1}{e^2}))=
    3(1,e)+3(-2,\frac{1}{e^2})\).
  \end{subactivity}

  \begin{TBLnote}
    Draw a visualization of these vectors by sketching ``curved''
    vectors in the \(xy\) plane.
  \end{TBLnote}

  \begin{subactivity}
    Prove the scalar distribution property for this space:
    \(c(\vect u+\vect v)=c\vect u+c\vect v\).
  \end{subactivity}
\end{activity}



\end{applicationActivities}
