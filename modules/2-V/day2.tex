%!TEX root =../../course-notes.tex
% ^ leave for LaTeXTools build functionality

\begin{applicationActivities}{Day 2}

\begin{remark}
  The following sets are examples of vector spaces, with the usual/natural
  operations for addition and scalar multiplication.
  \begin{itemize}
    \item \(\IR^n\): Euclidean vectors with \(n\) components.
    \item \(\IR^\infty\): Sequences of real numbers \((v_1,v_2,\dots)\).
    \item \(\IR^{m\times n}\): Matrices of real numbers with \(m\) rows and
          \(n\) columns.
    \item \(\IC\): Complex numbers.
    \item \(\IP^n\): Polynomials of degree \(n\) or less.
    \item \(\IP\): Polynomials of any degree.
    \item \(C(\IR)\): Real-valued continuous functions.
  \end{itemize}
\end{remark}

\begin{activity}{10}
  Prove that \(\IP^2\) satisfies the \textit{commutative addition} property
  of vector spaces,
  by showing that if \(f(x)=a_1x^2+b_1x+c_1\) and \(g(x)=a_2x^2+b_2x+c_2\),
  then \(f(x)+g(x)=g(x)+f(x)\).
\end{activity}

\begin{definition}
  A \term{linear combination} of a set of vectors
  \(\{\vect v_1,\vect v_2,\dots,\vect v_m\}\) is given by
  \(c_1\vect v_1+c_2\vect v_2+\dots+c_m\vect v_m\) for any choice of
  scalar multiples \(c_1,c_2,\dots,c_m\).
\end{definition}

\begin{definition}
  The \term{span} of a set of vectors is the collection of all linear
  combinations of that set:
  \[
    \vspan\{\vect v_1,\vect v_2,\dots,\vect v_m\} =
    \{c_1\vect v_1+c_2\vect v_2+\dots+c_m\vect v_m :
    c_i\text{ is a real number}\}
  \]
\end{definition}

\begin{activity}{10}
  Consider \(\vspan\left\{\begin{bmatrix}1\\2\end{bmatrix}\right\}\).
  \begin{subactivity}
    Sketch
    \(c\begin{bmatrix}1\\2\end{bmatrix}\) in the \(xy\) plane
    for \(c=1,3,0,-2\).
  \end{subactivity}
  \begin{subactivity}
    Sketch a representation of all the vectors given by
    \(\vspan\left\{\begin{bmatrix}1\\2\end{bmatrix}\right\}\)
    in the \(xy\) plane.
  \end{subactivity}
\end{activity}

\begin{activity}{10}
  Consider \(\vspan\left\{\begin{bmatrix}1\\2\end{bmatrix},
  \begin{bmatrix}-1\\1\end{bmatrix}\right\}\).
  \begin{subactivity}
    Sketch
    \(c_1\begin{bmatrix}1\\2\end{bmatrix}+
    c_2\begin{bmatrix}-1\\1\end{bmatrix}\) in the \(xy\) plane
    for \(\begin{bmatrix}c_1\\c_2\end{bmatrix}=
    \begin{bmatrix}1\\0\end{bmatrix},
    \begin{bmatrix}0\\1\end{bmatrix},
    \begin{bmatrix}2\\0\end{bmatrix},
    \begin{bmatrix}2\\1\end{bmatrix}\).
  \end{subactivity}
  \begin{subactivity}
    Sketch a representation of all the vectors given by
    \(\vspan\left\{\begin{bmatrix}1\\2\end{bmatrix},
     \begin{bmatrix}-1\\1\end{bmatrix}\right\}\)
    in the \(xy\) plane.
  \end{subactivity}
\end{activity}

\begin{activity}{5}
    Sketch a representation of all the vectors given by
    \(\vspan\left\{\begin{bmatrix}6\\-4\end{bmatrix},
     \begin{bmatrix}-2\\3\end{bmatrix}\right\}\)
    in the \(xy\) plane.
\end{activity}

\begin{activity}{10}
  Consider the following linear system.
    \begin{alignat*}{3}
       x_1 &\,-\,&  x_2 &\,=\,& -1 \\
           &\,-\,& 3x_2 &\,=\,& -6 \\
     -3x_1 &\,+\,& 2x_2 &\,=\,&  1
    \end{alignat*}
  \begin{subactivity}
    Solve this system by using a calculator to find
    \[\RREF
      \begin{bmatrix}[cc|c]
        1 & -1 & -1 \\
        0 & -3 & -6 \\
        -3 & 2 & 1
      \end{bmatrix}
    \]
  \end{subactivity}
  \begin{subactivity}
    Given this solution, does
    \(\begin{bmatrix}-1\\-6\\1\end{bmatrix}\) belong to
    \(\vspan\left\{\begin{bmatrix}1\\0\\-3\end{bmatrix},
    \begin{bmatrix}-1\\-3\\2\end{bmatrix}\right\}\)?
  \end{subactivity}
\end{activity}

\begin{fact}
  A vector \(\vect b\) belongs to
  \(\vspan\{\vect v_1,\dots,\vect v_n\}\) if and only if
  the linear system corresponding to
  \([\vect v_1\,\dots\,\vect v_n \,|\, \vect b]\)
  is consistent.
\end{fact}

\begin{remark}
  To determine if \(\vect b\) belongs to
  \(\vspan\{\vect v_1,\dots,\vect v_n\}\), find
  \(\RREF[\vect v_1\,\dots\,\vect v_n \,|\, \vect b]\).
\end{remark}

\begin{activity}{5}
  Determine if
  \(\begin{bmatrix}3\\-2\\1\end{bmatrix}\) belongs to
  \(\vspan\left\{\begin{bmatrix}1\\0\\-3\end{bmatrix},
  \begin{bmatrix}-1\\-3\\2\end{bmatrix}\right\}\)
  by row-reducing an appropriate matrix.
\end{activity}


\end{applicationActivities}
