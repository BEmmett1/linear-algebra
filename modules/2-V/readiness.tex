%!TEX root =../../course-notes.tex
% ^ leave for LaTeXTools build functionality

\begin{readinessAssuranceOutcomes}
\item Add Euclidean vectors and multiply Euclidean vectors by scalars.
\item Add complex numbers and multiply complex numbers by scalars.
\item Add polynomials and multiply polynomials by scalars.
% \item Understand and apply polar coordinates.
\end{readinessAssuranceOutcomes}

\begin{readinessAssuranceResources}
\item \url{https://www.khanacademy.org/math/precalculus/vectors-precalc/vector-addition-subtraction/v/adding-and-subtracting-vectors}
\item \url{https://www.khanacademy.org/math/precalculus/vectors-precalc/combined-vector-operations/v/combined-vector-operations-example}
\item \url{https://www.khanacademy.org/math/precalculus/imaginary-and-complex-numbers/adding-and-subtracting-complex-numbers/v/adding-complex-numbers}
\item \url{https://www.khanacademy.org/math/algebra/introduction-to-polynomial-expressions/adding-and-subtracting-polynomials/v/adding-and-subtracting-polynomials-1}
% \item \url{https://www.khanacademy.org/math/multivariable-calculus/integrating-multivariable-functions/double-integrals-a/v/polar-coordinates-1}
\end{readinessAssuranceResources}




\begin{readinessAssuranceTest}

\item Simplify the following vector expression.
  \[
  2
  \begin{bmatrix}
    3 \\ -1 \\ 0
  \end{bmatrix}-
  3
  \begin{bmatrix}
    0 \\ 2 \\ 1
  \end{bmatrix}
  \]

\begin{multicols}{4}
\begin{readinessAssuranceTestChoices}
\item \(
        \begin{bmatrix}
          0 \\ 4 \\ -7
        \end{bmatrix}
      \)
\item \(
        \begin{bmatrix}
          6 \\ -8 \\ -3
        \end{bmatrix}
      \)
\item \(
        \begin{bmatrix}
          3 \\ 2 \\ -5
        \end{bmatrix}
      \)
\item \(
        \begin{bmatrix}
          -2 \\ 0 \\ 1
        \end{bmatrix}
      \)
\end{readinessAssuranceTestChoices}
\end{multicols}

\item Simplify the complex number expression
      \(-4(3-2i)+2(5+i)\).

\begin{multicols}{4}
\begin{readinessAssuranceTestChoices}
\item \(3-7i\)
\item \(4+i\)
\item \(-2+10i\)
\item \(-1-5i\)
\end{readinessAssuranceTestChoices}
\end{multicols}

\item Simplify \(3f(x)-2g(x)\) where
      \(f(x)=7-x^2\) and
      \(g(x)=2x^3+x-1\).

\begin{multicols}{4}
\begin{readinessAssuranceTestChoices}
\item \(x^3+4x-5\)
\item \(-4x^3-3x^2-2x+23\)
\item \(3x^3+5x^2-3x+17\)
\item \(-x^3+19x^2-4\)
\end{readinessAssuranceTestChoices}
\end{multicols}

% \item Which of these graphs could represent the polar coordinate
%       \(p(r,\theta)=p(-4,2\pi/3)\)?
%
% \begin{multicols}{4}
% \begin{readinessAssuranceTestChoices}
% \item
% \begin{tikzpicture}[scale=0.25]
% \draw[thin,gray,<->] (-5,0) -- (5,0);
% \draw[thin,gray,<->] (0,-5) -- (0,5);
% \draw[thick,blue,fill=blue] (-3.46,2) circle (0.2);
% \end{tikzpicture}
% \item
% \begin{tikzpicture}[scale=0.25]
% \draw[thin,gray,<->] (-5,0) -- (5,0);
% \draw[thin,gray,<->] (0,-5) -- (0,5);
% \draw[thick,blue,fill=blue] (2,3.46) circle (0.2);
% \end{tikzpicture}
% \item
% \begin{tikzpicture}[scale=0.25]
% \draw[thin,gray,<->] (-5,0) -- (5,0);
% \draw[thin,gray,<->] (0,-5) -- (0,5);
% \draw[thick,blue,fill=blue] (-3.46,-2) circle (0.2);
% \end{tikzpicture}
% \item
% \begin{tikzpicture}[scale=0.25]
% \draw[thin,gray,<->] (-5,0) -- (5,0);
% \draw[thin,gray,<->] (0,-5) -- (0,5);
% \draw[thick,blue,fill=blue] (2,-3.46) circle (0.2);
% \end{tikzpicture}
% \end{readinessAssuranceTestChoices}
% \end{multicols}



\end{readinessAssuranceTest}
