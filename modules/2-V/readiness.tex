%!TEX root =../../course-notes.tex
% ^ leave for LaTeXTools build functionality

\begin{readinessAssuranceOutcomes}
\item Add Euclidean vectors and multiply Euclidean vectors by scalars.
\item Add complex numbers and multiply complex numbers by scalars.
\item Add polynomials and multiply polynomials by scalars.
\item Perform basic manipulations of augmented matrices and linear
systems \standardList{E1,E2,E3}.
\end{readinessAssuranceOutcomes}

\begin{readinessAssuranceResources}
\item \url{https://www.khanacademy.org/math/precalculus/vectors-precalc/vector-addition-subtraction/v/adding-and-subtracting-vectors}
\item \url{https://www.khanacademy.org/math/precalculus/vectors-precalc/combined-vector-operations/v/combined-vector-operations-example}
\item \url{https://www.khanacademy.org/math/precalculus/imaginary-and-complex-numbers/adding-and-subtracting-complex-numbers/v/adding-complex-numbers}
\item \url{https://www.khanacademy.org/math/algebra/introduction-to-polynomial-expressions/adding-and-subtracting-polynomials/v/adding-and-subtracting-polynomials-1}
\end{readinessAssuranceResources}




\begin{readinessAssuranceTest}

\item Simplify the following vector expression.
  \[
  2
  \begin{bmatrix}
    3 \\ -1 \\ 0
  \end{bmatrix}-
  3
  \begin{bmatrix}
    0 \\ 2 \\ 1
  \end{bmatrix}
  \]

\begin{multicols}{4}
\begin{readinessAssuranceTestChoices}
\item \(
        \begin{bmatrix}
          0 \\ 4 \\ -8
        \end{bmatrix}
      \)
\item \(
        \begin{bmatrix}
          6 \\ -8 \\ -3
        \end{bmatrix}
      \) % correct
\item \(
        \begin{bmatrix}
          3 \\ 2 \\ -5
        \end{bmatrix}
      \)
\item \(
        \begin{bmatrix}
          -2 \\ 0 \\ 1
        \end{bmatrix}
      \)
\end{readinessAssuranceTestChoices}
\end{multicols}

\item Simplify the following vector expression.
  \[
  2\left(
  \begin{bmatrix}
    1 \\ 1 \\ -1
  \end{bmatrix}+
  \begin{bmatrix}
    -1 \\ 1 \\ -3
  \end{bmatrix}\right)
  \]

\begin{multicols}{4}
\begin{readinessAssuranceTestChoices}
\item \(
        \begin{bmatrix}
          0 \\ 4 \\ -8
        \end{bmatrix}
      \) % correct
\item \(
        \begin{bmatrix}
          6 \\ -8 \\ -3
        \end{bmatrix}
      \)
\item \(
        \begin{bmatrix}
          3 \\ 2 \\ -5
        \end{bmatrix}
      \)
\item \(
        \begin{bmatrix}
          -2 \\ 0 \\ 1
        \end{bmatrix}
      \)
\end{readinessAssuranceTestChoices}
\end{multicols}

\item Simplify the complex number expression
      \(-4(3-2i)+2(5+i)\).

\begin{multicols}{4}
\begin{readinessAssuranceTestChoices}
\item \(3-7i\)
\item \(4+i\)
\item \(-2+10i\) % correct
\item \(-1-5i\)
\end{readinessAssuranceTestChoices}
\end{multicols}

\item Which of these complex numbers might be represented
      by the following vector
      plotted on the complex plane (where the horizontal axis gives the
      real part and the vertical axis gives the imaginary part)?

      \begin{center}
        \begin{tikzpicture}[scale=0.25]
        \draw[thin,gray,<->] (-5,0) -- (5,0);
        \draw[thin,gray,<->] (0,-5) -- (0,5);
        \draw[thick,blue,->] (0,0) -- (-2,3);
        \end{tikzpicture}
      \end{center}

\begin{multicols}{4}
\begin{readinessAssuranceTestChoices}
\item \(5+i\)
\item \(-3-9i\)
\item \(-2+3i\) % correct
\item \(4i\)
\end{readinessAssuranceTestChoices}
\end{multicols}

\item Simplify \(3f(x)-2g(x)\) where
      \(f(x)=7-x^2\) and
      \(g(x)=2x^3+x-1\).

\begin{multicols}{4}
\begin{readinessAssuranceTestChoices}
\item \(x^3+4x-5\)
\item \(-4x^3-3x^2-2x+23\) % correct
\item \(3x^3+5x^2-3x+17\)
\item \(-x^3+19x^2-4\)
\end{readinessAssuranceTestChoices}
\end{multicols}

\item Express the following system of linear equations as an augmented matrix.
\begin{alignat*}{5}
  x_1 &\,+\,& 2x_2 &\, \,&     &\,-\,&  x_4 &\,=\,& 3 \\
      &\, \,&      &\, \,& x_3 &\,+\,& 4x_4 &\,=\,& -2
\end{alignat*}

\begin{multicols}{4}
\begin{readinessAssuranceTestChoices}
\item \(
      \begin{bmatrix}[c|c]
        1 & 0 \\
        2 & 0 \\
        0 & 1 \\
        -1 & 4 \\
        -2 & 3
      \end{bmatrix}
    \)
\item \(
      \begin{bmatrix}[cccc|c]
        1 & 2 & 0 & -1 & 3 \\
        0 & 0 & 1 & 4 & -2
      \end{bmatrix}
    \) % correct
    \item \(
          \begin{bmatrix}[c|c]
            1 & 2 \\
            0 & -1 \\
            3 & 0 \\
            0 & 1 \\
            4 & -2
          \end{bmatrix}
        \)
\item \(
      \begin{bmatrix}[cccc|c]
        1 & 2 & 1 & 4 & 3 \\
        -2 & 1 & 3 & 4 & 5
      \end{bmatrix}
    \)
\end{readinessAssuranceTestChoices}
\end{multicols}

\item Which of the following matrices is equivalent to the following matrix?
  \[
    \begin{bmatrix}[cc|c]
      1 & 2 & 3 \\
      0 & 4 & -1 \\
      2 & 3 & 2 \\
    \end{bmatrix}
  \]
  (Hint: The correct answer was obtained from a single row operation.)

  \begin{multicols}{4}
  \begin{readinessAssuranceTestChoices}
  \item
    \(
      \begin{bmatrix}[cc|c]
        1 & 2 & 3 \\
        0 & 4 & -1 \\
        0 & 0 & 1 \\
      \end{bmatrix}
    \)
  \item
    \(
      \begin{bmatrix}[cc|c]
        1 & 2 & 3 \\
        1 & 3 & 4 \\
        2 & 3 & 2 \\
      \end{bmatrix}
    \)
  \item
    \(
      \begin{bmatrix}[cc|c]
        1 & 2 & 3 \\
        0 & 1 & 1 \\
        2 & 3 & 2 \\
      \end{bmatrix}
    \)
  \item
    \(
      \begin{bmatrix}[cc|c]
        1 & 2 & 3 \\
        0 & 4 & -1 \\
        0 & -1 & -4 \\
      \end{bmatrix}
    \) % correct
  \end{readinessAssuranceTestChoices}
  \end{multicols}

\item Find
  \(\RREF
    \begin{bmatrix}[cc|c]
      1 & 2 & 3 \\
      0 & 4 & -1 \\
      2 & 3 & 2 \\
    \end{bmatrix}
  \).

  \begin{multicols}{4}
  \begin{readinessAssuranceTestChoices}
  \item
    \(
      \begin{bmatrix}[cc|c]
        1 & 0 & 3 \\
        0 & 1 & -1 \\
        0 & 0 & 0 \\
      \end{bmatrix}
    \)
  \item
    \(
      \begin{bmatrix}[cc|c]
        1 & 2 & 3 \\
        1 & 3 & 4 \\
        0 & 0 & 0 \\
      \end{bmatrix}
    \)
  \item
    \(
      \begin{bmatrix}[cc|c]
        1 & 2 & 3 \\
        0 & 1 & 1 \\
        0 & 0 & 0 \\
      \end{bmatrix}
    \)
  \item
    \(
      \begin{bmatrix}[cc|c]
        1 & 0 & 0 \\
        0 & 1 & 0 \\
        0 & 0 & 1 \\
      \end{bmatrix}
    \) % correct
  \end{readinessAssuranceTestChoices}
  \end{multicols}

  \item Solve the following system of linear equations.
  \begin{alignat*}{4}
    2x_1 &\,+\,& x_2 &\,+\,& 4x_3 &\,=\,& 0 \\
     x_1 &\,+\,& x_2 &\,+\,&  x_3 &\,=\,& 1 \\
   -3x_1 &\,+\,& 4x_2 &\,+\,& x_3 &\,=\,& -7 \\
  \end{alignat*}

  \begin{multicols}{2}
  \begin{readinessAssuranceTestChoices}
  \item \(
          \begin{bmatrix}
            x_1 \\ x_2 \\ x_3
          \end{bmatrix}=
          \begin{bmatrix}
            2 \\ 0 \\ -1
          \end{bmatrix}
        \) % correct
  \item \(
          \begin{bmatrix}
            x_1 \\ x_2 \\ x_3
          \end{bmatrix}=
          \begin{bmatrix}
            1 \\ -6 \\ 1
          \end{bmatrix}
        \)
  \item \(
          \begin{bmatrix}
            x_1 \\ x_2 \\ x_3
          \end{bmatrix}=
          \begin{bmatrix}
            1 \\ -2 \\ 1
          \end{bmatrix}+
          a
          \begin{bmatrix}
            1 \\ 0 \\ 3
          \end{bmatrix}
        \) for all real numbers \(a\)
  \item No solutions
  \end{readinessAssuranceTestChoices}
  \end{multicols}

  \item Solve the following system of linear equations.
  \begin{alignat*}{4}
    2x_1 &\,+\,& x_2 &\,+\,& 4x_3 &\,=\,& 0 \\
     x_1 &\,+\,& x_2 &\,+\,&  x_3 &\,=\,& 0 \\
  \end{alignat*}

  \begin{multicols}{2}
  \begin{readinessAssuranceTestChoices}
  \item \(
          \begin{bmatrix}
            x_1 \\ x_2 \\ x_3
          \end{bmatrix}=
          \begin{bmatrix}
            4 \\ 1 \\ -5
          \end{bmatrix}
        \)
  \item \(
          \begin{bmatrix}
            x_1 \\ x_2 \\ x_3
          \end{bmatrix}=
          \begin{bmatrix}
            1 \\ 2 \\ -1
          \end{bmatrix}
        \)
  \item \(
          \begin{bmatrix}
            x_1 \\ x_2 \\ x_3
          \end{bmatrix}=
          a
          \begin{bmatrix}
            -3 \\ 2 \\ 1
          \end{bmatrix}
        \) for all real numbers \(a\) % correct
  \item No solutions
  \end{readinessAssuranceTestChoices}
  \end{multicols}



\end{readinessAssuranceTest}
