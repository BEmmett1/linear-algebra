%!TEX root =../../course-notes.tex
% ^ leave for LaTeXTools build functionality

\begin{applicationActivities}{3}{10}

\begin{fact}
  A vector \(\vect b\) belongs to
  \(\vspan\{\vect v_1,\dots,\vect v_n\}\) if and only if
  the linear system corresponding to
  \([\vect v_1\,\dots\,\vect v_n \,|\, \vect b]\)
  is consistent.
\end{fact}

\begin{remark}
  To determine if \(\vect b\) belongs to
  \(\vspan\{\vect v_1,\dots,\vect v_n\}\), find
  \(\RREF[\vect v_1\,\dots\,\vect v_n \,|\, \vect b]\).
\end{remark}

\begin{activity}{5}
  Determine if
  \(\begin{bmatrix}3\\-2\\1\end{bmatrix}\) belongs to
  \(\vspan\left\{\begin{bmatrix}1\\0\\-3\end{bmatrix},
  \begin{bmatrix}-1\\-3\\2\end{bmatrix}\right\}\)
  by row-reducing an appropriate matrix.
\end{activity}

\begin{activity}{5}
  Determine if
  \(\begin{bmatrix}-1\\-9\\0\end{bmatrix}\) belongs to
  \(\vspan\left\{\begin{bmatrix}1\\0\\-3\end{bmatrix},
  \begin{bmatrix}-1\\-3\\2\end{bmatrix}\right\}\)
  by row-reducing an appropriate matrix.
\end{activity}


\begin{observation}
  So far we've only discussed linear combinations of Euclidean vectors.
  Fortunately, many vector spaces of interest can be reinterpreted as an
  \term{isomorphic} Euclidean space \(\IR^n\); that is, a Euclidean space
  that mirrors the behavior of the vector space exactly.
\end{observation}

\begin{activity}{5}
  We previously checked that \(\begin{bmatrix}3\\-2\\1\end{bmatrix}\)
  does not belong to
  \(\vspan\left\{\begin{bmatrix}1\\0\\-3\end{bmatrix},
  \begin{bmatrix}-1\\-3\\2\end{bmatrix}\right\}\).
  Does \(f(x)=3x^2-2x+1\) belong to
  \(\vspan\{x^2-3,-x^2-3x+2\}\)?
\end{activity}

\begin{activity}{10}
  Does the matrix \(\begin{bmatrix}6&3\\2&-1\end{bmatrix}\) belong to
  \(\vspan\left\{\begin{bmatrix}1&0\\0&-1\end{bmatrix},
  \begin{bmatrix}4&3\\2&1\end{bmatrix}\right\}\)?
\end{activity}

\begin{activity}{10}
  Does the complex number \(2i\) belong to
  \(\vspan\{-3+i,6-2i\}\)?
\end{activity}

\begin{activity}{10}
  How many vectors are required to span \(\IR^2\)?
  Sketch a drawing in the \(xy\) plane to support your guess.
\end{activity}

\begin{activity}{5}
  How many vectors are required to span \(\IR^3\)?
\end{activity}


\end{applicationActivities}
