%!TEX root =../../course-notes.tex
% ^ leave for LaTeXTools build functionality

\begin{applicationActivities}{Day 3}

\begin{activity}{0}
  Let $T: \IR^n \rightarrow \IR^n$ be a linear map with matrix $A \in M_{n,n}$.

  If $T$ is a bijection, then $AX=B$ has a unique solution for all $B \in \IR^n$.  Thus we can define a map $T^{-1} : \IR^n \rightarrow \IR^n$ by defining $T^{-1}(B)$ to be this solution.  It follows immediately that $T \circ T^{-1}$ is the identity map.  The matrix corresponding to $T^{-1}$ is denoted $A^{-1}$.

  \begin{enumerate}[1)]
  \item Solve $AX=\vec{e}_1$ to determine $T^{-1}(\vec{e}_1)$
  \item Solve $AX=\vec{e}_1$ to determine $T^{-1}(\vec{e}_2)$
  \item Solve $AX=\vec{e}_1$ to determine $T^{-1}(\vec{e}_3)$
  \item Compute $A^{-1}$
  \end{enumerate}

  A (square) matrix is called {\em invertible} if it corresponds to an invertible linear transformation.

  \begin{enumerate}[1)]
  \item Find the inverse of the matrix $\begin{bmatrix} 1 & 3 \\ 0 & -2 \end{bmatrix}$
  \item Find the inverse of the matrix $\begin{bmatrix} 1 & -2 & 1 \\ -3 & 7 & 6 \\ 2 & -3 & 0 \end{bmatrix}$
  \end{enumerate}
\end{activity}

\end{applicationActivities}
