%!TEX root =../../course-notes.tex
% ^ leave for LaTeXTools build functionality

\begin{applicationActivities}{1}{21}

\begin{activity}{5}
Let $T: \IR^3 \rightarrow \IR^2$ be given by the standard matrix $B=\begin{bmatrix} 2 & 1 & -3 \\ 5 & -3 & 4 \end{bmatrix}$ and $S: \IR^2 \rightarrow \IR^4$ be given by the standard matrix $A=\begin{bmatrix} 1 & 2 \\ 0 & 1 \\ 3 & 5 \\ -1 & -2 \end{bmatrix}$.

What is the domain of the composition map $S \circ T$?
\begin{enumerate}[(a)]
\item $\IR$
\item $\IR^2$
\item $\IR^3$
\item $\IR^4$
\end{enumerate}
\end{activity}

\begin{activity}{2}
Let $T: \IR^3 \rightarrow \IR^2$ be given by the standard matrix $B=\begin{bmatrix} 2 & 1 & -3 \\ 5 & -3 & 4 \end{bmatrix}$ and $S: \IR^2 \rightarrow \IR^4$ be given by the standard matrix $A=\begin{bmatrix} 1 & 2 \\ 0 & 1 \\ 3 & 5 \\ -1 & -2 \end{bmatrix}$.

What is the codomain of the composition map $S \circ T$?
\begin{enumerate}[(a)]
\item $\IR$
\item $\IR^2$
\item $\IR^3$
\item $\IR^4$
\end{enumerate}
\end{activity}

\begin{activity}{2}
Let $T: \IR^3 \rightarrow \IR^2$ be given by thestandard  matrix $B=\begin{bmatrix} 2 & 1 & -3 \\ 5 & -3 & 4 \end{bmatrix}$ and $S: \IR^2 \rightarrow \IR^4$ be given by the standard matrix $A=\begin{bmatrix} 1 & 2 \\ 0 & 1 \\ 3 & 5 \\ -1 & -2 \end{bmatrix}$.

The standard matrix of $S \circ T$ will lie in which matrix space?
\begin{enumerate}[(a)]
\item $4 \times 3$ matrices
\item $4 \times 2$ matrices
\item $3 \times 2$ matrices
\item $2 \times 3$ matrices
\item $2 \times 4$ matrices
\item $3 \times 4$ matrices
%\item $M_{4,3}$
%\item $M_{4,2}$
%\item $M_{3,2}$
%\item $M_{2,3}$
%\item $M_{2,4}$
%\item $M_{3,4}$
\end{enumerate}
\end{activity}

\begin{activity}{15}
Let $T: \IR^3 \rightarrow \IR^2$ be given by the standard matrix $B=\begin{bmatrix} 2 & 1 & -3 \\ 5 & -3 & 4 \end{bmatrix}$ and $S: \IR^2 \rightarrow \IR^4$ be given by the standard matrix $A=\begin{bmatrix} 1 & 2 \\ 0 & 1 \\ 3 & 5 \\ -1 & -2 \end{bmatrix}$.

\begin{subactivity}
Compute $(S \circ T)(\vec{e}_1)$
\end{subactivity}
\begin{subactivity}
Compute  $(S \circ T)(\vec{e}_2)$
\end{subactivity}
\begin{subactivity}
Compute $(S \circ T)(\vec{e}_3)$.
\end{subactivity}
\begin{subactivity}
Find the standard matrix of $S \circ T$.
\end{subactivity}
\end{activity}


\begin{activity}{2}
Let $T: \IR^2 \rightarrow \IR^3$ be given by the matrix $B=\begin{bmatrix} 2 & 3 \\ 1 & -1 \\ 0 & -1 \end{bmatrix}$ and $S: \IR^3 \rightarrow \IR^2$ be given by the matrix $A=\begin{bmatrix} -4 & -2 & 3 \\ 0 & 1 & 1 \end{bmatrix}$.

What is the domain of the composition map $S \circ T$?
\begin{enumerate}[(a)]
\item $\IR$
\item $\IR^2$
\item $\IR^3$
\item $\IR^4$
\end{enumerate}
\end{activity}

\begin{activity}{2}
Let $T: \IR^2 \rightarrow \IR^3$ be given by the matrix $B=\begin{bmatrix} 2 & 3 \\ 1 & -1 \\ 0 & -1 \end{bmatrix}$ and $S: \IR^3 \rightarrow \IR^2$ be given by the matrix $A=\begin{bmatrix} -4 & -2 & 3 \\ 0 & 1 & 1 \end{bmatrix}$.

What is the codomain of the composition map $S \circ T$?
\begin{enumerate}[(a)]
\item $\IR$
\item $\IR^2$
\item $\IR^3$
\item $\IR^4$
\end{enumerate}
\end{activity}

\begin{activity}{2}
Let $T: \IR^2 \rightarrow \IR^3$ be given by the matrix $B=\begin{bmatrix} 2 & 3 \\ 1 & -1 \\ 0 & -1 \end{bmatrix}$ and $S: \IR^3 \rightarrow \IR^2$ be given by the matrix $A=\begin{bmatrix} -4 & -2 & 3 \\ 0 & 1 & 1 \end{bmatrix}$.

The standard matrix of $S \circ T$ will lie in which matrix space?
\begin{enumerate}[(a)]
%\item $M_{2,2}$
%\item $M_{2,3}$
%\item $M_{3,2}$
%\item $M_{3,3}$
\item $2 \times 2$ matrices
\item $2 \times 3$ matrices
\item $3 \times 2$ matrices
\item $3 \times 3$ matrices
\end{enumerate}
\end{activity}

%\begin{activity}{5}
%Let $T: \IR^2 \rightarrow \IR^3$ be given by the matrix $B=\begin{bmatrix} 2 & 3 \\ 1 & -1 \\ 0 & -1 \end{bmatrix}$ and $S: \IR^3 \rightarrow \IR^2$ be given by the matrix $A=\begin{bmatrix} -4 & -2 & 3 \\ 0 & 1 & 1 \end{bmatrix}$.
%
%Compute $(S \circ T)(\vec{e}_1)$ and $(S \circ T)(\vec{e}_2)$
%\end{activity}

\begin{activity}{10}
Let $T: \IR^2 \rightarrow \IR^3$ be given by the matrix $B=\begin{bmatrix} 2 & 3 \\ 1 & -1 \\ 0 & -1 \end{bmatrix}$ and $S: \IR^3 \rightarrow \IR^2$ be given by the matrix $A=\begin{bmatrix} -4 & -2 & 3 \\ 0 & 1 & 1 \end{bmatrix}$.

Find the standard matrix of $S \circ T$.
\end{activity}

%\begin{activity}{2}
%Let $T: \IR^1 \rightarrow \IR^4$ be given by the matrix $B=\begin{bmatrix} 3 \\\ -2 \\ 1 \\ -1\end{bmatrix}$ and $S: \IR^4 \rightarrow \IR^1$ be given by the matrix $A=\begin{bmatrix}2 & 3 & 2 & 5 \end{bmatrix}$.
%
%The matrix corresponding to $S \circ T$ will lie in which matrix space?
%\begin{enumerate}[(a)]
%\item $M_{1,1}$
%\item $M_{1,4}$
%\item $M_{4,1}$
%\item $M_{4,4}$
%\end{enumerate}
%\end{activity}

\begin{activity}{5}
Let $T: \IR^1 \rightarrow \IR^4$ be given by the matrix $B=\begin{bmatrix} 3 \\\ -2 \\ 1 \\ -1\end{bmatrix}$ and $S: \IR^4 \rightarrow \IR^1$ be given by the matrix $A=\begin{bmatrix}2 & 3 & 2 & 5 \end{bmatrix}$.

  Find the standard matrix of $S \circ T$ .
\end{activity}

\begin{definition}
We define the product of a $m \times n$ matrix $A$ and a $n \times k$ matrix $B$ to be the $m \times k$ standard matrix (denoted $AB$) of the composition map of the two corresponding linear functions.
\end{definition}

\begin{fact}
If $AB$ is defined, $BA$ need not be defined, and if it is defined, it is in general different from $AB$.
\end{fact}

\begin{activity}{10}
Let $A=\begin{bmatrix} 3 & 1 & -1 \\ 2 & 0 & 4  \end{bmatrix}$ and $B=\begin{bmatrix} a & b \\ c & d\\ e & f \end{bmatrix}$.  Compute $AB$.
\end{activity}


\begin{activity}{5}
Let $A=\begin{bmatrix} 3 & 1 & -1 \\ 2 & 0 & 4 \\ -1 & 3 & 5 \end{bmatrix}$ and $X=\begin{bmatrix} x \\ y \\ z \end{bmatrix}$.
Compute $AX$
\end{activity}

\begin{observation}
Consider the  system of equations
\begin{align*}
3x+y-z &= 5 \\ 2x+4z &= -7 \\ -x+3y+5z &=2
\end{align*}

We can interpret this as a \term{matrix equation} $AX=B$ where
\begin{align*}
A&=\begin{bmatrix} 3 & 1 & -1 \\ 2 & 0 & 4 \\ -1 & 3 & 5  \end{bmatrix} & X&=\begin{bmatrix} x  \\ y \\ z  \end{bmatrix} & B &= \begin{bmatrix} 5 \\ -7 \\ 2 \end{bmatrix}
\end{align*}

For this reason, we will swap out the use of Euclidean vectors
\(\vect x\in\IR^n\) and \(n\times 1\) matrices \(X\) whenever it is
convenient.
\end{observation}

\end{applicationActivities}
