%!TEX root =../../course-notes.tex
% ^ leave for LaTeXTools build functionality

\begin{applicationActivities}{3}{19}

\begin{observation}
Let $T: V \rightarrow W$.  We have previously defined the following
terms.
\begin{itemize}
\item  $T$ is called \term{injective} or \term{one-to-one} if $T$ does not map two distinct values to the same place.
\item $T$ is called \term{surjective} or \term{onto} if every element of $W$ is mapped to by some element of $V$.
\item The \term{kernel} of $T$ is the set of all things that are mapped to $\vec{0}$.  It is a subspace of $V$.
\item The \term{image} of $T$ is the set of all things in $W$ that are mapped to by something in $V$.  It is a subspace of $W$.
\end{itemize}
\end{observation}

\begin{activity}{5}
Let $T: V \rightarrow W$ be a linear transformation where
$\ker T = \{ \vec{0} \}$. Can you answer either of the following questions
about \(T\)?
\begin{enumerate}[(a)]
\item Is $T$ injective?
\item Is $T$ surjective?
\end{enumerate}
(Hint: If $T(\vec{v})=T(\vec{w})$, then what is $T(\vec{v}-\vec{w})$?)
\end{activity}

\begin{fact}
A linear transformation $T$ is injective \textbf{if and only if} $\ker T = \{\vec{0}\}$. Put another way, an injective linear transformation may be
recognized by its \term{trivial} kernal.
\end{fact}

\begin{activity}{5}
Let $T: V \rightarrow \IR^3$ be a linear transformation where
$\Im T = \vspan \left\{ \begin{bmatrix} 1 \\ 0 \\ 3 \end{bmatrix}, \begin{bmatrix} 3 \\ -1 \\ -1 \end{bmatrix} \right\}$.
Can you answer either of the following questions about \(T\)?
\begin{enumerate}[(a)]
\item Is $T$ injective?
\item Is $T$ surjective?
\end{enumerate}
\end{activity}

\begin{fact}
A linear transformation $T:V \rightarrow W$ is surjective \textbf{if and only if} $\Im T = W$. Put another way, a surjective linear transformation may be
recognized by its same codomain and image.
\end{fact}

\begin{activity}{15}
Let $T: \IR^n \rightarrow \IR^m$ be a linear map with matrix $A$ (for the standard bases). Sort the following claims into two groups of
equivalent statements.
\begin{multicols}{2}
\begin{enumerate}[(a)]
\item $T$ is injective
\item $T$ is surjective
\item The kernel of \(T\) is trivial.
\item The columns of $A$ span $\IR^m$
\item The columns of $A$ are linearly independent
\item Every column of $\RREF(A)$ has a pivot.
\item Every row of $\RREF(A)$ has a pivot.
\item The image of \(T\) equals its codomain.
\item The system of linear equations given by the augmented matrix $\begin{bmatrix}[c|c]A & \vec{b} \end{bmatrix}$ has a solution for all $\vec{b} \in \IR^m$
\item The system of linear equations given by the augmented matrix $\begin{bmatrix}[c|c] A & \vec{0} \end{bmatrix}$ has exactly one solution.
\end{enumerate}
\end{multicols}
\begin{TBLnote}
  This activity may be ran as a card sort.
\end{TBLnote}


\end{activity}

\begin{definition}
If $T: V \rightarrow W$ is both injective and surjective, it is called \term{bijective}.
\end{definition}

\begin{activity}{5}
Let $T: \IR^n \rightarrow \IR^m$ be a bijective linear map with matrix $A$ (for the standard bases). Label each of the following as true or false.
\begin{enumerate}[(a)]
\item The columns of $A$ form a basis for $\IR^m$
\item $\RREF(A)$ is the identity matrix.
\item The system of linear equations given by the augmented matrix $\begin{bmatrix}[c|c] A & \vec{b} \end{bmatrix}$ has exactly one solution
for all \(\vect b \in \IR^m\).
\end{enumerate}
\end{activity}

\begin{activity}{10}
Let $T: \IR^2 \rightarrow \IR^3$ be given by $$T\left(\begin{bmatrix} x \\ y \end{bmatrix} \right) = \begin{bmatrix} 2x+3y \\ x-y \\ x+3y\end{bmatrix}.$$  Which of the following must be true?
\begin{enumerate}[(a)]
\item $T$ is neither injective nor surjective
\item $T$ is injective but not surjective
\item $T$ is surjective but not injective
\item $T$ is bijective.
\end{enumerate}
\end{activity}

\begin{activity}{5}
Let $T: \IR^3 \rightarrow \IR^2$ be given by $$T\left(\begin{bmatrix} x \\ y  \\ z \end{bmatrix} \right) = \begin{bmatrix} 2x+y-z \\ 4x+y+z\end{bmatrix}.$$  Which of the following must be true?
\begin{enumerate}[(a)]
\item $T$ is neither injective nor surjective
\item $T$ is injective but not surjective
\item $T$ is surjective but not injective
\item $T$ is bijective.
\end{enumerate}
\end{activity}

\begin{activity}{5}
Let $T: \IR^3 \rightarrow \IR^3$ be given by $$T\left(\begin{bmatrix} x \\ y  \\ z \end{bmatrix} \right) = \begin{bmatrix} 2x+y-z \\ 4x+y+z \\ 6x+2y+z\end{bmatrix}.$$ Which of the following must be true?
\begin{enumerate}[(a)]
\item $T$ is neither injective nor surjective
\item $T$ is injective but not surjective
\item $T$ is surjective but not injective
\item $T$ is bijective.
\end{enumerate}
\end{activity}

\begin{activity}{5}
Let $T: \IR^3 \rightarrow \IR^3$ be given by $$T\left(\begin{bmatrix} x \\ y  \\ z \end{bmatrix} \right) = \begin{bmatrix} 2x+y-z \\ 4x+y+z \\ 6x+2y\end{bmatrix}.$$   Which of the following must be true?
\begin{enumerate}[(a)]
\item $T$ is neither injective nor surjective
\item $T$ is injective but not surjective
\item $T$ is surjective but not injective
\item $T$ is bijective.
\end{enumerate}
\end{activity}


\end{applicationActivities}
