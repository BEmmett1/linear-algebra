%!TEX root =../../course-notes.tex
% ^ leave for LaTeXTools build functionality

\begin{applicationActivities}{1}{17}

\begin{definition}
A \term{linear transformation} is a map between vector spaces that preserves the vector space operations.  More precisely, if $V$ and $W$ are vector spaces, a map $T:V\rightarrow W$ is called a linear transformation if
\begin{enumerate}
\item $T(\vec{v}+\vec{w}) = T(\vec{v})+T(\vec{w})$ for any $\vec{v},\vec{w} \in V$
\item $T(c\vec{v}) = cT(\vec{v})$ for any $c \in \IR$, $\vec{v} \in V$.
\end{enumerate}
In other words, a map is linear if one can do vector space operations before applying the map or after, and obtain the same answer.

$V$ is called the \term{domain} of $T$ and $W$ is called the \term{co-domain} of $T$.
\end{definition}

\begin{activity}{15}
Determine if each of the following maps are linear transformations
\begin{enumerate}[(a)]
\item $T_1 : \IR^2 \rightarrow \IR$ given by $T_1\left(\begin{bmatrix} a \\ b \end{bmatrix} \right) = \sqrt{a^2+b^2}$
\item $T_2 : \IR^3 \rightarrow \IR^2$ given by $T_2\left(\begin{bmatrix} x \\ y \\ z \end{bmatrix} \right) = \begin{bmatrix} x-z \\ y \end{bmatrix}$
\item $T_3: \P_d \rightarrow \P_{d-1}$ given by $T_3(f(x)) = f^\prime(x)$.
\item $T_4: C(\IR) \rightarrow C(\IR)$ given by $T_4(f(x)) = f(-x)$
\item $T_5: \P \rightarrow \P$ given by $T_5(f(x)) = f(x)+x^2$
\end{enumerate}
\end{activity}

\begin{activity}{15}
Suppose $T: \IR^3 \rightarrow \IR^2$ is a linear transformation, and you know $T\left(\begin{bmatrix} 1 \\ 0 \\ 0 \end{bmatrix} \right) = \begin{bmatrix} 2 \\ 1 \end{bmatrix} $ and $T\left(\begin{bmatrix} 0 \\ 0 \\ 1 \end{bmatrix} \right) = \begin{bmatrix} -3 \\ 2 \end{bmatrix} $.  Compute each of the following:
\begin{enumerate}[(a)]
\item $T\left(\begin{bmatrix} 3 \\ 0 \\ 0 \end{bmatrix}\right)$
\item $T\left(\begin{bmatrix} 0 \\ 0 \\ -2 \end{bmatrix}\right)$
\item $T\left(\begin{bmatrix} 1 \\ 0 \\ 1 \end{bmatrix}\right)$
\item $T\left(\begin{bmatrix} -2 \\ 0 \\ 5 \end{bmatrix}\right)$
\end{enumerate}
\end{activity}

\begin{activity}{5}
Suppose $T: \IR^4 \rightarrow \IR^3$ is a linear transformation.  What is the smallest number of vectors needed to determine $T$?  In other words, what is the smallest number $n$ such that there are $\vec{v}_1,\ldots,\vec{v}_n \in \IR^4$ and given  $T(\vec{v}_1), \ldots, T(\vec{v}_n)$ you can determine $T(\vec{w})$ for \textit{any} $\vec{w} \in \IR^2$?
\end{activity}

\begin{observation}
Fix an ordered basis for $V$.  Since every vector can be written \textit{uniquely} as a linear combination of basis vectors, a linear transformation $T:V \rightarrow W$ corresponds exactly to a choice of where each basis vector goes.  For convenience, we can thus encode a linear transformation as a matrix, with one column for the image of each basis vector (in order).
\end{observation}

\begin{example}
Let $T: \IR^3 \rightarrow \IR^2$ be a linear transformation with
\begin{align*}
T\left(\begin{bmatrix} 1 \\ 0 \\ 0 \end{bmatrix} \right) &= \begin{bmatrix} 3 \\ 2\end{bmatrix} &
T\left(\begin{bmatrix} 0 \\ 1 \\ 0 \end{bmatrix} \right) &= \begin{bmatrix} -1 \\ 4\end{bmatrix} &
T\left(\begin{bmatrix} 0 \\ 0 \\ 1 \end{bmatrix} \right) &= \begin{bmatrix} 5 \\ 0\end{bmatrix}
\end{align*}

Then the matrix corresponding to $T$ with respect to the standard bases is $$\begin{bmatrix}3 & -1 & 5 \\ 2 & 4 & 0 \end{bmatrix}.$$
\end{example}

\begin{activity}{10}
  Let $T: \IR^3 \rightarrow \IR^2$ be a linear transformation with
\begin{align*}
T\left(\begin{bmatrix} 1 \\ 0 \\ 0 \end{bmatrix} \right) &= \begin{bmatrix} 3 \\ 2\end{bmatrix} &
T\left(\begin{bmatrix} 0 \\ 1 \\ 0 \end{bmatrix} \right) &= \begin{bmatrix} -1 \\ 4\end{bmatrix} &
T\left(\begin{bmatrix} 0 \\ 0 \\ 1 \end{bmatrix} \right) &= \begin{bmatrix} 5 \\ 0\end{bmatrix}
\end{align*}
Write the matrix corresponding to this linear transformation with respect to the ordered basis
\[\left\{ \begin{bmatrix} 2 \\ 1 \\ 1 \end{bmatrix} , \begin{bmatrix} -1 \\ -1 \\ 3 \end{bmatrix} , \begin{bmatrix} 0 \\ 1 \\ 2 \end{bmatrix} \right\}\] for $\IR^3$ and the standard basis for $\IR^2$.
\end{activity}

\begin{activity}{5}
Let $D: \P^3 \rightarrow \P^2$ be the derivative map (recall this is a linear transformation).  Write the matrix corresponding to $D$ with respect to the ordered basis $\{1,x,x^2,x^3\}$ for $\P^3$ and $\{1,x,x^2\}$ for $\P^2$.
\end{activity}

\end{applicationActivities}
