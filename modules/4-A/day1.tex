%!TEX root =../../course-notes.tex
% ^ leave for LaTeXTools build functionality

\begin{applicationActivities}{1}{17}

\begin{definition}
A \term{linear transformation} is a map between vector spaces that preserves the vector space operations.  More precisely, if $V$ and $W$ are vector spaces, a map $T:V\rightarrow W$ is called a linear transformation if
\begin{enumerate}
\item $T(\vec{v}+\vec{w}) = T(\vec{v})+T(\vec{w})$ for any $\vec{v},\vec{w} \in V$
\item $T(c\vec{v}) = cT(\vec{v})$ for any $c \in \IR$, $\vec{v} \in V$.
\end{enumerate}
\ \\
\ \\
In other words, a map is linear if one can do vector space operations before applying the map or after, and obtain the same answer. \\

\ \\
\ \\

$V$ is called the \term{domain} of $T$ and $W$ is called the \term{co-domain} of $T$.
\end{definition}

\begin{example}
Let $T : \IR^3 \rightarrow \IR^2$ be given by $$T\left(\begin{bmatrix} x \\ y \\ z \end{bmatrix} \right) = \begin{bmatrix} x-z \\ y \end{bmatrix}$$

\pause We check: 
\begin{align*}
T\left(\begin{bmatrix} x_1 \\ y_1 \\ z_1 \end{bmatrix} + \begin{bmatrix}x_2 \\ y_2  \\ z_2 \end{bmatrix} \right) &= T\left(\begin{bmatrix} x_1+x_2 \\ y_1+y_2 \\ z_1+z_2 \end{bmatrix} \right) = \begin{bmatrix} (x_1+x_2)-(z_1+z_2) \\ (y_1+y_2) \end{bmatrix} \\
T\left(\begin{bmatrix} x_1 \\ y_1 \\ z_1 \end{bmatrix} \right) + T\left(\begin{bmatrix} x_2 \\ y_2 \\ z_2 \end{bmatrix} \right) &=  \begin{bmatrix} x_1-z_1 \\ y_1 \end{bmatrix}+\begin{bmatrix} x_2-z_2 \\ y_2 \end{bmatrix} 
\end{align*}
\pause
$$T\left(c\begin{bmatrix} x \\ y \\ z \end{bmatrix} \right) = \begin{bmatrix} cx-cz \\ cy \end{bmatrix} = c\begin{bmatrix} x-z \\ y \end{bmatrix} = cT\left(\begin{bmatrix} x \\ y \\ z \end{bmatrix} \right)$$

Therefore $T$ is a linear transformation.
\end{example}


\begin{activity}{15}
Determine if each of the following maps are linear transformations
\begin{subactivity}
$T_1: \IR^2 \rightarrow \IR$ given by $T_1 \left(\begin{bmatrix} x \\ y \end{bmatrix} \right) = \sqrt{x^2+y^2}$.
\end{subactivity}
\begin{subactivity}
$T_2: \IR^2 \rightarrow \IR^3 $ given by $T_2\left(\begin{bmatrix} x \\ y \\ z \end{bmatrix} \right)  = \begin{bmatrix} -x \\ -y \\ -z \end{bmatrix}$
\end{subactivity}
\begin{subactivity}
$T_3: \P^d \rightarrow \P^{d-1}$ given by $T_3(f(x)) = f^\prime(x)$.
\end{subactivity}
\begin{subactivity}
$T_4: \P \rightarrow \P$ given by $T_4(f(x)) = f(x)+x^2$
\end{subactivity}
\end{activity}

\begin{activity}{5}
Suppose $T: \IR^3 \rightarrow \IR^2$ is a linear transformation, and you know $T\left(\begin{bmatrix} 1 \\ 0 \\ 0 \end{bmatrix} \right) = \begin{bmatrix} 2 \\ 1 \end{bmatrix} $ and $T\left(\begin{bmatrix} 0 \\ 0 \\ 1 \end{bmatrix} \right) = \begin{bmatrix} -3 \\ 2 \end{bmatrix} $.  Compute $T\left(\begin{bmatrix} 3 \\ 0 \\ 0 \end{bmatrix}\right)$.
\begin{multicols}{2}
\begin{enumerate}[(a)]
\item $\begin{bmatrix} 6 \\ 2\end{bmatrix}$
\item $\begin{bmatrix} -9 \\ 6 \end{bmatrix}$
\item $\begin{bmatrix} -4 \\ -2 \end{bmatrix}$
\item $\begin{bmatrix} 6 \\ 4 \end{bmatrix}$
\end{enumerate}
\end{multicols}
\end{activity}

\begin{activity}{3}
Suppose $T: \IR^3 \rightarrow \IR^2$ is a linear transformation, and you know $T\left(\begin{bmatrix} 1 \\ 0 \\ 0 \end{bmatrix} \right) = \begin{bmatrix} 2 \\ 1 \end{bmatrix} $ and $T\left(\begin{bmatrix} 0 \\ 0 \\ 1 \end{bmatrix} \right) = \begin{bmatrix} -3 \\ 2 \end{bmatrix} $.  Compute $T\left(\begin{bmatrix} 0 \\ 0 \\ -2 \end{bmatrix}\right)$.
\begin{multicols}{2}
\begin{enumerate}[(a)]
\item $\begin{bmatrix} 6 \\ 2\end{bmatrix}$
\item $\begin{bmatrix} -9 \\ 6 \end{bmatrix}$
\item $\begin{bmatrix} -4 \\ -2 \end{bmatrix}$
\item $\begin{bmatrix} 6 \\ 4 \end{bmatrix}$
\end{enumerate}
\end{multicols}
\end{activity}

\begin{activity}{5}
Suppose $T: \IR^3 \rightarrow \IR^2$ is a linear transformation, and you know $T\left(\begin{bmatrix} 1 \\ 0 \\ 0 \end{bmatrix} \right) = \begin{bmatrix} 2 \\ 1 \end{bmatrix} $ and $T\left(\begin{bmatrix} 0 \\ 0 \\ 1 \end{bmatrix} \right) = \begin{bmatrix} -3 \\ 2 \end{bmatrix} $.  Compute $T\left(\begin{bmatrix} 1 \\ 0 \\ 1 \end{bmatrix}\right)$.
\begin{multicols}{2}
\begin{enumerate}[(a)]
\item $\begin{bmatrix} 2 \\ 1\end{bmatrix}$
\item $\begin{bmatrix} 3 \\ -1 \end{bmatrix}$
\item $\begin{bmatrix} -1 \\ 3 \end{bmatrix}$
\item $\begin{bmatrix} 5 \\ -8 \end{bmatrix}$
\end{enumerate}
\end{multicols}
\end{activity}

\begin{activity}{2}
Suppose $T: \IR^3 \rightarrow \IR^2$ is a linear transformation, and you know $T\left(\begin{bmatrix} 1 \\ 0 \\ 0 \end{bmatrix} \right) = \begin{bmatrix} 2 \\ 1 \end{bmatrix} $ and $T\left(\begin{bmatrix} 0 \\ 0 \\ 1 \end{bmatrix} \right) = \begin{bmatrix} -3 \\ 2 \end{bmatrix} $.  Compute $T\left(\begin{bmatrix} -2 \\ 0 \\ -3 \end{bmatrix}\right)$.
\begin{multicols}{2}
\begin{enumerate}[(a)]
\item $\begin{bmatrix} 2 \\ 1\end{bmatrix}$
\item $\begin{bmatrix} 3 \\ -1 \end{bmatrix}$
\item $\begin{bmatrix} -1 \\ 3 \end{bmatrix}$
\item $\begin{bmatrix} 5 \\ -8 \end{bmatrix}$
\end{enumerate}
\end{multicols}
\end{activity}

\begin{activity}{5}
Suppose $T: \IR^4 \rightarrow \IR^3$ is a linear transformation.  What is the smallest number of vectors needed to determine $T$?  \\ In other words, what is the smallest number $n$ such that there are $\vec{v}_1,\ldots,\vec{v}_n \in \IR^4$ and given  $T(\vec{v}_1), \ldots, T(\vec{v}_n)$ you can determine $T(\vec{w})$ for \textit{any} $\vec{w} \in \IR^4$?
\begin{enumerate}[(a)]
\item $2$
\item $3$
\item $4$
\item $5$
\item You need infinitely many
\end{enumerate}
\end{activity}

\begin{observation}
Fix an ordered basis for $V$.  Since every vector can be written \textit{uniquely} as a linear combination of basis vectors, a linear transformation $T:V \rightarrow W$ corresponds exactly to a choice of where each basis vector goes.  For convenience, we can thus encode a linear transformation as a matrix, with one column for the image of each basis vector (in order).
\end{observation}

\begin{definition}
The {\bf standard basis} of $\IR^n$ is the (ordered) basis $\{\vec{e}_1, \vec{e}_2, \ldots, \vec{e}_n\}$ where
\begin{align*}
\vec{e}_1 &= \begin{bmatrix} 1 \\ 0 \\ 0 \\\vdots \\ 0 \\ 0 \end{bmatrix}  & 
\vec{e}_2 &= \begin{bmatrix} 0 \\ 1 \\ 0 \\\vdots \\ 0 \\ 0 \end{bmatrix}  & \cdots  & & 
\vec{e}_n &= \begin{bmatrix} 0 \\ 0 \\ 0 \\\vdots \\ 0 \\ 1 \end{bmatrix} 
\end{align*}
\end{definition}


\begin{example}
Let $T: \IR^3 \rightarrow \IR^2$ be a linear transformation with
\begin{align*}
T\left(\begin{bmatrix} 1 \\ 0 \\ 0 \end{bmatrix} \right) &= \begin{bmatrix} 3 \\ 2\end{bmatrix} &
T\left(\begin{bmatrix} 0 \\ 1 \\ 0 \end{bmatrix} \right) &= \begin{bmatrix} -1 \\ 4\end{bmatrix} &
T\left(\begin{bmatrix} 0 \\ 0 \\ 1 \end{bmatrix} \right) &= \begin{bmatrix} 5 \\ 0\end{bmatrix}
\end{align*}

Then the matrix corresponding to $T$ with respect to the standard bases is $$\begin{bmatrix}3 & -1 & 5 \\ 2 & 4 & 0 \end{bmatrix}.$$
\end{example}

\begin{activity}{5}
  Let $T: \IR^3 \rightarrow \IR^2$ be the linear transformation given by
$$T\left(\begin{bmatrix} x\\ y \\ z \end{bmatrix} \right) = \begin{bmatrix} x+3z \\ 2x-y-4z \end{bmatrix}$$
Write the matrix corresponding to this linear transformation with respect to standard basis.
\end{activity}

\begin{activity}{5}
  Let $T: \IR^3 \rightarrow \IR^2$ be the linear transformation given by the matrix (with respect to the standard bases) $$\begin{bmatrix} 3  & -2 & -1  \\ 4 & 5 & 2 \end{bmatrix}.$$ 

Compute $T\left(\begin{bmatrix} x\\ y \\ z \end{bmatrix} \right) $.
\end{activity}

\begin{activity}{10}
Let $D: \P^3 \rightarrow \P^2$ be the derivative map (recall this is a linear transformation).  
\begin{subactivity}
Write down a  corresponding linear transformation $T: \IR^4 \rightarrow \IR^3$ 
\end{subactivity}
\begin{subactivity}
Write the matrix corresponding to $T$ with respect to the standard bases.
\end{subactivity}
\end{activity}

\end{applicationActivities}
