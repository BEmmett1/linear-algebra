%!TEX root =../../course-notes.tex
% ^ leave for LaTeXTools build functionality

\begin{applicationActivities}{4}{28}

\begin{activity}{5}
  If $A$ is a $4 \times 4$ matrix, what is the largest number of eigenvalues $A$ can have?
  \begin{enumerate}[(a)]
  \item $3$
  \item $4$
  \item $5$
  \item $6$
  \item It can have infinitely many
  \end{enumerate}
\end{activity}

\begin{activity}{10}
  $2$ is an eigenvalue of each of the matrices $A=\begin{bmatrix} 1 & -2 & 1 \\ -1 & 0 & 1 \\ -1 & -2 & 3\end{bmatrix}$ and $B=\begin{bmatrix} -3 & -9 & 5 \\ -2 & -2 & 2 \\ -7 & -13 & 9 \end{bmatrix}$.

  Compute the eigenspace associated to $2$ for both $A$ and $B$.
\end{activity}

\begin{definition}

\begin{itemize}
\item The \term{algebraic multiplicity} of an eigenvalue is its multiplicity as a root of the characteristic polynomial.
\item The \term{geometric multiplicity} of an eigenvalue is the dimension of the eigenspace.
\end{itemize}

\end{definition}

\begin{activity}{10} How are the algebraic and geometric multiplicities related?
\begin{enumerate}[(a)]
\item The algebraic multiplicity is always at least as big as than the geometric multiplicity.
\item The geometric multiplicity is always at least as big as the algebraic multiplicity.
\item Sometimes the algebraic multiplicity is larger and sometimes the geometric multiplicity is larger.
\end{enumerate}
\end{activity}

\begin{activity}{15}
   Find the eigenvalues, along with both their algebraic and geometric multiplicities, for the matrix $\begin{bmatrix} -3 & 1 & 2 & 1 \\ -9 & 5 & -2 & -1 \\ 31 & -17 & 6 & 3 \\ -69 & 39 & -18 & -9 \end{bmatrix}$
\end{activity}

\begin{activity}{10}
Let  $A=\begin{bmatrix}0 & -1 \\ 1 & 0 \end{bmatrix}$.
\begin{subactivity}
  Find the eigenvalues of $A$
  \end{subactivity}
  \begin{subactivity}
   Describe what this linear transformation is doing geometrically; draw a picture.
   \end{subactivity}
\end{activity}



\end{applicationActivities}
