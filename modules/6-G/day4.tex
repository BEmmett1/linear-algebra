%!TEX root =../../course-notes.tex
% ^ leave for LaTeXTools build functionality

\begin{applicationActivities}{4}{28}
\begin{observation}
Recall from last class:
\begin{itemize}
\item To find the eigenvalues of a matrix $A$, we need to find values of $\lambda$ such that $A-\lambda I$ has a nontrivial kernel. Equivalently,
we want values where $A-\lambda I$ is not invertible, so we want to know
the values of \(\lambda\) where $\det(A-\lambda I)=0$.
\item $\det(A-\lambda I)$ is a polynomial with variable \(\lambda\),
called the \term{characteristic polynomial} of $A$. Thus the roots of
the characteristic polynomial of \(A\) are exactly the eigenvalues of \(A\).
\item Once an eigenvalue \(\lambda\) is found, the \term{eigenspace}
containing all \term{eigenvectors} \(\vec x\) satisfying
\(A\vec x=\lambda\vec x\) is given by $\ker(A-\lambda I)$.
\end{itemize}
\end{observation}

\begin{activity}{5}
  If $A$ is a $4 \times 4$ matrix, what is the largest number of eigenvalues $A$ can have?
  \begin{enumerate}[(a)]
  \item $3$
  \item $4$
  \item $5$
  \item $6$
  \item It can have infinitely many
  \end{enumerate}
\end{activity}

\begin{activity}{5}
  $2$ is an eigenvalue of the matrix $A=\begin{bmatrix} 1 & -2 & 1 \\ -1 & 0 & 1 \\ -1 & -2 & 3\end{bmatrix}$.

  Compute the eigenspace of $A$ associated to the eigenvalue $2$ by
  solving for the kernel of
  \[
    A-2I
      =
    \begin{bmatrix}
      1-2 & -2 & 1 \\
      -1 & 0-2 & 1 \\
      -1 & -2 & 3-2
    \end{bmatrix}
      =
    \begin{bmatrix}
      -1 & -2 & 1 \\
      -1 & -2 & 1 \\
      -1 & -2 & 1
    \end{bmatrix}
  \]
\end{activity}

\begin{activity}{5}
  $2$ is an eigenvalue of the matrix $B=\begin{bmatrix} -3 & -9 & 5 \\ -2 & -2 & 2 \\ -7 & -13 & 9 \end{bmatrix}$.

  Compute the eigenspace of $B$ associated to the eigenvalue $2$ by
  solving for the kernel of \(B-2I\).
\end{activity}

\begin{definition}

\begin{itemize}
\item The \term{algebraic multiplicity} of an eigenvalue is its multiplicity as a root of the characteristic polynomial.
\item The \term{geometric multiplicity} of an eigenvalue is the dimension of the eigenspace.
\end{itemize}

\end{definition}

\begin{fact}
  The geometric multiplicity of an eigenvalue cannot exceed its
  algebraic multiplicity (but it \textit{can} be different).
\end{fact}

% \begin{activity}{5} How are the algebraic and geometric multiplicities related?
% \begin{enumerate}[(a)]
% \item The algebraic multiplicity is always at least as big as than the geometric multiplicity.
% \item The geometric multiplicity is always at least as big as the algebraic multiplicity.
% \item Sometimes the algebraic multiplicity is larger and sometimes the geometric multiplicity is larger.
% \end{enumerate}
% \end{activity}

\begin{activity}{20}
   Find all of the eigenvalues, along with both their algebraic and geometric multiplicities, for the matrix $\begin{bmatrix} -3 & 1 & 2 & 1 \\ -9 & 5 & -2 & -1 \\ 31 & -17 & 6 & 3 \\ -69 & 39 & -18 & -9 \end{bmatrix}$.  Use technology to help you!
\end{activity}



\begin{activity}{10}
Let  $A=\begin{bmatrix}0 & -1 \\ 1 & 0 \end{bmatrix}$.
\begin{subactivity}
  Find the eigenvalues of $A$
  \end{subactivity}
  \begin{subactivity}
   Describe what this linear transformation is doing geometrically; draw a picture.
   \end{subactivity}
\end{activity}



\end{applicationActivities}
