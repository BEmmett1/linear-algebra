%!TEX root =../../course-notes.tex
% ^ leave for LaTeXTools build functionality

\begin{applicationActivities}{3}{6}

\begin{definition}
  An algorithm that reduces \(A\) to \(\RREF(A)\) is called
  \term{Gauss-Jordan elimination}. For example:
  \begin{multicols}{3}
    \begin{enumerate}
      \item Circle the top-left-most cell that (a) is below any existing pivot
      positions and (b) has a nonzero term either in that position or below it.
      \item Ignoring any rows above this pivot position, use row operations
      to change the value of your pivot position to \(1\), and the terms below
      it to \(0\).
      \item Repeat these two steps as often as possible.
      \item Finally,
      zero out any terms above pivot positions.
    \end{enumerate}
  \end{multicols}
\end{definition}

\begin{observation}
  Here is an example of applying Gauss-Jordan elimination to a matrix:
  \[
    \begin{bmatrix}[ccc|c]
      \circledNumber{3} & -2 & 13 & 6 \\
      2 & -2 & 10 & 2 \\
      -1 & 3 & -6 & 11
    \end{bmatrix}\sim
    \begin{bmatrix}[ccc|c]
     \circledNumber{2} & -2 & 10 & 2 \\
      3 & -2 & 13 & 6 \\
      -1 & 3 & -6 & 11
    \end{bmatrix}\sim
    \begin{bmatrix}[ccc|c]
     \circledNumber{1} & -1 & 5 & 1 \\
      3 & -2 & 13 & 6 \\
      -1 & 3 & -6 & 11
    \end{bmatrix}
  \]
  \[\sim
    \begin{bmatrix}[ccc|c]
     \circledNumber{1} & -1 & 5 & 1 \\
      0 & \circledNumber{1} & -2 & 3 \\
      0 & 2 & -1 & 12
    \end{bmatrix}
    \sim
      \begin{bmatrix}[ccc|c]
       \circledNumber{1} & -1 & 5 & 1 \\
        0 & \circledNumber{1} & -2 & 3 \\
        0 & 0 & \circledNumber{3} & 6
      \end{bmatrix}
    \sim
    \begin{bmatrix}[ccc|c]
     \circledNumber{1} & -1 & 5 & 1 \\
      0 & \circledNumber{1} & -2 & 3 \\
      0 & 0 & \circledNumber{1} & 2
    \end{bmatrix}
  \]
  \[\sim
    \begin{bmatrix}[ccc|c]
     \circledNumber{1} & -1 & 5 & 1 \\
      0 & \circledNumber{1} & -2 & 3 \\
      0 & 0 & \circledNumber{1} & 2
    \end{bmatrix}
    \sim
    \begin{bmatrix}[ccc|c]
     \circledNumber{1} & -1 & 0 & -9 \\
      0 & \circledNumber{1} & 0 & 7 \\
      0 & 0 & \circledNumber{1} & 2
    \end{bmatrix}
    \sim
    \begin{bmatrix}[ccc|c]
     \circledNumber{1} & 0 & 0 & -2 \\
      0 & \circledNumber{1} & 0 & 7 \\
      0 & 0 & \circledNumber{1} & 2
    \end{bmatrix}
  \]
\end{observation}

\begin{activity}{15}
  Find \(\RREF(A)\) where
  \[A=
    \begin{bmatrix}[cccc|c]
      -1 &  1 & -3 &  2 &  0 \\
       2 & -1 &  5 &  3 & -11 \\
       3 &  2 &  4 &  1 &  1 \\
       0 &  1 & -1 &  1 &  1 \\
    \end{bmatrix}
  .\]
\end{activity}

\begin{definition}
  The columns of \(\RREF(A)\) without a leading term represent
  \term{free variables} of the linear system modeled by \(A\)
  that may be set equal to arbitrary parameters.
  The other \term{bounded variables} can then be expressed in terms
  of those parameters to describe the solution set
  to the linear system modeled by \(A\).
\end{definition}

\begin{activity}{10}
  Given the linear system and its equivalent row-reduced matrix
  \begin{multicols}{2}\noindent
    \begin{alignat*}{5}
      -x_1 &\,+\,&  x_2 &\,-\,&  3x_3 &\,+\,&  2x_4 &\,=\,& 0 \\
      2x_1 &\,-\,&  x_2 &\,+\,&  5x_3 &\,+\,&  3x_4 &\,=\,& -11 \\
      3x_1 &\,+\,& 2x_2 &\,+\,&  4x_3 &\,+\,&   x_4 &\,=\,& 1 \\
           &\, \,&  x_2 &\,-\,&   x_3 &\,+\,&   x_4 &\,=\,& 1 \\
    \end{alignat*}
  \[
    \begin{bmatrix}[cccc|c]
       1 &  0 &  2 &  0 & -1 \\
       0 &  1 & -1 &  0 &  3 \\
       0 &  0 &  0 &  1 & -2 \\
       0 &  0 &  0 &  0 &  0 \\
    \end{bmatrix}
  \]
  \end{multicols}
  circle the pivot positions and describe the solution set
  \(
    \begin{bmatrix}
      x_1 \\
      x_2 \\
      x_3 \\
      x_4
    \end{bmatrix}=
    \begin{bmatrix}
      p_1 \\
      p_2 \\
      p_3 \\
      p_4
    \end{bmatrix}
    +a\begin{bmatrix}
      s_1 \\
      s_2 \\
      s_3 \\
      s_4
    \end{bmatrix}
  \) by setting the free variable (the column without a pivot position)
  equal to \(a\), and expressing each of the other
  bounded variables equal to an expression in terms of \(a\).
\end{activity}

\begin{remark}
  It's not necessary to completely find \(\RREF(A)\) to
  deduce that a linear system is inconsistent.
\end{remark}

\begin{activity}{10} %TODO remove matrices, focus on augmented matrix
  Find a contradiction in the inconsistent linear system
    \begin{alignat*}{4}
      2x_1 &\,-\,& 3x_2 &\,=\,& 17 \\
       x_1 &\,+\,& 2x_2 &\,=\,& -2 \\
      -x_1 &\,-\,&  x_2 &\,=\,& 1
    \end{alignat*}
  by considering the following equivalent augmented matrices:
  \[
    \begin{bmatrix}[cc|c]
       2 & -3 & 17 \\
       1 &  2 & -2 \\
      -1 & -1 &  1 \\
    \end{bmatrix}\sim
    \begin{bmatrix}[cc|c]
       1 &  2 & -2 \\
       0 &  1 &  3 \\
       0 &  0 &  2 \\
    \end{bmatrix}
  .\]
\end{activity}

\begin{activity}{5}
  Show that all linear systems of the form
  \begin{alignat*}{5}
    a_{11}x_1 &\,+\,& a_{12}x_2 &\,+\,& \dots  &\,+\,& a_{1n}x_n &\,=\,& 0 \\
    a_{21}x_1 &\,+\,& a_{22}x_2 &\,+\,& \dots  &\,+\,& a_{2n}x_n &\,=\,& 0 \\
     \vdots&  &\vdots&   &&  &\vdots&&\vdots  \\
    a_{m1}x_1 &\,+\,& a_{m2}x_2 &\,+\,& \dots  &\,+\,& a_{mn}x_n &\,=\,& 0
  \end{alignat*}
  are consistent by finding a
  quickly verifiable solution.
\end{activity}

\begin{definition}
  A \term{homogeneous system} is a linear system satisfying \(b_i=0\), that is,
  it is a linear system of the form
  \begin{alignat*}{5}
    a_{11}x_1 &\,+\,& a_{12}x_2 &\,+\,& \dots  &\,+\,& a_{1n}x_n &\,=\,& 0 \\
    a_{21}x_1 &\,+\,& a_{22}x_2 &\,+\,& \dots  &\,+\,& a_{2n}x_n &\,=\,& 0 \\
     \vdots&  &\vdots&   &&  &\vdots&&\vdots  \\
    a_{m1}x_1 &\,+\,& a_{m2}x_2 &\,+\,& \dots  &\,+\,& a_{mn}x_n &\,=\,& 0
  \end{alignat*}
\end{definition}

\begin{fact}
  Because the zero vector is always a solution,
  the solution set to any homogeneous system with infinitely-many solutions
  may be generated by multiplying the parameters representing the free variables
  by a minimal set of Euclidean vectors, and adding these up. For example:
  \[
    \begin{bmatrix}
      x_1 \\
      x_2 \\
      x_3 \\
      x_4
    \end{bmatrix}=
    a\begin{bmatrix}
      3 \\
      1 \\
      -1 \\
      0
    \end{bmatrix}+
    b\begin{bmatrix}
      0 \\
      0 \\
      0 \\
      1
    \end{bmatrix}
  \]
\end{fact}

\begin{definition}
  A minimal set of Euclidean vectors generating the solution set to a
  homogeneous system is called a \textbf{basis} for the solution
  set of the homogeneous system. For example:
  \begin{multicols}{2}\noindent
  \[
    \begin{bmatrix}
      x_1 \\
      x_2 \\
      x_3 \\
      x_4
    \end{bmatrix}=
    a\begin{bmatrix}
      3 \\
      1 \\
      -1 \\
      0
    \end{bmatrix}+
    b\begin{bmatrix}
      0 \\
      0 \\
      0 \\
      1
    \end{bmatrix}
  \]
  \[
    \textrm{Basis}=\left\{
    \begin{bmatrix}
      3 \\
      1 \\
      -1 \\
      0
    \end{bmatrix},
    \begin{bmatrix}
      0 \\
      0 \\
      0 \\
      1
    \end{bmatrix}\right\}
  \]
  \end{multicols}
\end{definition}

\begin{activity}{10}
  Find a basis for the solution set of the following homogeneous linear
  system.
  \begin{alignat*}{5}
    x_1 &\,+\,& 2x_2 &\, \,&     &\,-\,&  x_4 &\,=\,& 0 \\
        &\, \,&      &\, \,& x_3 &\,+\,& 4x_4 &\,=\,& 0 \\
   2x_1 &\,+\,& 4x_2 &\,+\,& x_3 &\,+\,& 2x_4 &\,=\,& 0 \\
  \end{alignat*}
\end{activity}




\end{applicationActivities}
