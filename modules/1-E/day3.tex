%!TEX root =../../course-notes.tex
% ^ leave for LaTeXTools build functionality

\begin{applicationActivities}{3}{5}

\begin{definition}
  An algorithm that reduces \(A\) to \(\RREF(A)\) is called
  \term{Gauss-Jordan elimination}. For example:
  \begin{enumerate}
    \item Circle the cell that
      (a) is in the top-most row without a pivot position and
      (b) is in the left-most column with a nonzero term either in that
          position or below it.
      This position (not the number inside) is called a \term{pivot}.
  	\item Change the pivot's value to \(1\) by using row operations involving
      only the pivot row and rows below it.
  	\item Add or subtract multiples of the pivot row to zero out above and below the pivot.
    \item Return to Step 1 and repeat as needed until the matrix is
    in row reduced echelon form.
  \end{enumerate}
\end{definition}

\begin{observation}
  Here is an example of applying Gauss-Jordan elimination to a matrix:

\begin{small}
\begin{align*}
\begin{bmatrix}[cccc|c]
\circledNumber{2} & -2 & -6 & 1 & 3 \\
-1 & 1 & 3 & -1 & -3 \\
1 & -2 & -1 & 1 & 2
\end{bmatrix}
&\sim
\begin{bmatrix}[cccc|c]
\circledNumber{1} & -2 & -1 & 1 & 2 \\
-1 & 1 & 3 & -1 & -3 \\
2 & -2 & -6 & 1 & 3
\end{bmatrix} \sim
\begin{bmatrix}[cccc|c]
\circledNumber{1} & -2 & -1 & 1 & 2 \\
0 & \circledNumber{-1} & 2 & 0 & -1 \\
0 & 2 & -4 & -1 & -1
\end{bmatrix} \\
&\sim
\begin{bmatrix}[cccc|c]
\circledNumber{1} & -2 & -1 & 1 & 2 \\
0 & \circledNumber{1} & -2 & 0  & 1 \\
0 & 2 & -4 & -1 & -1
\end{bmatrix}
\sim
\begin{bmatrix}[cccc|c]
\circledNumber{1} & 0 & -5 & 1 & 4 \\
0 & \circledNumber{1} & -2 & 0  & 1 \\
0 & 0 & 0 & \circledNumber{-1} & -3
\end{bmatrix}  \\
&\sim
\begin{bmatrix}[cccc|c]
\circledNumber{1} & 0 & -5 & 1 & 4 \\
0 & \circledNumber{1} & -2 & 0  & 1 \\
0 & 0 & 0 & \circledNumber{1} & 3
\end{bmatrix}
\sim
\begin{bmatrix}[cccc|c]
\circledNumber{1} & 0 & -5 & 0 & 1 \\
0 & \circledNumber{1} & -2 & 0  & 1 \\
0 & 0 & 0 & \circledNumber{1} & 3
\end{bmatrix}
\end{align*}
\end{small}

\end{observation}

\begin{definition}
  The columns of \(\RREF(A)\) without a leading term represent
  \term{free variables} of the linear system modeled by \(A\)
  that may be set equal to arbitrary parameters.
  The other \term{bounded variables} can then be expressed in terms
  of those parameters to describe the solution set
  to the linear system modeled by \(A\).
\end{definition}

\begin{example}
Here, \(x_3\) is the free variable set equal to \(a\) since its
column lacks a pivot, and the
other bounded variables are put in terms of \(a\).

\begin{center}
\begin{tabular}{ccccc}
$\begin{aligned}
2x_1-2x_2-6x_3+x_4 &= 3 \\
-x_1+x_2+3x_3-x_4 &= -3 \\
x_1-2x_2-x_3+x_4 &= 1
\end{aligned}$ & &
$\begin{aligned}
x_1-5x_3 &= 1 \\
x_2-2x_3 &= 1 \\
x_4 &= 3
\end{aligned}$ &\(\Rightarrow\)&
$\begin{aligned}
x_1 &= 1+5a \\
x_2 &= 1+2a \\
x_3 &= a \\
x_4 &= 3
\end{aligned}$
\\
$\Downarrow$ & & $\Uparrow$ \\
$\begin{bmatrix}[cccc|c]
{2} & -2 & -6 & 1 & 3 \\
-1 & 1 & 3 & -1 & -3 \\
1 & -2 & -1 & 1 & 2
\end{bmatrix}$  & $\sim$ &
$\begin{bmatrix}[cccc|c]
\circledNumber{1} & 0 & -5 & 0 & 1 \\
0 & \circledNumber{1} & -2 & 0  & 1 \\
0 & 0 & 0 & \circledNumber{1} & 3
\end{bmatrix}$ \\
 & & \\
\end{tabular}
\end{center}

So the solution set is $\left\{\begin{bmatrix} 1+5a \\ 1+2a \\ a \\ 3 \end{bmatrix} \bigg|\ a \in \IR \right\}$.
\end{example}

%\begin{activity}{15}
%  Find \(\RREF(A)\) where
%  \[A=
%    \begin{bmatrix}[cccc|c]
%      -1 &  1 & -3 &  2 &  0 \\
%       2 & -1 &  5 &  3 & -11 \\
%       3 &  2 &  4 &  1 &  1 \\
%       0 &  1 & -1 &  1 &  1 \\
%    \end{bmatrix}
%  .\]
%\end{activity}



\begin{activity}{20}
  Solve the system of linear equations,
  circling the pivot positions in your augmented matrices as you work.
    \begin{alignat*}{5}
      -x_1 &\,+\,&  x_2 &\,-\,&  3x_3 &\,+\,&  2x_4 &\,=\,& 0 \\
      2x_1 &\,-\,&  x_2 &\,+\,&  5x_3 &\,+\,&  3x_4 &\,=\,& -11 \\
      3x_1 &\,+\,& 2x_2 &\,+\,&  4x_3 &\,+\,&   x_4 &\,=\,& 1 \\
           &\, \,&  x_2 &\,-\,&   x_3 &\,+\,&   x_4 &\,=\,& 1 \\
    \end{alignat*}

  Remember to find the solution set of the system
  by setting the free variable (the column without a pivot position)
  equal to \(a\), and then express each of the other
  bounded variables equal to an expression in terms of \(a\).

  \begin{TBLnote}
    The resulting RREF matrix is
    \[
      \begin{bmatrix}[cccc|c]
         1 &  0 &  2 &  0 & -1 \\
         0 &  1 & -1 &  0 &  3 \\
         0 &  0 &  0 &  1 & -2 \\
         0 &  0 &  0 &  0 &  0 \\
      \end{bmatrix}
    \]
  \end{TBLnote}
\end{activity}

\begin{remark}
  From now on, unless specified, there's no need to show your work in
  finding \(\RREF(A)\), so you may use a calculator to speed up your work.
\end{remark}

\begin{activity}{10}  Solve the linear system
    \begin{alignat*}{4}
      2x_1 &\,-\,& 3x_2 &\,=\,& 17 \\
       x_1 &\,+\,& 2x_2 &\,=\,& -2 \\
      -x_1 &\,-\,&  x_2 &\,=\,& 1
    \end{alignat*}

    \begin{TBLnote} This is an inconsistent solution.  Point out to the students that one need not go all the way to RREF to discover a system is inconsistent.
    \end{TBLnote}
\end{activity}

\begin{activity}{5}
  Show that all linear systems of the form
  \begin{alignat*}{5}
    a_{11}x_1 &\,+\,& a_{12}x_2 &\,+\,& \dots  &\,+\,& a_{1n}x_n &\,=\,& 0 \\
    a_{21}x_1 &\,+\,& a_{22}x_2 &\,+\,& \dots  &\,+\,& a_{2n}x_n &\,=\,& 0 \\
     \vdots&  &\vdots&   &&  &\vdots&&\vdots  \\
    a_{m1}x_1 &\,+\,& a_{m2}x_2 &\,+\,& \dots  &\,+\,& a_{mn}x_n &\,=\,& 0
  \end{alignat*}
  are consistent by finding a
  quickly verifiable solution.
\end{activity}

\begin{definition}
  A \term{homogeneous system} is a linear system satisfying \(b_i=0\), that is,
  it is a linear system of the form
  \begin{alignat*}{5}
    a_{11}x_1 &\,+\,& a_{12}x_2 &\,+\,& \dots  &\,+\,& a_{1n}x_n &\,=\,& 0 \\
    a_{21}x_1 &\,+\,& a_{22}x_2 &\,+\,& \dots  &\,+\,& a_{2n}x_n &\,=\,& 0 \\
     \vdots&  &\vdots&   &&  &\vdots&&\vdots  \\
    a_{m1}x_1 &\,+\,& a_{m2}x_2 &\,+\,& \dots  &\,+\,& a_{mn}x_n &\,=\,& 0
  \end{alignat*}
\end{definition}

\begin{fact}
  Because the zero vector is always a solution,
  the solution set to any homogeneous system with infinitely-many solutions
  may be generated by multiplying the parameters representing the free variables
  by a minimal set of Euclidean vectors, and adding these up. For example:
  \[
    \begin{bmatrix}
      x_1 \\
      x_2 \\
      x_3 \\
      x_4
    \end{bmatrix}=
    a\begin{bmatrix}
      3 \\
      1 \\
      -1 \\
      0
    \end{bmatrix}+
    b\begin{bmatrix}
      0 \\
      0 \\
      0 \\
      1
    \end{bmatrix}
  \]
\end{fact}

\begin{definition}
  A minimal set of Euclidean vectors generating the solution set to a
  homogeneous system is called a \textbf{basis} for the solution
  set of the homogeneous system. For example:
  \begin{multicols}{2}\noindent
  \[
    \begin{bmatrix}
      x_1 \\
      x_2 \\
      x_3 \\
      x_4
    \end{bmatrix}=
    a\begin{bmatrix}
      3 \\
      1 \\
      -1 \\
      0
    \end{bmatrix}+
    b\begin{bmatrix}
      0 \\
      0 \\
      0 \\
      1
    \end{bmatrix}
  \]
  \[
    \textrm{Basis}=\left\{
    \begin{bmatrix}
      3 \\
      1 \\
      -1 \\
      0
    \end{bmatrix},
    \begin{bmatrix}
      0 \\
      0 \\
      0 \\
      1
    \end{bmatrix}\right\}
  \]
  \end{multicols}
\end{definition}

\begin{activity}{10}
  Find a basis for the solution set of the following homogeneous linear
  system.
  \begin{alignat*}{5}
    x_1 &\,+\,& 2x_2 &\, \,&     &\,-\,&  x_4 &\,=\,& 0 \\
        &\, \,&      &\, \,& x_3 &\,+\,& 4x_4 &\,=\,& 0 \\
   2x_1 &\,+\,& 4x_2 &\,+\,& x_3 &\,+\,& 2x_4 &\,=\,& 0 \\
  \end{alignat*}
\end{activity}




\end{applicationActivities}
