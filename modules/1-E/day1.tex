%!TEX root =../../course-notes.tex
% ^ leave for LaTeXTools build functionality

\begin{applicationActivities}{1}{3}



\begin{remark}
  The only important information in a linear system are its coefficients and
  constants.

  \begin{multicols}{3}\noindent
    Original linear system:
    \begin{alignat*}{2}
       x_1 + 3x_3 &\,=\,& 3 \\
      3x_1 - 2x_2 + 4x_3 &\,=\,& 0 \\
      -x_2 +  x_3 &\,=\,& -2
    \end{alignat*}
    Verbose standard form:
    \begin{alignat*}{4}
      1x_1 &\,+\,& 0x_2 &\,+\,& 3x_3 &\,=\,& 3 \\
      3x_1 &\,-\,& 2x_2 &\,+\,& 4x_3 &\,=\,& 0 \\
      0x_1 &\,-\,& 1x_2 &\,+\,& 1x_3 &\,=\,& -2
    \end{alignat*}
    Coefficients/constants:
    \begin{alignat*}{4}
       1 &     &  0 &\,\,& 3 &\,|\,& 3 \\
       3 &\, \,& -2 &\,\,& 4 &\,|\,& 0 \\
       0 &\, \,& -1 &\,\,& 1 &\,|\,& -2
    \end{alignat*}
  \end{multicols}
\end{remark}

\begin{definition}
  A system of \(m\) linear equations with \(n\) variables is often represented
  by writing its coefficients and constants in an \term{augmented matrix}.
  \begin{multicols}{2}\noindent
  \begin{alignat*}{5}
    a_{11}x_1 &\,+\,& a_{12}x_2 &\,+\,& \dots  &\,+\,& a_{1n}x_n &\,=\,& b_1 \\
    a_{21}x_1 &\,+\,& a_{22}x_2 &\,+\,& \dots  &\,+\,& a_{2n}x_n &\,=\,& b_2 \\
     \vdots&  &\vdots&   &&  &\vdots&&\vdots  \\
    a_{m1}x_1 &\,+\,& a_{m2}x_2 &\,+\,& \dots  &\,+\,& a_{mn}x_n &\,=\,& b_m
  \end{alignat*}
  \[
    \begin{bmatrix}[cccc|c]
      a_{11} & a_{12} & \cdots & a_{1n} & b_1\\
      a_{21} & a_{22} & \cdots & a_{2n} & b_2\\
      \vdots & \vdots & \ddots & \vdots & \vdots\\
      a_{m1} & a_{m2} & \cdots & a_{mn} & b_m
    \end{bmatrix}
  \]
  \end{multicols}
\end{definition}

\begin{definition}
  Two systems of linear equations (and their corresponding augmented
  matrices) are said to be \term{equivalent} if they have the same
  solution set.

  For example, both of these systems have a single solution:
  \(\begin{bmatrix} x_1 \\ x_2 \end{bmatrix} = \begin{bmatrix} 1 \\ 1\end{bmatrix}\).

  \begin{multicols}{2}\noindent
    \begin{alignat*}{3}
      3x_1 &\,-\,& 2x_2 &\,=\,& 1 \\
      x_1 &\,+\,& 4x_2 &\,=\,& 5 \\
    \end{alignat*}
    \begin{alignat*}{3}
      3x_1 &\,-\,& 2x_2 &\,=\,& 1 \\
      4x_1 &\,+\,& 2x_2 &\,=\,& 6 \\
    \end{alignat*}
  \end{multicols}

  Therefore these augmented matrices are equivalent:

  \begin{multicols}{2}\noindent
    \[
      \begin{bmatrix}[cc|c]
        3 & -2 & 1\\
        1 &  4 & 5\\
      \end{bmatrix}
    \]
    \[
      \begin{bmatrix}[cc|c]
        3 & -2 & 1\\
        4 &  2 & 6\\
      \end{bmatrix}
    \]
  \end{multicols}
\end{definition}

\begin{activity}{10}
  Following are six procedures used to manipulate an augmented matrix.
  Label the procedures that would result in an equivalent augmented
  matrix as \textbf{valid}, and label the procedures that would
  change the solution set of the corresponding linear system as
  \textbf{invalid}.
  \begin{multicols}{2}
    \begin{enumerate}[a)]
      \item Swap two rows.
      \item Swap two columns.
      \item Add a constant to every term in a row.
      \item Multiply a row by a nonzero constant.
      \item Add a constant multiple of one row to another row.
      \item Replace a column with zeros.
    \end{enumerate}
  \end{multicols}
  \begin{TBLnote}
    This activity could be ran as a card sort.  Allow 5 additional minutes for intra team discussion.
  \end{TBLnote}
\end{activity}


\begin{definition}
  The following \term{row operations} produce equivalent
  augmented matrices:
  \begin{enumerate}
    \item Swap two rows.
    \item Multiply a row by a nonzero constant.
    \item Add a constant multiple of one row to another row.
  \end{enumerate}
  Whenever two matrices \(A,B\) are equivalent (so whenever we do any of
  these operations), we write \(A\sim B\).
\end{definition}

\begin{activity}{10}
  Consider the following  linear systems.
  \begin{multicols}{3}\noindent
  \begin{enumerate}[(A)]
    \item \(
		\begin{alignedat}{4}
   		  3x_1 &\,-\,& 2x_2 &\,+\,& 13x_3 &\,=\,& 6 \\
   		  2x_1 &\,-\,& 2x_2 &\,+\,& 10x_3 &\,=\,& 2 \\
   		  -x_1 &\,+\,& 3x_2 &\,-\,&  6x_3 &\,=\,& 11
   		\end{alignedat} 
	\)

	\item \(
		\begin{alignedat}{4}
	   		x_1 &     &      &\,+\,& 9x_3 &\,=\,& 16 \\
	       		&     & x_2 &\,+\,& x_3 &\,=\,& 9 \\
	       		&     &      &\,-\,& 12x_3 &\,=\,& -24 
		\end{alignedat}
	\)   
   
   \item \(
		\begin{alignedat}{4}
    		x_1 &\,-\,& 3x_2 &\,+\,&  6x_3 &\,=\,& -11 \\
		   2x_1 &\,-\,& 2x_2 &\,+\,& 10x_3 &\,=\,& 2 \\
		   3x_1 &\,-\,& 2x_2 &\,+\,& 13x_3 &\,=\,& 6 
    	\end{alignedat}
	\)


   	\item \(
		\begin{alignedat}{4}
	   		x_1 &\,-\,& 3x_2 &\,+\,& 6x_3 &\,=\,& -11 \\
	   		    &     & x_2 &\,+\,& x_3 &\,=\,& 9 \\
	   		    &     & 7x_2 &\,-\,& 5x_3 &\,=\,& 39 \\
		\end{alignedat}
	\)

   	\item \(
		\begin{alignedat}{4}
	   		x_1 &\,-\,& 3x_2 &\,+\,& 6x_3 &\,=\,& -11 \\
	   		    &     & 4x_2 &\,-\,& 2x_3 &\,=\,& 24 \\
	   		    &     & 7x_2 &\,-\,& 5x_3 &\,=\,& 39 \\
		\end{alignedat}
	\)

 	
	
	\item \(
		\begin{alignedat}{4}
	   		x_1 &     &      &\,+\,& 9x_3 &\,=\,& 16 \\
	       		&     & x_2 &\,+\,& x_3 &\,=\,& 9 \\
	       		&     &      &\,\,& x_3 &\,=\,& 2 
		\end{alignedat}
	\)
	\end{enumerate}
    \end{multicols}
  \begin{subactivity}
	Which system can be obtained from System (A) in the fewest number of row operations?
  \end{subactivity}
  \begin{subactivity}
    Rank the six linear systems from easiest to solve to hardest to solve.
  \end{subactivity}
\end{activity}

\begin{activity}{10}
Consider the following augmented matrices.
\begin{multicols}{3}
\begin{enumerate}[(A)]
\item \( \begin{bmatrix}[ccc|c] 3 & -2 & 13 & 6 \\ 2 & -2 & 10 & 2 \\ -1 & 3 & -6 & 11 \end{bmatrix} \)
\item \( \begin{bmatrix}[ccc|c] 1 & 0 & 9 & 16 \\ 0 & 1 & 1 & 9 \\ 0 & 0 & -12 & -24 \end{bmatrix} \)
\item \( \begin{bmatrix}[ccc|c] 1 & -3 & 6 & -11 \\ 2 & -2 & 10 & 2 \\ 3 & -2 & 13 & 6 \end{bmatrix} \)
\item \( \begin{bmatrix}[ccc|c] 1 & -3 & 6 & -11 \\ 0 & 1 & 1 & 9 \\ 0 & 7 & -5 & 39 \end{bmatrix} \)
\item \( \begin{bmatrix}[ccc|c] 1 & -3 & 6 & -11 \\ 0 & 4 & -2 & 24 \\ 0 & 7 & -5 & 39 \end{bmatrix} \)
\item \( \begin{bmatrix}[ccc|c] 1 & 0 & 9 & 16 \\ 0 & 1 & 1 & 9 \\ 0 & 0 & 1 & 2 \end{bmatrix} \)
\end{enumerate}
\end{multicols}
\begin{subactivity}
Rank the six matrices from farthest from a reduced row echelon form (RREF) matrix to closest to a RREF matrix.
\end{subactivity}
\begin{subactivity}
These matrices are all \term{row equivalent} and represent equivalent linear systems.
Write down one of these linear systems and solve it.
\end{subactivity}
\end{activity}

\begin{remark}
It is important to understand the \term{Gauss-Jordan elimination} algorithm that converts a matrix into reduced row echelon form, but in practice we don't do thsi by hand; we use technology to do this for us.
\end{remark}

\begin{activity}{10}
\begin{itemize}
\item Go to {\tt http://www.cocalc.com} and create an account.
\item Create a project titled ``Linear Algebra Team X'' with your appropriate team number.  Add all team members as collaborators.
\item Open the project and click on ``New''
\item Give it an appropriate name such as ``Class 4 workbook''.  Make a new Jupyter notebook.
\item Click on ``Kernel'' and make sure ``Octave'' is selected.
\item Type {\tt A=[1 3 4 ; 2 5 7]} to store the matrix $\begin{bmatrix} 1 & 3 & 4 \\ 2 & 5 & 7\end{bmatrix}$ in the variable $A$; hold shift when you press enter.
\item Type {\tt rref(A)} to compute the reduced row echelon form of $A$.
\end{itemize}
\end{activity}

\begin{remark}
If you need to find the reduced row echelon form of a matrix during class, you should feel free to use CoCalc/Octave.
\ \\

You can change a cell from ``Code'' to ``Markdown'' or ``Raw'' to put comments around your calculations such as Activity numbers.
\end{remark}

\begin{activity}{5}
Consider our system of equations from above.
 \[
		\begin{alignedat}{4}
   		  3x_1 &\,-\,& 2x_2 &\,+\,& 13x_3 &\,=\,& 6 \\
   		  2x_1 &\,-\,& 2x_2 &\,+\,& 10x_3 &\,=\,& 2 \\
   		  -x_1 &\,+\,& 3x_2 &\,-\,&  6x_3 &\,=\,& 11
   		\end{alignedat} 
\]

Convert this to an augmented matrix, use CoCalc to compute the reduced row echelon form, and convert back to a simpler system of equations to solve this system.  Write your solution on your whiteboard.
\end{activity}



\end{applicationActivities}
