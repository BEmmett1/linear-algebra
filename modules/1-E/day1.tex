%!TEX root =../../course-notes.tex
% ^ leave for LaTeXTools build functionality

\begin{applicationActivities}{Day 1}

\begin{definition}
A \term{linear equation} is an equation of the variables \(x_i\) of the form
\[
a_1x_1+a_2x_2+\dots+a_nx_n=b
.\]
A \term{solution}
for a linear equation is expressed in terms of the Euclidean vectors
\[
  \begin{bmatrix}
    x_1 \\
    x_2 \\
    \vdots \\
    x_n
  \end{bmatrix}=
  \begin{bmatrix}
    s_1 \\
    s_2 \\
    \vdots \\
    s_n
  \end{bmatrix}
\]
and must satisfy
\[
a_1s_1+a_2s_2+\dots+a_ns_n=b
.\]
\end{definition}

\begin{observation}
The linear equation \(3x-5y=-2\) may be graphed as a line in the \(xy\) plane.

\begin{center}
\begin{tikzpicture}[scale=0.3]
\draw[thin,gray,<->] (-5,0) -- (5,0);
\draw[thin,gray,<->] (0,-5) -- (0,5);
\draw[thick,blue] (-5,-2.6) -- (5,3.4);
% \draw[thick,red] (-5,-3.67) -- (5,3);
\end{tikzpicture}
\end{center}

The linear equation \(x+2y-z=4\) may be graphed as a plane in \(xyz\) space.
\end{observation}

\begin{remark}
In previous classes you likely assumed \(x=x_1\), \(y=x_2\), and \(z=x_3\).
However, since this course often deals with equations of four or more
variables, we will almost always write our variables as \(x_i\).
\end{remark}

\begin{definition}
A \term{system of linear equations} (or a \term{linear system} for short)
is a collection of one or more linear equations.
  \begin{alignat*}{5}
    a_{11}x_1 &\,+\,& a_{12}x_2 &\,+\,& \dots  &\,+\,& a_{1n}x_n &\,=\,& b_1 \\
    a_{21}x_1 &\,+\,& a_{22}x_2 &\,+\,& \dots  &\,+\,& a_{2n}x_n &\,=\,& b_2 \\
     \vdots&  &\vdots&   &&  &\vdots&&\vdots  \\
    a_{m1}x_1 &\,+\,& a_{m2}x_2 &\,+\,& \dots  &\,+\,& a_{mn}x_n &\,=\,& b_m
  \end{alignat*}
A \term{solution}
\[
  \begin{bmatrix}
    x_1 \\
    x_2 \\
    \vdots \\
    x_n
  \end{bmatrix}=
  \begin{bmatrix}
    s_1 \\
    s_2 \\
    \vdots \\
    s_n
  \end{bmatrix}
\]
for a linear system satisfies
\[
a_{i1}s_1+a_{i2}s_2+\dots+a_{in}s_n=b_i
\]
for \(1\leq i\leq m\) (that is, the solution satisfies all equations
in the system).
\end{definition}

\begin{remark}
  When variables in a large linear system are missing, we prefer to
  write the system in one of the following standard forms:

  \begin{multicols}{3}\noindent
    Original linear system:
    \begin{alignat*}{2}
       x_1 + 3x_3 &\,=\,& 3 \\
      3x_1 - 2x_2 + 4x_3 &\,=\,& 0 \\
      -x_2 +  x_3 &\,=\,& -2
    \end{alignat*}
    Verbose standard form:
    \begin{alignat*}{4}
       x_1 &\,+\,& 0x_2 &\,+\,& 3x_3 &\,=\,& 3 \\
      3x_1 &\,-\,& 2x_2 &\,+\,& 4x_3 &\,=\,& 0 \\
      0x_1 &\,-\,&  x_2 &\,+\,&  x_3 &\,=\,& -2
    \end{alignat*}
    Concise standard form:
    \begin{alignat*}{4}
       x_1 &     &      &\,+\,& 3x_3 &\,=\,& 3 \\
      3x_1 &\,-\,& 2x_2 &\,+\,& 4x_3 &\,=\,& 0 \\
           &\,-\,&  x_2 &\,+\,&  x_3 &\,=\,& -2
    \end{alignat*}
  \end{multicols}
\end{remark}

\begin{definition}
  A linear system is \term{consistent} if there exists a solution for the
  system. Otherwise it is \term{inconsistent}.
\end{definition}

\begin{fact}
  All linear systems are either \textbf{consistent with one solution},
  \textbf{consistent with infinitely-many solutions}, or
  \textbf{inconsistent}.
\end{fact}

\begin{activity}
  Consider the following graphs representing linear systems of two variables.
  Label each graph with \textbf{consistent with one solution},
  \textbf{consistent with infinitely-many solutions}, or
  \textbf{inconsistent}.
  \begin{multicols}{4}
  \begin{center}
    \systemWithInfinitelyManySolutions
    \systemWithOneSolutionB
    \systemWithNoSolutions
    \systemWithOneSolutionA
  \end{center}
  \end{multicols}
\end{activity}

\begin{activity}
  All inconsistent linear systems contain a logical \term{contradiction}.
  Find a contradiction in this system by solving for \(x_1\) in the first
  equation, substituting the resulting expression into the
  second equation, and then simplifying.
  \begin{align*}
  -x_1+2x_2  &=  5 \\
  2x_1-4x_2  &=  6
  \end{align*}
\end{activity}

\begin{activity}
  Consider the following consistent linear system.
  \begin{align*}
  -x_1+2x_2  &= -3 \\
  2x_1-4x_2  &=  6
  \end{align*}
\begin{subactivity}
  Find three different solutions
  \(
    \begin{bmatrix}
      x_1 \\
      x_2
    \end{bmatrix}=
    \begin{bmatrix}
      r_1 \\
      r_2
    \end{bmatrix},
    \begin{bmatrix}
      s_1 \\
      s_2
    \end{bmatrix},
    \begin{bmatrix}
      t_1 \\
      t_2
    \end{bmatrix}
  \)
  for this system.
\end{subactivity}
\begin{subactivity}
  Let \(x_2=a\) where \(a\) is an arbitrary real number, then find an
  expression for \(x_1\) in terms of \(a\). Use this to describe \textit{all}
  solutions (the \term{solution set})
  \(
    \begin{bmatrix}
      x_1 \\
      x_2
    \end{bmatrix}=
    \begin{bmatrix}
      ? \\
      a
    \end{bmatrix}
  \)
  for the linear system in terms of \(a\).
\end{subactivity}
\end{activity}

\begin{remark}
  The solution set to a consistent linear system with infinitely many solutions
  may be described by setting
  certain variables equal to arbitrary parameters, and expressing the
  other variables in terms of those parameters.
\end{remark}

\begin{activity}
  Consider the following linear system.
  \begin{alignat*}{5}
    x_1 &\,+\,& 2x_2 &\, \,&     &\,-\,&  x_4 &\,=\,& 3 \\
        &\, \,&      &\, \,& x_3 &\,+\,& 4x_4 &\,=\,& -2
  \end{alignat*}
  Describe the solution set
  \[
    \begin{bmatrix}
      x_1 \\
      x_2 \\
      x_3 \\
      x_4
    \end{bmatrix}=
    \begin{bmatrix}
      ? \\
      a \\
      ? \\
      b
    \end{bmatrix}=
    a\begin{bmatrix}
      ? \\
      1 \\
      ? \\
      0
    \end{bmatrix}+
    b\begin{bmatrix}
      ? \\
      0 \\
      ? \\
      1
    \end{bmatrix}+
    \begin{bmatrix}
      t_1 \\
      0 \\
      t_3 \\
      0
    \end{bmatrix}
  \] to the linear system
  by setting \(x_2=a\) and \(x_4=b\), and then solving for \(x_1\) and
  \(x_3\).
\end{activity}

\begin{observation}
  Solving linear systems of two variables by graphing or substitution is
  reasonable for two-variable systems, but won't cut it for equations with
  more variables. Linear Algebra provides us several tools to make this
  process more efficient.
\end{observation}

% \begin{definition}
%   Let \(\mathbb R^{m\times n}\) denote the space of \(m\times n\)
%   dimensional \term{Euclidean matrices} of the form:
%   \[
%   \begin{bmatrix}
%     a_{11} & a_{12} & \cdots & a_{1n} \\
%     a_{21} & a_{22} & \cdots & a_{2n} \\
%     \vdots & \vdots & \ddots & \vdots \\
%     a_{m1} & a_{m2} & \cdots & a_{mn}
%   \end{bmatrix}
%   \]
%   Let \(\mathbb R^{n}=\mathbb R^{1\times n}\) denote the space of
%   \(n\)-dimensional \term{Euclidean vectors} of the form:
%   \[
%   \begin{bmatrix}
%     a_{1} \\
%     a_{2} \\
%     \vdots \\
%     a_{m}
%   \end{bmatrix}
%   \]
%   Note that a Euclidean vector is just a Euclidean matrix with a single column.
% \end{definition}
%
% \begin{remark}
%   There are other types of matrices/vectors, but since the above situation is
%   the most common/useful for our course, we will often refer to a
%   ``Euclidean matrix'' as just a ``matrix'', and a ``Euclidean vector'' as
%   just a ``vector''.
% \end{remark}
%
% \begin{definition}
%   \textbf{Matrix addition} is done coordinate-wise:
%   \[
%   \begin{bmatrix}
%     a_{11} & a_{12} & \cdots & a_{1n} \\
%     a_{21} & a_{22} & \cdots & a_{2n} \\
%     \vdots & \vdots & \ddots & \vdots \\
%     a_{m1} & a_{m2} & \cdots & a_{mn}
%   \end{bmatrix} +
%   \begin{bmatrix}
%     b_{11} & b_{12} & \cdots & b_{1n} \\
%     b_{21} & b_{22} & \cdots & b_{2n} \\
%     \vdots & \vdots & \ddots & \vdots \\
%     b_{m1} & b_{m2} & \cdots & b_{mn}
%   \end{bmatrix} =
%   \begin{bmatrix}
%     a_{11}+b_{11} & a_{12}+b_{12} & \cdots & a_{1n}+b_{1n} \\
%     a_{21}+b_{21} & a_{22}+b_{22} & \cdots & a_{2n}+b_{2n} \\
%     \vdots & \vdots & \ddots & \vdots \\
%     a_{m1}+b_{m1} & a_{m2}+b_{m2} & \cdots & a_{mn}+b_{mn}
%   \end{bmatrix}
%   \]
%   \textbf{Scalar multiplication} of a matrix is done by distribution:
%   \[
%   c
%   \begin{bmatrix}
%     a_{11} & a_{12} & \cdots & a_{1n} \\
%     a_{21} & a_{22} & \cdots & a_{2n} \\
%     \vdots & \vdots & \ddots & \vdots \\
%     a_{m1} & a_{m2} & \cdots & a_{mn}
%   \end{bmatrix} =
%   \begin{bmatrix}
%     ca_{11} & ca_{12} & \cdots & ca_{1n} \\
%     ca_{21} & ca_{22} & \cdots & ca_{2n} \\
%     \vdots & \vdots & \ddots & \vdots \\
%     ca_{m1} & ca_{m2} & \cdots & ca_{mn}
%   \end{bmatrix}
%   \]
% \end{definition}
%
% \begin{activity}
%   Simplify the following matrix expression. (Subtracting a matrix is the
%   same as multiplying it by \(-1\) and adding.)
%   \[
%   2
%   \begin{bmatrix}
%     3 & -1 & 0 \\
%     2 & 4  & -1
%   \end{bmatrix}-
%   3
%   \begin{bmatrix}
%     0 & 2 & 1 \\
%     1 & -3  & 0
%   \end{bmatrix}
%   \]
% \end{activity}
%
%
% \begin{activity}
%   Two matrices/vectors are equal if and only if the values at every coordinate
%   are equal. Show that the following linear system and the following vector
%   equation must have the same solutions by simplifying the vector equation.
%   \begin{multicols}{2}\noindent
%   \[
%     x_1
%     \begin{bmatrix}
%       1 \\
%       3 \\
%       0
%     \end{bmatrix}+
%     x_2
%     \begin{bmatrix}
%       0 \\
%       -2 \\
%       -1
%     \end{bmatrix}+
%     x_3
%     \begin{bmatrix}
%       3 \\
%       4 \\
%       1
%     \end{bmatrix}=
%     \begin{bmatrix}
%       3 \\
%       0 \\
%       -2
%     \end{bmatrix}
%   \]
%   \begin{alignat*}{4}
%      x_1 &     &      &\,+\,& 3x_3 &\,=\,& 3 \\
%     3x_1 &\,-\,& 2x_2 &\,+\,& 4x_3 &\,=\,& 0 \\
%          &\,-\,&  x_2 &\,+\,&  x_3 &\,=\,& -2
%   \end{alignat*}
%   \end{multicols}
% \end{activity}

\begin{definition}
  A system of \(m\) linear equations with \(n\) variables is often represented
  by writing its coefficients and constants in an \term{augmented matrix}.
  % (The vertical bar in the matrix is just for decoration, to represent the
  % equal signs in the original system and separate the coefficients from the
  % constant.)
  \begin{multicols}{2}\noindent
  \begin{alignat*}{5}
    a_{11}x_1 &\,+\,& a_{12}x_2 &\,+\,& \dots  &\,+\,& a_{1n}x_n &\,=\,& b_1 \\
    a_{21}x_1 &\,+\,& a_{22}x_2 &\,+\,& \dots  &\,+\,& a_{2n}x_n &\,=\,& b_2 \\
     \vdots&  &\vdots&   &&  &\vdots&&\vdots  \\
    a_{m1}x_1 &\,+\,& a_{m2}x_2 &\,+\,& \dots  &\,+\,& a_{mn}x_n &\,=\,& b_m
  \end{alignat*}
  \[
    \begin{bmatrix}[cccc|c]
      a_{11} & a_{12} & \cdots & a_{1n} & b_1\\
      a_{21} & a_{22} & \cdots & a_{2n} & b_2\\
      \vdots & \vdots & \ddots & \vdots & \vdots\\
      a_{m1} & a_{m2} & \cdots & a_{mn} & b_m
    \end{bmatrix}
  \]
  \end{multicols}
\end{definition}

\begin{definition}
  Two systems of linear equations (and their corresponding augmented
  matrices) are said to be \term{equivalent} if they share exactly the
  same solutions.
\end{definition}

\begin{activity}
  Following are six procedures used to manipulate an augmented matrix.
  Label the procedures that would result in an equivalent augmented
  matrix as \textbf{valid}, and label the procedures that would
  change the solution set of the corresponding linear system as
  \textbf{invalid}.
  \begin{multicols}{2}
    \begin{enumerate}[a)]
      \item Swap two rows.
      \item Swap two columns.
      \item Add a constant to every term in a row.
      \item Multiply a row by a conzero constant.
      \item Add a constant multiple of one row to another row.
      \item Replace a column with zeros.
    \end{enumerate}
  \end{multicols}
  \begin{TBLnote}
    This activity could be ran as a card sort.
  \end{TBLnote}
\end{activity}

% \begin{activity}
%   Use the linear system and its augmented matrix
%   \begin{multicols}{2}\noindent
%   \begin{alignat*}{3}
%      x_1 &\,+\,& 2x_2 &\,=\,& 4 \\
%     2x_1 &\,-\,& 3x_2 &\,=\,& 1
%   \end{alignat*}
%   \[
%     \begin{bmatrix}[cc|c]
%       1 & 2 & 4 \\
%       2 & -3 & 1
%     \end{bmatrix}
%   \]
%   \end{multicols}
%   to demonstrate why each of these operations is invalid
%   \begin{multicols}{2}
%     \begin{itemize}
%       \item Interchange two columns.
%       \item Add a constant to every term in a row.
%       \item Multiply a row by a conzero constant.
%       \item Add a constant multiple of one row to another row.
%       \item Replace a column with zeros.
%     \end{itemize}
%   \end{multicols}
% \end{activity}

\begin{definition}
  The following \term{row operations} produce equivalent
  augmented matrices:
  \begin{enumerate}
    \item Swap two rows.
    \item Multiply a row by a conzero constant.
    \item Add a constant multiple of one row to another row.
  \end{enumerate}
  Whenever two matrices \(A,B\) are equivalent (so whenever we do any of
  these operations), we write \(A\sim B\).
\end{definition}

\begin{activity}
  Show that the following two linear systems:
  \begin{multicols}{2}\noindent
    \begin{alignat*}{4}
      3x_1 &\,-\,& 2x_2 &\,+\,& 13x_3 &\,=\,& 6 \\
      2x_1 &\,-\,& 2x_2 &\,+\,& 10x_3 &\,=\,& 2 \\
     -1x_1 &\,+\,& 3x_2 &\,-\,&  6x_3 &\,=\,& 11
    \end{alignat*}
    \begin{alignat*}{4}
       x_1 &\,-\,&  x_2  &\,+\,&  5x_3 &\,=\,& 1 \\
           &\, \,&  x_2 &\,-\,&  2x_3 &\,=\,& 3 \\
           &\, \,&      &\, \,&   x_3 &\,=\,& 2
    \end{alignat*}
  \end{multicols}
  are equivalent by converting the first system to an augmented matrix,
  and then performing the following row operations to obtain
  an augmented matrix equivalent to the second system.
  \begin{multicols}{2}\noindent
  \begin{enumerate}
    \item Swap \(R_1\) (first row) and \(R_2\) (second row).
    \item Multiply \(R_2\) by \(\frac{1}{2}\).
    \item Add \(R_1\) to \(R_3\).
    \item Add \(-3R_1\) to \(R_2\).
    \item Add \(-2R_2\) to \(R_3\).
    \item Multiply \(R_3\) by \(\frac{1}{3}\).
  \end{enumerate}
  \end{multicols}
\end{activity}

\begin{definition}
  The \term{leading term} of a matrix row is its first nonzero term.
  A matrix is in \term{row echelon form} if all leading terms are \(1\),
  the leading term of every row
  is farther right than every leading term on a higher row, and all zero
  rows are at the bottom of the matrix.
\end{definition}

\begin{activity}
  Reproduce the steps that manipulated the matrix
  \[
    \begin{bmatrix}[ccc|c]
      3 & -2 & 13 & 6 \\
      2 & -2 & 10 & 2 \\
      -1 & 3 & -6 & 11
    \end{bmatrix}\sim
    \begin{bmatrix}[ccc|c]
      1 & -1 &  5 & 1 \\
      0 &  1 & -2 & 3 \\
      0 &  0 &  1 & 2
    \end{bmatrix}
  \]
  into row echelon form by using the following algorithm.
  \begin{multicols}{2}\noindent
  \begin{enumerate}
    \item Identify
    the top cell of the first non-zero column as your \textbf{pivot position};
    you will ignore anything in the matrix that is above or left of your
    current pivot position.
    \item If the pivot position contains a \(0\), swap its row with a lower
          row that does not contain a \(0\) in its column.
    \item Divide the pivot row by the term in pivot position to change the
          pivot term to \(1\). (If convenient, you can first swap the
          pivot row with a lower row to make this division easier.)
    \item Add multiples of the pivot row to all lower rows so that all terms
          below pivot position become \(0\).
    \item Move your pivot position down and right one step.
    \item If all terms in and below pivot position are zero, move your
          pivot position right. Repeat this step as needed.
    \item If the matrix is not yet in row echelon form, return to Step 2.
  \end{enumerate}
  \end{multicols}
\end{activity}

\begin{definition}
  A matrix is in \term{reduced row echelon form} if it is in row echelon form
  and all terms above leading terms are \(0\).
\end{definition}

\begin{activity}
  Show that the following two linear systems:
  \begin{multicols}{2}\noindent
    \begin{alignat*}{4}
       x_1 &\,-\,&  x_2  &\,+\,&  5x_3 &\,=\,& 1 \\
           &\, \,&  x_2 &\,-\,&  2x_3 &\,=\,& 3 \\
           &\, \,&      &\, \,&   x_3 &\,=\,& 2
    \end{alignat*}
      \begin{alignat*}{4}
         x_1 &\, \,&      &\, \,&       &\,=\,& -2 \\
             &\, \,&  x_2 &\, \,&       &\,=\,& 7 \\
             &\, \,&      &\, \,&   x_3 &\,=\,& 2
      \end{alignat*}
  \end{multicols}
  are equivalent by converting the first system to an augmented matrix,
  and then performing the following row operations to obtain
  an augmented matrix equivalent to the second system.
  \begin{enumerate}
    \item Add \(2R_3\) to \(R_2\).
    \item Add \(-5R_3\) to \(R_1\).
    \item Add \(R_2\) to \(R_1\).
  \end{enumerate}
  Then write the solution to the linear system.
\end{activity}

\begin{activity}
  Verify that \((x_1,x_2,x_3)=(-2,7,2)\) is a solution to the linear system
    \begin{alignat*}{4}
      3x_1 &\,-\,& 2x_2 &\,+\,& 13x_3 &\,=\,& 6 \\
      2x_1 &\,-\,& 2x_2 &\,+\,& 10x_3 &\,=\,& 2 \\
     -1x_1 &\,+\,& 3x_2 &\,-\,&  6x_3 &\,=\,& 11
    \end{alignat*}
  by plugging the solution into each equation.
\end{activity}

\begin{activity}
  Reproduce the steps that manipulated the matrix
  \[
    \begin{bmatrix}[ccc|c]
      1 & -1 &  5 & 1 \\
      0 &  1 & -2 & 3 \\
      0 &  0 &  1 & 2
    \end{bmatrix}\sim
    \begin{bmatrix}[ccc|c]
      1 &  0 &  0 & -2 \\
      0 &  1 &  0 & 7 \\
      0 &  0 &  1 & 2
    \end{bmatrix}
  \]
  into reduced row echelon form by adding multiples of lower rows into
  higher rows, zeroing out all terms above leading terms.
\end{activity}

\begin{fact}
  Every augmented matrix \(A\) reduces to a unique reduced row echelon form
  matrix. This matrix is denoted as \(\RREF(A)\).
\end{fact}

\begin{definition}
  The following algorithm that reduces \(A\) to \(\RREF(A)\) is known as
  \term{Gauss-Jordan elimination}.

  \begin{multicols}{2}\noindent
  \begin{enumerate}
    \item Identify
    the top cell of the first non-zero column as your \textbf{pivot position};
    you will ignore anything in the matrix that is above or left of your
    current pivot position.
    \item If the pivot position contains a \(0\), swap its row with a lower
          row that does not contain a \(0\) in its column.
    \item Divide the pivot row by the term in pivot position to change the
          pivot term to \(1\). (If convenient, you can first swap the
          pivot row with a lower row to make this division easier.)
    \item Add multiples of the pivot row to all lower rows so that all terms
          below pivot position become \(0\).
    \item Move your pivot position down and right one step.
    \item If all terms in and below pivot position are zero, move your
          pivot position right. Repeat this step as needed.
    \item If the matrix is not yet in row echelon form, return to Step 2.
    \item Finally, add multiples of lower rows into higher rows, zeroing out
          all terms above leading terms.
  \end{enumerate}
  \end{multicols}
\end{definition}

\begin{activity}
  Find \(\RREF(A)\) where
  \[A=
    \begin{bmatrix}[cccc|c]
      -1 &  1 & -3 &  2 &  0 \\
       2 & -1 &  5 &  3 & -11 \\
       3 &  2 &  4 &  1 &  1 \\
       0 &  1 & -1 &  1 &  1 \\
    \end{bmatrix}
  .\]
\end{activity}

\begin{definition}
  The columns of \(\RREF(A)\) without a leading term represent
  \term{free variables} of the linear system modeled by \(A\)
  that may be set equal to arbitrary parameters.
  The other \term{bounded variables} can then be expressed in terms
  of those parameters to describe the solution set
  to the linear system modeled by \(A\).
\end{definition}

\begin{activity}
  Given the linear system and its equivalent augmented matrices
  \begin{multicols}{2}\noindent
    \begin{alignat*}{5}
      -x_1 &\,+\,&  x_2 &\,-\,&  3x_3 &\,+\,&  2x_4 &\,=\,& 0 \\
      2x_1 &\,-\,&  x_2 &\,+\,&  5x_3 &\,+\,&  3x_4 &\,=\,& -11 \\
      3x_1 &\,+\,& 2x_2 &\,+\,&  4x_3 &\,+\,&   x_4 &\,=\,& 1 \\
           &\, \,&  x_2 &\,-\,&   x_3 &\,+\,&   x_4 &\,=\,& 1 \\
    \end{alignat*}
  \[
    \begin{bmatrix}[cccc|c]
      -1 &  1 & -3 &  2 &  0 \\
       2 & -1 &  5 &  3 & -11 \\
       3 &  2 &  4 &  1 &  1 \\
       0 &  1 & -1 &  1 &  1 \\
    \end{bmatrix}\sim
    \begin{bmatrix}[cccc|c]
       1 &  0 &  2 &  0 & -1 \\
       0 &  1 & -1 &  0 &  3 \\
       0 &  0 &  0 &  1 & -2 \\
       0 &  0 &  0 &  0 &  0 \\
    \end{bmatrix}
  \]
  \end{multicols}
  describe the solution set
  \(
    \begin{bmatrix}
      x_1 \\
      x_2 \\
      x_3 \\
      x_4
    \end{bmatrix}=
    a\begin{bmatrix}
      r_1 \\
      r_2 \\
      r_3 \\
      r_4
    \end{bmatrix}+
    \begin{bmatrix}
      s_1 \\
      s_2 \\
      s_3 \\
      s_4
    \end{bmatrix}
  \) to the linear system by setting the free variable
  \(x_3=a\), and then expressing each of the
  bounded variables \(x_1,x_2,x_4\) equal to an expression in terms
  of \(a\).
\end{activity}

\begin{remark}
  It's not necessary to completely find \(\RREF(A)\) to
  deduce that a linear system is inconsistent.
\end{remark}

\begin{activity}
  Find the contradiction in the inconsistent linear system
    \begin{alignat*}{4}
      2x_1 &\,-\,& 3x_2 &\,=\,& 17 \\
       x_1 &\,+\,& 2x_2 &\,=\,& -2 \\
      -x_1 &\,-\,&  x_2 &\,=\,& 1
    \end{alignat*}
  by considering the following equivalent augmented matrices:
  \[
    \begin{bmatrix}[cc|c]
       2 & -3 & 17 \\
       1 &  2 & -2 \\
      -1 & -1 &  1 \\
    \end{bmatrix}\sim
    \begin{bmatrix}[cc|c]
       1 &  2 & -2 \\
       0 &  1 &  3 \\
       0 &  0 &  2 \\
    \end{bmatrix}
  .\]
\end{activity}

\begin{definition}
  A \term{homogeneous system} is a linear system satisfying \(b_i=0\), that is,
  it is a linear system of the form
  \begin{alignat*}{5}
    a_{11}x_1 &\,+\,& a_{12}x_2 &\,+\,& \dots  &\,+\,& a_{1n}x_n &\,=\,& 0 \\
    a_{21}x_1 &\,+\,& a_{22}x_2 &\,+\,& \dots  &\,+\,& a_{2n}x_n &\,=\,& 0 \\
     \vdots&  &\vdots&   &&  &\vdots&&\vdots  \\
    a_{m1}x_1 &\,+\,& a_{m2}x_2 &\,+\,& \dots  &\,+\,& a_{mn}x_n &\,=\,& 0
  \end{alignat*}
\end{definition}

\begin{activity}
  Prove that all homogeneous systems are consistent.
  (Hint: Find an obvious solution
  \(
    \begin{bmatrix}
      x_1 \\
      x_2 \\
      \vdots \\
      x_n
    \end{bmatrix}=
    \begin{bmatrix}
      s_1 \\
      s_2 \\
      \vdots \\
      s_n
    \end{bmatrix}
  \) that satisfies every homogeneous
  system.)
\end{activity}

\begin{fact}
  The solution set to any homogeneous system with infinitely-many solutions
  is generated by multiplying the parameters representing the free variables
  by Euclidean vectors, and adding these up. For example:
  \[
    \begin{bmatrix}
      x_1 \\
      x_2 \\
      x_3 \\
      x_4
    \end{bmatrix}=
    a\begin{bmatrix}
      3 \\
      1 \\
      -1 \\
      0
    \end{bmatrix}+
    b\begin{bmatrix}
      0 \\
      0 \\
      0 \\
      1
    \end{bmatrix}
  \]
\end{fact}

\begin{definition}
  A set of Euclidean vectors generating the solution set to a homogeneous
  system is called a \textbf{basis} for the solution
  set of the homogeneous system. For example:
  \begin{multicols}{2}\noindent
  \[
    \begin{bmatrix}
      x_1 \\
      x_2 \\
      x_3 \\
      x_4
    \end{bmatrix}=
    a\begin{bmatrix}
      3 \\
      1 \\
      -1 \\
      0
    \end{bmatrix}+
    b\begin{bmatrix}
      0 \\
      0 \\
      0 \\
      1
    \end{bmatrix}
  \]
  \[
    \textrm{Basis}=\left\{
    \begin{bmatrix}
      3 \\
      1 \\
      -1 \\
      0
    \end{bmatrix},
    \begin{bmatrix}
      0 \\
      0 \\
      0 \\
      1
    \end{bmatrix}\right\}
  \]
  \end{multicols}
\end{definition}

\begin{activity}
  Find a basis for the solution set of the following homogeneous linear
  system.
  \begin{alignat*}{5}
    x_1 &\,+\,& 2x_2 &\, \,&     &\,-\,&  x_4 &\,=\,& 0 \\
        &\, \,&      &\, \,& x_3 &\,+\,& 4x_4 &\,=\,& 0 \\
   2x_1 &\,+\,& 4x_2 &\,+\,& x_3 &\,+\,& 2x_4 &\,=\,& 0 \\
  \end{alignat*}
\end{activity}




\end{applicationActivities}
