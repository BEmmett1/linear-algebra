%!TEX root =../../course-notes.tex
% ^ leave for LaTeXTools build functionality

\begin{applicationActivities}{Day 1}

\begin{definition}
A \term{linear equation} is an equation of the variables \(x_i\) of the form
\[
a_1x_1+a_2x_2+\dots+a_nx_n=b
.\]
A \term{solution}
for a linear equation is expressed in terms of the Euclidean vectors
\[
  \begin{bmatrix}
    x_1 \\
    x_2 \\
    \vdots \\
    x_n
  \end{bmatrix}=
  \begin{bmatrix}
    s_1 \\
    s_2 \\
    \vdots \\
    s_n
  \end{bmatrix}
\]
and must satisfy
\[
a_1s_1+a_2s_2+\dots+a_ns_n=b
.\]
\end{definition}

\begin{observation}
The linear equation \(3x-5y=-2\) may be graphed as a line in the \(xy\) plane.

\begin{center}
\begin{tikzpicture}[scale=0.3]
\draw[thin,gray,<->] (-5,0) -- (5,0);
\draw[thin,gray,<->] (0,-5) -- (0,5);
\draw[thick,blue] (-5,-2.6) -- (5,3.4);
% \draw[thick,red] (-5,-3.67) -- (5,3);
\end{tikzpicture}
\end{center}

The linear equation \(x+2y-z=4\) may be graphed as a plane in \(xyz\) space.
\end{observation}

\begin{remark}
In previous classes you likely assumed \(x=x_1\), \(y=x_2\), and \(z=x_3\).
However, since this course often deals with equations of four or more
variables, we will almost always write our variables as \(x_i\).
\end{remark}

\begin{definition}
A \term{system of linear equations} (or a \term{linear system} for short)
is a collection of one or more linear equations.
  \begin{alignat*}{5}
    a_{11}x_1 &\,+\,& a_{12}x_2 &\,+\,& \dots  &\,+\,& a_{1n}x_n &\,=\,& b_1 \\
    a_{21}x_1 &\,+\,& a_{22}x_2 &\,+\,& \dots  &\,+\,& a_{2n}x_n &\,=\,& b_2 \\
     \vdots&  &\vdots&   &&  &\vdots&&\vdots  \\
    a_{m1}x_1 &\,+\,& a_{m2}x_2 &\,+\,& \dots  &\,+\,& a_{mn}x_n &\,=\,& b_m
  \end{alignat*}
A \term{solution}
\[
  \begin{bmatrix}
    x_1 \\
    x_2 \\
    \vdots \\
    x_n
  \end{bmatrix}=
  \begin{bmatrix}
    s_1 \\
    s_2 \\
    \vdots \\
    s_n
  \end{bmatrix}
\]
for a linear system satisfies
\[
a_{i1}s_1+a_{i2}s_2+\dots+a_{in}s_n=b_i
\]
for \(1\leq i\leq m\) (that is, the solution satisfies all equations
in the system).
\end{definition}

\begin{remark}
  When variables in a large linear system are missing, we prefer to
  write the system in one of the following standard forms:

  \begin{multicols}{3}\noindent
    Original linear system:
    \begin{alignat*}{2}
       x_1 + 3x_3 &\,=\,& 3 \\
      3x_1 - 2x_2 + 4x_3 &\,=\,& 0 \\
      -x_2 +  x_3 &\,=\,& -2
    \end{alignat*}
    Verbose standard form:
    \begin{alignat*}{4}
       x_1 &\,+\,& 0x_2 &\,+\,& 3x_3 &\,=\,& 3 \\
      3x_1 &\,-\,& 2x_2 &\,+\,& 4x_3 &\,=\,& 0 \\
      0x_1 &\,-\,&  x_2 &\,+\,&  x_3 &\,=\,& -2
    \end{alignat*}
    Concise standard form:
    \begin{alignat*}{4}
       x_1 &     &      &\,+\,& 3x_3 &\,=\,& 3 \\
      3x_1 &\,-\,& 2x_2 &\,+\,& 4x_3 &\,=\,& 0 \\
           &\,-\,&  x_2 &\,+\,&  x_3 &\,=\,& -2
    \end{alignat*}
  \end{multicols}
\end{remark}

\begin{definition}
  A linear system is \term{consistent} if there exists a solution for the
  system. Otherwise it is \term{inconsistent}.
\end{definition}

\begin{fact}
  All linear systems are either \textbf{consistent with one solution},
  \textbf{consistent with infinitely-many solutions}, or
  \textbf{inconsistent}.
\end{fact}

\begin{activity}{5}
  Consider the following graphs representing linear systems of two variables.
  Label each graph with \textbf{consistent with one solution},
  \textbf{consistent with infinitely-many solutions}, or
  \textbf{inconsistent}.
  \begin{multicols}{4}
  \begin{center}
    \systemWithInfinitelyManySolutions
    \systemWithOneSolutionB
    \systemWithNoSolutions
    \systemWithOneSolutionA
  \end{center}
  \end{multicols}
\end{activity}

\begin{activity}{10}
  All inconsistent linear systems contain a logical \term{contradiction}.
  Find a contradiction in this system by solving for \(x_1\) in the first
  equation, substituting the resulting expression into the
  second equation, and then simplifying. %TODO less handholding
  \begin{align*}
  -x_1+2x_2  &=  5 \\
  2x_1-4x_2  &=  6
  \end{align*}
\end{activity}

\begin{activity}{10}
  Consider the following consistent linear system.
  \begin{align*}
  -x_1+2x_2  &= -3 \\
  2x_1-4x_2  &=  6
  \end{align*}
\begin{subactivity}
  Find three different solutions
  \(
    \begin{bmatrix}
      x_1 \\
      x_2
    \end{bmatrix}=
    \begin{bmatrix}
      r_1 \\
      r_2
    \end{bmatrix},
    \begin{bmatrix}
      s_1 \\
      s_2
    \end{bmatrix},
    \begin{bmatrix}
      t_1 \\
      t_2
    \end{bmatrix}
  \)
  for this system.
\end{subactivity}
\begin{subactivity}
  Let \(x_2=a\) where \(a\) is an arbitrary real number, then find an
  expression for \(x_1\) in terms of \(a\). Use this to describe \textit{all}
  solutions (the \term{solution set})
  \(
    \begin{bmatrix}
      x_1 \\
      x_2
    \end{bmatrix}=
    \begin{bmatrix}
      ? \\
      a
    \end{bmatrix}
  \)
  for the linear system in terms of \(a\).
\end{subactivity}
\end{activity}

\begin{remark}
  The solution set of a consistent linear system with infinitely many solutions
  may be described by setting each
  certain variable equal to an arbitrary parameter, and expressing the
  other variables in terms of those parameters. (Later we will learn
  how to do this methodically.)
\end{remark}

\begin{activity}{10}
  Consider the following linear system.
  \begin{alignat*}{5}
    x_1 &\,+\,& 2x_2 &\, \,&     &\,-\,&  x_4 &\,=\,& 3 \\
        &\, \,&      &\, \,& x_3 &\,+\,& 4x_4 &\,=\,& -2
  \end{alignat*}
  Describe the solution set
  \[
    \begin{bmatrix}
      x_1 \\
      x_2 \\
      x_3 \\
      x_4
    \end{bmatrix}=
    \begin{bmatrix}
      ? \\
      a \\
      ? \\
      b
    \end{bmatrix}=
    a\begin{bmatrix}
      ? \\
      1 \\
      ? \\
      0
    \end{bmatrix}+
    b\begin{bmatrix}
      ? \\
      0 \\
      ? \\
      1
    \end{bmatrix}+
    \begin{bmatrix}
      t_1 \\
      0 \\
      t_3 \\
      0
    \end{bmatrix}
  \] to the linear system
  by setting \(x_2=a\) and \(x_4=b\), and then solving for \(x_1\) and
  \(x_3\).
\end{activity}

\begin{observation}
  Solving linear systems of two variables by graphing or substitution is
  reasonable for two-variable systems, but won't cut it for equations with
  more variables. %TODO motivate coming definitions
\end{observation}

% \begin{definition}
%   Let \(\mathbb R^{m\times n}\) denote the space of \(m\times n\)
%   dimensional \term{Euclidean matrices} of the form:
%   \[
%   \begin{bmatrix}
%     a_{11} & a_{12} & \cdots & a_{1n} \\
%     a_{21} & a_{22} & \cdots & a_{2n} \\
%     \vdots & \vdots & \ddots & \vdots \\
%     a_{m1} & a_{m2} & \cdots & a_{mn}
%   \end{bmatrix}
%   \]
%   Let \(\mathbb R^{n}=\mathbb R^{1\times n}\) denote the space of
%   \(n\)-dimensional \term{Euclidean vectors} of the form:
%   \[
%   \begin{bmatrix}
%     a_{1} \\
%     a_{2} \\
%     \vdots \\
%     a_{m}
%   \end{bmatrix}
%   \]
%   Note that a Euclidean vector is just a Euclidean matrix with a single column.
% \end{definition}
%
% \begin{remark}
%   There are other types of matrices/vectors, but since the above situation is
%   the most common/useful for our course, we will often refer to a
%   ``Euclidean matrix'' as just a ``matrix'', and a ``Euclidean vector'' as
%   just a ``vector''.
% \end{remark}
%
% \begin{definition}
%   \textbf{Matrix addition} is done coordinate-wise:
%   \[
%   \begin{bmatrix}
%     a_{11} & a_{12} & \cdots & a_{1n} \\
%     a_{21} & a_{22} & \cdots & a_{2n} \\
%     \vdots & \vdots & \ddots & \vdots \\
%     a_{m1} & a_{m2} & \cdots & a_{mn}
%   \end{bmatrix} +
%   \begin{bmatrix}
%     b_{11} & b_{12} & \cdots & b_{1n} \\
%     b_{21} & b_{22} & \cdots & b_{2n} \\
%     \vdots & \vdots & \ddots & \vdots \\
%     b_{m1} & b_{m2} & \cdots & b_{mn}
%   \end{bmatrix} =
%   \begin{bmatrix}
%     a_{11}+b_{11} & a_{12}+b_{12} & \cdots & a_{1n}+b_{1n} \\
%     a_{21}+b_{21} & a_{22}+b_{22} & \cdots & a_{2n}+b_{2n} \\
%     \vdots & \vdots & \ddots & \vdots \\
%     a_{m1}+b_{m1} & a_{m2}+b_{m2} & \cdots & a_{mn}+b_{mn}
%   \end{bmatrix}
%   \]
%   \textbf{Scalar multiplication} of a matrix is done by distribution:
%   \[
%   c
%   \begin{bmatrix}
%     a_{11} & a_{12} & \cdots & a_{1n} \\
%     a_{21} & a_{22} & \cdots & a_{2n} \\
%     \vdots & \vdots & \ddots & \vdots \\
%     a_{m1} & a_{m2} & \cdots & a_{mn}
%   \end{bmatrix} =
%   \begin{bmatrix}
%     ca_{11} & ca_{12} & \cdots & ca_{1n} \\
%     ca_{21} & ca_{22} & \cdots & ca_{2n} \\
%     \vdots & \vdots & \ddots & \vdots \\
%     ca_{m1} & ca_{m2} & \cdots & ca_{mn}
%   \end{bmatrix}
%   \]
% \end{definition}
%
% \begin{activity}{0}
%   Simplify the following matrix expression. (Subtracting a matrix is the
%   same as multiplying it by \(-1\) and adding.)
%   \[
%   2
%   \begin{bmatrix}
%     3 & -1 & 0 \\
%     2 & 4  & -1
%   \end{bmatrix}-
%   3
%   \begin{bmatrix}
%     0 & 2 & 1 \\
%     1 & -3  & 0
%   \end{bmatrix}
%   \]
% \end{activity}
%
%
% \begin{activity}{0}
%   Two matrices/vectors are equal if and only if the values at every coordinate
%   are equal. Show that the following linear system and the following vector
%   equation must have the same solutions by simplifying the vector equation.
%   \begin{multicols}{2}\noindent
%   \[
%     x_1
%     \begin{bmatrix}
%       1 \\
%       3 \\
%       0
%     \end{bmatrix}+
%     x_2
%     \begin{bmatrix}
%       0 \\
%       -2 \\
%       -1
%     \end{bmatrix}+
%     x_3
%     \begin{bmatrix}
%       3 \\
%       4 \\
%       1
%     \end{bmatrix}=
%     \begin{bmatrix}
%       3 \\
%       0 \\
%       -2
%     \end{bmatrix}
%   \]
%   \begin{alignat*}{4}
%      x_1 &     &      &\,+\,& 3x_3 &\,=\,& 3 \\
%     3x_1 &\,-\,& 2x_2 &\,+\,& 4x_3 &\,=\,& 0 \\
%          &\,-\,&  x_2 &\,+\,&  x_3 &\,=\,& -2
%   \end{alignat*}
%   \end{multicols}
% \end{activity}

\begin{definition}
  A system of \(m\) linear equations with \(n\) variables is often represented
  by writing its coefficients and constants in an \term{augmented matrix}.
  % (The vertical bar in the matrix is just for decoration, to represent the
  % equal signs in the original system and separate the coefficients from the
  % constant.)
  \begin{multicols}{2}\noindent
  \begin{alignat*}{5}
    a_{11}x_1 &\,+\,& a_{12}x_2 &\,+\,& \dots  &\,+\,& a_{1n}x_n &\,=\,& b_1 \\
    a_{21}x_1 &\,+\,& a_{22}x_2 &\,+\,& \dots  &\,+\,& a_{2n}x_n &\,=\,& b_2 \\
     \vdots&  &\vdots&   &&  &\vdots&&\vdots  \\
    a_{m1}x_1 &\,+\,& a_{m2}x_2 &\,+\,& \dots  &\,+\,& a_{mn}x_n &\,=\,& b_m
  \end{alignat*}
  \[
    \begin{bmatrix}[cccc|c]
      a_{11} & a_{12} & \cdots & a_{1n} & b_1\\
      a_{21} & a_{22} & \cdots & a_{2n} & b_2\\
      \vdots & \vdots & \ddots & \vdots & \vdots\\
      a_{m1} & a_{m2} & \cdots & a_{mn} & b_m
    \end{bmatrix}
  \]
  \end{multicols}
\end{definition}

\begin{definition}
  Two systems of linear equations (and their corresponding augmented
  matrices) are said to be \term{equivalent} if they have the same
  solution set.
\end{definition}

\begin{activity}{10}
  Following are six procedures used to manipulate an augmented matrix.
  Label the procedures that would result in an equivalent augmented
  matrix as \textbf{valid}, and label the procedures that would
  change the solution set of the corresponding linear system as
  \textbf{invalid}.
  \begin{multicols}{2}
    \begin{enumerate}[a)]
      \item Swap two rows.
      \item Swap two columns.
      \item Add a constant to every term in a row.
      \item Multiply a row by a conzero constant.
      \item Add a constant multiple of one row to another row.
      \item Replace a column with zeros.
    \end{enumerate}
  \end{multicols}
  \begin{TBLnote}
    This activity could be ran as a card sort.
  \end{TBLnote}
\end{activity}

% \begin{activity}{0}
%   Use the linear system and its augmented matrix
%   \begin{multicols}{2}\noindent
%   \begin{alignat*}{3}
%      x_1 &\,+\,& 2x_2 &\,=\,& 4 \\
%     2x_1 &\,-\,& 3x_2 &\,=\,& 1
%   \end{alignat*}
%   \[
%     \begin{bmatrix}[cc|c]
%       1 & 2 & 4 \\
%       2 & -3 & 1
%     \end{bmatrix}
%   \]
%   \end{multicols}
%   to demonstrate why each of these operations is invalid
%   \begin{multicols}{2}
%     \begin{itemize}
%       \item Interchange two columns.
%       \item Add a constant to every term in a row.
%       \item Multiply a row by a conzero constant.
%       \item Add a constant multiple of one row to another row.
%       \item Replace a column with zeros.
%     \end{itemize}
%   \end{multicols}
% \end{activity}



\end{applicationActivities}
