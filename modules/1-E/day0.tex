%!TEX root =../../course-notes.tex
% ^ leave for LaTeXTools build functionality

\begin{applicationActivities}{0}{2}

\begin{definition}
A \term{linear equation} is an equation of the variables \(x_i\) of the form
\[
a_1x_1+a_2x_2+\dots+a_nx_n=b
.\]
A \term{solution}
for a linear equation is expressed in terms of the Euclidean vectors
\[
  \begin{bmatrix}
    x_1 \\
    x_2 \\
    \vdots \\
    x_n
  \end{bmatrix}=
  \begin{bmatrix}
    s_1 \\
    s_2 \\
    \vdots \\
    s_n
  \end{bmatrix}
\]
and must satisfy
\[
a_1s_1+a_2s_2+\dots+a_ns_n=b
.\]
\end{definition}



\begin{remark}
In previous classes you likely assumed \(x=x_1\), \(y=x_2\), and \(z=x_3\).
However, since this course often deals with equations of four or more
variables, we will almost always write our variables as \(x_i\).
\end{remark}

\begin{definition}
A \term{system of linear equations} (or a \term{linear system} for short)
is a collection of one or more linear equations.
  \begin{alignat*}{5}
    a_{11}x_1 &\,+\,& a_{12}x_2 &\,+\,& \dots  &\,+\,& a_{1n}x_n &\,=\,& b_1 \\
    a_{21}x_1 &\,+\,& a_{22}x_2 &\,+\,& \dots  &\,+\,& a_{2n}x_n &\,=\,& b_2 \\
     \vdots&  &\vdots&   &&  &\vdots&&\vdots  \\
    a_{m1}x_1 &\,+\,& a_{m2}x_2 &\,+\,& \dots  &\,+\,& a_{mn}x_n &\,=\,& b_m
  \end{alignat*}
A \term{solution}
\[
  \begin{bmatrix}
    x_1 \\
    x_2 \\
    \vdots \\
    x_n
  \end{bmatrix}=
  \begin{bmatrix}
    s_1 \\
    s_2 \\
    \vdots \\
    s_n
  \end{bmatrix}
\]
for a linear system satisfies
\[
a_{i1}s_1+a_{i2}s_2+\dots+a_{in}s_n=b_i
\]
for \(1\leq i\leq m\) (that is, the solution satisfies all equations
in the system).
\end{definition}

\begin{remark}
  When variables in a large linear system are missing, we prefer to
  write the system in one of the following standard forms:

  \begin{multicols}{3}\noindent
    Original linear system:
    \begin{alignat*}{2}
       x_1 + 3x_3 &\,=\,& 3 \\
      3x_1 - 2x_2 + 4x_3 &\,=\,& 0 \\
      -x_2 +  x_3 &\,=\,& -2
    \end{alignat*}
    Verbose standard form:
    \begin{alignat*}{4}
      1x_1 &\,+\,& 0x_2 &\,+\,& 3x_3 &\,=\,& 3 \\
      3x_1 &\,-\,& 2x_2 &\,+\,& 4x_3 &\,=\,& 0 \\
      0x_1 &\,-\,& 1x_2 &\,+\,& 1x_3 &\,=\,& -2
    \end{alignat*}
    Concise standard form:
    \begin{alignat*}{4}
       x_1 &     &      &\,+\,& 3x_3 &\,=\,& 3 \\
      3x_1 &\,-\,& 2x_2 &\,+\,& 4x_3 &\,=\,& 0 \\
           &\,-\,&  x_2 &\,+\,&  x_3 &\,=\,& -2
    \end{alignat*}
  \end{multicols}
\end{remark}

\begin{definition}
  A linear system is \term{consistent} if there exists a solution for the
  system. Otherwise it is \term{inconsistent}.
\end{definition}

\begin{fact}
  All linear systems are either \textbf{consistent with one solution},
  \textbf{consistent with infinitely-many solutions}, or
  \textbf{inconsistent}.
\end{fact}

\begin{activity}{10}
  All inconsistent linear systems contain a logical \term{contradiction}.
  Find a contradiction in this system.
  \begin{align*}
  -x_1+2x_2  &=  5 \\
  2x_1-4x_2  &=  6
  \end{align*}
\end{activity}

\begin{activity}{10}
  Consider the following consistent linear system.
  \begin{align*}
  -x_1+2x_2  &= -3 \\
  2x_1-4x_2  &=  6
  \end{align*}
\begin{subactivity}
  Find three different solutions
  for this system.
\end{subactivity}
\begin{subactivity}
  Let \(x_2=a\) where \(a\) is an arbitrary real number, then find an
  expression for \(x_1\) in terms of \(a\). Use this to write \textit{all}
  solutions (the \term{solution set})
  \(
    \left\{
	\begin{bmatrix}
      ? \\
      a
    \end{bmatrix} \,\middle|\, a \in \IR 
	\right\}
  \)
  for the linear system in terms of \(a\).
\end{subactivity}
\end{activity}

% \begin{remark}
%   The solution set of a consistent linear system with infinitely many solutions
%   may be described by setting each
%   certain variable equal to an arbitrary parameter, and expressing the
%   other variables in terms of those parameters. (Later we will learn
%   how to do this methodically.)
% \end{remark}

\begin{activity}{10}
  Consider the following linear system.
  \begin{alignat*}{5}
    x_1 &\,+\,& 2x_2 &\, \,&     &\,-\,&  x_4 &\,=\,& 3 \\
        &\, \,&      &\, \,& x_3 &\,+\,& 4x_4 &\,=\,& -2
  \end{alignat*}
  Describe the solution set
  \[
  	\left\{
	\begin{bmatrix}
      ? \\
      a \\
      ? \\
      b
	\end{bmatrix} \,\middle|\, a,b \in \IR \right\}
  \] 
  to the linear system
  by setting \(x_2=a\) and \(x_4=b\), and then solving for \(x_1\) and
  \(x_3\).
\end{activity}

\begin{observation}
  Solving linear systems of two variables by graphing or substitution is
  reasonable for two-variable systems, but these simple techniques
  won't cut it for equations with
  more than two variables or more than two equations.
\end{observation}
\end{applicationActivities}
