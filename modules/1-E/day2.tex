%!TEX root =../../course-notes.tex
% ^ leave for LaTeXTools build functionality

\begin{applicationActivities}{2}{4}

\begin{definition}
  The following \term{row operations} produce equivalent
  augmented matrices:
  \begin{enumerate}
    \item Swap two rows.
    \item Multiply a row by a conzero constant.
    \item Add a constant multiple of one row to another row.
  \end{enumerate}
  Whenever two matrices \(A,B\) are equivalent (so whenever we do any of
  these operations), we write \(A\sim B\).
\end{definition}

\begin{activity}{10}
  Consider the following two linear systems.
  \begin{multicols}{2}\noindent
    \begin{alignat*}{4}
      3x_1 &\,-\,& 2x_2 &\,+\,& 13x_3 &\,=\,& 6 \\
      2x_1 &\,-\,& 2x_2 &\,+\,& 10x_3 &\,=\,& 2 \\
     -1x_1 &\,+\,& 3x_2 &\,-\,&  6x_3 &\,=\,& 11
    \end{alignat*}
    \begin{alignat*}{4}
       x_1 &\,-\,&  x_2  &\,+\,&  5x_3 &\,=\,& 1 \\
           &\, \,&  x_2 &\,-\,&  2x_3 &\,=\,& 3 \\
           &\, \,&      &\, \,&   x_3 &\,=\,& 2
    \end{alignat*}
  \end{multicols}
  \begin{subactivity}
    Show these are equivalent by converting the first system to an augmented
    matrix, and then performing the following row operations to obtain
    an augmented matrix equivalent to the second system.
    \begin{multicols}{2}\noindent
    \begin{enumerate}
      \item Swap \(R_1\) (first row) and \(R_2\) (second row).
      \item Multiply \(R_2\) by \(\frac{1}{2}\).
      \item Add \(R_1\) to \(R_3\).
      \item Add \(-3R_1\) to \(R_2\).
      \item Add \(-2R_2\) to \(R_3\).
      \item Multiply \(R_3\) by \(\frac{1}{3}\).
    \end{enumerate}
    \end{multicols}
  \end{subactivity}
  \begin{subactivity}
    Which linear system would you rather solve?
  \end{subactivity}
\end{activity}

\begin{definition}
  The \term{leading term} of a matrix row is its first nonzero term.
  A matrix is in \term{row echelon form} if all leading terms are \(1\),
  the leading term of every row
  is farther right than every leading term on a higher row, and all zero
  rows are at the bottom of the matrix. Examples:
  \begin{multicols}{3}\noindent
    \[
      \begin{bmatrix}[ccc|c]
        1 & -1 &  5 & 1 \\
        0 &  1 & -2 & 3 \\
        0 &  0 &  1 & 2
      \end{bmatrix}
    \]
    \[
      \begin{bmatrix}[ccc|c]
        1 & -1 &  5 & 1 \\
        0 &  0 &  1 & 3 \\
        0 &  0 &  0 & 1
      \end{bmatrix}
    \]
    \[
      \begin{bmatrix}[ccc|c]
        1 & -1 &  5 & 1 \\
        0 &  0 &  1 & 3 \\
        0 &  0 &  0 & 0
      \end{bmatrix}
    \]
  \end{multicols}
\end{definition}

\begin{activity}{10}
  Find your own sequence of row operations to manipulate the matrix
  \[
    \begin{bmatrix}[ccc|c]
      3 & -2 & 13 & 6 \\
      2 & -2 & 10 & 2 \\
      -1 & 3 & -6 & 11
    \end{bmatrix}
  \]
  into row echelon form. (Note that row echelon form is not unique.)

  The most efficient way to do this is by circling \term{pivot positions}
  in your matrix:
  \begin{enumerate}
    \item Circle the top-left-most cell that (a) is below any existing pivot
    positions and (b) has a nonzero term either in that position or below it.
    \item Ignoring any rows above this pivot position, use row operations
    to change the value of your pivot position to \(1\), and the terms below
    it to \(0\).
    \item Repeat these two steps as often as possible.
  \end{enumerate}
\end{activity}

\begin{activity}{10}
  Solve this simplifed linear system:
    \begin{alignat*}{4}
       x_1 &\,-\,&  x_2  &\,+\,&  5x_3 &\,=\,& 1 \\
           &\, \,&  x_2 &\,-\,&  2x_3 &\,=\,& 3 \\
           &\, \,&      &\, \,&   x_3 &\,=\,& 2
    \end{alignat*}
\end{activity}

\begin{observation}
  The consise standard form of the solution to this linear system
  corresponds to a simplified row echelon form matrix:
  \begin{multicols}{2}\noindent
    \begin{alignat*}{4}
       x_1 &\, \,&      &\, \,&       &\,=\,& -2 \\
           &\, \,&  x_2 &\, \,&       &\,=\,& 7 \\
           &\, \,&      &\, \,&   x_3 &\,=\,& 2
    \end{alignat*}
    \[
      \begin{bmatrix}[ccc|c]
        1 & 0 & 0 & -2 \\
        0 & 1 & 0 & 7 \\
        0 & 0 & 1 & 2
      \end{bmatrix}
    \]
  \end{multicols}
\end{observation}

\begin{definition}
  A matrix is in \term{reduced row echelon form} if it is in row echelon form
  and all terms above leading terms are \(0\). Examples:
  \begin{multicols}{3}\noindent
    \[
      \begin{bmatrix}[ccc|c]
        1 & 0 & 0 & -2 \\
        0 & 1 & 0 & 7 \\
        0 & 0 & 1 & 2
      \end{bmatrix}
    \]
    \[
      \begin{bmatrix}[ccc|c]
        1 & 0 & -2 & 0 \\
        0 & 1 & 3 & 0 \\
        0 & 0 & 0 & 1
      \end{bmatrix}
    \]
    \[
      \begin{bmatrix}[ccc|c]
        1 & 3 & 0 & -2 \\
        0 & 0 & 1 & 7 \\
        0 & 0 & 0 & 0
      \end{bmatrix}
    \]
  \end{multicols}
\end{definition}

\begin{activity}{10}
  Show that the following two linear systems:
  \begin{multicols}{2}\noindent
    \begin{alignat*}{4}
       x_1 &\,-\,&  x_2  &\,+\,&  5x_3 &\,=\,& 1 \\
           &\, \,&  x_2 &\,-\,&  2x_3 &\,=\,& 3 \\
           &\, \,&      &\, \,&   x_3 &\,=\,& 2
    \end{alignat*}
      \begin{alignat*}{4}
         x_1 &\, \,&      &\, \,&       &\,=\,& -2 \\
             &\, \,&  x_2 &\, \,&       &\,=\,& 7 \\
             &\, \,&      &\, \,&   x_3 &\,=\,& 2
      \end{alignat*}
  \end{multicols}
  are equivalent by converting the first system to an augmented matrix,
  and then zeroing out all terms above pivot positions (the leading terms).
\end{activity}

\begin{remark}
  We may verify that \(\begin{bmatrix}x_1\\x_2\\x_3\end{bmatrix}=
  \begin{bmatrix}-2\\7\\2\end{bmatrix}\) is a solution to the
  original linear system
    \begin{alignat*}{4}
      3x_1 &\,-\,& 2x_2 &\,+\,& 13x_3 &\,=\,& 6 \\
      2x_1 &\,-\,& 2x_2 &\,+\,& 10x_3 &\,=\,& 2 \\
     -1x_1 &\,+\,& 3x_2 &\,-\,&  6x_3 &\,=\,& 11
    \end{alignat*}
  by plugging the solution into each equation.
\end{remark}

\begin{fact}
  Every augmented matrix \(A\) reduces to a unique reduced row echelon form
  matrix. This matrix is denoted as \(\RREF(A)\).
\end{fact}

\begin{activity}{10}
  Consider the following matrix.
  \[
    A = \begin{bmatrix}[ccc|c]
      1 & 2 & 3 & 1\\
      2 & 4 & 8 & 0
    \end{bmatrix}
  \]
  \begin{subactivity}
    Find \(\RREF(A)\).
  \end{subactivity}
  \begin{subactivity}
    How many solutions does the corresponding linear system have?
  \end{subactivity}
\end{activity}





\end{applicationActivities}
