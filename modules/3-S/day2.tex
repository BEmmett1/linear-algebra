%!TEX root =../../course-notes.tex
% ^ leave for LaTeXTools build functionality

\begin{applicationActivities}{Day 2}

\begin{activity}{10}
  (take basis shown to be linearly independent in previous day, and
  show that it spans)
\end{activity}

\begin{definition}
  A \term{basis} is a linearly independent set that spans a vector space.
\end{definition}

\begin{observation}
  A basis may be thought of as building blocks for a vector space, since
  every vector in the space can be expressed as a unique linear combination
  of basis vectors.
\end{observation}

\begin{activity}{10}
  (given four sets of general vectors, identify which are bases and which aren't)
\end{activity}

\begin{activity}{10}
  If \(\{\vect v_1,\vect v_2,\vect v_3,\vect v_4\}\) is a basis for
  \(\IR^4\), that means \(\RREF[\vect v_1\,\vect v_2\,\vect v_3\,\vect v_4]\)
  doesn't have a column without a pivot position, and doesn't have a
  row of zeros. What is \(\RREF[\vect v_1\,\vect v_2\,\vect v_3\,\vect v_4]\)?
\end{activity}

\begin{fact}
  The set \(\{\vect v_1,\dots,\vect v_m\}\) is a basis for \(\IR^n\) if and
  only if \(m=n\) and
  \(\RREF[\vect v_1\,\dots\,\vect v_n]=
  \begin{bmatrix}
    1&0&\dots&0\\
    0&1&\dots&0\\
    \vdots&\vdots&\ddots&\vdots\\
    0&0&\dots&1
  \end{bmatrix}
  \)
\end{fact}

\begin{activity}{10}
  (given four sets of \(IR^5\) vectors, identify which are bases and which
  aren't)
\end{activity}

\begin{activity}{10}
  How can \{u,v,u+v\} (but with numbers) be changed to make it
  linearly independent?
\end{activity}

\end{applicationActivities}
