%!TEX root =../../course-notes.tex
% ^ leave for LaTeXTools build functionality

\begin{applicationActivities}{Day 4}

\begin{fact}
  All bases for a vector space are the same size.
\end{fact}

\begin{definition}
  The \term{dimension} of a vector space is given by the cardinality/size
  of any basis for the vector space.
\end{definition}

\begin{activity}{15}
  Reduce a bunch of spans to bases to find their dimension.
\end{activity}

\begin{activity}{5}
  What is the dimension of the vector space of \(7\)th-degree polynomials
  \(\P^7\)?
\end{activity}

\begin{activity}{10}
  What is the dimension of the vector space of polynomials
  \(\P\)?
\end{activity}

\begin{observation}
  Several interesting vector spaces are infinite-dimensional:
  \begin{itemize}
    \item The space of polynomials \(\P\)
    \item The space of real number sequences \(\IR^\infty\)
    \item The space of continuous functions \(C(\IR)\)
  \end{itemize}
\end{observation}

\begin{fact}
  Every vector space with dimension \(n<\infty\) is isomorphic to \(\IR^n\).
\end{fact}

\end{applicationActivities}
