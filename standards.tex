\documentclass{article}

\usepackage{tbil-la}

\usepackage[left=1in,right=1in,top=1in,bottom=1in]{geometry}
\usepackage{enumerate,amssymb}

\begin{document}
\pagestyle{empty}
\noindent \course \hfill \sem \hfill \prof
\vspace{0.3in}
\hrule
\begin{center}{\large \bf Linear Algebra Standards}\end{center}
\hrule


\begin{description}
\item[How can we solve systems of linear equations?] \hfill
\begin{enumerate}[{$\Box\ \Box$ \bf {E}1.}]
\item {\bf Systems as matrices}. I can translate back and forth between a system of linear equations and the corresponding augmented matrix.
\item {\bf Row reduction}.  I can put a matrix in reduced row echelon form.
\item {\bf Systems of linear equations}. I can solve a system of linear equations.
\item {\bf Homogeneous systems}. I can find a basis for the solution set of a homogeneous system of equations.
\end{enumerate}

\item[What is a vector space?] \hfill
\begin{enumerate}[{$\Box\ \Box$ \bf {V}1.}]
\item {\bf Vector space}. I can determine if a set with given operations forms a vector space.
\item {\bf Linear combinations}. I can determine if a vector can be written as a linear combination of a given set of vectors.
\item {\bf Spanning sets}. I can determine if a set of vectors spans a vector space.
\item {\bf Subspaces}. I can determine if a subset of a vector space is a subspace or not.

\end{enumerate}

\item [What structure do vector spaces have?] \hfill
\begin{enumerate}[{$\Box\ \Box$ \bf {S}1.}]

\item {\bf Linear independence}. I can determine if a set of vectors is linearly dependent or independent.
\item {\bf Basis verification}. I can determine if a set of vectors is a basis of a vector space.
\item {\bf Basis construction}.  I can compute a basis for the subspace spanned by a given set of vectors.
\item {\bf Dimension}.  I can compute the dimension of a vector space.
\end{enumerate}

\item[How can we understand linear maps algebraically?] \hfill
\begin{enumerate}[{$\Box\ \Box$ \bf {A}1.}]
\item {\bf Linear maps as matrices}.  I can write the matrix (with respect to the standard bases) corresponding to a linear transformation between Euclidean spaces.
\item {\bf Linear map verification}. I can determine if a map between vector spaces is linear or not.
\item {\bf Injectivity and surjectivity}.  I can determine if a given linear map is injective and/or surjective.
\item {\bf Kernel and Image}. I can compute the kernel and image of a linear map, including finding bases.
\end{enumerate}


\item[What algebraic structure do matrices have?] \hfill
\begin{enumerate}[{$\Box\ \Box$ \bf {M}1.}]
\item {\bf Matrix Multiplication}. I can multiply matrices.
\item {\bf Invertible Matrices}. I can determine if a square matrix is invertible or not.
\item {\bf Matrix inverses}.  I can compute the inverse matrix of an invertible matrix.
\end{enumerate}


\item[How can we understand linear maps geometrically?] \hfill
\begin{enumerate}[{$\Box\ \Box$ \bf {G}1.}]
\item {\bf Determinants}. I can compute the determinant of a square matrix.
\item {\bf Eigenvalues}. I can find the eigenvalues of a square matrix, along with their algebraic multiplicities.
\item {\bf Eigenvectors}. I can find the eigenspace of a square matrix associated to a given eigenvalue.
\item {\bf Geometric multiplicity}. I can compute the geometric multiplicity of an eigenvalue of a square matrix.
\end{enumerate}

\end{description}



\end{document}
