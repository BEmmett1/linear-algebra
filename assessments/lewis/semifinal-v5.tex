%% 
%% This is file, `semifinal-v5.tex',
%% generated with the extract package.
%% 
%% Generated on :  2017/11/15,12:30
%% From source  :  semifinal-v5_solutions.tex
%% Using options:  active,generate=semifinal-v5,extract-env={problem}
%% 
\documentclass{sbgLAsemi}
\usepackage{amsmath,amssymb,amsthm,enumerate}
\coursetitle{Math 237}
\courselabel{Linear Algebra}
\calculatorpolicy{You may use a calculator, but you must show all relevant work to receive credit for a standard.}


\newcommand{\IR}{\mathbb{R}}
\newcommand{\IC}{\mathbb{C}}
\renewcommand{\P}{\mathcal{P}}
\renewcommand{\Im}{{\rm Im\ }}
\DeclareMathOperator{\RREF}{RREF}
\DeclareMathOperator{\vspan}{span}

\makeatletter
\renewcommand*\env@matrix[1][*\c@MaxMatrixCols c]{%
  \hskip -\arraycolsep
  \let\@ifnextchar\new@ifnextchar
  \array{#1}}
\makeatother

\title{Semifinal}
\version{5}

\begin{document}

\begin{problem}{E1}
Write an augmented matrix corresponding to the following system of linear equations.
\begin{align*}
x_1+3x_2-4x_3 +x_4 &= 5 \\
3x_1+9x_2+x_3-7x_4 &= 0 \\
x_1-x_3 +x_4 &= 1
\end{align*}
\end{problem}

\begin{problem}{E2}
Put the following matrix in reduced row echelon form.
$$\begin{bmatrix}
 3 & -1 & 0 \\
 -1 & 0 & -1 \\
 -1 & 1 & 2 \\
 0 & 2 & 6
\end{bmatrix}$$
\end{problem}

\begin{problem}{E3}
Solve the system of linear equations.
\begin{align*}
2x+y-z+w &=5 \\
3x-y-2w &= 0 \\
-x+5z+3w&=-1
\end{align*}
\end{problem}

\begin{problem}{E4}
Find a basis for the solution set to the homogeneous system of equations
given by
\begin{align*}
3x+2y+z &= 0 \\
x+y+z &= 0
\end{align*}
\end{problem}

\begin{problem}{V1}
Let $V$ be the set of all real numbers with the operations, for any $x, y \in V$, $c \in \IR$,
\begin{align*}
x \oplus y &= \sqrt{x^2+y^2} \\
c \odot x &= c x
\end{align*}
\begin{enumerate}[(a)]
\item Show that the vector \textbf{addition} $\oplus$ is \textbf{associative}:
      \(x \oplus (y \oplus z)=(x\oplus y)\oplus z\).
\item Determine if $V$ is a vector space or not.  Justify your answer.
\end{enumerate}
\end{problem}

\begin{problem}{V2} Determine if $\begin{bmatrix} 1 \\ 4 \\ 3 \end{bmatrix}$ is a linear combination of the vectors $\begin{bmatrix} 3 \\ 0 \\ -1 \end{bmatrix}$, $\begin{bmatrix} 1 \\ -1 \\ 4 \end{bmatrix}$, and $\begin{bmatrix} 5 \\ 1 \\  -6 \end{bmatrix}$.
\end{problem}

\begin{problem}{V3}
Determine if the vectors $\begin{bmatrix} 1 \\ 1 \\ 2 \\1 \end{bmatrix}$, $\begin{bmatrix} 3 \\ 3 \\ 6 \\ 3 \end{bmatrix}$, $\begin{bmatrix}3 \\ -1 \\ 3 \\ -2\end{bmatrix}$, and $\begin{bmatrix} 7 \\ -1 \\ 8 \\ -3 \end{bmatrix}$  span $\IR^4$.
\end{problem}

\begin{problem}{V4} Let \(W\) be the set of all complex numbers
that are purely real (i.e of the form $a+0i$)  or purely imaginary (i.e. of the form $0+bi$).
Determine if \(W\) is a subspace of \(\IC\).
\end{problem}

\begin{problem}{S1}
Determine if the set of matrices $\left\{\begin{bmatrix} 3 & -1 \\ 0 & 4 \end{bmatrix}, \begin{bmatrix} 1  & 2 \\ -2 & 1 \end{bmatrix}, \begin{bmatrix} 3 & -8 \\ 6 & 5 \end{bmatrix} \right\}$  is linearly dependent or linearly independent.
\end{problem}

\begin{problem}{S2}
  Determine if the set $\left\{ 2x^2-x+3, 2x^2+2, -x^2+4x+1 \right\}$
  is a basis of $\P^2$.
\end{problem}

\begin{problem}{S3}
Let \(
  W={\rm span}\left\{
    \begin{bmatrix} 2 & 0 \\ -2 & 0 \end{bmatrix},
    \begin{bmatrix} 3 & 1 \\ 3 & 6 \end{bmatrix},
    \begin{bmatrix} 0 & 0 \\ 1 & 1 \end{bmatrix},
    \begin{bmatrix} 1 & 2 \\ 0 & 1 \end{bmatrix}
  \right\}
\). Find a basis for this vector space.
\end{problem}

\begin{problem}{S4}
Let $W$ be the subspace of $\IR^{2\times2}$ given by $W={\rm span}\left(\left\{\begin{bmatrix} 2 & 0 \\ -2 & 0 \end{bmatrix}, \begin{bmatrix} 3 & 1 \\ 3 & 6 \end{bmatrix}, \begin{bmatrix} 0 & 0 \\ 1 & 1 \end{bmatrix}, \begin{bmatrix}1 & 2 \\ 0 & 1 \end{bmatrix}\right\}\right)$. Compute the dimension of $W$.
\end{problem}

\begin{problem}{A1}
Let $T: \IR^3 \rightarrow \IR$ be the linear transformation given by $$T\left(\begin{bmatrix} x_1 \\ x_2 \\ x_3  \end{bmatrix} \right) = \begin{bmatrix} x_3+3x_1 \end{bmatrix}.$$ Write the matrix for $T$ with respect to the standard bases of $\IR^3$ and $\IR$.
\end{problem}

\begin{problem}{A2}
 Determine if $T: \IR^2 \rightarrow \IR^2$ given by $T\left(\begin{bmatrix} x \\ y \end{bmatrix}\right) = \begin{bmatrix} e^{x} \\ e^y \end{bmatrix}$ is a linear transformation.
\end{problem}

\begin{problem}{A3}
Determine if each of the following linear transformations is injective (one-to-one) and/or surjective (onto).
\begin{enumerate}[(a)]
\item
  \(S: \IR^4 \rightarrow \IR^3\) where
  \(S(\vec e_1)=\begin{bmatrix}
    2  \\
    1 \\
    0
  \end{bmatrix}\),
  \(S(\vec e_2)=\begin{bmatrix}
    1  \\
    2\\
    1
  \end{bmatrix}\),
  \(S(\vec e_3)=\begin{bmatrix}
    0  \\
    -1 \\
    0
  \end{bmatrix}\), and
  \(S(\vec e_4)=\begin{bmatrix}
    3 \\
    2\\
    1
  \end{bmatrix}\),
\item
  \(T: \IR^3 \rightarrow \IR^3\) where
  \(T(\vec e_1)=\begin{bmatrix}
    2  \\
    2 \\
    1\\
  \end{bmatrix}\),
  \(T(\vec e_2)=\begin{bmatrix}
     1 \\
     0 \\
     4 \\
  \end{bmatrix}\), and
  \(T(\vec e_3)=\begin{bmatrix}
     1 \\
     2 \\
     -3 \\
  \end{bmatrix}\).
\end{enumerate}
\end{problem}

\begin{problem}{A4}
Let $T: \IR^3 \rightarrow \IR^3$ be the linear map given by \(
  T\left(\begin{bmatrix} x \\ y \\ z \end{bmatrix} \right) =
  \begin{bmatrix}
    8x-3y-z \\
    y+3z \\
    -7x+3y+2z
  \end{bmatrix}
\). Compute a basis for the kernel and a basis for the image of $T$.
\end{problem}

\begin{problem}{M1}
Let
\begin{align*}
A &= \begin{bmatrix} 0 & 0 & -1 & -1 \\ 1 & 3 & 7 & 2 \end{bmatrix} & B &= \begin{bmatrix} 0 & 1 & 7 & 7 \\ -1 & -2 & 0 & 4 \\ 0 & 0 & 1 & 5 \end{bmatrix} & C&=\begin{bmatrix} 3 & 2 \\ 0 & 1 \\ -2 & -1 \end{bmatrix}
\end{align*}
Exactly one of the six products $AB$, $AC$, $BA$, $BC$, $CA$, $CB$ can be computed.  Determine which one, and compute it.
\end{problem}

\begin{problem}{M2}
Determine if the matrix $\begin{bmatrix} 1 & 3 & 3 & 7 \\ 1 & 3 & -1 & -1 \\ 2 & 6 & 3 & 8 \\ 1 & 3 & -2 & -3 \end{bmatrix}$ is invertible.
\end{problem}

\begin{problem}{M3}
Compute the inverse of the matrix $\begin{bmatrix} 1 & 2 & 3 & 0 \\ 0 & -1 & 4 & -2 \\ 0 & 0 & 1 & 3 \\ 0 & 0 & 0 & 1 \end{bmatrix}$.
\end{problem}

\begin{problem}{G1}
Compute the determinant of the matrix
\[
  \begin{bmatrix}
    1 & 3 & 2 & 4 \\
    5 & 0 & -4 & 0 \\
    -2 & 3 & -1 & 1 \\
    0 & 1 & 0 & 1
  \end{bmatrix}
.\]
\end{problem}

\begin{problem}{G2}
Compute the eigenvalues, along with their algebraic multiplicities, of the matrix $ \begin{bmatrix} 9 & -3 & 2 \\ 19 & -6 & 5 \\ -11 & 4 & -2\end{bmatrix}$.
\end{problem}

\begin{problem}{G3}
Compute the eigenspace associated to the eigenvalue $2$ in the matrix $\begin{bmatrix} -1 & 1 & 0 \\ -9 & 5 & 0 \\ 15 & -5 & 2 \end{bmatrix}$.
\end{problem}

\begin{problem}{G4}
Compute the geometric multiplicity of the eigenvalue $2$ in the matrix $\begin{bmatrix} -1 & 1 & 0 \\ -9 & 5 & 0 \\ 15 & -5 & 2 \end{bmatrix}$.
\end{problem}

\end{document}
