\documentclass{sbgLAsemi}

\begin{extract*}
\usepackage{amsmath,amssymb,amsthm,enumerate}
\coursetitle{Math 237}
\courselabel{Linear Algebra}
\calculatorpolicy{You may use a calculator, but you must show all relevant work to receive credit for a standard.}


\newcommand{\IR}{\mathbb{R}}
\newcommand{\IC}{\mathbb{C}}
\renewcommand{\P}{\mathcal{P}}
\renewcommand{\Im}{{\rm Im\ }}
\DeclareMathOperator{\RREF}{RREF}
\DeclareMathOperator{\vspan}{span}

\makeatletter
\renewcommand*\env@matrix[1][*\c@MaxMatrixCols c]{%
  \hskip -\arraycolsep
  \let\@ifnextchar\new@ifnextchar
  \array{#1}}
\makeatother

\title{Semifinal}
\version{4}
\end{extract*}

\begin{document}

\begin{problem}{E1}
Write a system of linear equations corresponding to the following
augmented matrix.
\[
\begin{bmatrix}[ccc|c]
-4 & -1 & 3 & 2  \\
1 & 2 & -1 & 0 \\
-1 & 4 & 1 & 4
\end{bmatrix}
\]
\end{problem}
\begin{solution}
\begin{align*}
-4x_1-x_2+3x_3&=2 \\
x_1+2x_2-x_3 &= 0 \\
-x_1+4x_2+x_3&=4
\end{align*}
\end{solution}

\begin{problem}{E2}
Find \(\RREF A\), where
\[
  A =
  \begin{bmatrix}[cccc|c]
    3 & -2 & 1 & 8 & -5 \\
    2 & 2 & 0 & 6 & -2 \\
    -1 & 1 & 1 & -4 & 6 \\
  \end{bmatrix}
\]
\end{problem}
\begin{solution}
\[
  \RREF A =
  \begin{bmatrix}[cccc|c]
    1 & 0 & 0 & 3 & -2 \\
    0 & 1 & 0 & 0 & 1 \\
    0 & 0 & 1 & -1 & 3
  \end{bmatrix}
\]
\end{solution}

\begin{problem}{E3}
Solve the following linear system.
\begin{align*}
3x+2y+z &= 7 \\
x+y+z &= 1 \\
-2x+3z &= -11
\end{align*}
\end{problem}
\begin{solution}
Let \(A =
  \begin{bmatrix}[ccc|c]
    3 & 2 & 1 & 7 \\
    1 & 1 & 1 & 1 \\
    -2 & 0 & 3 & 11
  \end{bmatrix}
\), so \(\RREF A =
  \begin{bmatrix}[ccc|c]
    1 & 0 & 0 & 4 \\
    0 & 1 & 0 & -2 \\
    0 & 0 & 1 & -1
  \end{bmatrix}
\). It follows that the system has exactly one solution:
\(\begin{bmatrix}
  4 & -2 & -1
\end{bmatrix}\)
\end{solution}
\begin{problem}{E4}
Find a basis for the solution set to the homogeneous system of equations
given by
\begin{align*}
3x+2y+z &= 0 \\
x+y+z &= 0
\end{align*}
\end{problem}
\begin{solution}
Let \(A =
  \begin{bmatrix}[ccc|c]
    3 & 2 & 1 & 0 \\
    1 & 1 & 1 & 0
  \end{bmatrix}
\), so \(\RREF A =
  \begin{bmatrix}[ccc|c]
    1 & 0 & -1 & 0 \\
    0 & 1 & 2 & 0
  \end{bmatrix}
\).
It follows that the basis for the solution set is given by \(\left\{
  \begin{bmatrix}
    1 \\
    -2 \\
    1
  \end{bmatrix}
\right\}\).
\end{solution}

\begin{problem}{V1}
Let $V$ be the set of all real numbers with the operations, for any $x, y \in V$, $c \in \IR$,
\begin{align*}
x \oplus y &= \sqrt{x^2+y^2} \\
c \odot x &= c x
\end{align*}
\begin{enumerate}[(a)]
\item Show that the vector \textbf{addition} $\oplus$ is \textbf{associative}:
      \(x \oplus (y \oplus z)=(x\oplus y)\oplus z\).
\item Determine if $V$ is a vector space or not.  Justify your answer.
\end{enumerate}
\end{problem}
\begin{solution}
Let $x,y,z \in \IR$.  Then
\begin{align*}
(x\oplus y) \oplus z &= \sqrt{x^2+y^2} \oplus z \\&= \sqrt{ (\sqrt{x^2+y^2})^2+z^2} \\&= \sqrt{x^2+y^2+z^2} \\
&= \sqrt{x^2+(\sqrt{y^2+z^2})^2} \\ &= x \oplus \sqrt{y^2+z^2} \\ &= x \oplus (y \oplus z)
\end{align*}
However, this is not a vector space, as there is no zero vector.
\end{solution}
\begin{problem}{V2}
Determine if $\begin{bmatrix} 1 \\ 4 \\ 3 \end{bmatrix}$ is a linear combination of the vectors $\begin{bmatrix} 2 \\ 3 \\ -1 \end{bmatrix}$, $\begin{bmatrix} 1 \\ -1 \\ 0 \end{bmatrix}$, and $\begin{bmatrix} -3 \\ -2 \\ 5 \end{bmatrix}$.
\end{problem}
\begin{solution}
$$\RREF\left( \begin{bmatrix}[ccc|c] 2 & 1 & -3  & 1 \\ 3 & -1 & -2 & 4 \\ -1 & 0 & 5 & 3 \end{bmatrix} \right) = \begin{bmatrix}[ccc|c] 1 & 0 & 0 & 2 \\ 0 & 1 & 0 & 0 \\ 0 & 0 &  1 & 1 \end{bmatrix}$$
Since this system has a solution,  $\begin{bmatrix} 1 \\ 4 \\ 3 \end{bmatrix}$ is a linear combination of the three vectors.
\end{solution}

\begin{problem}{V3}
Determine if the vectors $\begin{bmatrix} 2 \\ 0 \\ -2 \\ 0 \end{bmatrix}$, $\begin{bmatrix} 3 \\ 1 \\ 3 \\ 6 \end{bmatrix}$, $\begin{bmatrix} 0 \\ 0 \\ 1 \\ 1 \end{bmatrix}$, and $\begin{bmatrix}1 \\ 2 \\ 0 \\ 1 \end{bmatrix}$ span $\IR^4$.
\end{problem}
\begin{solution}
$$\RREF\left( \begin{bmatrix} 2 & 3 & 0 & 1 \\ 0 & 1 & 0 & 2 \\ -2 & 3 & 1 & 0 \\ 0 & 6 & 1 & 1\end{bmatrix} \right) = \begin{bmatrix}1 & 0 & 0 & -\frac{5}{2} \\ 0 & 1 & 0 & 2 \\ 0 & 0 & 1 & -11\\ 0 & 0 & 0 & 0  \end{bmatrix} $$
Since there is a zero row, the vectors do not span $\IR^4$.
\end{solution}

\begin{problem}{V4} Let $W$ be the set of all polynomials of the form
\(ax^3+bx\).  Determine if $W$ is a subspace of \(\P^3\).
\end{problem}
\begin{solution}
Yes because \(s(a_1x^3+b_1x)+t(a_2x^3+b_2x)=
(sa_1+ta_2)x^3+(sb_1+tb_2)x\) also belongs to \(W\).
Alternately, yes because \(W\) is isomorphic to \(\IR^2\).
\end{solution}


\begin{problem}{S1}
Determine if the set of vectors $\left\{\begin{bmatrix} 3 \\ -1 \\ 0 \\ 4 \end{bmatrix}, \begin{bmatrix} 1  \\ 2 \\ -2 \\ 1 \end{bmatrix}, \begin{bmatrix} 3 \\ -8 \\ 6 \\ 5 \end{bmatrix} \right\}$  is linearly dependent or linearly independent.
\end{problem}
\begin{solution}
$$\RREF\left(\begin{bmatrix} 3 & 1 & 3 \\ -1 & 2 & -8 \\ 0 & -2 & 6 \\ 4 & 1 & 5 \end{bmatrix} \right) = \begin{bmatrix} 1 & 0 & 2 \\ 0 & 1 & -3 \\ 0 & 0 & 0 \\ 0 & 0 & 0 \end{bmatrix}$$
Since the reduced row echelon form has a nonpivot column, the vectors are linearly dependent.
\end{solution}

\begin{problem}{S2}
Determine if the set $\left\{ x^3-3x^2+2x+2, -x^3+4x^2-x+1, -x^3+2x+1, 3x^2+3x+9 \right\}$ is a basis of $\P^3$ or not.
\end{problem}

\begin{solution}
$$\RREF \begin{bmatrix} 1 & -1 & -1 & 0 \\ -3 & 4 & 0 & 3 \\ 2 & -1 & 2 & 3 \\ 2 & 1 & 1 & 9 \end{bmatrix}=\begin{bmatrix} 1 &0 & 0 & 3 \\ 0 & 1 & 0 & 3 \\ 0 & 0 & 1 & 0 \\ 0 & 0 & 0 & 0 \end{bmatrix}$$
Since this is not the identity matrix, the set is not a basis.
\end{solution}
\begin{problem}{S3}
Let $W={\rm span}\left(\left\{\begin{bmatrix} 2 \\ 0 \\ -2 \\ 0 \end{bmatrix}, \begin{bmatrix} 3 \\ 1 \\ 3 \\ 6 \end{bmatrix}, \begin{bmatrix} 0 \\ 0 \\ 1 \\ 1 \end{bmatrix}, \begin{bmatrix}1 \\ 2 \\ 0 \\ 1 \end{bmatrix}\right\}\right)$.  Find a basis of $W$.
\end{problem}
\begin{solution}
$$\RREF\left( \begin{bmatrix} 2 & 3 & 0 & 1 \\ 0 & 1 & 0 & 2 \\ -2 & 3 & 1 & 0 \\ 0 & 6 & 1 & 1\end{bmatrix} \right) = \begin{bmatrix}1 & 0 & 0 & -\frac{5}{2} \\ 0 & 1 & 0 & 2 \\ 0 & 0 & 1 & -11\\ 0 & 0 & 0 & 0  \end{bmatrix} $$
Then $\left\{\begin{bmatrix} 2 \\ 0 \\ -2 \\ 0 \end{bmatrix}, \begin{bmatrix} 3 \\ 1 \\ 3 \\ 6 \end{bmatrix}, \begin{bmatrix} 0 \\ 0 \\ 1 \\ 1 \end{bmatrix}\right\}$ is a basis of $W$.
\end{solution}


\begin{problem}{S4}
Let $W = {\rm span} \left( \left\{ \begin{bmatrix} 1 \\ 1 \\ 2 \\ 1 \end{bmatrix}, \begin{bmatrix} 3 \\ 3 \\ 6 \\ 3 \end{bmatrix}, \begin{bmatrix} 3 \\ -1 \\ 3 \\ -2 \end{bmatrix}, \begin{bmatrix} 7 \\ -1 \\ 8 \\ -3 \end{bmatrix} \right\} \right)$.  Find the dimension of $W$.
\end{problem}
\begin{solution}
$$\RREF\left(\begin{bmatrix} 1 & 3 & 3 & 7 \\ 1 & 3 & -1 & -1 \\ 2 & 6 & 3 & 8 \\ 1 & 3 & -2 & -3 \end{bmatrix}\right) = \begin{bmatrix} 1 & 3 & 0 & 1 \\ 0 & 0 & 1 & 2 \\ 0 & 0 & 0 & 0 \\  0 & 0 & 0 & 0 \end{bmatrix}$$

This has two pivot columns, so $W$ has dimension 2.
\end{solution}


\begin{problem}{A1}
Let $T: \IR^4 \rightarrow \IR^2$ be the linear transformation given by $$T\left(\begin{bmatrix} x_1 \\ x_2 \\ x_3 \\ x_4 \end{bmatrix} \right) = \begin{bmatrix} x_1+3x_3 \\ 3x_2-x_3 \end{bmatrix}.$$ Write the matrix for $T$ with respect to the standard bases of $\IR^4$ and $\IR^2$.
\end{problem}
\begin{solution}
$$\begin{bmatrix} 1 & 0 & 3 & 0 \\ 0 & 3 & -1 & 0 \end{bmatrix}$$
\end{solution}


\begin{problem}{A2}
Determine if the map $T: \P^3 \rightarrow \P^4$ given by $T(f(x))=xf(x)-f(x)$ is a linear transformation or not.
\end{problem}
\begin{problem}{A3}
Determine if the following linear maps are injective (one-to-one) and/or surjective (onto).
\begin{enumerate}[(a)]
\item $S: \IR^2 \rightarrow \IR^3$ given by $S\left(\begin{bmatrix} x \\ y  \end{bmatrix} \right) = \begin{bmatrix} 3x+2y \\ x-y \\ x+4y \end{bmatrix} $
\item $T: \IR^3 \rightarrow \IR^3$ given by $T\left(\begin{bmatrix} x \\ y \\ z  \end{bmatrix} \right) = \begin{bmatrix} x+y+z \\ 2y+3z \\ x-y-2z \end{bmatrix} $
\end{enumerate}
\end{problem}

\begin{solution}
\begin{enumerate}[(a)]
\item $$\RREF\left( \begin{bmatrix} 1 &  1 & 1 \\ 0  & 2 & 3 \\ 1  & -1 & -2 \end{bmatrix} \right) = \begin{bmatrix} 1 &  0 & -\frac{1}{2} \\ 0  & 1 & \frac{3}{2} \\ 0& 0 & 0  \end{bmatrix}$$
Since there is a nonpivot column, $T$ is not injective.  Since there is a zero row, $T$ is not surjective.
\item $$\RREF \left( \begin{bmatrix} 3 & 2 \\ 1 & -1 \\ 1 & 4 \end{bmatrix} \right) = \begin{bmatrix} 1 & 0 \\ 0 & 1 \\ 0 & 0 \end{bmatrix}$$
Since all columns are pivot columns, $S$ is injective.  Since there is a zero row, $S$ is not surjective.
\end{enumerate}
\end{solution}



\begin{problem}{A4}
Let $T: \IR^{2\times 2} \rightarrow \IR^3$ be the linear map given by $T\left(\begin{bmatrix} x & y \\ z & w \end{bmatrix} \right) = \begin{bmatrix}  8x-3y-z+4w \\ y+3z-4w \\ -7x+3y+2z-5w\end{bmatrix} $.
Compute a basis for the kernel and a basis for the image of $T$.
\end{problem}
\begin{solution}
$$\RREF \left( \begin{bmatrix} 8 & -3 & -1 & 4 \\ 0 & 1 & 3 & -4 \\ -7 & 3 & 2 & -5 \end{bmatrix} \right) = \begin{bmatrix} 1 & 0 & 1 & -1 \\ 0 & 1 & 3 & -4 \\ 0 & 0 & 0 & 0 \end{bmatrix}$$

Thus \(\left\{ \begin{bmatrix} 8 \\ 0 \\ -7 \end{bmatrix}, \begin{bmatrix} -3 \\ 1 \\ 3 \end{bmatrix} \right\}\) is a basis for the image, and \( \left\{ \begin{bmatrix} -1 & -3 \\ 1 & 0 \end{bmatrix}, \begin{bmatrix} 1 & 4 \\ 0 & 1 \end{bmatrix} \right\} \) is a basis for the kernel.
\end{solution}


\begin{problem}{M1}
Let 
\begin{align*}
A &= \begin{bmatrix} 1 & 3 & -1 & -1 \\ 0 & 0 & 7 & 2 \end{bmatrix} & B &= \begin{bmatrix} 0 & 1 & 7 & 7 \\ -1 & -2 & 0 & 4 \\ 0 & 0 & 1 & 5 \end{bmatrix} & C&=\begin{bmatrix} 3 & 2 \\ 0 & 1 \\ -2 & -1 \end{bmatrix}
\end{align*}
Exactly one of the six products $AB$, $AC$, $BA$, $BC$, $CA$, $CB$ can be computed.  Determine which one, and compute it.
\end{problem}
\begin{solution}
$CA$ is the only one that can be computed, and 
$$CA = \begin{bmatrix} 3 & 9 & 11 & 1 \\ 0 & 0 & 7 & 2 \\ -2 & - 6 & -5 & 0 \end{bmatrix}$$
\end{solution}

\begin{problem}{M2}
Determine if the matrix $\begin{bmatrix} 3 & -1 & 0 & 4 \\ 2 & 1 & 1 & 1 \\ 0 & 1 & 1 & -1 \\ 1 & -2 & 0 & 3 \end{bmatrix}$ is invertible.
\end{problem}
\begin{solution}
$$\RREF \begin{bmatrix} 3 & -1 & 0 & 4 \\ 2 & 1 & 1 & 1 \\ 0 & 1 & 1 & -1 \\ 1 & -2 & 0 & 3 \end{bmatrix} = \begin{bmatrix} 1 & 0 & 0 & 1 \\ 0 & 1 & 0 & -1 \\ 0 & 0 & 1 & 0 \\ 0 & 0 & 0 & 0 \end{bmatrix}$$
This matrix is not row equivalent to the identity matrix, so it is not invertible.
\end{solution}


\begin{problem}{M3}
  Find the inverse of the matrix
  \(\begin{bmatrix}
    1 & -4 & 5  \\
    -5 & 24 & -28  \\
    1 & -5 & 6
  \end{bmatrix}\).
\end{problem}
\begin{solution}
\(\begin{bmatrix}[ccc|ccc]
  1 & -4 & 5 & 1 & 0 & 0  \\
  -5 & 24 & -28 & 0 & 1 & 0 \\
  1 & -5 & 6 & 0 & 0 & 1
\end{bmatrix}\sim\begin{bmatrix}[ccc|ccc]
  1 & 0 & 0 & 4 & -1 & -8  \\
  0 & 1 & 0 & 2 & 1 & 3  \\
  0 & 0 & 1 & 1 & 1 & 4
\end{bmatrix}\). Thus the inverse is
\(\begin{bmatrix}
  4 & -1 & -8  \\
  2 & 1 & 3  \\
  1 & 1 & 4
\end{bmatrix}\).
\end{solution}


\begin{problem}{G1}
Compute the determinant of the matrix $\begin{bmatrix} 1 & 3 & 0 & -1 \\ 0 & 1 & 3 & 1 \\ 2 & 0 & 0 & -1 \\ 1 & -3 & -2 & -1 \end{bmatrix}$.
\end{problem}
\begin{solution}
\begin{align*}
\det \begin{bmatrix} 1 & 3 & 0 & -1 \\ 0 & 1 & 3 & 1 \\ 2 & 0 & 0 & -1 \\ 1 & -3 & -2 & -1 \end{bmatrix} &= 2 \det \begin{bmatrix} 3 & 0 & -1 \\ 1 & 3 & 1 \\ -3 & -2 & -1 \end{bmatrix} -(-1) \det \begin{bmatrix} 1 & 3 & 0 \\ 0 & 1 & 3 \\ 1 & -3 & -2 \end{bmatrix} \\
&=2\left( 3 \det \begin{bmatrix} 3 & 1 \\ -2 & -1 \end{bmatrix} + (-1) \det \begin{bmatrix} 1 & 3 \\ -3 & -2 \end{bmatrix}\right) + \left( 1 \det \begin{bmatrix} 1 & 3 \\ -3 & -2 \end{bmatrix} -3 \begin{bmatrix} 0 & 3 \\ 1 & -2 \end{bmatrix} \right) \\
&=2 \left( 3(-1)+(-1)(7) \right) + \left( (1)(7)-3(-3) \right) \\
&= 2(-10)+16 \\
&=-4
\end{align*}
\end{solution}


\begin{problem}{G2} 
Compute the eigenvalues, along with their algebraic multiplicities, of the matrix $ \begin{bmatrix}2 & -3 & 2 \\ 8 & -9 & 5 \\ 8 & -7 & 3 \end{bmatrix}$.
\end{problem}
\begin{solution}
The eigenvalues are $0$ with multiplicity 1 and $-2$, with algebraic multiplicity 2.
\end{solution}

\begin{problem}{G3}
Compute the eigenspace associated to the eigenvalue $2$ in the matrix $\begin{bmatrix} -1 & 1 & 0 \\ -9 & 5 & 0 \\ 15 & -5 & 2 \end{bmatrix}$.
\end{problem}

\begin{solution}
The eigenspace is the solution space of the system $(B-2I)X=0$.
$$\RREF(B-2I)=\RREF\left(\begin{bmatrix} -3 & 1 & 0 \\ -9 & 3 & 0 \\ 15 & - 5 & 0 \end{bmatrix} \right) = \begin{bmatrix} 1 & -\frac{1}{3} & 0 \\ 0 & 0 & 0 \\ 0 & 0 & 0 \end{bmatrix}$$
So the system simplifies to $x-\frac{y}{3}=0$, or $3x=y$.  Thus the eigenspace is $$E_2 = {\rm span}\left( \left\{ \begin{bmatrix} 1 \\ 3 \\ 0 \end{bmatrix}, \begin{bmatrix} 0 \\ 0 \\ 1\end{bmatrix} \right\} \right)$$
\end{solution}
\begin{problem}{G4}
Compute the geometric multiplicity of the eigenvalue $-1$ in the matrix $\begin{bmatrix} 4 & -2 & -1 \\ 15 & -7 & -3 \\ -5 & 2 & 0 \end{bmatrix}$.  \end{problem}
\begin{solution}
$$\RREF\left(A+I\right) = \begin{bmatrix} 1 & - \frac{2}{5} & -\frac{1}{5} \\ 0 & 0 & 0 \\ 0 & 0 & 0 \end{bmatrix}$$
So the geometric multiplicity is $2$.
\end{solution}


\end{document}