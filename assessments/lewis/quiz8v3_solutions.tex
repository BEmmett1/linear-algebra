\documentclass{sbgLAquiz}

\begin{extract*}
\usepackage{amsmath,amssymb,amsthm,enumerate}
\coursetitle{Math 237}
\courselabel{Linear Algebra}
\calculatorpolicy{You may use a calculator, but you must show all relevant work to receive credit for a standard.}


\newcommand{\IR}{\mathbb{R}}
\newcommand{\IC}{\mathbb{C}}
\renewcommand{\P}{\mathcal{P}}
\renewcommand{\Im}{{\rm Im\ }}
\DeclareMathOperator{\RREF}{RREF}
\DeclareMathOperator{\vspan}{span}

\makeatletter
\renewcommand*\env@matrix[1][*\c@MaxMatrixCols c]{%
  \hskip -\arraycolsep
  \let\@ifnextchar\new@ifnextchar
  \array{#1}}
\makeatother

\title{Mastery Quiz Day 8 }
\standard{E1,E3,E4,V1,E2}
\version{3}
\end{extract*}

\begin{document}

\begin{problem}{E1}
Write a system of linear equations corresponding to the following
augmented matrix.
\[
\begin{bmatrix}[ccc|c]
2 & -1 & 0 & 1  \\
-1 & 4 & 1 & -7  \\
1 & 2 & -1 & 0
\end{bmatrix}
\]
\end{problem}
\begin{solution}
\begin{align*}
2x_1-x_2&=1 \\
-x_1+4x_2+x_3&=-7 \\
x_1+2x_2-x_3 &= 0
\end{align*}
\end{solution}

\begin{problem}{E3}
Find the solution set for the following system of linear equations.
\begin{align*}
2x_1-2x_2+6x_3-x_4 &=-1 \\
3x_1+6x_3+x_4 &= 5 \\
-4x_1+x_2-9x_3+2x_4&=-7
\end{align*}
\end{problem}
\begin{solution}
Let \(A =
  \begin{bmatrix}[cccc|c]
    2 & -2 & 6 & -1 & -1 \\
    3 & 0 & 6 & 1 & 5 \\
    -4 & 1 & -9 & 2 & -7
  \end{bmatrix}
\), so \(\RREF A =
  \begin{bmatrix}[cccc|c]
    1 & 0 & 2 & 0 & 2 \\
    0 & 1 & -1 & 0 & 3 \\
    0 & 0 & 0 & 1 & -1
  \end{bmatrix}
\). It follows that the solution set is given by \(
  \begin{bmatrix}
    2 - 2a \\
    3 + a \\
    a \\
    -1
  \end{bmatrix}
\) for all real numbers \(a\).
\end{solution}

\begin{extract}\newpage\end{extract}
\begin{problem}{E4}
Find a basis for the solution set of the system of equations
\begin{align*}
x+3y+3z+7w &= 0 \\
 x+3y-z-w &= 0 \\
  2x+6y+3z+8w &= 0 \\
   x+3y-2z-3w &= 0
\end{align*}
\end{problem}
\begin{solution}
$$\RREF\left(\begin{bmatrix} 1 & 3 & 3 & 7 \\ 1 & 3 & -1 & -1 \\ 2 & 6 & 3 & 8 \\ 1 & 3 & -2 & -3 \end{bmatrix}\right) = \begin{bmatrix} 1 & 3 & 0 & 1 \\ 0 & 0 & 1 & 2 \\ 0 & 0 & 0 & 0 \\ 0 & 0 & 0 & 0 \end{bmatrix}$$
Then the solution set is
$$\left\{ \begin{bmatrix} -3a-b \\ a \\ -2b \\ b \end{bmatrix}\ \big|\ a,b \in \IR\right\}$$
So a basis for the solution set is $$ \left\{ \begin{bmatrix} 3 \\ -1 \\ 0 \\ 0 \end{bmatrix} , \begin{bmatrix} 1\\0 \\ 2 \\ -1 \end{bmatrix} \right\} $$
\end{solution}


\begin{problem}{V1}
Let $V$ be the set of all real numbers with the operations, for any $x, y \in V$, $c \in \IR$,
\begin{align*}
x \oplus y &= \sqrt{x^2+y^2} \\
c \odot x &= c x
\end{align*}
\begin{enumerate}[(a)]
\item Show that the vector addition $\oplus$ is associative.
\item Determine if $V$ is a vector space or not.  Justify your answer.
\end{enumerate}
\end{problem}
\begin{solution}
Let $x,y,z \in \IR$.  Then
\begin{align*}
(x\oplus y) \oplus z &= \sqrt{x^2+y^2} \oplus z \\&= \sqrt{ (\sqrt{x^2+y^2})^2+z^2} \\&= \sqrt{x^2+y^2+z^2} \\
&= \sqrt{x^2+(\sqrt{y^2+z^2})^2} \\ &= x \oplus \sqrt{y^2+z^2} \\ &= x \oplus (y \oplus z)
\end{align*}
However, this is not a vector space, as there is no zero vector.
\end{solution}
\end{document}