\documentclass{sbgLAsemi}

\begin{extract*}
\usepackage{amsmath,amssymb,amsthm,enumerate}
\coursetitle{Math 237}
\courselabel{Linear Algebra}
\calculatorpolicy{You may use a calculator, but you must show all relevant work to receive credit for a standard.}


\newcommand{\IR}{\mathbb{R}}
\newcommand{\IC}{\mathbb{C}}
\renewcommand{\P}{\mathcal{P}}
\renewcommand{\Im}{{\rm Im\ }}
\DeclareMathOperator{\RREF}{RREF}
\DeclareMathOperator{\vspan}{span}

\makeatletter
\renewcommand*\env@matrix[1][*\c@MaxMatrixCols c]{%
  \hskip -\arraycolsep
  \let\@ifnextchar\new@ifnextchar
  \array{#1}}
\makeatother

\title{Semifinal}
\version{1}
\end{extract*}

\begin{document}

\begin{problem}{E1}
Write an augmented matrix corresponding to the following system of linear equations.
\begin{align*}
x_1+4x_3 &= 1 \\
x_2-x_3 &= 7 \\
x_1-x_2+3x_3 &= -1
\end{align*}
\end{problem}
\begin{solution}
\[
\begin{bmatrix}[ccc|c]
1 & 0 & 4 & 1 \\
0 & 1 & -1 & 7 \\
1 & -1 & 3 & -1
\end{bmatrix}
\]
\end{solution}

\begin{problem}{E2}
Put the following matrix in reduced row echelon form.
$$\begin{bmatrix}[ccc|c] -3 & 1 & 0 & 2 \\ -8 & 2 & -1 & 6 \\ 0 & 2 & 3 & -2 \end{bmatrix}$$
\end{problem}
\begin{solution}
$$\begin{bmatrix}[ccc|c]
-3 & 1 & 0 & 2 \\
 -8 & 2 & -1 & 6 \\
 0 & 2 & 3 & -2
\end{bmatrix} \sim
\begin{bmatrix}[ccc|c]
1 & -\frac{1}{3} & 0 & -\frac{2}{3} \\
 -8 & 2 & -1 & 6 \\
 0 & 2 & 3 & -2
\end{bmatrix} \sim
\begin{bmatrix}[ccc|c]
1 & -\frac{1}{3} & 0 & -\frac{2}{3} \\
 0 & -\frac{2}{3} & -1 & \frac{2}{3} \\
 0 & 2 & 3 & -2
\end{bmatrix} $$
$$\sim
\begin{bmatrix}[ccc|c]
1 & -\frac{1}{3} & 0 & -\frac{2}{3} \\
 0 & 1 & \frac{3}{2} & -1 \\
 0 & 2 & 3 & -2
\end{bmatrix} \sim
\begin{bmatrix}[ccc|c]
1 & 0 & \frac{1}{2} & -1 \\
 0 & 1 & \frac{3}{2} & -1 \\
 0 & 0 & 0 & 0
\end{bmatrix}$$
\end{solution}

\begin{problem}{E3}
Solve the system of equations
\begin{align*}
-3x +y &= 2\\
-8x+2y-z &= 6 \\
2y+3z &= -2
\end{align*}


\end{problem}

\begin{solution}
$$\RREF\left(\begin{bmatrix}[ccc|c] -3 & 1 & 0 & 2 \\ -8 & 2 & -1 & 6 \\ 0 & 2 & 3 & -2 \end{bmatrix} \right) = \begin{bmatrix}[ccc|c] 1 & 0 & \frac{1}{2} & -1 \\ 0 & 1 & \frac{3}{2} & -1 \\ 0 & 0 & 0 & 0 \end{bmatrix}$$
The solutions are $$\left\{ \begin{bmatrix} -1-\frac{c}{2} \\ -1-\frac{3c}{2} \\ c \end{bmatrix}\ \bigg|\ c\in \IR\right\} = \left\{ \begin{bmatrix} c-1 \\ 3c-1 \\ -2c  \end{bmatrix}\ \bigg|\ c\in \IR\right\}$$
\end{solution}

\begin{problem}{E4}
Find a basis for the solution set to the homogeneous system of equations
\begin{align*}
2x_1+3x_2-5x_3+14x_4 &= 0\\
x_1+x_2-x_3+5x_4 &= 0
\end{align*}
\end{problem}
\begin{solution}
Let \(A =
  \begin{bmatrix}[cccc|c]
    2 & 3 & -5 & 14 & 0 \\
    1 & 1 & -1 & 5 & 0
  \end{bmatrix}
\), so \(\RREF A =
  \begin{bmatrix}[cccc|c]
    1 & 0 & 2 & 1 & 1 \\
    0 & 1 & -3 & 4 & 2 \\
  \end{bmatrix}
\).
It follows that the basis for the solution set is given by \(\left\{
  \begin{bmatrix}
    - 2 \\
    3 \\
    1 \\
    0
  \end{bmatrix},
  \begin{bmatrix}
    -1 \\
    - 4 \\
    0 \\
    1
  \end{bmatrix}
\right\}\).
\end{solution}

\begin{problem}{V1}
Let $V$ be the set of all points on the line $x+y=2$ with the operations, for any $(x_1,y_1), (x_2,y_2) \in V$, $c \in \IR$,
\begin{align*}
(x_1,y_1) \oplus (x_2,y_2) &= (x_1+x_2-1,y_1+y_2-1) \\
c \odot (x_1,y_1) &= (cx_1-(c-1), cy_1-(c-1))
\end{align*}
\begin{enumerate}[(a)]
\item Show that this vector space has an \textbf{additive identity} element
      \(\mathbf{0}\) satisfying \((x,y)\oplus\mathbf{0}=(x,y)\).
\item Determine if $V$ is a vector space or not.  Justify your answer.
\end{enumerate}
\end{problem}
\begin{solution}
Let $(x_1,y_1) \in V$; then $(x_1,y_1) \oplus (1,1) = (x_1,y_1)$, so $(1,1)$ is an additive identity element.

Now we will show the other seven properties.  Let $(x_1,y_1), (x_2,y_2) \in V$, and let $c,d \in \IR$.
\begin{enumerate}[1)]
\item Since real addition is associative, $\oplus$ is associative.
\item Since real addition is commutative, $\oplus$ is commutative.
\item The additive identity is $(1,1)$.
\item $(x_1,y_1) \oplus (2-x_1,2-y_1) = (1,1)$, so $(2-x_1,2-y_1)$ is the additive inverse of $(x_1,y_1)$.
\item \begin{align*} c\odot \left(d \odot (x_1,y_1) \right) &=c\odot \left( dx_1-(d-1),dy_1-(d-1)\right) \\
&= \left( c\left(dx_1-(d-1) \right)-(c-1), c\left(dy_1-(d-1) \right) \right) \\
&= \left(cdx_1-cd+c-(c-1), cdy_1-cd+c-(c-1) \right) \\
&= \left(cdx_1-(cd-1), cdy_1-(cd-1) \right) \\
&= (cd) \odot (x_1,y_1)
\end{align*}
\item $1 \odot (x_1,y_1) = (x_1-(1-1),y_1-(1-1)=(x_1,y_1)$
\item \begin{align*} c \odot \left( (x_1,y_1)\oplus(x_2,y_2) \right) &=
c\odot \left( x_1+y_1-1,x_2+y_2-1 \right) \\
&= \left( c(x_1+y_1-1)-(c-1), c(x_2+y_2-1)-(c-1) \right) \\
&= (cx_1+cx_2-2c+1, cy_1+cy_2-2c+1) \\
&= \left(cx_1-(c-1),cy_1-(c-1) \right) \oplus (cx_2-(c-1),cy_2-(c-1)) \\
&=c \odot (x_1,y_1) \oplus c\odot (x_2,y_2)
\end{align*}
\item \begin{align*} (c+d) \odot (x_1,y_1) &=
\left( (c+d)x_1-(c+d-1), (c+d)y_1-(c+d-1) \right) \\
&= \left( cx_1-(c-1), cy_1-(c-1) \right) \oplus (dx_1-(d-1), dy_1-(d-1) ) \\
&= c\odot (x_1,y_1) \oplus c \odot (x_2,y_2)
\end{align*}
\end{enumerate}
Therefore $V$ is a vector space.
\end{solution}

\begin{problem}{V2}
  Determine if
  \(\begin{bmatrix} 3 \\ -2 \\ 4 \end{bmatrix}\)
  belongs to the span of the set
  \(\left\{
    \begin{bmatrix} 1 \\ 2 \\ -3 \end{bmatrix},
    \begin{bmatrix} 2 \\ 4 \\ -6 \end{bmatrix},
    \begin{bmatrix} 0 \\ 0 \\ 0 \end{bmatrix}
    \right\}
  \).
\end{problem}
\begin{solution}
  Since
  \[
    \RREF\left(
      \begin{bmatrix}[ccc|c]
        1 & 2 & 0 & 3 \\
        2 & 4 & 0 & -2 \\
        -3 & -6 & 0 & 4
      \end{bmatrix}
    \right) =
    \begin{bmatrix}[ccc|c]
      1 & 2 & 0 & 0 \\
      0 & 0 & 0 & 1 \\
      0 & 0 & 0 & 0
    \end{bmatrix}
  \]
  contains the contradiction \(0=1\),
  \(\begin{bmatrix} 3 \\ -2 \\ 4 \end{bmatrix}\) is
  not a linear combination of the three vectors.
\end{solution}
\begin{problem}{V3}
Does
\(
  \operatorname{span}\left\{
    \begin{bmatrix} 2 \\ -1 \\ 4 \\ 2 \\ 1 \end{bmatrix},
    \begin{bmatrix} -1 \\ 3 \\ 5 \\ 2 \\ 0 \end{bmatrix},
    \begin{bmatrix} 1 \\ 0 \\ 5 \\ 1 \\ -3 \end{bmatrix}
  \right\} = \IR^5
\)?
\end{problem}
\begin{solution}
Since there are only three vectors, they cannot span \(\IR^5\).
\end{solution}
\begin{problem}{V4} Let \(W\) be the set of all \(\IR^3\) vectors
\(\begin{bmatrix} x \\ y \\ z \end{bmatrix}\)
satisfying \(x+y+z=1\) (this forms a plane).
Determine if \(W\) is a subspace of \(\IR^3\).
\end{problem}
\begin{solution}
No, because \(\mathbf{0}\) does not belong to \(W\).
\end{solution}


\begin{problem}{S1}
Determine if the set of matrices $\left\{\begin{bmatrix} 3 & -1 \\ 0 & 4 \end{bmatrix}, \begin{bmatrix} 1  & 2 \\ -2 & 1 \end{bmatrix}, \begin{bmatrix} 3 & -8 \\ 6 & 5 \end{bmatrix} \right\}$  is linearly dependent or linearly independent.
\end{problem}
\begin{solution}
$$\RREF\left(\begin{bmatrix} 3 & 1 & 3 \\ -1 & 2 & -8 \\ 0 & -2 & 6 \\ 4 & 1 & 5 \end{bmatrix} \right) = \begin{bmatrix} 1 & 0 & 2 \\ 0 & 1 & -3 \\ 0 & 0 & 0 \\ 0 & 0 & 0 \end{bmatrix}$$
Since the reduced row echelon form has a nonpivot column, the vectors are linearly dependent.
\end{solution}
\begin{problem}{S2}
Determine if the set $\left\{\begin{bmatrix} 1 \\ 1 \\ -1 \end{bmatrix}, \begin{bmatrix} 3 \\ -1 \\ 1 \end{bmatrix},\begin{bmatrix} 2 \\ 0 \\ -2 \end{bmatrix}\right\}$ is a basis of $\IR^3$.
\end{problem}
\begin{solution}
$$\RREF\left(\begin{bmatrix} 1 & 3 & 2 \\ 1 & -1 & 0 \\ -1 & 1 & -2 \end{bmatrix} \right)= \begin{bmatrix} 1 & 0 &0 \\ 0 & 1 & 0 \\ 0 & 0 & 1\end{bmatrix}$$
Since the resulting matrix is the identity matrix, it is a basis.
\end{solution}


\begin{problem}{S3}
Let $W$ be the subspace of $\P^2$ given by $W = {\rm span} \left( \left\{  -3x^2-8x, x^2+2x+2, -x+3\right\} \right)$.   Find a basis for $W$.
\end{problem}
\begin{solution}
Let $A= \begin{bmatrix}-3 & 1 & 0 \\ -8 & 2 & -1 \\ 0 & 2 & 3\end{bmatrix}$, and compute $\RREF(A) = \begin{bmatrix} 1 & 0 & \frac{1}{2} \\ 0 & 1 & \frac{3}{2} \\ 0 & 0 & 0 \end{bmatrix}$.
Since the first two columns are pivot columns, $\left\{ -3x^2-8x, x^2+2x+2\right\} $ is a basis for $W$.
\end{solution}


\begin{problem}{S4}
Let $W = {\rm span} \left( \left\{  \begin{bmatrix} -3 \\ -8 \\ 0 \end{bmatrix}, \begin{bmatrix} 1 \\ 2 \\ 2 \end{bmatrix}, \begin{bmatrix} 0 \\ -1 \\ 3 \end{bmatrix} \right\} \right)$.  Compute the dimension of $W$.
\end{problem}
\begin{solution}
Let $A= \begin{bmatrix}-3 & 1 & 0 \\ -8 & 2 & -1 \\ 0 & 2 & 3\end{bmatrix}$, and compute $\RREF(A) = \begin{bmatrix} 1 & 0 & \frac{1}{2} \\ 0 & 1 & \frac{3}{2} \\ 0 & 0 & 0 \end{bmatrix}$.
Since there are two pivot columns, ${\rm dim}\ W = 2$.
\end{solution}


\begin{problem}{A1}
Let $T: \IR^3 \rightarrow \IR$ be the linear transformation given by $$T\left(\begin{bmatrix} x_1 \\ x_2 \\ x_3  \end{bmatrix} \right) = \begin{bmatrix} x_2+3x_3 \end{bmatrix}.$$ Write the matrix for $T$ with respect to the standard bases of $\IR^3$ and $\IR$.
\end{problem}
\begin{solution}
$$\begin{bmatrix} 0 & 1 & 3 \end{bmatrix}$$
\end{solution}


\begin{problem}{A2}
Determine if the map $T: \P^3 \rightarrow \P^4$ given by $T(f(x))=xf(x)-f(x)$ is a linear transformation or not.
\end{problem}
\begin{problem}{A3}
Determine if the following linear maps are injective (one-to-one) and/or surjective (onto).
\begin{enumerate}[(a)]
\item $S: \IR^2 \rightarrow \IR^3$ given by $S\left(\begin{bmatrix} x \\ y  \end{bmatrix} \right) = \begin{bmatrix} 3x+2y \\ x-y \\ x+4y \end{bmatrix} $
\item $T: \IR^3 \rightarrow \IR^3$ given by $T\left(\begin{bmatrix} x \\ y \\ z  \end{bmatrix} \right) = \begin{bmatrix} x+y+z \\ 2y+3z \\ x-y-2z \end{bmatrix} $
\end{enumerate}
\end{problem}

\begin{solution}
\begin{enumerate}[(a)]
\item $$\RREF\left( \begin{bmatrix} 1 &  1 & 1 \\ 0  & 2 & 3 \\ 1  & -1 & -2 \end{bmatrix} \right) = \begin{bmatrix} 1 &  0 & -\frac{1}{2} \\ 0  & 1 & \frac{3}{2} \\ 0& 0 & 0  \end{bmatrix}$$
Since there is a nonpivot column, $T$ is not injective.  Since there is a zero row, $T$ is not surjective.
\item $$\RREF \left( \begin{bmatrix} 3 & 2 \\ 1 & -1 \\ 1 & 4 \end{bmatrix} \right) = \begin{bmatrix} 1 & 0 \\ 0 & 1 \\ 0 & 0 \end{bmatrix}$$
Since all columns are pivot columns, $S$ is injective.  Since there is a zero row, $S$ is not surjective.
\end{enumerate}
\end{solution}



\begin{problem}{A4}
Let $T: \IR^4\rightarrow \IR^4$ be the linear transformation given by $$T\left(\begin{bmatrix} x \\ y \\ z \\ w \end{bmatrix} \right) = \begin{bmatrix} x+3y+3z+7w \\ x+3y-z-w \\ 2x+6y+3z+8w \\ x+3y-2z-3w \end{bmatrix}$$
Compute a basis for the kernel and a basis for the image of $T$.
\end{problem}
\begin{solution}

$$\RREF\left(\begin{bmatrix} 1 & 3 & 3 & 7 \\ 1 & 3 & -1 & -1 \\ 2 & 6 & 3 & 8 \\ 1 & 3 & -2 & -3 \end{bmatrix}\right) = \begin{bmatrix} 1 & 3 & 0 & 1 \\ 0 & 0 & 1 & 2 \\ 0 & 0 & 0 & 0 \end{bmatrix}$$

Then a basis for the kernel is
$$\left\{ \begin{bmatrix} -3 \\ 1 \\ 0 \\ 0 \end{bmatrix} , \begin{bmatrix} -1\\0 \\ -2 \\ 1 \end{bmatrix} \right\}$$
and a basis for the image is
$$\left\{ \begin{bmatrix} 1 \\ 1 \\ 2 \\ 1 \end{bmatrix} , \begin{bmatrix} 3 \\ -1 \\ 3 \\ -2 \end{bmatrix}\right\} $$
\end{solution}

\begin{problem}{M1}
Let 
\begin{align*}
A &= \begin{bmatrix} 1 & 3 & -1 & -1 \\ 0 & 0 & 7 & 2 \end{bmatrix} & B &= \begin{bmatrix} 0 & 1 & 7 & 7 \\ -1 & -2 & 0 & 4 \\ 0 & 0 & 1 & 5 \end{bmatrix} & C&=\begin{bmatrix} 3 & 2 \\ 0 & 1 \\ -2 & -1 \end{bmatrix}
\end{align*}
Exactly one of the six products $AB$, $AC$, $BA$, $BC$, $CA$, $CB$ can be computed.  Determine which one, and compute it.
\end{problem}
\begin{solution}
$CA$ is the only one that can be computed, and 
$$CA = \begin{bmatrix} 3 & 9 & 11 & 1 \\ 0 & 0 & 7 & 2 \\ -2 & - 6 & -5 & 0 \end{bmatrix}$$
\end{solution}

\begin{problem}{M2}
Determine if the matrix $\begin{bmatrix} 3 & -1 & 0 & 4 \\ 2 & 1 & 1 & 1 \\ 0 & 1 & 1 & -1 \\ 1 & -2 & 0 & 3 \end{bmatrix}$ is invertible.
\end{problem}
\begin{solution}
$$\RREF \begin{bmatrix} 3 & -1 & 0 & 4 \\ 2 & 1 & 1 & 1 \\ 0 & 1 & 1 & -1 \\ 1 & -2 & 0 & 3 \end{bmatrix} = \begin{bmatrix} 1 & 0 & 0 & 1 \\ 0 & 1 & 0 & -1 \\ 0 & 0 & 1 & 0 \\ 0 & 0 & 0 & 0 \end{bmatrix}$$
This matrix is not row equivalent to the identity matrix, so it is not invertible.
\end{solution}


\begin{problem}{M3}
Find the inverse of the matrix $\begin{bmatrix} 3 & -1 & 0  \\ 2 & 1 & 1  \\ 0 & 1 & 1   \end{bmatrix}$.
\end{problem}
\begin{solution}
$$\begin{bmatrix} 3 & -1 & 0  \\ 2 & 1 & 1  \\ 0 & 1 & 1   \end{bmatrix}^{-1} = \begin{bmatrix} 0 & \frac{1}{2} & -\frac{1}{2} \\ -1 & \frac{3}{2} & -\frac{3}{2} \\ 1 & -\frac{3}{2} & \frac{5}{2}\end{bmatrix}$$
\end{solution}



\begin{problem}{G1}
Compute the determinant of the matrix $\begin{bmatrix} 8 & 5 & 3 & 0 \\ 3 & 2 & 1 & 1 \\ 5 & -3 & 1 & -2 \\ -1 & 2 & 0 & 1\end{bmatrix} $.
\end{problem}
\begin{solution}
$-1$.
\end{solution}

\begin{problem}{G2} 
Compute the eigenvalues, along with their algebraic multiplicities, of the matrix $ \begin{bmatrix} 9 & -3 & 2 \\ 23 & -8 & 5 \\  2 & -1 & 1 \end{bmatrix}$.
\end{problem}
\begin{solution}
The eigenvalues are $-1$, $1$, and $2$, each with multiplicity 1.
\end{solution}

\begin{problem}{G3}
Compute the eigenspace of the eigenvalue $-1$ in the matrix $\begin{bmatrix} 4 & -2 & -1 \\ 15 & -7 & -3 \\ -5 & 2 & 0 \end{bmatrix}$. 
\end{problem}
\begin{solution}
$$\RREF\left(A+I\right) = \begin{bmatrix} 1 & - \frac{2}{5} & -\frac{1}{5} \\ 0 & 0 & 0 \\ 0 & 0 & 0 \end{bmatrix}$$
So the eigenspace is spanned by $\begin{bmatrix} 2 \\5 \\  0 \end{bmatrix}$ and $\begin{bmatrix} 1 \\ 0 \\ 5 \end{bmatrix}$.
\end{solution}


\begin{problem}{G4}
Compute the geometric multiplicity of the eigenvalue $2$ in the matrix $\begin{bmatrix} -1 & 1 & 0 \\ -9 & 5 & 0 \\ 15 & -5 & 2 \end{bmatrix}$.
\end{problem}

\begin{solution}
The eigenspace is the solution space of the system $(B-2I)X=0$.
$$\RREF(B-2I)=\RREF\left(\begin{bmatrix} -3 & 1 & 0 \\ -9 & 3 & 0 \\ 15 & - 5 & 0 \end{bmatrix} \right) = \begin{bmatrix} 1 & -\frac{1}{3} & 0 \\ 0 & 0 & 0 \\ 0 & 0 & 0 \end{bmatrix}$$
Thus the geometric multiplicity is 2.
\end{solution}
\end{document}