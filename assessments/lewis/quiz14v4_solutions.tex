\documentclass{sbgLAquiz}

\begin{extract*}
\usepackage{amsmath,amssymb,amsthm,enumerate}
\coursetitle{Math 237}
\courselabel{Linear Algebra}
\calculatorpolicy{You may use a calculator, but you must show all relevant work to receive credit for a standard.}


\newcommand{\IR}{\mathbb{R}}
\newcommand{\IC}{\mathbb{C}}
\renewcommand{\P}{\mathcal{P}}
\renewcommand{\Im}{{\rm Im\ }}
\DeclareMathOperator{\RREF}{RREF}
\DeclareMathOperator{\vspan}{span}

\makeatletter
\renewcommand*\env@matrix[1][*\c@MaxMatrixCols c]{%
  \hskip -\arraycolsep
  \let\@ifnextchar\new@ifnextchar
  \array{#1}}
\makeatother

\title{Mastery Quiz Day 14 }
\standard{V1,V3,V4,S2}
\version{4}
\end{extract*}

\begin{document}

\begin{problem}{V1}
Let $V$ be the set of all pairs of real numbers with the operations, for any $(x_1,y_1), (x_2,y_2) \in V$, $c \in \IR$,
\begin{align*}
(x_1,y_1) \oplus (x_2,y_2) &= (x_1+x_2,y_1+y_2) \\
c \odot (x_1,y_1) &= (0, cy_1)
\end{align*}
\begin{enumerate}[(a)]
\item Show that scalar multiplication
      \textbf{distributes vectors} over scalar addition:
      \((c+d)\odot(x,y)=
      c\odot(x,y) \oplus d\odot(x,y)\).
\item Determine if $V$ is a vector space or not.  Justify your answer.
\end{enumerate}
\end{problem}
\begin{solution}
Let $(x_1,y_1) \in V$, and let $c,d \in \IR$.  Then
$$(c+d)\odot (x_1,y_1)=(0, (c+d)y_1) = (0,cy_1) \oplus (0,dy_1) = c \odot (x_1,y_1) \oplus d \odot (x_1,y_1).$$
However, $V$ is not a vector space, as $1 \odot (x_1,y_1) = (0,y_1) \neq (x_1,y_1)$.
\end{solution}


\begin{problem}{V3}
Does
\(
  \operatorname{span}\left\{
    \begin{bmatrix} 2 \\ -1 \\ 4 \end{bmatrix},
    \begin{bmatrix} 3 \\ 12 \\ -9 \end{bmatrix},
    \begin{bmatrix} 1 \\ 2 \\ 3 \end{bmatrix},
    \begin{bmatrix} -4 \\ 2 \\ -8 \end{bmatrix}
  \right\} = \IR^3
\)?
\end{problem}
\begin{solution}
Since
\[
  \RREF\begin{bmatrix}
    2 & 3 & 1 & -4 \\
    -1 & 12 & 2 & 2 \\
    4 & -9 & 3 & -8
  \end{bmatrix} =
  \begin{bmatrix}
    1 & 0 & 0 & -2 \\
    0 & 1 & 0 & 0 \\
    0 & 0 & 1 & 0
  \end{bmatrix}
\]
lacks a zero row, the vectors span \(\IR^3\).
\end{solution}

\begin{extract}\newpage\end{extract}
\begin{problem}{V4}
Determine if the set of all lattice points, i.e. $\{(x,y)\ \big|\ \text{$x$ and $y$ are integers} \}$ is a subspace of $\IR^2$.
\end{problem}
\begin{solution}
This set is closed under addition, but not under scalar multiplication so it is not a subspace.
\end{solution}

\begin{problem}{S2}
Determine if the set $\left\{ x^2+x-1, 3x^2-x+1, 2x-2 \right\}$ is a basis of $\P_2$
\end{problem}
\begin{solution}
$$\RREF\left(\begin{bmatrix} 1 & 3 & 2 \\ 1 & -1 & 0 \\ -1 & 1 & -2 \end{bmatrix} \right)= \begin{bmatrix} 1 & 0 &0 \\ 0 & 1 & 0 \\ 0 & 0 & 1\end{bmatrix}$$
Since the resulting matrix is the identity matrix, it is a basis.
\end{solution}


\end{document}