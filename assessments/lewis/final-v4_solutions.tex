\documentclass{sbgLAexam}

\begin{extract*}
\usepackage{amsmath,amssymb,amsthm,enumerate}
\coursetitle{Math 237}
\courselabel{Linear Algebra}
\calculatorpolicy{You may use a calculator, but you must show all relevant work to receive credit for a standard.}


\newcommand{\IR}{\mathbb{R}}
\newcommand{\IC}{\mathbb{C}}
\renewcommand{\P}{\mathcal{P}}
\renewcommand{\Im}{{\rm Im\ }}
\DeclareMathOperator{\RREF}{RREF}
\DeclareMathOperator{\vspan}{span}

\makeatletter
\renewcommand*\env@matrix[1][*\c@MaxMatrixCols c]{%
  \hskip -\arraycolsep
  \let\@ifnextchar\new@ifnextchar
  \array{#1}}
\makeatother

\title{Final Exam}
\standard{E1,E2,E3,E4,V1,V2,V3,V4,S1,S2,S3,S4,A1,A2,A3,A4,M1,M2,M3,G1,G2,G3,G4}
\version{4}
\end{extract*}

\begin{document}

\begin{problem}{E1}
Write a system of linear equations corresponding to the following
augmented matrix.
\[
\begin{bmatrix}[ccc|c]
2 & -1 & 0 & 1  \\
-1 & 4 & 1 & -7  \\
1 & 2 & -1 & 0
\end{bmatrix}
\]
\end{problem}
\begin{solution}
\begin{align*}
2x_1-x_2&=1 \\
-x_1+4x_2+x_3&=-7 \\
x_1+2x_2-x_3 &= 0
\end{align*}
\end{solution}

\begin{problem}{E2}
Find \(\RREF A\), where
\[
  A =
  \begin{bmatrix}[cc|c]
    2 & -7 & 4 \\
    1 & -3 & 2 \\
    3 & 0 & 3
  \end{bmatrix}
\]
\end{problem}
\begin{solution}
\[
  \RREF A =
  \begin{bmatrix}[cc|c]
    1 & 0 & 0 \\
    0 & 1 & 0 \\
    0 & 0 & 1
  \end{bmatrix}
\]
\end{solution}

\begin{extract}\newpage\end{extract}
\begin{problem}{E3}
Find the solution set for the following system of linear equations.
\begin{align*}
2x_1-2x_2+6x_3-x_4 &=-1 \\
3x_1+6x_3+x_4 &= 5 \\
-4x_1+x_2-9x_3+2x_4&=-7
\end{align*}
\end{problem}
\begin{solution}
Let \(A =
  \begin{bmatrix}[cccc|c]
    2 & -2 & 6 & -1 & -1 \\
    3 & 0 & 6 & 1 & 5 \\
    -4 & 1 & -9 & 2 & -7
  \end{bmatrix}
\), so \(\RREF A =
  \begin{bmatrix}[cccc|c]
    1 & 0 & 2 & 0 & 2 \\
    0 & 1 & -1 & 0 & 3 \\
    0 & 0 & 0 & 1 & -1
  \end{bmatrix}
\). It follows that the solution set is given by \(
  \begin{bmatrix}
    2 - 2a \\
    3 + a \\
    a \\
    -1
  \end{bmatrix}
\) for all real numbers \(a\).
\end{solution}

\begin{problem}{E4}
Find a basis for the solution set to the system of equations
\begin{align*}
x+2y-3z &= 0 \\
2x+y-4z &= 0 \\
3y -2z & = 0 \\
x -y -z &= 0
\end{align*}
\end{problem}
\begin{solution}
$$\RREF \left(\begin{bmatrix} 1 & 2 & -3 \\ 2 & 1 & -4 \\ 0 & 3 & -2 \\ 1 & -1 & -1 \end{bmatrix} \right) = \begin{bmatrix} 1 & 0 & -\frac{5}{3} \\ 0 & 1 & -\frac{2}{3} \\ 0 & 0 & 0 \\ 0 & 0 & 0 \end{bmatrix}$$
Then the solution set is
$$\left\{ \begin{bmatrix} \frac{5}{3}a \\ \frac{2}{3}a \\ a \end{bmatrix} \bigg|\ a \in \IR \right\}$$
So a basis is $\left\{ \begin{bmatrix} \frac{5}{3} \\ \frac{2}{3} \\ 1 \end{bmatrix} \right\}$ or $\left\{ \begin{bmatrix} 5 \\  2 \\ 3 \end{bmatrix} \right\}$.
\end{solution}

\begin{extract}\newpage\end{extract}
\begin{problem}{V1}
Let $V$ be the set of all polynomials with the operations, for any $f, g \in V$, $c \in \IR$,
\begin{align*}
f \oplus g &= f^\prime + g^\prime \\
c \odot f &= c f^\prime
\end{align*}
(here $f^\prime$ denotes the derivative of $f$).
\begin{enumerate}[(a)]
\item Show that scalar multiplication \textbf{distributes scalars} over
      vector addition:
      \(c\odot(f \oplus g)=
      c\odot f \oplus c\odot g\).
\item Determine if $V$ is a vector space or not.  Justify your answer.
\end{enumerate}
\end{problem}
\begin{solution}
Let $f,g \in \mathcal{P}$, and let $c \in \IR$.
$$c \odot (f \oplus g) = c \odot (f^\prime+g^\prime) =
c(f^\prime+g^\prime)^\prime = cf^{\prime\ \prime}+cg^{\prime\ \prime} =
cf^\prime\oplus cg^\prime= c \odot f \oplus c \odot g.$$
However, this is not a vector space, as there is no zero vector.  Additionally, $1 \odot f \neq f$ for any nonzero polynomial $f$.
\end{solution}


\begin{problem}{V2} Determine if $\begin{bmatrix} 1 \\ 4 \\ 3 \end{bmatrix}$ is a linear combination of the vectors $\begin{bmatrix} 3 \\ 0 \\ -1 \end{bmatrix}$, $\begin{bmatrix} 1 \\ -1 \\ 4 \end{bmatrix}$, and $\begin{bmatrix} 5 \\ 1 \\  -6 \end{bmatrix}$.
\end{problem}
\begin{solution}
$$\RREF\left(\begin{bmatrix}[ccc|c] 3 & 1 & 5 & 1 \\ 0 & -1 & 1 & 4 \\ -1 & 4 & -6 & 3 \end{bmatrix} \right) = \begin{bmatrix}[ccc|c] 1 & 0 & 2 & 0 \\ 0 & 1 & -1 & 0 \\ 0 & 0 & 0 & 1 \end{bmatrix}$$
So $\begin{bmatrix} 1 \\ 4 \\ 3 \end{bmatrix}$ is not a linear combination of the three vectors.
\end{solution}


\begin{extract}\newpage\end{extract}
\begin{problem}{V3}
Determine if the vectors $\begin{bmatrix} 1 \\ 1 \\ 2 \\1 \end{bmatrix}$, $\begin{bmatrix} 3 \\ 3 \\ 6 \\ 3 \end{bmatrix}$, $\begin{bmatrix}3 \\ -1 \\ 3 \\ -2\end{bmatrix}$, and $\begin{bmatrix} 7 \\ -1 \\ 8 \\ -3 \end{bmatrix}$  span $\IR^4$.
\end{problem}
\begin{solution}
$$\RREF\left(\begin{bmatrix} 1 & 3 & 3 & 7 \\ 1 & 3 & -1 & -1 \\ 2 & 6 & 3 & 8 \\ 1 & 3 & -2 & -3 \end{bmatrix}\right) = \begin{bmatrix} 1 & 3 & 0 & 1 \\ 0 & 0 & 1 & 2 \\ 0 & 0 & 0 & 0 \\ 0 & 0 & 0 & 0  \end{bmatrix}$$
Since there are zero rows, they do not span.  Alternatively, by inspection $\begin{bmatrix} 3 \\ 3 \\ 6 \\ 3 \end{bmatrix}=3\begin{bmatrix} 1 \\ 1 \\ 2 \\1 \end{bmatrix}$, so the set is linearly dependent, so it spans a subspace of dimension at most 3, therefore it does not span $\IR^4$.
\end{solution}

\begin{problem}{V4} Let $W$ be the set of all polynomials of the form
\(ax^3+bx\).  Determine if $W$ is a subspace of \(\P^3\).
\end{problem}
\begin{solution}
Yes because \(s(a_1x^3+b_1x)+t(a_2x^3+b_2x)=
(sa_1+ta_2)x^3+(sb_1+tb_2)x\) also belongs to \(W\).
Alternately, yes because \(W\) is isomorphic to \(\IR^2\).
\end{solution}


\begin{extract}\newpage\end{extract}
\begin{problem}{S1}
Determine if the set of polynomials  $\left\{x^2+x, x^2+2x-1, x^2+3x-2\right\}$ is  linearly dependent or linearly independent
\end{problem}
\begin{solution}
$$\RREF\left( \begin{bmatrix} 1 &  1 & 1 \\ 1  & 2 & 3 \\ 0  & -1 & -2 \end{bmatrix} \right) = \begin{bmatrix} 1 &  0 & -1 \\ 0  & 1 & 2 \\ 0& 0 & 0  \end{bmatrix}$$
Since there is a nonpivot column, the set is linearly dependent.
\end{solution}


\begin{problem}{S2}
  Determine if the set \(\left\{
    \begin{bmatrix} 3 \\ -1 \\ 2 \\3 \end{bmatrix},
    \begin{bmatrix} 2 \\ 0 \\ 2 \\ 4\end{bmatrix},
    \begin{bmatrix} 1 \\ -1 \\ 0 \\ -1\end{bmatrix},
    \begin{bmatrix} -1 \\ 3 \\ 0 \\ 5\end{bmatrix}
  \right\}\) is a basis of $\IR^4$.
\end{problem}
\begin{solution}
  \[\RREF\left(
    \begin{bmatrix}
      3 & 2 & 1 & -1\\
      -1 & 0 & -1 & 3\\
      2 & 2 & 0 & 0\\
      3 & 4 & -1 & 5\\
    \end{bmatrix} \right)= \begin{bmatrix}
      1 & 0 & 1 & 0 \\
      0 & 1 & -1 & 0 \\
      0 & 0 & 0 & 1 \\
      0 & 0 & 0 & 0
    \end{bmatrix}
  \]
Since the resulting matrix is not the identity matrix, it is not a basis.
\end{solution}


\begin{extract}\newpage\end{extract}
\begin{problem}{S3}
Let $W = {\rm span} \left( \left\{ \begin{bmatrix} 1 \\ 1 \\ 2 \\ 1 \end{bmatrix}, \begin{bmatrix} 3 \\ 3 \\ 6 \\ 3 \end{bmatrix}, \begin{bmatrix} 3 \\ -1 \\ 3 \\ -2 \end{bmatrix}, \begin{bmatrix} 7 \\ -1 \\ 8 \\ -3 \end{bmatrix} \right\} \right)$.  Find a basis for $W$.
\end{problem}
\begin{solution}
$$\RREF\left(\begin{bmatrix} 1 & 3 & 3 & 7 \\ 1 & 3 & -1 & -1 \\ 2 & 6 & 3 & 8 \\ 1 & 3 & -2 & -3 \end{bmatrix}\right) = \begin{bmatrix} 1 & 3 & 0 & 1 \\ 0 & 0 & 1 & 2 \\ 0 & 0 & 0 & 0 \\  0 & 0 & 0 & 0 \end{bmatrix}$$

Then a basis is
$ \left\{ \begin{bmatrix} 1 \\ 1 \\ 2 \\ 1 \end{bmatrix} , \begin{bmatrix} 3 \\ -1 \\ 3 \\ -2 \end{bmatrix}\right\} $.
\end{solution}


\begin{problem}{S4}
Let $W={\rm span}\left(\left\{\begin{bmatrix} 2 \\ 0 \\ -2 \\ 0 \end{bmatrix}, \begin{bmatrix} 3 \\ 1 \\ 3 \\ 6 \end{bmatrix}, \begin{bmatrix} 0 \\ 0 \\ 1 \\ 1 \end{bmatrix}, \begin{bmatrix}1 \\ 2 \\ 0 \\ 1 \end{bmatrix}\right\}\right)$. Compute the dimension of $W$.
\end{problem}
\begin{solution}
$$\RREF\left( \begin{bmatrix} 2 & 3 & 0 & 1 \\ 0 & 1 & 0 & 2 \\ -2 & 3 & 1 & 0 \\ 0 & 6 & 1 & 1\end{bmatrix} \right) = \begin{bmatrix}1 & 0 & 0 & -\frac{5}{2} \\ 0 & 1 & 0 & 2 \\ 0 & 0 & 1 & -11\\ 0 & 0 & 0 & 0  \end{bmatrix} $$
This has 3 pivot columns so  $\dim(W) =3$.
\end{solution}


\begin{extract}\newpage\end{extract}
\begin{problem}{A1}
Let $T: \IR^3\rightarrow \IR^4$ be the linear transformation given by $$T\left(\begin{bmatrix} x \\ y \\ z \\  \end{bmatrix} \right) = \begin{bmatrix} -3x+y \\ -8x+2y-z \\ 2y+3z \\ 7x \end{bmatrix}.$$  Write the matrix for $T$ with respect to the standard bases of $\IR^3$ and $\IR^4$.
\end{problem}
\begin{solution}
$$\begin{bmatrix} 3 & 1 & 0 \\ -8 & 2 & -1 \\ 0 & 2 & 3 \\ 7 & 0 & 0 \end{bmatrix}$$
\end{solution}

\begin{problem}{A2}
Determine if $D: \IR^{2\times 2} \rightarrow \IR$ given by $D\left(\begin{bmatrix} a & b \\ c & d \end{bmatrix} \right) = ad-bc$ is a linear transformation or not.
\end{problem}
\begin{solution}
$D(I)=1$ but $D(2I)=4 \neq 2D(I)$, so $D$ is not linear.
\end{solution}

\begin{extract}\newpage\end{extract}
\begin{problem}{A3}
Determine if each of the following linear transformations is injective (one-to-one) and/or surjective (onto).
\begin{enumerate}[(a)]
\item $T: \IR^3 \rightarrow \IR^3$ given by $T\left(\begin{bmatrix} x \\ y \\ z  \end{bmatrix} \right) = \begin{bmatrix} x+y+z \\ 2y+3z \\ x-y-2z \end{bmatrix} $
\item $S: \IR^2 \rightarrow \IR^3$ given by $S\left(\begin{bmatrix} x \\ y  \end{bmatrix} \right) = \begin{bmatrix} 3x+2y \\ x-y \\ x+4y \end{bmatrix} $
\end{enumerate}
\end{problem}
\begin{solution}
\begin{enumerate}[(a)]
\item $$\RREF\left( \begin{bmatrix} 1 &  1 & 1 \\ 0  & 2 & 3 \\ 1  & -1 & -2 \end{bmatrix} \right) = \begin{bmatrix} 1 &  0 & -\frac{1}{2} \\ 0  & 1 & \frac{3}{2} \\ 0& 0 & 0  \end{bmatrix}$$
Since there is a nonpivot column, $T$ is not injective.  Since there is a zero row, $T$ is not surjective.
\item $$\RREF \left( \begin{bmatrix} 3 & 2 \\ 1 & -1 \\ 1 & 4 \end{bmatrix} \right) = \begin{bmatrix} 1 & 0 \\ 0 & 1 \\ 0 & 0 \end{bmatrix}$$
Since all columns are pivot columns, $S$ is injective.  Since there is a zero row, $S$ is not surjective.
\end{enumerate}
\end{solution}



\begin{problem}{A4}
Let $T: \IR^{2\times 2} \rightarrow \IR^3$ be the linear map given by $T\left(\begin{bmatrix} x & y \\ z & w \end{bmatrix} \right) = \begin{bmatrix}  8x-3y-z+4w \\ y+3z-4w \\ -7x+3y+2z-5w\end{bmatrix} $.
Compute a basis for the kernel and a basis for the image of $T$.
\end{problem}
\begin{solution}
$$\RREF \left( \begin{bmatrix} 8 & -3 & -1 & 4 \\ 0 & 1 & 3 & -4 \\ -7 & 3 & 2 & -5 \end{bmatrix} \right) = \begin{bmatrix} 1 & 0 & 1 & -1 \\ 0 & 1 & 3 & -4 \\ 0 & 0 & 0 & 0 \end{bmatrix}$$

Thus \(\left\{ \begin{bmatrix} 8 \\ 0 \\ -7 \end{bmatrix}, \begin{bmatrix} -3 \\ 1 \\ 3 \end{bmatrix} \right\}\) is a basis for the image, and \( \left\{ \begin{bmatrix} -1 & -3 \\ 1 & 0 \end{bmatrix}, \begin{bmatrix} 1 & 4 \\ 0 & 1 \end{bmatrix} \right\} \) is a basis for the kernel.
\end{solution}


\begin{extract}\newpage\end{extract}
\begin{problem}{M1}
Let 
\begin{align*}
A &= \begin{bmatrix} 3 \\ 5 \\ -1  \end{bmatrix} & B&=\begin{bmatrix} 1 & -1 & 3 & -3 \\ 2 & 1 & -1 & 2 \end{bmatrix} & C &= \begin{bmatrix} 2 & -1 \\ 0 & 4 \\ 3 & 1 \end{bmatrix} \end{align*}
Exactly one of the six products $AB$, $AC$, $BA$, $BC$, $CA$, $CB$ can be computed.  Determine which one, and compute it.
\end{problem}
\begin{solution}
$CB$ is the only one that can be computed, and
$$CB=\begin{bmatrix} 0 & -3 & 7 & -8 \\ 8 & 4 & -4 & 8 \\ 5 & -2 & 8 & -7 \end{bmatrix}$$
\end{solution}

\begin{problem}{M2}
Determine if the matrix $\begin{bmatrix} 3 & -1 & 0  \\ 2 & 1 & 1  \\ 0 & 1 & 1   \end{bmatrix}$ is invertible.
\end{problem}
\begin{solution}
It is row equivalent to the identity matrix, so it is invertible.
\end{solution}


\begin{extract}\newpage\end{extract}
\begin{problem}{M3}
  Find the inverse of the matrix
  \(\begin{bmatrix}
    4 & -1 & -8  \\
    2 & 1 & 3  \\
    1 & 1 & 4
  \end{bmatrix}\).
\end{problem}
\begin{solution}
\(\begin{bmatrix}[ccc|ccc]
  4 & -1 & -8 & 1 & 0 & 0  \\
  2 & 1 & 3   & 0 & 1 & 0 \\
  1 & 1 & 4   & 0 & 0 & 1
\end{bmatrix}\sim\begin{bmatrix}[ccc|ccc]
  1 & 0 & 0 & 1 & -4 & 5 \\
  0 & 1 & 0 & -5 & 24 & -28 \\
  0 & 0 & 1 & 1 & -5 & 6
\end{bmatrix}\). Thus the inverse is
\(\begin{bmatrix}
  1 & -4 & 5  \\
  -5 & 24 & -28  \\
  1 & -5 & 6
\end{bmatrix}\).
\end{solution}


\begin{problem}{G1}
Compute the determinant of the matrix $\begin{bmatrix} 2 & 3 & 0 & 1 \\ -1 & 3 & 1 & 4 \\ 0 & 2 & 0 & 3 \\ 1 & -1 & 3 & 5 \end{bmatrix}$.
\end{problem}
\begin{solution}
$-60$.
\end{solution}

\begin{extract}\newpage\end{extract}
\begin{problem}{G2}
Compute the eigenvalues, along with their algebraic multiplicities, of the matrix $ \begin{bmatrix} 9 & -3 & 2 \\ 19 & -6 & 5 \\ -5 & 2 & 0 \end{bmatrix}$.
\end{problem}
\begin{solution}
1 (with algebraic multiplicity 3)
\end{solution}

\begin{problem}{G3}
Find the eigenspace associated to the eigenvalue $2$ in the matrix $A=\begin{bmatrix}0 & -2 & -1 & 0 \\ -4 & -2 & -2 & 0 \\ 14 & 12 & 10 & 2 \\ -13 & -10 & -8 & -1 \end{bmatrix}$.
\end{problem}
\begin{solution}
The eigenspace is spanned by $\begin{bmatrix} -1 \\ \frac{1}{2} \\ 1 \\ 0 \end{bmatrix}$ and  $\begin{bmatrix} -1 \\ 1 \\ 0 \\ 1 \end{bmatrix}$.
\end{solution}

\begin{extract}\newpage\end{extract}
\begin{problem}{G4}
Compute the geometric multiplicity of the eigenvalue $2$ in the matrix $A=\begin{bmatrix} 0 & 1 & 0 & 0 \\ -4 & 4 & 0 & 0 \\ 11 & -6 & 1 & -1 \\ -9 & 5 & 1 & 3 \end{bmatrix}$.
\end{problem}
\begin{solution}
The eigenspace is spanned by $\begin{bmatrix} -1 \\ -2 \\ 1 \\ 0 \end{bmatrix}$ and $\begin{bmatrix} -1 \\ -2 \\ 0 \\ 1 \end{bmatrix}$, so the geometric multiplicity is $2$.
\end{solution}

\end{document}