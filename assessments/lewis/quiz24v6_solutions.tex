\documentclass{sbgLAquiz}

\begin{extract*}
\usepackage{amsmath,amssymb,amsthm,enumerate}
\coursetitle{Math 237}
\courselabel{Linear Algebra}
\calculatorpolicy{You may use a calculator, but you must show all relevant work to receive credit for a standard.}


\newcommand{\IR}{\mathbb{R}}
\newcommand{\IC}{\mathbb{C}}
\renewcommand{\P}{\mathcal{P}}
\renewcommand{\Im}{{\rm Im\ }}
\DeclareMathOperator{\RREF}{RREF}
\DeclareMathOperator{\vspan}{span}

\makeatletter
\renewcommand*\env@matrix[1][*\c@MaxMatrixCols c]{%
  \hskip -\arraycolsep
  \let\@ifnextchar\new@ifnextchar
  \array{#1}}
\makeatother

\title{Mastery Quiz Day 24 }
\standard{A1,A2,M1,M2,M3}
\version{6}
\end{extract*}

\begin{document}

\begin{problem}{A1}
Let $T: \IR^3\rightarrow \IR^4$ be the linear transformation given by $$T\left(\begin{bmatrix} x \\ y \\ z \\  \end{bmatrix} \right) = \begin{bmatrix} -3x+y \\ -8x+2y-z \\ 2y+3z \\ 0 \end{bmatrix}.$$  Write the matrix for $T$ with respect to the standard bases of $\IR^3$ and $\IR^4$.
\end{problem}
\begin{solution}
$$\begin{bmatrix} 3 & 1 & 0 \\ -8 & 2 & -1 \\ 0 & 2 & 3 \\ 0 & 0 & 0 \end{bmatrix}$$
\end{solution}

\begin{problem}{A2} 
Let $T: \IR^2 \rightarrow \IR^2$ be given by $T\left(\begin{bmatrix}x \\ y  \end{bmatrix} \right) = \begin{bmatrix} x+y \\ \sqrt{x}+\sqrt{y} \end{bmatrix}$.  Determine if $T$ is a linear transformation.
\end{problem}
\begin{solution}
$$T\left(\begin{bmatrix} 0 \\ 4 \end{bmatrix}\right) = \begin{bmatrix} 4 \\ 2 \end{bmatrix} \neq \begin{bmatrix} 4 \\ 4 \end{bmatrix} = 4T\left(\begin{bmatrix} 0 \\ 1 \end{bmatrix} \right)$$
So $T$ is not a linear transformation.
\end{solution}

\begin{extract}\newpage\end{extract}
\begin{problem}{M1}
Let 
\begin{align*}
A &= \begin{bmatrix} 3 \\ 5 \\ -1  \end{bmatrix} & B &= \begin{bmatrix} 2 & -1 \\ 0 & 4 \\ 3 & 1 \end{bmatrix} & C&=\begin{bmatrix} 1 & -1 & 3 & -3 \\ 2 & 1 & -1 & 2 \end{bmatrix}
\end{align*}
Exactly one of the six products $AB$, $AC$, $BA$, $BC$, $CA$, $CB$ can be computed.  Determine which one, and compute it.
\end{problem}
\begin{solution}
$BC$ is the only one that can be computed, and
$$BC=\begin{bmatrix} 0 & -3 & 7 & -8 \\ 8 & 4 & -4 & 8 \\ 5 & -2 & 8 & -7 \end{bmatrix}$$
\end{solution}

\begin{problem}{M2}
Determine if the matrix $\begin{bmatrix} 3 & -1 & 0  \\ 2 & 1 & 1  \\ 0 & 1 & 1   \end{bmatrix}$ is invertible.
\end{problem}
\begin{solution}
It is row equivalent to the identity matrix, so it is invertible.
\end{solution}


\begin{extract}\newpage\end{extract}
\begin{problem}{M3}
  Find the inverse of the matrix
  \(\begin{bmatrix}
    3 & 1 & 3  \\
    2 & -1 & -6  \\
    1 & 1 & 4
  \end{bmatrix}\).
\end{problem}
\begin{solution}
\(\begin{bmatrix}[ccc|ccc]
  3 & 1 & 3 & 1 & 0 & 0 \\
  2 & -1 & -6 & 0 & 1 & 0 \\
  1 & 1 & 4 & 0 & 0 & 1
\end{bmatrix}\sim\begin{bmatrix}[ccc|ccc]
  1 & 0 & 0 & 2 & -1 & -3  \\
  0 & 1 & 0 & -14 & 9 & 24  \\
  0 & 0 & 1 & 3 & -2 & -5
\end{bmatrix}\). Thus the inverse is
\(\begin{bmatrix}
  2 & -1 & -3  \\
  -14 & 9 & 24  \\
  3 & -2 & -5
\end{bmatrix}\).
\end{solution}


\end{document}