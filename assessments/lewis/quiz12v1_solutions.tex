\documentclass{sbgLAquiz}

\begin{extract*}
\usepackage{amsmath,amssymb,amsthm,enumerate}
\coursetitle{Math 237}
\courselabel{Linear Algebra}
\calculatorpolicy{You may use a calculator, but you must show all relevant work to receive credit for a standard.}


\newcommand{\IR}{\mathbb{R}}
\newcommand{\IC}{\mathbb{C}}
\renewcommand{\P}{\mathcal{P}}
\renewcommand{\Im}{{\rm Im\ }}
\DeclareMathOperator{\RREF}{RREF}
\DeclareMathOperator{\vspan}{span}

\makeatletter
\renewcommand*\env@matrix[1][*\c@MaxMatrixCols c]{%
  \hskip -\arraycolsep
  \let\@ifnextchar\new@ifnextchar
  \array{#1}}
\makeatother

\title{Mastery Quiz Day 12 }
\standard{V1,V3,V4}
\version{1}
\end{extract*}

\begin{document}

\begin{problem}{V1}
Let $V$ be the set of all pairs of real numbers with the operations, for any $(x_1,y_1), (x_2,y_2) \in V$, $c \in \IR$,
\begin{align*}
(x_1,y_1) \oplus (x_2,y_2) &= (x_1+x_2,y_1+y_2) \\
c \odot (x_1,y_1) &= (c^2x_1, c^3y_1)
\end{align*}
\begin{enumerate}[(a)]
\item Show that scalar multiplication \textbf{distributes scalars} over
      vector addition:
      \(c\odot((x_1,y_1) \oplus (x_2,y_2))=
      c\odot(x_1,y_1) \oplus c\odot(x_2,y_2)\).
\item Determine if $V$ is a vector space or not.  Justify your answer.
\end{enumerate}
\end{problem}
\begin{solution}
Let $(x_1,y_1), (x_2,y_2) \in V$ and let $c \in \IR$.
\begin{align*}
c \odot \left( (x_1,y_1) \oplus (x_2,y_2) \right) &= c \odot (x_1+x_2,y_1+y_2) \\
&= (c^2(x_1+x_2), c^3(y_1+y_2) ) \\
&= (c^2x_1,c^3y_1) \oplus (c^2x_2,c^3y_2) \\
&= c \odot (x_1,y_1) \oplus c \odot (x_2,y_2)
\end{align*}
However, $V$ is not a vector space, as the other distributive law fails:
$$(c+d) \odot (x_1,y_1) = ( (c+d)^2 x_1, (c+d)^3 y_1) \neq ((c^2+d^2)x_1, (c^3+d^3)y_1) = c \odot (x_1,y_1) \oplus d \odot (x_1,y_1).$$
\end{solution}


\begin{problem}{V3}
Determine if the vectors $\begin{bmatrix} 1 \\ 1 \\ 2 \\1 \end{bmatrix}$, $\begin{bmatrix} 3 \\ 3 \\ 6 \\ 3 \end{bmatrix}$, $\begin{bmatrix}3 \\ -1 \\ 3 \\ -2\end{bmatrix}$, and $\begin{bmatrix} 7 \\ -1 \\ 8 \\ -3 \end{bmatrix}$  span $\IR^4$.
\end{problem}
\begin{solution}
$$\RREF\left(\begin{bmatrix} 1 & 3 & 3 & 7 \\ 1 & 3 & -1 & -1 \\ 2 & 6 & 3 & 8 \\ 1 & 3 & -2 & -3 \end{bmatrix}\right) = \begin{bmatrix} 1 & 3 & 0 & 1 \\ 0 & 0 & 1 & 2 \\ 0 & 0 & 0 & 0 \\ 0 & 0 & 0 & 0  \end{bmatrix}$$
Since there are zero rows, they do not span.  Alternatively, by inspection $\begin{bmatrix} 3 \\ 3 \\ 6 \\ 3 \end{bmatrix}=3\begin{bmatrix} 1 \\ 1 \\ 2 \\1 \end{bmatrix}$, so the set is linearly dependent, so it spans a subspace of dimension at most 3, therefore it does not span $\IR^4$.
\end{solution}

\begin{extract}\newpage\end{extract}
\begin{problem}{V4} Let \(W\) be the set of all complex numbers
that are purely real (i.e of the form $a+0i$)  or purely imaginary (i.e. of the form $0+bi$).
Determine if \(W\) is a subspace of \(\IC\).
\end{problem}
\begin{solution}
No, because \(1\) is purely real and \(i\) is purely imaginary, but
the linear combination \(1+i\) is neither.
\end{solution}


\end{document}