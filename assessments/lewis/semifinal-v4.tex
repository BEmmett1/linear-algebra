%% 
%% This is file, `semifinal-v4.tex',
%% generated with the extract package.
%% 
%% Generated on :  2017/11/20,9:43
%% From source  :  semifinal-v4_solutions.tex
%% Using options:  active,generate=semifinal-v4,extract-env={problem}
%% 
\documentclass{sbgLAsemi}
\usepackage{amsmath,amssymb,amsthm,enumerate}
\coursetitle{Math 237}
\courselabel{Linear Algebra}
\calculatorpolicy{You may use a calculator, but you must show all relevant work to receive credit for a standard.}


\newcommand{\IR}{\mathbb{R}}
\newcommand{\IC}{\mathbb{C}}
\renewcommand{\P}{\mathcal{P}}
\renewcommand{\Im}{{\rm Im\ }}
\DeclareMathOperator{\RREF}{RREF}
\DeclareMathOperator{\vspan}{span}

\makeatletter
\renewcommand*\env@matrix[1][*\c@MaxMatrixCols c]{%
  \hskip -\arraycolsep
  \let\@ifnextchar\new@ifnextchar
  \array{#1}}
\makeatother

\title{Semifinal}
\version{4}

\begin{document}

\begin{problem}{E1}
Write an augmented matrix corresponding to the following system of linear equations.
\begin{align*}
x_1+3x_2-4x_3 +x_4 &= 5 \\
3x_1+9x_2+x_3-7x_4 &= 0 \\
x_1-x_3 +x_4 &= 1
\end{align*}
\end{problem}

\begin{problem}{E2}
Put the following matrix in reduced row echelon form.
$$\begin{bmatrix}
 3 & -1 & 0 \\
 -1 & 0 & -1 \\
 -1 & 1 & 2 \\
 0 & 2 & 6
\end{bmatrix}$$
\end{problem}

\begin{problem}{E3}
Find the solution set for the following system of linear equations.
\begin{align*}
2x_1-2x_2+6x_3-x_4 &=-1 \\
3x_1+6x_3+x_4 &= 5 \\
-4x_1+x_2-9x_3+2x_4&=-7
\end{align*}
\end{problem}

\begin{problem}{E4}
Find a basis for the solution set of the system of equations
\begin{align*}
x+3y+3z+7w &= 0 \\
 x+3y-z-w &= 0 \\
  2x+6y+3z+8w &= 0 \\
   x+3y-2z-3w &= 0
\end{align*}
\end{problem}

\begin{problem}{V1}
Let $V$ be the set of all points on the parabola $y=x^2$ with the operations, for any $(x_1,y_1), (x_2,y_2) \in V$, $c \in \IR$,
\begin{align*}
(x_1,y_1) \oplus (x_2,y_2) &= (x_1+x_2,y_1+y_2+2x_1x_2) \\
c \odot (x_1,y_1) &= (cx_1, c^2y_1)
\end{align*}
\begin{enumerate}[(a)]
\item Show that the vector \textbf{addition} $\oplus$ is \textbf{associative}:
      \((x_1,y_1) \oplus ((x_2,y_2) \oplus (x_3,y_3))=((x_1,y_1)\oplus (x_2,y_2))\oplus (x_3,y_3)\).
\item Determine if $V$ is a vector space or not.  Justify your answer.
\end{enumerate}
\end{problem}

\begin{problem}{V2}
Determine if $\begin{bmatrix} 0 \\ 1 \\ -2 \\ 1 \end{bmatrix}$ can be written as a linear combination of the vectors $\begin{bmatrix} 5 \\ 2 \\ -3 \\ 2 \end{bmatrix}$, $\begin{bmatrix} 3 \\ 1 \\ 1 \\ 0 \end{bmatrix}$, and $\begin{bmatrix} 8 \\ 3 \\ 5 \\ -1 \end{bmatrix}$.
\end{problem}

\begin{problem}{V3}
Does
\(
  \operatorname{span}\left\{
    \begin{bmatrix} 2 \\ -1 \\ 4 \end{bmatrix},
    \begin{bmatrix} 3 \\ 12 \\ -9 \end{bmatrix},
    \begin{bmatrix} 1 \\ 2 \\ 3 \end{bmatrix},
    \begin{bmatrix} -4 \\ 2 \\ -8 \end{bmatrix}
  \right\} = \IR^3
\)?
\end{problem}

\begin{problem}{V4} Let \(W\) be the set of all \(\IR^3\) vectors
\(\begin{bmatrix} x \\ y \\ z \end{bmatrix}\)
satisfying \(x+y+z=1\) (this forms a plane).
Determine if \(W\) is a subspace of \(\IR^3\).
\end{problem}

\begin{problem}{S1}
Determine if the set of matrices $\left\{\begin{bmatrix} 3 & -1 \\ 0 & 4 \end{bmatrix}, \begin{bmatrix} 1  & 2 \\ -2 & 1 \end{bmatrix}, \begin{bmatrix} 3 & -8 \\ 6 & 5 \end{bmatrix} \right\}$  is linearly dependent or linearly independent.
\end{problem}

\begin{problem}{S2}
Determine if the set $\left\{ x^2+x-1, 3x^2-x+1, 2x^2-2 \right\}$ is a basis of $\P^2$.
\end{problem}

\begin{problem}{S3}
Let $W$ be the subspace of $\P^3$ given by $W = {\rm span} \left( \left\{ x^3+x^2+2x+1, 3x^3+3x^2+6x+3, 3x^3-x^2+3x-2, 7x^3-x^2+8x-3 \right\} \right)$.  Find a basis for $W$.
\end{problem}

\begin{problem}{S4}
Let $W$ be the subspace of $\IR^{2\times2}$ given by $W={\rm span}\left(\left\{\begin{bmatrix} 2 & 0 \\ -2 & 0 \end{bmatrix}, \begin{bmatrix} 3 & 1 \\ 3 & 6 \end{bmatrix}, \begin{bmatrix} 0 & 0 \\ 1 & 1 \end{bmatrix}, \begin{bmatrix}1 & 2 \\ 0 & 1 \end{bmatrix}\right\}\right)$. Compute the dimension of $W$.
\end{problem}

\begin{problem}{A1}
Let $T: \IR^3\rightarrow \IR^4$ be the linear transformation given by $$T\left(\begin{bmatrix} x \\ y \\ z \\  \end{bmatrix} \right) = \begin{bmatrix} -3x+y \\ -8x+2y-z \\ 2y+3z \\ 0 \end{bmatrix}.$$  Write the matrix for $T$ with respect to the standard bases of $\IR^3$ and $\IR^4$.
\end{problem}

\begin{problem}{A2}
Determine if the map $T: \P^3 \rightarrow \P^4$ given by $T(f(x))=xf(x)-f(x)$ is a linear transformation or not.
\end{problem}

\begin{problem}{A3}
Determine if each of the following linear transformations is injective (one-to-one) and/or surjective (onto).
\begin{enumerate}[(a)]
\item
  \(S: \IR^2 \rightarrow \IR^3\) where
  \(S(\vec e_1)=\begin{bmatrix}
    2 \\
    1 \\
    0
  \end{bmatrix}\) and
  \(S(\vec e_2)=\begin{bmatrix}
    1 \\
    2 \\
    1
  \end{bmatrix}\).
\item
  \(T: \IR^3 \rightarrow \IR^2\) where
  \(T(\vec e_1)=\begin{bmatrix}
    2 \\
    2
  \end{bmatrix}\),
  \(T(\vec e_2)=\begin{bmatrix}
   1  \\
   0
  \end{bmatrix}\), and
  \(T(\vec e_3)=\begin{bmatrix}
    1 \\
    1
  \end{bmatrix}\).
\end{enumerate}
\end{problem}

\begin{problem}{A4}
Let $T: \IR^4 \rightarrow \IR^3$ be the linear map given by $T\left(\begin{bmatrix} x \\ y \\ z \\ w \end{bmatrix} \right) = \begin{bmatrix}  8x-3y-z+4w \\ y+3z-4w \\ -7x+3y+2z-5w\end{bmatrix} $.
Compute a basis for the kernel and a basis for the image of $T$.
\end{problem}

\begin{problem}{M1}
Let
\begin{align*}
A &= \begin{bmatrix} 2 & 3 \\ 0 & 1 \end{bmatrix} & B&= \begin{bmatrix} 3 & 1 & 0 \end{bmatrix} & C&= \begin{bmatrix} 0 & -1 & 4 \\ 1 & -1 & 2 \end{bmatrix}
\end{align*}

Exactly one of the six products $AB$, $AC$, $BA$, $BC$, $CA$, $CB$ can be computed.  Determine which one, and compute it.
\end{problem}

\begin{problem}{M2}
Determine if the matrix $\begin{bmatrix} 3 & -1 & 0 & 4 \\ 2 & 1 & 1 & 1 \\ 0 & 1 & 1 & -1 \\ 1 & -2 & 0 & 3 \end{bmatrix}$ is invertible.
\end{problem}

\begin{problem}{M3} Find the inverse of the matrix $\begin{bmatrix} 6 & 0 & 1 \\ -14 & 3 & -4 \\ -23 & 4 & -6\end{bmatrix}$.
\end{problem}

\begin{problem}{G1}
Compute the determinant of the matrix
\[
  \begin{bmatrix}
    1 & 3 & 2 & 4 \\
    5 & 0 & -4 & 0 \\
    -2 & 3 & -1 & 1 \\
    0 & 1 & 0 & 1
  \end{bmatrix}
.\]
\end{problem}

\begin{problem}{G2}
Compute the eigenvalues, along with their algebraic multiplicities, of the matrix $ \begin{bmatrix} 9 & -3 & 2 \\ 19 & -6 & 5 \\ -11 & 4 & -2\end{bmatrix}$.
\end{problem}

\begin{problem}{G3}
Find the eigenspace associated to the eigenvalue $1$ in the matrix $A=\begin{bmatrix} -3 & 1 & 0 \\ -8 & 2 & -1 \\ 0 & 2 & 3 \end{bmatrix}$
\end{problem}

\begin{problem}{G4}
Compute the geometric multiplicity of the eigenvalue $2$ in the matrix $\begin{bmatrix} -1 & 1 & 0 \\ -9 & 5 & 0 \\ 15 & -5 & 2 \end{bmatrix}$.
\end{problem}

\end{document}
