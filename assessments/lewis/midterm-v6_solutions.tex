\documentclass{sbgLAexam}

\begin{extract*}
\usepackage{amsmath,amssymb,amsthm,enumerate}
\coursetitle{Math 237}
\courselabel{Linear Algebra}
\calculatorpolicy{You may use a calculator, but you must show all relevant work to receive credit for a standard.}


\newcommand{\IR}{\mathbb{R}}
\newcommand{\IC}{\mathbb{C}}
\renewcommand{\P}{\mathcal{P}}
\renewcommand{\Im}{{\rm Im\ }}
\DeclareMathOperator{\RREF}{RREF}
\DeclareMathOperator{\vspan}{span}

\makeatletter
\renewcommand*\env@matrix[1][*\c@MaxMatrixCols c]{%
  \hskip -\arraycolsep
  \let\@ifnextchar\new@ifnextchar
  \array{#1}}
\makeatother

\title{Midterm Exam}
\standard{E1,E2,E3,E4,V1,V2,V3,V4,S1,S2,S3,S4}
\version{6}
\end{extract*}

\begin{document}

\begin{problem}{E1}
Write an augmented matrix corresponding to the following system of linear equations.
\begin{align*}
x+3y-4z &= 5 \\
3x+9y+z &= 0 \\
x-z &= 1
\end{align*}
\end{problem}
\begin{solution}
\[
\begin{bmatrix}[ccc|c]
1 & 3 & -4 & 5 \\
3 & 9 & 1 & 0 \\
1 & 0 & -1 & 1
\end{bmatrix}
\]
\end{solution}

\begin{problem}{E2}
Put the following matrix in reduced row echelon form.
$$\begin{bmatrix}[ccc|c] -3 & 1 & 0 & 2 \\ -8 & 2 & -1 & 6 \\ 0 & 2 & 3 & -2 \end{bmatrix}$$
\end{problem}
\begin{solution}
$$\begin{bmatrix}[ccc|c]
-3 & 1 & 0 & 2 \\
 -8 & 2 & -1 & 6 \\
 0 & 2 & 3 & -2
\end{bmatrix} \sim
\begin{bmatrix}[ccc|c]
1 & -\frac{1}{3} & 0 & -\frac{2}{3} \\
 -8 & 2 & -1 & 6 \\
 0 & 2 & 3 & -2
\end{bmatrix} \sim
\begin{bmatrix}[ccc|c]
1 & -\frac{1}{3} & 0 & -\frac{2}{3} \\
 0 & -\frac{2}{3} & -1 & \frac{2}{3} \\
 0 & 2 & 3 & -2
\end{bmatrix} $$
$$\sim
\begin{bmatrix}[ccc|c]
1 & -\frac{1}{3} & 0 & -\frac{2}{3} \\
 0 & 1 & \frac{3}{2} & -1 \\
 0 & 2 & 3 & -2
\end{bmatrix} \sim
\begin{bmatrix}[ccc|c]
1 & 0 & \frac{1}{2} & -1 \\
 0 & 1 & \frac{3}{2} & -1 \\
 0 & 0 & 0 & 0
\end{bmatrix}$$
\end{solution}

\begin{extract}\newpage\end{extract}
\begin{problem}{E3}
Solve the system of equations
\begin{align*}
x+3y-4z &= 5 \\
3x+9y+z &= 2
\end{align*}
\end{problem}
\begin{solution}
$$\RREF \left(\begin{bmatrix}[ccc|c] 1 & 3 & -4 & 5 \\ 3 & 9 & 1 & 2 \end{bmatrix} \right) = \begin{bmatrix}[ccc|c] 1 & 3 & 0 & 1 \\ 0 & 0 & 1 & -1\end{bmatrix}$$
So the solution set is
$$\left\{ \begin{bmatrix} 1-3c \\ c \\ -1 \end{bmatrix} \bigg|\ c \in \IR \right\}$$
\end{solution}



\begin{problem}{E4}
Find a basis for the solution set to the homogeneous system of equations
given by
\begin{align*}
2x_1-2x_2+6x_3-x_4 &=0 \\
3x_1+6x_3+x_4 &=0 \\
-4x_1+x_2-9x_3+2x_4&=0
\end{align*}
\end{problem}
\begin{solution}
Let \(A =
  \begin{bmatrix}[cccc|c]
    2 & -2 & 6 & -1 & 0 \\
    3 & 0 & 6 & 1 & 0 \\
    -4 & 1 & -9 & 2 & 0
  \end{bmatrix}
\), so \(\RREF A =
  \begin{bmatrix}[cccc|c]
    1 & 0 & 2 & 0 & 0 \\
    0 & 1 & -1 & 0 & 0 \\
    0 & 0 & 0 & 1 & 0
  \end{bmatrix}
\).
It follows that the basis for the solution set is given by \(\left\{
  \begin{bmatrix}
    -2 \\
    1 \\
    1 \\
    0
  \end{bmatrix}
\right\}\).
\end{solution}
\begin{extract}\newpage\end{extract}
\begin{problem}{V1}
Let $V$ be the set of all pairs of real numbers with the operations, for any $(x_1,y_1), (x_2,y_2) \in V$, $c \in \IR$,
\begin{align*}
(x_1,y_1) \oplus (x_2,y_2) &= (x_1+x_2,y_1+y_2) \\
c \odot (x_1,y_1) &= (c^2x_1, c^3y_1)
\end{align*}
\begin{enumerate}[(a)]
\item Show that scalar multiplication \textbf{distributes scalars} over
      vector addition:
      \(c\odot((x_1,y_1) \oplus (x_2,y_2))=
      c\odot(x_1,y_1) \oplus c\odot(x_2,y_2)\).
\item Determine if $V$ is a vector space or not.  Justify your answer.
\end{enumerate}
\end{problem}
\begin{solution}
Let $(x_1,y_1), (x_2,y_2) \in V$ and let $c \in \IR$.
\begin{align*}
c \odot \left( (x_1,y_1) \oplus (x_2,y_2) \right) &= c \odot (x_1+x_2,y_1+y_2) \\
&= (c^2(x_1+x_2), c^3(y_1+y_2) ) \\
&= (c^2x_1,c^3y_1) \oplus (c^2x_2,c^3y_2) \\
&= c \odot (x_1,y_1) \oplus c \odot (x_2,y_2)
\end{align*}
However, $V$ is not a vector space, as the other distributive law fails:
$$(c+d) \odot (x_1,y_1) = ( (c+d)^2 x_1, (c+d)^3 y_1) \neq ((c^2+d^2)x_1, (c^3+d^3)y_1) = c \odot (x_1,y_1) \oplus d \odot (x_1,y_1).$$
\end{solution}


\begin{problem}{V2}
Determine if  $\begin{bmatrix} 0 \\ 0 \\ 2 \end{bmatrix}$ can be written as a linear combination of the vectors $\begin{bmatrix} -1 \\ -9 \\ 15 \end{bmatrix}$ and $\begin{bmatrix} 1 \\ 5 \\ -5 \end{bmatrix}$.
\end{problem}
\begin{solution}
$$\RREF\left(\begin{bmatrix}[cc|c] -1 & 1 & 0 \\ -9 & 5 & 0 \\ 15 & -5 & 2 \end{bmatrix} \right) = \begin{bmatrix}[cc|c] 1 & 0 & 0 \\ 0 & 1 & 0 \\ 0 & 0 & 1 \end{bmatrix}$$
Since this system has no solution, $\begin{bmatrix} 0 \\ 0 \\ 2 \end{bmatrix}$ cannot be written as a linear combination of the vectors $\begin{bmatrix} -1 \\ -9 \\ 15 \end{bmatrix}$ and $\begin{bmatrix} 1 \\ 5 \\ -5 \end{bmatrix}$.

\end{solution}


\begin{extract}\newpage\end{extract}
\begin{problem}{V3}
Does
\(
  \operatorname{span}\left\{
    \begin{bmatrix} 2 \\ -1 \\ 4 \\ 2 \\ 1 \end{bmatrix},
    \begin{bmatrix} -1 \\ 3 \\ 5 \\ 2 \\ 0 \end{bmatrix},
    \begin{bmatrix} 1 \\ 0 \\ 5 \\ 1 \\ -3 \end{bmatrix}
  \right\} = \IR^5
\)?
\end{problem}
\begin{solution}
Since there are only three vectors, they cannot span \(\IR^5\).
\end{solution}
\begin{problem}{V4}
Determine if the set of all lattice points, i.e. $\{(x,y)\ \big|\ \text{$x$ and $y$ are integers} \}$ is a subspace of $\IR^2$.
\end{problem}
\begin{solution}
This set is closed under addition, but not under scalar multiplication so it is not a subspace.
\end{solution}

\begin{extract}\newpage\end{extract}
\begin{problem}{S1}
Determine if the set of vectors $\left\{ \begin{bmatrix} -3 \\ -8 \\ 0 \end{bmatrix}, \begin{bmatrix} 1 \\ 2 \\ 2 \end{bmatrix}, \begin{bmatrix} 0 \\ -1 \\ 3 \end{bmatrix} \right\}$ is  linearly dependent or linearly independent
\end{problem}
\begin{solution}
$$\RREF\left( \begin{bmatrix}-3 & 1 & 0 \\ -8 & 2 & -1 \\ 0 & 2 & 3 \end{bmatrix}\right) = \begin{bmatrix} 1 & 0 & \frac{1}{2} \\ 0 & 1 & \frac{3}{2} \\ 0 & 0 & 0 \end{bmatrix}$$ 
This has a non pivot column, therefore the set is linearly dependent.
\end{solution}

\begin{problem}{S2}
Determine if the set $\left\{ x^2+x-1, 3x^2-x+1, 2x-2 \right\}$ is a basis of $\P_2$
\end{problem}
\begin{solution}
$$\RREF\left(\begin{bmatrix} 1 & 3 & 2 \\ 1 & -1 & 0 \\ -1 & 1 & -2 \end{bmatrix} \right)= \begin{bmatrix} 1 & 0 &0 \\ 0 & 1 & 0 \\ 0 & 0 & 1\end{bmatrix}$$
Since the resulting matrix is the identity matrix, it is a basis.
\end{solution}


\begin{extract}\newpage\end{extract}
\begin{problem}{S3}
Let $W = {\rm span} \left( \left\{ \begin{bmatrix} 1 \\ 1 \\ 2 \\ 1 \end{bmatrix}, \begin{bmatrix} 3 \\ 3 \\ 6 \\ 3 \end{bmatrix}, \begin{bmatrix} 3 \\ -1 \\ 3 \\ -2 \end{bmatrix}, \begin{bmatrix} 7 \\ -1 \\ 8 \\ -3 \end{bmatrix} \right\} \right)$.  Find a basis for $W$.
\end{problem}
\begin{solution}
$$\RREF\left(\begin{bmatrix} 1 & 3 & 3 & 7 \\ 1 & 3 & -1 & -1 \\ 2 & 6 & 3 & 8 \\ 1 & 3 & -2 & -3 \end{bmatrix}\right) = \begin{bmatrix} 1 & 3 & 0 & 1 \\ 0 & 0 & 1 & 2 \\ 0 & 0 & 0 & 0 \\  0 & 0 & 0 & 0 \end{bmatrix}$$

Then a basis is 
$ \left\{ \begin{bmatrix} 1 \\ 1 \\ 2 \\ 1 \end{bmatrix} , \begin{bmatrix} 3 \\ -1 \\ 3 \\ -2 \end{bmatrix}\right\} $.
\end{solution}


\begin{problem}{S4}
Let $W = {\rm span} \left( \left\{ \begin{bmatrix} 1 \\ 1 \\ 2 \\ 1 \end{bmatrix}, \begin{bmatrix} 3 \\ 3 \\ 6 \\ 3 \end{bmatrix}, \begin{bmatrix} 3 \\ -1 \\ 3 \\ -2 \end{bmatrix}, \begin{bmatrix} 7 \\ -1 \\ 8 \\ -3 \end{bmatrix} \right\} \right)$.  Find the dimension of $W$.
\end{problem}
\begin{solution}
$$\RREF\left(\begin{bmatrix} 1 & 3 & 3 & 7 \\ 1 & 3 & -1 & -1 \\ 2 & 6 & 3 & 8 \\ 1 & 3 & -2 & -3 \end{bmatrix}\right) = \begin{bmatrix} 1 & 3 & 0 & 1 \\ 0 & 0 & 1 & 2 \\ 0 & 0 & 0 & 0 \\  0 & 0 & 0 & 0 \end{bmatrix}$$

This has two pivot columns, so $W$ has dimension 2.
\end{solution}


\end{document}