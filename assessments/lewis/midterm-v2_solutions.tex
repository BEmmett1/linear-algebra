\documentclass{sbgLAexam}

\begin{extract*}
\usepackage{amsmath,amssymb,amsthm,enumerate}
\coursetitle{Math 237}
\courselabel{Linear Algebra}
\calculatorpolicy{You may use a calculator, but you must show all relevant work to receive credit for a standard.}


\newcommand{\IR}{\mathbb{R}}
\newcommand{\IC}{\mathbb{C}}
\renewcommand{\P}{\mathcal{P}}
\renewcommand{\Im}{{\rm Im\ }}
\DeclareMathOperator{\RREF}{RREF}
\DeclareMathOperator{\vspan}{span}

\makeatletter
\renewcommand*\env@matrix[1][*\c@MaxMatrixCols c]{%
  \hskip -\arraycolsep
  \let\@ifnextchar\new@ifnextchar
  \array{#1}}
\makeatother

\title{Midterm Exam}
\standard{E1,E2,E3,E4,V1,V2,V3,V4,S1,S2,S3,S4}
\version{2}
\end{extract*}

\begin{document}

\begin{problem}{E1}
Write a system of linear equations corresponding to the following
augmented matrix.
\[
\begin{bmatrix}[cccc|c]
3 & -1 & 0 & 1 & 5 \\
-1 & 9 & 1 & -7 & 0 \\
1 & 0 & -1 & 0 & -3
\end{bmatrix}
\]
\end{problem}
\begin{solution}
\begin{align*}
3x_1-x_2+x_4 &= 5 \\
-x_1+9x_2+x_3-7x_4 &= 0 \\
x_1-x_3 &= -3
\end{align*}
\end{solution}

\begin{problem}{E2}
Find \(\RREF A\), where
\[
  A =
  \begin{bmatrix}[cc|c]
    2 & -7 & 4 \\
    1 & -3 & 2 \\
    3 & 0 & 3
  \end{bmatrix}
\]
\end{problem}
\begin{solution}
\[
  \RREF A =
  \begin{bmatrix}[cc|c]
    1 & 0 & 0 \\
    0 & 1 & 0 \\
    0 & 0 & 1
  \end{bmatrix}
\]
\end{solution}

\begin{extract}\newpage\end{extract}
\begin{problem}{E3}
Find the solution set for the following system of linear equations.
\begin{align*}
2x_1-2x_2+6x_3-x_4 &=-1 \\
3x_1+6x_3+x_4 &= 5 \\
-4x_1+x_2-9x_3+2x_4&=-7
\end{align*}
\end{problem}
\begin{solution}
Let \(A =
  \begin{bmatrix}[cccc|c]
    2 & -2 & 6 & -1 & -1 \\
    3 & 0 & 6 & 1 & 5 \\
    -4 & 1 & -9 & 2 & -7
  \end{bmatrix}
\), so \(\RREF A =
  \begin{bmatrix}[cccc|c]
    1 & 0 & 2 & 0 & 2 \\
    0 & 1 & -1 & 0 & 3 \\
    0 & 0 & 0 & 1 & -1
  \end{bmatrix}
\). It follows that the solution set is given by \(
  \begin{bmatrix}
    2 - 2a \\
    3 + a \\
    a \\
    -1
  \end{bmatrix}
\) for all real numbers \(a\).
\end{solution}

\begin{problem}{E4}
Find a basis for the solution set to the homogeneous system of equations
\begin{align*}
2x_1+3x_2-5x_3+14x_4 &= 0\\
x_1+x_2-x_3+5x_4 &= 0
\end{align*}
\end{problem}
\begin{solution}
Let \(A =
  \begin{bmatrix}[cccc|c]
    2 & 3 & -5 & 14 & 0 \\
    1 & 1 & -1 & 5 & 0
  \end{bmatrix}
\), so \(\RREF A =
  \begin{bmatrix}[cccc|c]
    1 & 0 & 2 & 1 & 1 \\
    0 & 1 & -3 & 4 & 2 \\
  \end{bmatrix}
\).
It follows that the basis for the solution set is given by \(\left\{
  \begin{bmatrix}
    - 2 \\
    3 \\
    1 \\
    0
  \end{bmatrix},
  \begin{bmatrix}
    -1 \\
    - 4 \\
    0 \\
    1
  \end{bmatrix}
\right\}\).
\end{solution}

\begin{extract}\newpage\end{extract}
\begin{problem}{V1}
Let $V$ be the set of all points on the line $x+y=2$ with the operations, for any $(x_1,y_1), (x_2,y_2) \in V$, $c \in \IR$,
\begin{align*}
(x_1,y_1) \oplus (x_2,y_2) &= (x_1+x_2-1,y_1+y_2-1) \\
c \odot (x_1,y_1) &= (cx_1-(c-1), cy_1-(c-1))
\end{align*}
\begin{enumerate}[(a)]
\item Show that this vector space has an \textbf{additive identity} element
      \(\mathbf{0}\) satisfying \((x,y)\oplus\mathbf{0}=(x,y)\).
\item Determine if $V$ is a vector space or not.  Justify your answer.
\end{enumerate}
\end{problem}
\begin{solution}
Let $(x_1,y_1) \in V$; then $(x_1,y_1) \oplus (1,1) = (x_1,y_1)$, so $(1,1)$ is an additive identity element.

Now we will show the other seven properties.  Let $(x_1,y_1), (x_2,y_2) \in V$, and let $c,d \in \IR$.
\begin{enumerate}[1)]
\item Since real addition is associative, $\oplus$ is associative.
\item Since real addition is commutative, $\oplus$ is commutative.
\item The additive identity is $(1,1)$.
\item $(x_1,y_1) \oplus (2-x_1,2-y_1) = (1,1)$, so $(2-x_1,2-y_1)$ is the additive inverse of $(x_1,y_1)$.
\item \begin{align*} c\odot \left(d \odot (x_1,y_1) \right) &=c\odot \left( dx_1-(d-1),dy_1-(d-1)\right) \\
&= \left( c\left(dx_1-(d-1) \right)-(c-1), c\left(dy_1-(d-1) \right) \right) \\
&= \left(cdx_1-cd+c-(c-1), cdy_1-cd+c-(c-1) \right) \\
&= \left(cdx_1-(cd-1), cdy_1-(cd-1) \right) \\
&= (cd) \odot (x_1,y_1)
\end{align*}
\item $1 \odot (x_1,y_1) = (x_1-(1-1),y_1-(1-1)=(x_1,y_1)$
\item \begin{align*} c \odot \left( (x_1,y_1)\oplus(x_2,y_2) \right) &=
c\odot \left( x_1+y_1-1,x_2+y_2-1 \right) \\
&= \left( c(x_1+y_1-1)-(c-1), c(x_2+y_2-1)-(c-1) \right) \\
&= (cx_1+cx_2-2c+1, cy_1+cy_2-2c+1) \\
&= \left(cx_1-(c-1),cy_1-(c-1) \right) \oplus (cx_2-(c-1),cy_2-(c-1)) \\
&=c \odot (x_1,y_1) \oplus c\odot (x_2,y_2)
\end{align*}
\item \begin{align*} (c+d) \odot (x_1,y_1) &=
\left( (c+d)x_1-(c+d-1), (c+d)y_1-(c+d-1) \right) \\
&= \left( cx_1-(c-1), cy_1-(c-1) \right) \oplus (dx_1-(d-1), dy_1-(d-1) ) \\
&= c\odot (x_1,y_1) \oplus c \odot (x_2,y_2)
\end{align*}
\end{enumerate}
Therefore $V$ is a vector space.
\end{solution}

\begin{problem}{V2}
Determine if $\begin{bmatrix}0 \\ -1 \\ 2 \\ 6 \end{bmatrix}$ can be written as a linear combination of the vectors $\begin{bmatrix} 3 \\ -1 \\ -1 \\ 0 \end{bmatrix}$ and $\begin{bmatrix} -1 \\ 0 \\ 1 \\ 2 \end{bmatrix}$.
\end{problem}
\begin{solution}
$$\RREF\left(\left[\begin{array}{cc|c} 3 & -1 & 0 \\ -1 & 0 & -1 \\ -1 & 1 & 2 \\ 0 & 2 & 6\end{array} \right] \right)= \left[ \begin{array}{cc|c} 1 & 0 & 1 \\ 0 & 1 & 3 \\ 0 & 0 & 0 \\ 0 & 0 & 0 \end{array} \right]$$

Since this system has a solution, $\begin{bmatrix}0 \\ -1 \\ 2 \\ 6 \end{bmatrix}$ can be written as a linear combination of the vectors $\begin{bmatrix} 3 \\ -1 \\ -1 \\ 0 \end{bmatrix}$ and $\begin{bmatrix} -1 \\ 0 \\ 1 \\ 2 \end{bmatrix}$, namely
$$\begin{bmatrix}0 \\ -1 \\ 2 \\ 6 \end{bmatrix}=\begin{bmatrix} 3 \\ -1 \\ -1 \\ 0 \end{bmatrix}+3\begin{bmatrix} -1 \\ 0 \\ 1 \\ 2 \end{bmatrix}.$$
\end{solution}

\begin{extract}\newpage\end{extract}
\begin{problem}{V3}
Determine if the vectors $\begin{bmatrix} 1 \\ 1 \\ 2 \\1 \end{bmatrix}$, $\begin{bmatrix} 3 \\ 3 \\ 6 \\ 3 \end{bmatrix}$, $\begin{bmatrix}3 \\ -1 \\ 3 \\ -2\end{bmatrix}$, and $\begin{bmatrix} 7 \\ -1 \\ 8 \\ -3 \end{bmatrix}$  span $\IR^4$.
\end{problem}
\begin{solution}
$$\RREF\left(\begin{bmatrix} 1 & 3 & 3 & 7 \\ 1 & 3 & -1 & -1 \\ 2 & 6 & 3 & 8 \\ 1 & 3 & -2 & -3 \end{bmatrix}\right) = \begin{bmatrix} 1 & 3 & 0 & 1 \\ 0 & 0 & 1 & 2 \\ 0 & 0 & 0 & 0 \\ 0 & 0 & 0 & 0  \end{bmatrix}$$
Since there are zero rows, they do not span.  Alternatively, by inspection $\begin{bmatrix} 3 \\ 3 \\ 6 \\ 3 \end{bmatrix}=3\begin{bmatrix} 1 \\ 1 \\ 2 \\1 \end{bmatrix}$, so the set is linearly dependent, so it spans a subspace of dimension at most 3, therefore it does not span $\IR^4$.
\end{solution}

\begin{problem}{V4} Let \(W\) be the set of all \(\IR^3\) vectors
\(\begin{bmatrix} x \\ y \\ z \end{bmatrix}\) satisfying \(x+y+z=0\) (this forms a plane).
Determine if \(W\) is a subspace of \(\IR^3\).
\end{problem}
\begin{solution}
Yes, because \(z=-x-y\) and
\(
  a\begin{bmatrix} x_1 \\ y_1 \\ -x_1-y_1 \end{bmatrix}+
  b\begin{bmatrix} x_2 \\ y_2 \\ -x_2-y_2 \end{bmatrix}=
  \begin{bmatrix}
    ax_1+bx_2 \\
    ay_1+by_2 \\
    -(ax_1+bx_2)-(ay_1+by_2)
  \end{bmatrix}
\).
Alternately, yes because \(W\) is isomorphic to \(\IR^2\).
\end{solution}


\begin{extract}\newpage\end{extract}
\begin{problem}{S1}
Determine if the set of polynomials  $\left\{x^2+x, x^2+2x-1, x^2+3x-2\right\}$ is  linearly dependent or linearly independent
\end{problem}
\begin{solution}
$$\RREF\left( \begin{bmatrix} 1 &  1 & 1 \\ 0  & 2 & 3 \\ 1  & -1 & -2 \end{bmatrix} \right) = \begin{bmatrix} 1 &  0 & -\frac{1}{2} \\ 0  & 1 & \frac{3}{2} \\ 0& 0 & 0  \end{bmatrix}$$
Since there is a nonpivot column, the set is linearly dependent.
\end{solution}


\begin{problem}{S2}
Determine if the set $\left\{ 
\begin{bmatrix} 1 & -3 \\ 2 & 2 \end{bmatrix}, 
\begin{bmatrix} -1 & 4 \\ -1 & 1 \end{bmatrix}, 
\begin{bmatrix} -1 & 0 \\ 2 & 1 \end{bmatrix}, 
\begin{bmatrix} 0 & 3 \\ 3 & 9 \end{bmatrix} 
\right\}$ is a basis of $M_{2,2}$ or not.
\end{problem}

\begin{solution}
$$\RREF \begin{bmatrix} 1 & -1 & -1 & 0 \\ -3 & 4 & 0 & 3 \\ 2 & -1 & 2 & 3 \\ 2 & 1 & 1 & 9 \end{bmatrix}=\begin{bmatrix} 1 &0 & 0 & 3 \\ 0 & 1 & 0 & 3 \\ 0 & 0 & 1 & 0 \\ 0 & 0 & 0 & 0 \end{bmatrix}$$
Since this is not the identity matrix, the set is not a basis.
\end{solution}

\begin{extract}\newpage\end{extract}
\begin{problem}{S3}
Let $W = {\rm span} \left( \left\{ \begin{bmatrix} 1 \\ 1 \\ 2 \\ 1 \end{bmatrix}, \begin{bmatrix} 3 \\ 3 \\ 6 \\ 3 \end{bmatrix}, \begin{bmatrix} 3 \\ -1 \\ 3 \\ -2 \end{bmatrix}, \begin{bmatrix} 7 \\ -1 \\ 8 \\ -3 \end{bmatrix} \right\} \right)$.  Find a basis for $W$.
\end{problem}
\begin{solution}
$$\RREF\left(\begin{bmatrix} 1 & 3 & 3 & 7 \\ 1 & 3 & -1 & -1 \\ 2 & 6 & 3 & 8 \\ 1 & 3 & -2 & -3 \end{bmatrix}\right) = \begin{bmatrix} 1 & 3 & 0 & 1 \\ 0 & 0 & 1 & 2 \\ 0 & 0 & 0 & 0 \\  0 & 0 & 0 & 0 \end{bmatrix}$$

Then a basis is 
$ \left\{ \begin{bmatrix} 1 \\ 1 \\ 2 \\ 1 \end{bmatrix} , \begin{bmatrix} 3 \\ -1 \\ 3 \\ -2 \end{bmatrix}\right\} $.
\end{solution}


\begin{problem}{S4}
Let $W = {\rm span} \left( \left\{  \begin{bmatrix} -3 \\ -8 \\ 0 \end{bmatrix}, \begin{bmatrix} 1 \\ 2 \\ 2 \end{bmatrix}, \begin{bmatrix} 0 \\ -1 \\ 3 \end{bmatrix} \right\} \right)$.  Compute the dimension of $W$.
\end{problem}
\begin{solution}
Let $A= \begin{bmatrix}-3 & 1 & 0 \\ -8 & 2 & -1 \\ 0 & 2 & 3\end{bmatrix}$, and compute $\RREF(A) = \begin{bmatrix} 1 & 0 & \frac{1}{2} \\ 0 & 1 & \frac{3}{2} \\ 0 & 0 & 0 \end{bmatrix}$.
Since there are two pivot columns, ${\rm dim}\ W = 2$.  
\end{solution}


\end{document}