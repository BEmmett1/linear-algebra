\documentclass{sbgLAexam}

\begin{extract*}
\usepackage{amsmath,amssymb,amsthm,enumerate}
\coursetitle{Math 237}
\courselabel{Linear Algebra}
\calculatorpolicy{You may use a calculator, but you must show all relevant work to receive credit for a standard.}


\newcommand{\IR}{\mathbb{R}}
\newcommand{\IC}{\mathbb{C}}
\renewcommand{\P}{\mathcal{P}}
\renewcommand{\Im}{{\rm Im\ }}
\DeclareMathOperator{\RREF}{RREF}
\DeclareMathOperator{\vspan}{span}

\makeatletter
\renewcommand*\env@matrix[1][*\c@MaxMatrixCols c]{%
  \hskip -\arraycolsep
  \let\@ifnextchar\new@ifnextchar
  \array{#1}}
\makeatother

\title{Midterm Exam}
\standard{E1,E2,E3,E4,V1,V2,V3,V4,S1,S2,S3,S4}
\version{2}
\end{extract*}

\begin{document}

\begin{problem}{E1}
Write an augmented matrix corresponding to the following system of linear equations.
\begin{align*}
x_1+4x_3 &= 1 \\
x_2-x_3 &= 7 \\
x_1-x_2+3x_4 &= -1
\end{align*}
\end{problem}
\begin{solution}
\[
\begin{bmatrix}[cccc|c]
1 & 0 & 4 & 0 & 1 \\
0 & 1 & -1 & 0 & 7 \\
1 & -1 & 0 & 3 & -1
\end{bmatrix}
\]
\end{solution}

\begin{problem}{E2}
Find \(\RREF A\), where
\[
  A =
  \begin{bmatrix}[cccc|c]
    3 & -2 & 1 & 8 & -5 \\
    2 & 2 & 0 & 6 & -2 \\
    -1 & 1 & 1 & -4 & 6 \\
  \end{bmatrix}
\]
\end{problem}
\begin{solution}
\[
  \RREF A =
  \begin{bmatrix}[cccc|c]
    1 & 0 & 0 & 3 & -2 \\
    0 & 1 & 0 & 0 & 1 \\
    0 & 0 & 1 & -1 & 3
  \end{bmatrix}
\]
\end{solution}

\begin{extract}\newpage\end{extract}
\begin{problem}{E3}
Solve the following linear system.
\begin{align*}
3x+2y+z &= 7 \\
x+y+z &= 1 \\
-2x+3z &= -11
\end{align*}
\end{problem}
\begin{solution}
Let \(A =
  \begin{bmatrix}[ccc|c]
    3 & 2 & 1 & 7 \\
    1 & 1 & 1 & 1 \\
    -2 & 0 & 3 & 11
  \end{bmatrix}
\), so \(\RREF A =
  \begin{bmatrix}[ccc|c]
    1 & 0 & 0 & 4 \\
    0 & 1 & 0 & -2 \\
    0 & 0 & 1 & -1
  \end{bmatrix}
\). It follows that the system has exactly one solution:
\(\begin{bmatrix}
  4 & -2 & -1
\end{bmatrix}\)
\end{solution}
\begin{problem}{E4}
Find a basis for the solution set of the system of equations
\begin{align*}
x+2y+3z+w &= 0 \\
3x-y+z+w &= 0 \\
2x-3y-2z &= 0
\end{align*}
\end{problem}
\begin{solution}
$$\RREF \left(\begin{bmatrix} 1 & -2 & 3 & 1 \\ 3 & -1 & 1 & 1 \\ 2 & -3 & -2 & 0  \end{bmatrix} \right) = \begin{bmatrix} 1 & 0 & \frac{5}{7} & \frac{3}{7} \\ 0 & 1 & \frac{8}{7} & \frac{2}{7} \\ 0 & 0 & 0 & 0\end{bmatrix}$$
Then the solution set is
$$\left\{ \begin{bmatrix} -\frac{5}{7}a-\frac{3}{7}b \\ -\frac{8}{7}a-\frac{2}{7}b \\ a \\ b \end{bmatrix} \bigg |\ a,b \in \IR \right\}$$
So a basis for the solution set is $\left\{\begin{bmatrix} -\frac{5}{7} \\ -\frac{8}{7} \\ 1 \\ 0\end{bmatrix}, \begin{bmatrix} - \frac{3}{7} \\ -\frac{2}{7} \\ 0 \\ 1 \end{bmatrix} \right\}$, or $\left\{\begin{bmatrix} 5 \\ 8 \\ -7 \\ 0 \end{bmatrix}, \begin{bmatrix} 3 \\ 2 \\ 0 \\ -7 \end{bmatrix}\right\}$.
\end{solution}

\begin{extract}\newpage\end{extract}
\begin{problem}{V1}
Let $V$ be the set of all points on the line $x+y=2$ with the operations, for any $(x_1,y_1), (x_2,y_2) \in V$, $c \in \IR$,
\begin{align*}
(x_1,y_1) \oplus (x_2,y_2) &= (x_1+x_2-1,y_1+y_2-1) \\
c \odot (x_1,y_1) &= (cx_1-(c-1), cy_1-(c-1))
\end{align*}
\begin{enumerate}[(a)]
\item Show that this vector space has an \textbf{additive identity} element
      \(\mathbf{0}\) satisfying \((x,y)\oplus\mathbf{0}=(x,y)\).
\item Determine if $V$ is a vector space or not.  Justify your answer.
\end{enumerate}
\end{problem}
\begin{solution}
Let $(x_1,y_1) \in V$; then $(x_1,y_1) \oplus (1,1) = (x_1,y_1)$, so $(1,1)$ is an additive identity element.

Now we will show the other seven properties.  Let $(x_1,y_1), (x_2,y_2) \in V$, and let $c,d \in \IR$.
\begin{enumerate}[1)]
\item Since real addition is associative, $\oplus$ is associative.
\item Since real addition is commutative, $\oplus$ is commutative.
\item The additive identity is $(1,1)$.
\item $(x_1,y_1) \oplus (2-x_1,2-y_1) = (1,1)$, so $(2-x_1,2-y_1)$ is the additive inverse of $(x_1,y_1)$.
\item \begin{align*} c\odot \left(d \odot (x_1,y_1) \right) &=c\odot \left( dx_1-(d-1),dy_1-(d-1)\right) \\
&= \left( c\left(dx_1-(d-1) \right)-(c-1), c\left(dy_1-(d-1) \right) \right) \\
&= \left(cdx_1-cd+c-(c-1), cdy_1-cd+c-(c-1) \right) \\
&= \left(cdx_1-(cd-1), cdy_1-(cd-1) \right) \\
&= (cd) \odot (x_1,y_1)
\end{align*}
\item $1 \odot (x_1,y_1) = (x_1-(1-1),y_1-(1-1)=(x_1,y_1)$
\item \begin{align*} c \odot \left( (x_1,y_1)\oplus(x_2,y_2) \right) &=
c\odot \left( x_1+y_1-1,x_2+y_2-1 \right) \\
&= \left( c(x_1+y_1-1)-(c-1), c(x_2+y_2-1)-(c-1) \right) \\
&= (cx_1+cx_2-2c+1, cy_1+cy_2-2c+1) \\
&= \left(cx_1-(c-1),cy_1-(c-1) \right) \oplus (cx_2-(c-1),cy_2-(c-1)) \\
&=c \odot (x_1,y_1) \oplus c\odot (x_2,y_2)
\end{align*}
\item \begin{align*} (c+d) \odot (x_1,y_1) &=
\left( (c+d)x_1-(c+d-1), (c+d)y_1-(c+d-1) \right) \\
&= \left( cx_1-(c-1), cy_1-(c-1) \right) \oplus (dx_1-(d-1), dy_1-(d-1) ) \\
&= c\odot (x_1,y_1) \oplus c \odot (x_2,y_2)
\end{align*}
\end{enumerate}
Therefore $V$ is a vector space.
\end{solution}

\begin{problem}{V2}
Determine if  $\begin{bmatrix} 0 \\ 0 \\ 2 \end{bmatrix}$ can be written as a linear combination of the vectors $\begin{bmatrix} -1 \\ -9 \\ 15 \end{bmatrix}$ and $\begin{bmatrix} 1 \\ 5 \\ -5 \end{bmatrix}$.
\end{problem}
\begin{solution}
$$\RREF\left(\begin{bmatrix}[cc|c] -1 & 1 & 0 \\ -9 & 5 & 0 \\ 15 & -5 & 2 \end{bmatrix} \right) = \begin{bmatrix}[cc|c] 1 & 0 & 0 \\ 0 & 1 & 0 \\ 0 & 0 & 1 \end{bmatrix}$$
Since this system has no solution, $\begin{bmatrix} 0 \\ 0 \\ 2 \end{bmatrix}$ cannot be written as a linear combination of the vectors $\begin{bmatrix} -1 \\ -9 \\ 15 \end{bmatrix}$ and $\begin{bmatrix} 1 \\ 5 \\ -5 \end{bmatrix}$.

\end{solution}


\begin{extract}\newpage\end{extract}
\begin{problem}{V3}
Determine if the vectors $\begin{bmatrix} 1 \\ 1 \\ 2 \\1 \end{bmatrix}$, $\begin{bmatrix} 3 \\ 3 \\ 6 \\ 3 \end{bmatrix}$, $\begin{bmatrix}3 \\ -1 \\ 3 \\ -2\end{bmatrix}$, and $\begin{bmatrix} 7 \\ -1 \\ 8 \\ -3 \end{bmatrix}$  span $\IR^4$.
\end{problem}
\begin{solution}
$$\RREF\left(\begin{bmatrix} 1 & 3 & 3 & 7 \\ 1 & 3 & -1 & -1 \\ 2 & 6 & 3 & 8 \\ 1 & 3 & -2 & -3 \end{bmatrix}\right) = \begin{bmatrix} 1 & 3 & 0 & 1 \\ 0 & 0 & 1 & 2 \\ 0 & 0 & 0 & 0 \\ 0 & 0 & 0 & 0  \end{bmatrix}$$
Since there are zero rows, they do not span.  Alternatively, by inspection $\begin{bmatrix} 3 \\ 3 \\ 6 \\ 3 \end{bmatrix}=3\begin{bmatrix} 1 \\ 1 \\ 2 \\1 \end{bmatrix}$, so the set is linearly dependent, so it spans a subspace of dimension at most 3, therefore it does not span $\IR^4$.
\end{solution}

\begin{problem}{V4}
Determine if $\left\{ \begin{bmatrix} x \\ y \\ 0 \\ z \end{bmatrix}\  \bigg|\ x,y,z \in \IR\right\}$  a subspace of $\IR^4$.
\end{problem}
\begin{solution}
It is closed under addition and scalar multiplication, so it is a subspace.  Alternatively, it is the image of the linear transformation from $\IR^3 \rightarrow \IR^4$ given by $$\begin{bmatrix} x \\ y \\ z \end{bmatrix} \mapsto  \begin{bmatrix} x \\ y \\ 0 \\ z \end{bmatrix}.$$
\end{solution}


\begin{extract}\newpage\end{extract}
\begin{problem}{S1}
Determine if the set of polynomials $\left\{ x^3-8x, x^3+2x^2+2, -x^2+3\right\}$ is  linearly dependent or linearly independent
\end{problem}
\begin{solution}
$$\RREF\left( \begin{bmatrix}-3 & 1 & 0 \\ -8 & 2 & -1 \\ 0 & 2 & 3 \\ 0 & 0 & 0\end{bmatrix}\right) = \begin{bmatrix} 1 & 0 & \frac{1}{2} \\ 0 & 1 & \frac{3}{2} \\ 0 & 0 & 0 \\ 0 & 0 & 0 \end{bmatrix}$$ 
This has a non pivot column, therefore the set is linearly dependent.
\end{solution}


\begin{problem}{S2}
Determine if the set $\left\{ x^2+x-1, 3x^2-x+1, 2x-2 \right\}$ is a basis of $\P_2$
\end{problem}
\begin{solution}
$$\RREF\left(\begin{bmatrix} 1 & 3 & 2 \\ 1 & -1 & 0 \\ -1 & 1 & -2 \end{bmatrix} \right)= \begin{bmatrix} 1 & 0 &0 \\ 0 & 1 & 0 \\ 0 & 0 & 1\end{bmatrix}$$
Since the resulting matrix is the identity matrix, it is a basis.
\end{solution}


\begin{extract}\newpage\end{extract}
\begin{problem}{S3}  
Let $W={\rm span}\left(\left\{\begin{bmatrix} 2 \\ 0 \\ -2 \\ 0 \end{bmatrix}, \begin{bmatrix} 3 \\ 1 \\ 3 \\ 6 \end{bmatrix}, \begin{bmatrix} 0 \\ 0 \\ 1 \\ 1 \end{bmatrix}, \begin{bmatrix}1 \\ 2 \\ 0 \\ 1 \end{bmatrix}\right\}\right)$.  Find a basis of $W$.
\end{problem}
\begin{solution}
$$\RREF\left( \begin{bmatrix} 2 & 3 & 0 & 1 \\ 0 & 1 & 0 & 2 \\ -2 & 3 & 1 & 0 \\ 0 & 6 & 1 & 1\end{bmatrix} \right) = \begin{bmatrix}1 & 0 & 0 & -\frac{5}{2} \\ 0 & 1 & 0 & 2 \\ 0 & 0 & 1 & -11\\ 0 & 0 & 0 & 0  \end{bmatrix} $$
Then $\left\{\begin{bmatrix} 2 \\ 0 \\ -2 \\ 0 \end{bmatrix}, \begin{bmatrix} 3 \\ 1 \\ 3 \\ 6 \end{bmatrix}, \begin{bmatrix} 0 \\ 0 \\ 1 \\ 1 \end{bmatrix}\right\}$ is a basis of $W$.
\end{solution}


\begin{problem}{S4}
Let $W = {\rm span} \left( \left\{ \begin{bmatrix} 1 \\ 1 \\ 2 \\ 1 \end{bmatrix}, \begin{bmatrix} 3 \\ 3 \\ 6 \\ 3 \end{bmatrix}, \begin{bmatrix} 3 \\ -1 \\ 3 \\ -2 \end{bmatrix}, \begin{bmatrix} 7 \\ -1 \\ 8 \\ -3 \end{bmatrix} \right\} \right)$.  Find the dimension of $W$.
\end{problem}
\begin{solution}
$$\RREF\left(\begin{bmatrix} 1 & 3 & 3 & 7 \\ 1 & 3 & -1 & -1 \\ 2 & 6 & 3 & 8 \\ 1 & 3 & -2 & -3 \end{bmatrix}\right) = \begin{bmatrix} 1 & 3 & 0 & 1 \\ 0 & 0 & 1 & 2 \\ 0 & 0 & 0 & 0 \\  0 & 0 & 0 & 0 \end{bmatrix}$$

This has two pivot columns, so $W$ has dimension 2.
\end{solution}


\end{document}