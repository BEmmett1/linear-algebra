\documentclass{sbgLAsemi}

\begin{extract*}
\usepackage{amsmath,amssymb,amsthm,enumerate}
\coursetitle{Math 237}
\courselabel{Linear Algebra}
\calculatorpolicy{You may use a calculator, but you must show all relevant work to receive credit for a standard.}


\newcommand{\IR}{\mathbb{R}}
\newcommand{\IC}{\mathbb{C}}
\renewcommand{\P}{\mathcal{P}}
\renewcommand{\Im}{{\rm Im\ }}
\DeclareMathOperator{\RREF}{RREF}
\DeclareMathOperator{\vspan}{span}

\makeatletter
\renewcommand*\env@matrix[1][*\c@MaxMatrixCols c]{%
  \hskip -\arraycolsep
  \let\@ifnextchar\new@ifnextchar
  \array{#1}}
\makeatother

\title{Semifinal}
\version{2}
\end{extract*}

\begin{document}

\begin{problem}{E1}
Write an augmented matrix corresponding to the following system of linear equations.
\begin{align*}
x_1+4x_3 &= 1 \\
x_2-x_3 &= 7 \\
x_1-x_2+3x_3 &= -1
\end{align*}
\end{problem}
\begin{solution}
\[
\begin{bmatrix}[ccc|c]
1 & 0 & 4 & 1 \\
0 & 1 & -1 & 7 \\
1 & -1 & 3 & -1
\end{bmatrix}
\]
\end{solution}

\begin{problem}{E2}
Find \(\RREF A\), where
\[
  A =
  \begin{bmatrix}[cccc|c]
    2 & 2 & 1 & 2 & -1 \\
    1 & 1 & 2 & 4 & 5 \\
    3 & 3 & -1 & -2 & 1
  \end{bmatrix}
\]
\end{problem}
\begin{solution}
\[
  \RREF A =
  \begin{bmatrix}[cccc|c]
    1 & 1 & 0 & 0 & 0 \\
    0 & 0 & 1 & 2 & 0 \\
    0 & 0 & 0 & 0 & 1
  \end{bmatrix}
\]
\end{solution}
\begin{problem}{E3}
Solve the system of linear equations.
\begin{align*}
2x+y-z+w &=5 \\
3x-y-2w &= 0 \\
-x+5z+3w&=-1
\end{align*}
\end{problem}
\begin{solution}
$$\RREF\left( \begin{bmatrix}[cccc|c] 2 & 1 & -1 & 0 & 5 \\ 3 & -1 & 0 & -2 & 0 \\ -1 & 0 & 5 & 0 & -1 \end{bmatrix} \right) = \begin{bmatrix}[cccc|c] 1 & 0 & 0 & -\frac{1}{12} & 1 \\ 0 & 1 & 0 & \frac{7}{4} & 3 \\ 0 & 0 & 1 & \frac{7}{12} & 0 \end{bmatrix}$$
So the solutions are $$\left\{ \begin{bmatrix} 1+a \\ 3-21a \\ -7a \\ 12a \end{bmatrix}\ \big|\ a \in \IR\right\}$$
\end{solution}

\begin{problem}{E4}
Find a basis for the solution set to the system of equations
\begin{align*}
x+2y-3z &= 0 \\
2x+y-4z &= 0 \\
3y -2z & = 0 \\
x -y -z &= 0
\end{align*}
\end{problem}
\begin{solution}
$$\RREF \left(\begin{bmatrix} 1 & 2 & -3 \\ 2 & 1 & -4 \\ 0 & 3 & -2 \\ 1 & -1 & -1 \end{bmatrix} \right) = \begin{bmatrix} 1 & 0 & -\frac{5}{3} \\ 0 & 1 & -\frac{2}{3} \\ 0 & 0 & 0 \\ 0 & 0 & 0 \end{bmatrix}$$
Then the solution set is
$$\left\{ \begin{bmatrix} \frac{5}{3}a \\ \frac{2}{3}a \\ a \end{bmatrix} \bigg|\ a \in \IR \right\}$$
So a basis is $\left\{ \begin{bmatrix} \frac{5}{3} \\ \frac{2}{3} \\ 1 \end{bmatrix} \right\}$ or $\left\{ \begin{bmatrix} 5 \\  2 \\ 3 \end{bmatrix} \right\}$.
\end{solution}

\begin{problem}{V1}
Let $V$ be the set of all points on the parabola $y=x^2$ with the operations, for any $(x_1,y_1), (x_2,y_2) \in V$, $c \in \IR$,
\begin{align*}
(x_1,y_1) \oplus (x_2,y_2) &= (x_1+x_2,y_1+y_2+2x_1x_2) \\
c \odot (x_1,y_1) &= (cx_1, c^2y_1)
\end{align*}
\begin{enumerate}[(a)]
\item Show that the vector \textbf{addition} $\oplus$ is \textbf{associative}:
      \((x_1,y_1) \oplus ((x_2,y_2) \oplus (x_3,y_3))=((x_1,y_1)\oplus (x_2,y_2))\oplus (x_3,y_3)\).
\item Determine if $V$ is a vector space or not.  Justify your answer.
\end{enumerate}
\end{problem}

\begin{problem}{V2}
  Determine if
  \(\begin{bmatrix} 3 \\ -2 \\ 4 \end{bmatrix}\)
  belongs to the span of the set
  \(\left\{
    \begin{bmatrix} 1 \\ 2 \\ -3 \end{bmatrix},
    \begin{bmatrix} 2 \\ 4 \\ -6 \end{bmatrix},
    \begin{bmatrix} 0 \\ 0 \\ 0 \end{bmatrix}
    \right\}
  \).
\end{problem}
\begin{solution}
  Since
  \[
    \RREF\left(
      \begin{bmatrix}[ccc|c]
        1 & 2 & 0 & 3 \\
        2 & 4 & 0 & -2 \\
        -3 & -6 & 0 & 4
      \end{bmatrix}
    \right) =
    \begin{bmatrix}[ccc|c]
      1 & 2 & 0 & 0 \\
      0 & 0 & 0 & 1 \\
      0 & 0 & 0 & 0
    \end{bmatrix}
  \]
  contains the contradiction \(0=1\),
  \(\begin{bmatrix} 3 \\ -2 \\ 4 \end{bmatrix}\) is
  not a linear combination of the three vectors.
\end{solution}
\begin{problem}{V3}
Does
\(
  \operatorname{span}\left\{
    \begin{bmatrix} 2 \\ -1 \\ 4 \\ 2 \\ 1 \end{bmatrix},
    \begin{bmatrix} -1 \\ 3 \\ 5 \\ 2 \\ 0 \end{bmatrix},
    \begin{bmatrix} 1 \\ 0 \\ 5 \\ 1 \\ -3 \end{bmatrix}
  \right\} = \IR^5
\)?
\end{problem}
\begin{solution}
Since there are only three vectors, they cannot span \(\IR^5\).
\end{solution}
\begin{problem}{V4} Let \(W\) be the set of all \(\IR^3\) vectors
\(\begin{bmatrix} x \\ y \\ z \end{bmatrix}\)
satisfying \(x+y+z=1\) (this forms a plane).
Determine if \(W\) is a subspace of \(\IR^3\).
\end{problem}
\begin{solution}
No, because \(\mathbf{0}\) does not belong to \(W\).
\end{solution}


\begin{problem}{S1}
Determine if the set of vectors  $\left\{\begin{bmatrix} 1 \\ 0 \\ 1 \end{bmatrix}, \begin{bmatrix} 1 \\ 2 \\ -1 \end{bmatrix}, \begin{bmatrix} 1 \\ 3 \\ -2 \end{bmatrix}\right\}$ is  linearly dependent or linearly independent
\end{problem}
\begin{solution}
$$\RREF\left( \begin{bmatrix} 1 &  1 & 1 \\ 0  & 2 & 3 \\ 1  & -1 & -2 \end{bmatrix} \right) = \begin{bmatrix} 1 &  0 & -\frac{1}{2} \\ 0  & 1 & \frac{3}{2} \\ 0& 0 & 0  \end{bmatrix}$$
Since there is a nonpivot column, the set is linearly dependent.
\end{solution}

\begin{problem}{S2}
Determine if the set $\left\{ x^3-3x^2+2x+2, -x^3+4x^2-x+1, -x^3+2x+1, 3x^2+3x+9 \right\}$ is a basis of $\P^3$ or not.
\end{problem}

\begin{solution}
$$\RREF \begin{bmatrix} 1 & -1 & -1 & 0 \\ -3 & 4 & 0 & 3 \\ 2 & -1 & 2 & 3 \\ 2 & 1 & 1 & 9 \end{bmatrix}=\begin{bmatrix} 1 &0 & 0 & 3 \\ 0 & 1 & 0 & 3 \\ 0 & 0 & 1 & 0 \\ 0 & 0 & 0 & 0 \end{bmatrix}$$
Since this is not the identity matrix, the set is not a basis.
\end{solution}
\begin{problem}{S3}
Let $W = {\rm span} \left( \left\{  \begin{bmatrix} -3 \\ -8 \\ 0 \end{bmatrix}, \begin{bmatrix} 1 \\ 2 \\ 2 \end{bmatrix}, \begin{bmatrix} 0 \\ -1 \\ 3 \end{bmatrix} \right\} \right)$.   Find a basis for $W$.
\end{problem}
\begin{solution}
Let $A= \begin{bmatrix}-3 & 1 & 0 \\ -8 & 2 & -1 \\ 0 & 2 & 3\end{bmatrix}$, and compute $\RREF(A) = \begin{bmatrix} 1 & 0 & \frac{1}{2} \\ 0 & 1 & \frac{3}{2} \\ 0 & 0 & 0 \end{bmatrix}$.
Since the first two columns are pivot columns, $\left\{ \begin{bmatrix} -3 \\ -8 \\ 0 \end{bmatrix}, \begin{bmatrix} 1 \\ 2 \\ 2 \end{bmatrix} \right\} $ is a basis for $W$.
\end{solution}


\begin{problem}{S4}
Let $W$ be the subspace of $\P_3$ given by $W={\rm span}\left( \left\{ x^3-x^2+3x-3, 2x^3+x+1, 3x^3-x^2+4x-2, x^3+x^2+x-7\right\}\right)$.  Compute the dimension of $W$.
\end{problem}
\begin{solution}
$$ \RREF \left( \begin{bmatrix} 1 & 2 & 3 & 1 \\ -1 & 0 & -1 & 1 \\ 3 & 1 & 4 & 1 \\ -3 & 1 & -2 & -7 \end{bmatrix} \right) =  \begin{bmatrix} 1 & 0 & 1 & 0 \\ 0 & 1 & 1 & 0 \\ 0 & 0 & 0 & 1 \\ 0 & 0 & 0 & 0\end{bmatrix}$$
This has 3 pivot columns so $\dim(W)=3$.
\end{solution}


\begin{problem}{A1}
Let $T: \IR^3\rightarrow \IR^4$ be the linear transformation given by $$T\left(\begin{bmatrix} x \\ y \\ z \\  \end{bmatrix} \right) = \begin{bmatrix} -3x+y \\ -8x+2y-z \\ 2y+3z \\ 7x \end{bmatrix}.$$  Write the matrix for $T$ with respect to the standard bases of $\IR^3$ and $\IR^4$.
\end{problem}
\begin{solution}
$$\begin{bmatrix} 3 & 1 & 0 \\ -8 & 2 & -1 \\ 0 & 2 & 3 \\ 7 & 0 & 0 \end{bmatrix}$$
\end{solution}

\begin{problem}{A2}
Determine if the map $T: \P  \rightarrow \P$ given by $T(f) = f^{\prime}-f^{\prime \prime}$ is a linear transformation or not.
\end{problem}


\begin{problem}{A3}
Determine if the following linear maps are injective (one-to-one) and/or surjective (onto).
\begin{enumerate}[(a)]
\item $S: \IR^2 \rightarrow \IR^3$ given by $S\left(\begin{bmatrix} x \\ y  \end{bmatrix} \right) = \begin{bmatrix} 3x+2y \\ x-y \\ x+4y \end{bmatrix} $
\item $T: \IR^3 \rightarrow \IR^3$ given by $T\left(\begin{bmatrix} x \\ y \\ z  \end{bmatrix} \right) = \begin{bmatrix} x+y+z \\ 2y+3z \\ x-y-2z \end{bmatrix} $
\end{enumerate}
\end{problem}

\begin{solution}
\begin{enumerate}[(a)]
\item $$\RREF\left( \begin{bmatrix} 1 &  1 & 1 \\ 0  & 2 & 3 \\ 1  & -1 & -2 \end{bmatrix} \right) = \begin{bmatrix} 1 &  0 & -\frac{1}{2} \\ 0  & 1 & \frac{3}{2} \\ 0& 0 & 0  \end{bmatrix}$$
Since there is a nonpivot column, $T$ is not injective.  Since there is a zero row, $T$ is not surjective.
\item $$\RREF \left( \begin{bmatrix} 3 & 2 \\ 1 & -1 \\ 1 & 4 \end{bmatrix} \right) = \begin{bmatrix} 1 & 0 \\ 0 & 1 \\ 0 & 0 \end{bmatrix}$$
Since all columns are pivot columns, $S$ is injective.  Since there is a zero row, $S$ is not surjective.
\end{enumerate}
\end{solution}



\begin{problem}{A4}
Let $T: \IR^{2\times 2} \rightarrow \IR^3$ be the linear map given by $T\left(\begin{bmatrix} x & y \\ z & w \end{bmatrix} \right) = \begin{bmatrix}  8x-3y-z+4w \\ y+3z-4w \\ -7x+3y+2z-5w\end{bmatrix} $.
Compute a basis for the kernel and a basis for the image of $T$.
\end{problem}
\begin{solution}
$$\RREF \left( \begin{bmatrix} 8 & -3 & -1 & 4 \\ 0 & 1 & 3 & -4 \\ -7 & 3 & 2 & -5 \end{bmatrix} \right) = \begin{bmatrix} 1 & 0 & 1 & -1 \\ 0 & 1 & 3 & -4 \\ 0 & 0 & 0 & 0 \end{bmatrix}$$

Thus \(\left\{ \begin{bmatrix} 8 \\ 0 \\ -7 \end{bmatrix}, \begin{bmatrix} -3 \\ 1 \\ 3 \end{bmatrix} \right\}\) is a basis for the image, and \( \left\{ \begin{bmatrix} -1 & -3 \\ 1 & 0 \end{bmatrix}, \begin{bmatrix} 1 & 4 \\ 0 & 1 \end{bmatrix} \right\} \) is a basis for the kernel.
\end{solution}


\begin{problem}{M1}
Let 
\begin{align*}
A &= \begin{bmatrix} 3 \\ 5 \\ -1  \end{bmatrix} & B&=\begin{bmatrix} 1 & -1 & 3 & -3 \\ 2 & 1 & -1 & 2 \end{bmatrix} & C &= \begin{bmatrix} 2 & -1 \\ 0 & 4 \\ 3 & 1 \end{bmatrix} \end{align*}
Exactly one of the six products $AB$, $AC$, $BA$, $BC$, $CA$, $CB$ can be computed.  Determine which one, and compute it.
\end{problem}
\begin{solution}
$CB$ is the only one that can be computed, and
$$CB=\begin{bmatrix} 0 & -3 & 7 & -8 \\ 8 & 4 & -4 & 8 \\ 5 & -2 & 8 & -7 \end{bmatrix}$$
\end{solution}

\begin{problem}{M2}
Determine if the matrix $\begin{bmatrix} 3 & -1 & 0 & 4 \\ 2 & 1 & 1 & 1 \\ 0 & 1 & 1 & -1 \\ 1 & -2 & 0 & 3 \end{bmatrix}$ is invertible.
\end{problem}
\begin{solution}
$$\RREF \begin{bmatrix} 3 & -1 & 0 & 4 \\ 2 & 1 & 1 & 1 \\ 0 & 1 & 1 & -1 \\ 1 & -2 & 0 & 3 \end{bmatrix} = \begin{bmatrix} 1 & 0 & 0 & 1 \\ 0 & 1 & 0 & -1 \\ 0 & 0 & 1 & 0 \\ 0 & 0 & 0 & 0 \end{bmatrix}$$
This matrix is not row equivalent to the identity matrix, so it is not invertible.
\end{solution}


\begin{problem}{M3}
  Find the inverse of the matrix
  \(\begin{bmatrix}
    4 & -1 & -8  \\
    2 & 1 & 3  \\
    1 & 1 & 4
  \end{bmatrix}\).
\end{problem}
\begin{solution}
\(\begin{bmatrix}[ccc|ccc]
  4 & -1 & -8 & 1 & 0 & 0  \\
  2 & 1 & 3   & 0 & 1 & 0 \\
  1 & 1 & 4   & 0 & 0 & 1
\end{bmatrix}\sim\begin{bmatrix}[ccc|ccc]
  1 & 0 & 0 & 1 & -4 & 5 \\
  0 & 1 & 0 & -5 & 24 & -28 \\
  0 & 0 & 1 & 1 & -5 & 6
\end{bmatrix}\). Thus the inverse is
\(\begin{bmatrix}
  1 & -4 & 5  \\
  -5 & 24 & -28  \\
  1 & -5 & 6
\end{bmatrix}\).
\end{solution}


\begin{problem}{G1}
Compute the determinant of the matrix
\[
  \begin{bmatrix}
    0 & -4 & 1 & 1 \\
    0 & 1 & 0 & 1 \\
    -2 & 3 & -1 & 1 \\
    5 & 0 & -4 & 0 \\
  \end{bmatrix}
.\]
\end{problem}
\begin{solution}
\(55\).
\end{solution}

\begin{problem}{G2} 
Compute the eigenvalues, along with their algebraic multiplicities, of the matrix $ \begin{bmatrix} 9 & -3 & 2 \\ 23 & -8 & 5 \\  2 & -1 & 1 \end{bmatrix}$.
\end{problem}
\begin{solution}
The eigenvalues are $-1$, $1$, and $2$, each with multiplicity 1.
\end{solution}

\begin{problem}{G3}
Compute the eigenspace of the eigenvalue $-1$ in the matrix $\begin{bmatrix} 4 & -2 & -1 \\ 15 & -7 & -3 \\ -5 & 2 & 0 \end{bmatrix}$. 
\end{problem}
\begin{solution}
$$\RREF\left(A+I\right) = \begin{bmatrix} 1 & - \frac{2}{5} & -\frac{1}{5} \\ 0 & 0 & 0 \\ 0 & 0 & 0 \end{bmatrix}$$
So the eigenspace is spanned by $\begin{bmatrix} 2 \\5 \\  0 \end{bmatrix}$ and $\begin{bmatrix} 1 \\ 0 \\ 5 \end{bmatrix}$.
\end{solution}


\begin{problem}{G4}
Compute the geometric multiplicity of the eigenvalue $2$ in the matrix $\begin{bmatrix} -1 & 1 & 0 \\ -9 & 5 & 0 \\ 15 & -5 & 2 \end{bmatrix}$.
\end{problem}

\begin{solution}
The eigenspace is the solution space of the system $(B-2I)X=0$.
$$\RREF(B-2I)=\RREF\left(\begin{bmatrix} -3 & 1 & 0 \\ -9 & 3 & 0 \\ 15 & - 5 & 0 \end{bmatrix} \right) = \begin{bmatrix} 1 & -\frac{1}{3} & 0 \\ 0 & 0 & 0 \\ 0 & 0 & 0 \end{bmatrix}$$
Thus the geometric multiplicity is 2.
\end{solution}
\end{document}