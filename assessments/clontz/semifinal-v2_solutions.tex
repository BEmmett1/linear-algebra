\documentclass{sbgLAsemi}

\begin{extract*}
\usepackage{amsmath,amssymb,amsthm,enumerate}
\coursetitle{Math 237}
\courselabel{Linear Algebra}
\calculatorpolicy{You may use a calculator, but you must show all relevant work to receive credit for a standard.}


\newcommand{\IR}{\mathbb{R}}
\newcommand{\IC}{\mathbb{C}}
\renewcommand{\P}{\mathcal{P}}
\renewcommand{\Im}{{\rm Im\ }}
\DeclareMathOperator{\RREF}{RREF}
\DeclareMathOperator{\vspan}{span}

\makeatletter
\renewcommand*\env@matrix[1][*\c@MaxMatrixCols c]{%
  \hskip -\arraycolsep
  \let\@ifnextchar\new@ifnextchar
  \array{#1}}
\makeatother

\title{Semifinal}
\version{2}
\end{extract*}

\begin{document}

\begin{problem}{E1}
Write an augmented matrix corresponding to the following system of linear equations.
\begin{align*}
x_1+4x_3 &= 1 \\
x_2-x_3 &= 7 \\
x_1-x_2+3x_3 &= -1
\end{align*}
\end{problem}
\begin{solution}
\[
\begin{bmatrix}[ccc|c]
1 & 0 & 4 & 1 \\
0 & 1 & -1 & 7 \\
1 & -1 & 3 & -1
\end{bmatrix}
\]
\end{solution}

\begin{problem}{E2}
Find \(\RREF A\), where
\[
  A =
  \begin{bmatrix}[cccc|c]
    2 & 2 & 1 & 2 & -1 \\
    1 & 1 & 2 & 4 & 5 \\
    3 & 3 & -1 & -2 & 1
  \end{bmatrix}
\]
\end{problem}
\begin{solution}
\[
  \RREF A =
  \begin{bmatrix}[cccc|c]
    1 & 1 & 0 & 0 & 0 \\
    0 & 0 & 1 & 2 & 0 \\
    0 & 0 & 0 & 0 & 1
  \end{bmatrix}
\]
\end{solution}
\begin{problem}{E3}
Find the solution set for the following system of linear equations.
\begin{align*}
2x_1+3x_2-5x_3+14x_4 &= 8 \\
x_1+x_2-x_3+5x_4&= 3
\end{align*}
\end{problem}
\begin{solution}
Let \(A =
  \begin{bmatrix}[cccc|c]
    2 & 3 & -5 & 14 & 8 \\
    1 & 1 & -1 & 5 & 3
  \end{bmatrix}
\), so \(\RREF A =
  \begin{bmatrix}[cccc|c]
    1 & 0 & 2 & 1 & 1 \\
    0 & 1 & -3 & 4 & 2 \\
  \end{bmatrix}
\). It follows that the solution set is given by \(
  \begin{bmatrix}
    1 - 2a - b \\
    2 + 3a - 4b \\
    a \\
    b
  \end{bmatrix}
\) for all real numbers \(a,b\).
\end{solution}

\begin{problem}{E4}
Find a basis for the solution set of the system of equations
\begin{align*}
x+3y+3z+7w &= 0 \\
 x+3y-z-w &= 0 \\
  2x+6y+3z+8w &= 0 \\
   x+3y-2z-3w &= 0
\end{align*}
\end{problem}
\begin{solution}
$$\RREF\left(\begin{bmatrix} 1 & 3 & 3 & 7 \\ 1 & 3 & -1 & -1 \\ 2 & 6 & 3 & 8 \\ 1 & 3 & -2 & -3 \end{bmatrix}\right) = \begin{bmatrix} 1 & 3 & 0 & 1 \\ 0 & 0 & 1 & 2 \\ 0 & 0 & 0 & 0 \\ 0 & 0 & 0 & 0 \end{bmatrix}$$
Then the solution set is
$$\left\{ \begin{bmatrix} -3a-b \\ a \\ -2b \\ b \end{bmatrix}\ \big|\ a,b \in \IR\right\}$$
So a basis for the solution set is $$ \left\{ \begin{bmatrix} 3 \\ -1 \\ 0 \\ 0 \end{bmatrix} , \begin{bmatrix} 1\\0 \\ 2 \\ -1 \end{bmatrix} \right\} $$
\end{solution}


\begin{problem}{V1}
Let $V$ be the set of all pairs of real numbers with the operations, for any $(x_1,y_1), (x_2,y_2) \in V$, $c \in \IR$,
\begin{align*}
(x_1,y_1) \oplus (x_2,y_2) &= (x_1+x_2,y_1+y_2) \\
c \odot (x_1,y_1) &= (c^2x_1, c^3y_1)
\end{align*}
\begin{enumerate}[(a)]
\item Show that scalar multiplication \textbf{distributes scalars} over
      vector addition:
      \(c\odot((x_1,y_1) \oplus (x_2,y_2))=
      c\odot(x_1,y_1) \oplus c\odot(x_2,y_2)\).
\item Determine if $V$ is a vector space or not.  Justify your answer.
\end{enumerate}
\end{problem}
\begin{solution}
Let $(x_1,y_1), (x_2,y_2) \in V$ and let $c \in \IR$.
\begin{align*}
c \odot \left( (x_1,y_1) \oplus (x_2,y_2) \right) &= c \odot (x_1+x_2,y_1+y_2) \\
&= (c^2(x_1+x_2), c^3(y_1+y_2) ) \\
&= (c^2x_1,c^3y_1) \oplus (c^2x_2,c^3y_2) \\
&= c \odot (x_1,y_1) \oplus c \odot (x_2,y_2)
\end{align*}
However, $V$ is not a vector space, as the other distributive law fails:
$$(c+d) \odot (x_1,y_1) = ( (c+d)^2 x_1, (c+d)^3 y_1) \neq ((c^2+d^2)x_1, (c^3+d^3)y_1) = c \odot (x_1,y_1) \oplus d \odot (x_1,y_1).$$
\end{solution}


\begin{problem}{V2}
Determine if $\begin{bmatrix} 0 \\ 1 \\ -2 \\ 1 \end{bmatrix}$ can be written as a linear combination of the vectors $\begin{bmatrix} 5 \\ 2 \\ -3 \\ 2 \end{bmatrix}$, $\begin{bmatrix} 3 \\ 1 \\ 1 \\ 0 \end{bmatrix}$, and $\begin{bmatrix} 8 \\ 3 \\ 5 \\ -1 \end{bmatrix}$.
\end{problem}
\begin{solution}

$$\RREF \left(\begin{bmatrix}[ccc|c] 8 & 5 & 3 & 0\\ 3 & 2 & 1 & 1 \\ 5 & -3 & 1 & -2  \\ -1 & 2 & 0 & 1 \end{bmatrix} \right) = \begin{bmatrix}[ccc|c] 1 & 0 & 0 & 0  \\ 0 &  1 & 0 & 0  \\ 0 & 0 & 1 & 0 \\ 0 & 0 & 0 & 1  \end{bmatrix}$$
The system has no solution, so $\begin{bmatrix} 0 \\ 1 \\ -2 \\ 1 \end{bmatrix}$ is not a linear combination of the three other vectors.
\end{solution}


\begin{problem}{V3}
Does
\(
  \operatorname{span}\left\{
    \begin{bmatrix} 2 \\ -1 \\ 4 \end{bmatrix},
    \begin{bmatrix} 3 \\ 12 \\ -9 \end{bmatrix},
    \begin{bmatrix} 1 \\ 4 \\ -3 \end{bmatrix},
    \begin{bmatrix} -4 \\ 2 \\ -8 \end{bmatrix}
  \right\} = \IR^3
\)?
\end{problem}
\begin{solution}
Since
\[
  \RREF\begin{bmatrix}
    2 & 3 & 1 & -4 \\
    -1 & 12 & 4 & 2 \\
    4 & -9 & -3 & -8
  \end{bmatrix} =
  \begin{bmatrix}
    1 & 0 & 0 & -2 \\
    0 & 1 & 1/3 & 0 \\
    0 & 0 & 0 & 0
  \end{bmatrix}
\]
has a zero row, the vectors fail to span \(\IR^3\).
\end{solution}

\begin{problem}{V4}
Determine if $\left\{ \begin{bmatrix} x \\ y \\ 0 \\ z \end{bmatrix}\  \bigg|\ x,y,z \in \IR\right\}$  a subspace of $\IR^4$.
\end{problem}
\begin{solution}
It is closed under addition and scalar multiplication, so it is a subspace.  Alternatively, it is the image of the linear transformation from $\IR^3 \rightarrow \IR^4$ given by $$\begin{bmatrix} x \\ y \\ z \end{bmatrix} \mapsto  \begin{bmatrix} x \\ y \\ 0 \\ z \end{bmatrix}.$$
\end{solution}


\begin{problem}{S1}
Determine if the set of vectors $\left\{ \begin{bmatrix} -3 \\ 8 \\ 0 \end{bmatrix}, \begin{bmatrix} 1 \\ 2 \\ 2 \end{bmatrix}, \begin{bmatrix} 0 \\ -1 \\ 3 \end{bmatrix} \right\}$ is  linearly dependent or linearly independent
\end{problem}
\begin{solution}
$$\RREF\left( \begin{bmatrix}-3 & 1 & 0 \\ 8 & 2 & -1 \\ 0 & 2 & 3 \end{bmatrix}\right) = \begin{bmatrix} 1 & 0 & 0 \\ 0 & 1 & 0 \\ 0 & 0 & 1 \end{bmatrix}$$
Every column is a pivot column, therefore the set is linearly independent.
\end{solution}

\begin{problem}{S2}
  Determine if the set \(\left\{
    \begin{bmatrix} 3 \\ -1 \\ 2 \\3 \end{bmatrix},
    \begin{bmatrix} 2 \\ 0 \\ 2 \\ 4\end{bmatrix},
    \begin{bmatrix} 1 \\ -1 \\ 0 \\ -1\end{bmatrix},
    \begin{bmatrix} -1 \\ 3 \\ 0 \\ 5\end{bmatrix}
  \right\}\) is a basis of $\IR^4$.
\end{problem}
\begin{solution}
  \[\RREF\left(
    \begin{bmatrix}
      3 & 2 & 1 & -1\\
      -1 & 0 & -1 & 3\\
      2 & 2 & 0 & 0\\
      3 & 4 & -1 & 5\\
    \end{bmatrix} \right)= \begin{bmatrix}
      1 & 0 & 1 & 0 \\
      0 & 1 & -1 & 0 \\
      0 & 0 & 0 & 1 \\
      0 & 0 & 0 & 0
    \end{bmatrix}
  \]
Since the resulting matrix is not the identity matrix, it is not a basis.
\end{solution}


\begin{problem}{S3}
Let $W$ be the subspace of $\P^2$ given by $W = {\rm span} \left( \left\{  -3x^2-8x, x^2+2x+2, -x+3\right\} \right)$.   Find a basis for $W$.
\end{problem}
\begin{solution}
Let $A= \begin{bmatrix}-3 & 1 & 0 \\ -8 & 2 & -1 \\ 0 & 2 & 3\end{bmatrix}$, and compute $\RREF(A) = \begin{bmatrix} 1 & 0 & \frac{1}{2} \\ 0 & 1 & \frac{3}{2} \\ 0 & 0 & 0 \end{bmatrix}$.
Since the first two columns are pivot columns, $\left\{ -3x^2-8x, x^2+2x+2\right\} $ is a basis for $W$.
\end{solution}


\begin{problem}{S4}
  Let \(
    W={\rm span}\left\{ 2x^2-x+3, 2x^2+2, -x^2+4x+1 \right\}\).
  Find the dimension of \(W\).
\end{problem}
\begin{solution}
  \[\RREF\left(
    \begin{bmatrix}
      2 & 2 & -1 \\
      -1 & 0 & 4 \\
      3 & 2 & 1
    \end{bmatrix} \right)= \begin{bmatrix}
      1 & 0 &0 \\
      0 & 1 & 0 \\
      0 & 0 & 1
    \end{bmatrix}
  \]
  Since it has three pivot columns, its dimension is \(3\).
\end{solution}
\begin{problem}{A1}
Let $T: \IR^4 \rightarrow \IR^2$ be the linear transformation given by $$T\left(\begin{bmatrix} x_1 \\ x_2 \\ x_3 \\ x_4 \end{bmatrix} \right) = \begin{bmatrix} x_1+3x_3 \\ 3x_2-x_3 \end{bmatrix}.$$ Write the matrix for $T$ with respect to the standard bases of $\IR^4$ and $\IR^2$.
\end{problem}
\begin{solution}
$$\begin{bmatrix} 1 & 0 & 3 & 0 \\ 0 & 3 & -1 & 0 \end{bmatrix}$$
\end{solution}


\begin{problem}{A2}
Determine if $D: \IR^{2\times 2} \rightarrow \IR$ given by $D\left(\begin{bmatrix} a & b \\ c & d \end{bmatrix} \right) = a-3c$ is a linear transformation or not.
\end{problem}

\begin{problem}{A3}
Determine if the following linear maps are injective (one-to-one) and/or surjective (onto).
\begin{enumerate}[(a)]
\item $S: \IR^2 \rightarrow \IR^3$ given by $S\left(\begin{bmatrix} x \\ y  \end{bmatrix} \right) = \begin{bmatrix} 3x+2y \\ x-y \\ x+4y \end{bmatrix} $
\item $T: \IR^3 \rightarrow \IR^3$ given by $T\left(\begin{bmatrix} x \\ y \\ z  \end{bmatrix} \right) = \begin{bmatrix} x+y+z \\ 2y+3z \\ x-y-2z \end{bmatrix} $
\end{enumerate}
\end{problem}

\begin{solution}
\begin{enumerate}[(a)]
\item $$\RREF\left( \begin{bmatrix} 1 &  1 & 1 \\ 0  & 2 & 3 \\ 1  & -1 & -2 \end{bmatrix} \right) = \begin{bmatrix} 1 &  0 & -\frac{1}{2} \\ 0  & 1 & \frac{3}{2} \\ 0& 0 & 0  \end{bmatrix}$$
Since there is a nonpivot column, $T$ is not injective.  Since there is a zero row, $T$ is not surjective.
\item $$\RREF \left( \begin{bmatrix} 3 & 2 \\ 1 & -1 \\ 1 & 4 \end{bmatrix} \right) = \begin{bmatrix} 1 & 0 \\ 0 & 1 \\ 0 & 0 \end{bmatrix}$$
Since all columns are pivot columns, $S$ is injective.  Since there is a zero row, $S$ is not surjective.
\end{enumerate}
\end{solution}



\begin{problem}{A4}
Let $T: \IR^{2\times 2} \rightarrow \IR^3$ be the linear map given by $T\left(\begin{bmatrix} x & y \\ z & w \end{bmatrix} \right) = \begin{bmatrix}  8x-3y-z+4w \\ y+3z-4w \\ -7x+3y+2z-5w\end{bmatrix} $.
Compute a basis for the kernel and a basis for the image of $T$.
\end{problem}
\begin{solution}
$$\RREF \left( \begin{bmatrix} 8 & -3 & -1 & 4 \\ 0 & 1 & 3 & -4 \\ -7 & 3 & 2 & -5 \end{bmatrix} \right) = \begin{bmatrix} 1 & 0 & 1 & -1 \\ 0 & 1 & 3 & -4 \\ 0 & 0 & 0 & 0 \end{bmatrix}$$

Thus \(\left\{ \begin{bmatrix} 8 \\ 0 \\ -7 \end{bmatrix}, \begin{bmatrix} -3 \\ 1 \\ 3 \end{bmatrix} \right\}\) is a basis for the image, and \( \left\{ \begin{bmatrix} -1 & -3 \\ 1 & 0 \end{bmatrix}, \begin{bmatrix} 1 & 4 \\ 0 & 1 \end{bmatrix} \right\} \) is a basis for the kernel.
\end{solution}


\begin{problem}{M1}
Let 
\begin{align*}
A &= \begin{bmatrix} 2 & 3 \\ 0 & 1 \end{bmatrix} & B&= \begin{bmatrix} 3 & 1 & 0 \end{bmatrix} & C&= \begin{bmatrix} 0 & -1 & 4 \\ 1 & -1 & 2 \end{bmatrix}
\end{align*}

Exactly one of the six products $AB$, $AC$, $BA$, $BC$, $CA$, $CB$ can be computed.  Determine which one, and compute it.
\end{problem}
\begin{solution}
$AC$ is the only one that can be computed, and 
$$AC = \begin{bmatrix} 3 & -5 & 11 \\ 1 & -1 & 2 \end{bmatrix}$$
\end{solution}

\begin{problem}{M2}
Determine if the matrix $\begin{bmatrix} 3 & -1 & 0  \\ 2 & 1 & 1  \\ 0 & 1 & 1   \end{bmatrix}$ is invertible.
\end{problem}
\begin{solution}
It is row equivalent to the identity matrix, so it is invertible.
\end{solution}


\begin{problem}{M3}
Find the inverse of the matrix $\begin{bmatrix} 8 & 5 & 3 & 0 \\ 3 & 2 & 1 & 1 \\ 5 & -3 & 1 & -2 \\ -1 & 2 & 0 & 1\end{bmatrix} $.
\end{problem}
\begin{solution}
$$\RREF \left(\begin{bmatrix}[cccc|cccc] 8 & 5 & 3 & 0 & 1 & 0 & 0 & 0\\ 3 & 2 & 1 & 1 & 0 & 1 & 0 & 0 \\ 5 & -3 & 1 & -2 & 0 & 0 & 1 & 0 \\ -1 & 2 & 0 & 1 & 0 & 0 & 0 & 1 \end{bmatrix} \right) = \begin{bmatrix}[cccc|cccc] 1 & 0 & 0 & 0 & 1 & 2 & -5 & 12 \\ 0 &  1 & 0 & 0 & 1 & 1 & -4 & -9 \\ 0 & 0 & 1 & 0 & -4 & -7 & 20 & 47 \\ 0 & 0 & 0 & 1 & -1 & 0 & 3 & 7 \end{bmatrix}$$

So the inverse is $\begin{bmatrix} 1 & 2 & -5 & 12 \\  1 & 1 & -4 & -9 \\  -4 & -7 & 20 & 47 \\-1 & 0 & 3 & 7 \end{bmatrix}$.
\end{solution}


\begin{problem}{G1}
Compute the determinant of the matrix
\[
  \begin{bmatrix}
    0 & -4 & 1 & 1 \\
    -2 & 3 & -1 & 1 \\
    0 & 1 & 0 & 1 \\
    5 & 0 & -4 & 0 \\
  \end{bmatrix}
.\]
\end{problem}
\begin{solution}
\(-55\).
\end{solution}
\begin{problem}{G2}
Compute the eigenvalues, along with their algebraic multiplicities, of the matrix $ \begin{bmatrix} 8 & -3 & -1 \\ 21 & -8 & -3 \\ -7 & 3 & 2\end{bmatrix}$.
\end{problem}
\begin{solution}
\begin{align*}
\det(A-\lambda I) &= (8-\lambda) \det \begin{bmatrix} -8-\lambda & -3 \\ 3 & 2-\lambda \end{bmatrix}-(-3) \det \begin{bmatrix} 21 & -3 \\ -7 & 2-\lambda \end{bmatrix} +(-1) \det \begin{bmatrix} 21 & -8-\lambda \\ -7 & 3 \end{bmatrix} \\
&=(8-\lambda)\left(\lambda ^2+6\lambda-7\right)+3\left(-21\lambda + 21 \right)-\left(-7\lambda +7 \right) \\
&=(\lambda -1) \left( (8-\lambda)(\lambda+7)-63+7 \right) \\
&= (\lambda-1)(\lambda -\lambda ^2) \\
&= -\lambda (\lambda-1)^2
\end{align*}
So the eigenvalues are $0$ (with algebraic multiplicity 1) and $1$ (with algebraic multiplicity 2).
\end{solution}


\begin{problem}{G3}
Find the eigenspace associated to the eigenvalue $3$ in the matrix $A=\begin{bmatrix}1 & -2 & -1 & 0 \\ -4 & -1 & -2 & 0 \\ 14 & 12 & 11 & 2 \\ -14 & -10 & -9 & -1\end{bmatrix}$.
\end{problem}
\begin{solution}
The eigenspace is spanned by $\begin{bmatrix} -1 \\ \frac{1}{2} \\ 1 \\ 0 \end{bmatrix}$ and  $\begin{bmatrix} -1 \\ 1 \\ 0 \\ 1 \end{bmatrix}$.
\end{solution}


\begin{problem}{G4}
Compute the geometric multiplicity of the eigenvalue $1$ in the matrix $A=\begin{bmatrix}9 & -3 & 2 \\ 19 & -6 & 5 \\ -11 & 4 & -2 \end{bmatrix}$
\end{problem}
\begin{solution}
The eigenspace is spanned by $\begin{bmatrix} -1 \\ -2 \\ 1 \end{bmatrix}$, so the geometric multiplicity is $1$.
\end{solution}

\end{document}