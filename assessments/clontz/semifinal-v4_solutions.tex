\documentclass{sbgLAsemi}

\begin{extract*}
\usepackage{amsmath,amssymb,amsthm,enumerate}
\coursetitle{Math 237}
\courselabel{Linear Algebra}
\calculatorpolicy{You may use a calculator, but you must show all relevant work to receive credit for a standard.}


\newcommand{\IR}{\mathbb{R}}
\newcommand{\IC}{\mathbb{C}}
\renewcommand{\P}{\mathcal{P}}
\renewcommand{\Im}{{\rm Im\ }}
\DeclareMathOperator{\RREF}{RREF}
\DeclareMathOperator{\vspan}{span}

\makeatletter
\renewcommand*\env@matrix[1][*\c@MaxMatrixCols c]{%
  \hskip -\arraycolsep
  \let\@ifnextchar\new@ifnextchar
  \array{#1}}
\makeatother

\title{Semifinal}
\version{4}
\end{extract*}

\begin{document}

\begin{problem}{E1}
Write an augmented matrix corresponding to the following system of linear equations.
\begin{align*}
x_1+4x_3 &= 1 \\
x_2-x_3 &= 7 \\
x_1-x_2+3x_4 &= -1
\end{align*}
\end{problem}
\begin{solution}
\[
\begin{bmatrix}[cccc|c]
1 & 0 & 4 & 0 & 1 \\
0 & 1 & -1 & 0 & 7 \\
1 & -1 & 0 & 3 & -1
\end{bmatrix}
\]
\end{solution}

\begin{problem}{E2}
Find the reduced row echelon form of the matrix below.
$$\begin{bmatrix}[cccc|c] 2 & 1 & -1 & 0 & 5 \\ 3 & -1 & 0 & -2 & 0 \\ -1 & 0 & 5 & 0 & -1 \end{bmatrix}$$
\end{problem}
\begin{solution}
$$\begin{bmatrix}[cccc|c]
 2 & 1 & -1 & 0 & 5 \\
 3 & -1 & 0 & -2 & 0 \\
 -1 & 0 & 5 & 0 & -1
\end{bmatrix} \sim
\begin{bmatrix}[cccc|c]
 -1 & 0 & 5 & 0 & -1  \\
 2 & 1 & -1 & 0 & 5 \\
 3 & -1 & 0 & -2 & 0
\end{bmatrix} \sim
\begin{bmatrix}[cccc|c]
 1 & 0 & -5 & 0 & 1 \\
 2 & 1 & -1 & 0 & 5 \\
 3 & -1 & 0 & -2 & 0
\end{bmatrix} $$
$$\sim
\begin{bmatrix}[cccc|c]
 1 & 0 & -5 & 0 & 1  \\
 0 & 1 & 9 & 0 & 3 \\
 0 & -1 & 15 & -2 & -3 \\
\end{bmatrix} \sim
\begin{bmatrix}[cccc|c]
 1 & 0 & -5 & 0 & 1  \\
 0 & 1 & 9 & 0 & 3 \\
 0 & 0 & 24 & -2 & 0 \\
\end{bmatrix} \sim
\begin{bmatrix}[cccc|c]
 1 & 0 & -5 & 0 & 1 \\
 0 & 1 & 9 & 0 & 3 \\
 0 & 0 & 1 & -\frac{1}{12} & 0 \\
\end{bmatrix} \sim
\begin{bmatrix}[cccc|c]
1 & 0 & 0 & -\frac{5}{12} & 1 \\
 0 & 1 & 0 & \frac{3}{4} & 3 \\
 0 & 0 & 1 & -\frac{1}{12} & 0
\end{bmatrix}$$
\end{solution}

\begin{problem}{E3}
Find the solution set for the following system of linear equations.
\begin{align*}
2x_1+3x_2-5x_3+14x_4 &= 8 \\
x_1+x_2-x_3+5x_4&= 3
\end{align*}
\end{problem}
\begin{solution}
Let \(A =
  \begin{bmatrix}[cccc|c]
    2 & 3 & -5 & 14 & 8 \\
    1 & 1 & -1 & 5 & 3
  \end{bmatrix}
\), so \(\RREF A =
  \begin{bmatrix}[cccc|c]
    1 & 0 & 2 & 1 & 1 \\
    0 & 1 & -3 & 4 & 2 \\
  \end{bmatrix}
\). It follows that the solution set is given by \(
  \begin{bmatrix}
    1 - 2a - b \\
    2 + 3a - 4b \\
    a \\
    b
  \end{bmatrix}
\) for all real numbers \(a,b\).
\end{solution}

\begin{problem}{E4}
Find a basis for the solution set of the system of equations
\begin{align*}
x+2y+3z+w &= 0 \\
3x-y+z+w &= 0 \\
2x-3y-2z &= 0 \\
-x+2z+5w &=0
\end{align*}
\end{problem}
\begin{solution}
$$\RREF \left(\begin{bmatrix} 1 & -2 & 3 & 1 \\ 3 & -1 & 1 & 1 \\ 2 & -3 & -2 & 0 \\ -1 & 0 & 2 & 5 \end{bmatrix} \right) = \begin{bmatrix} 1 & 0 & 0 & -1 \\ 0 & 1 & 0 & -2 \\ 0 & 0 & 1 & 2 \\ 0 & 0 & 0 & 0\end{bmatrix}$$
Then the solution set is
$$\left\{ \begin{bmatrix} a \\ 2a \\ -2a \\ a \end{bmatrix} \bigg |\ a \in \IR \right\}$$
So a basis for the solution set is $\left\{\begin{bmatrix} 1 \\ 2 \\ -2 \\ 1\end{bmatrix} \right\}$.
\end{solution}


\begin{problem}{V1}
Let $V$ be the set of all pairs of real numbers with the operations, for any $(x_1,y_1), (x_2,y_2) \in V$, $c \in \IR$,
\begin{align*}
(x_1,y_1) \oplus (x_2,y_2) &= (x_1+x_2,y_1+y_2) \\
c \odot (x_1,y_1) &= (0, cy_1)
\end{align*}
\begin{enumerate}[(a)]
\item Show that scalar multiplication
      \textbf{distributes vectors} over scalar addition:
      \((c+d)\odot(x,y)=
      c\odot(x,y) \oplus d\odot(x,y)\).
\item Determine if $V$ is a vector space or not.  Justify your answer.
\end{enumerate}
\end{problem}
\begin{solution}
Let $(x_1,y_1) \in V$, and let $c,d \in \IR$.  Then
$$(c+d)\odot (x_1,y_1)=(0, (c+d)y_1) = (0,cy_1) \oplus (0,dy_1) = c \odot (x_1,y_1) \oplus d \odot (x_1,y_1).$$
However, $V$ is not a vector space, as $1 \odot (x_1,y_1) = (0,y_1) \neq (x_1,y_1)$.
\end{solution}


\begin{problem}{V2}
  Determine if
  \(\begin{bmatrix} 3 \\ -2 \\ 4 \end{bmatrix}\)
  belongs to the span of the set
  \(\left\{
    \begin{bmatrix} 1 \\ 2 \\ -3 \end{bmatrix},
    \begin{bmatrix} 2 \\ 4 \\ -6 \end{bmatrix},
    \begin{bmatrix} 0 \\ 0 \\ 0 \end{bmatrix}
    \right\}
  \).
\end{problem}
\begin{solution}
  Since
  \[
    \RREF\left(
      \begin{bmatrix}[ccc|c]
        1 & 2 & 0 & 3 \\
        2 & 4 & 0 & -2 \\
        -3 & -6 & 0 & 4
      \end{bmatrix}
    \right) =
    \begin{bmatrix}[ccc|c]
      1 & 2 & 0 & 0 \\
      0 & 0 & 0 & 1 \\
      0 & 0 & 0 & 0
    \end{bmatrix}
  \]
  contains the contradiction \(0=1\),
  \(\begin{bmatrix} 3 \\ -2 \\ 4 \end{bmatrix}\) is
  not a linear combination of the three vectors.
\end{solution}
\begin{problem}{V3}
Determine if the vectors  $\begin{bmatrix} 8 \\ 21 \\ -7 \end{bmatrix}$, $\begin{bmatrix} -3 \\ -8 \\ 3 \end{bmatrix}$, $\begin{bmatrix} -1 \\ -3 \\ 2 \end{bmatrix}$, and $\begin{bmatrix} 4 \\ 11 \\ -5 \end{bmatrix}$ span $\IR^3$.
\end{problem}
\begin{solution}
$$\RREF\left(\begin{bmatrix} 8 & -3 & -1 & 4 \\ 21 & -8 & -3 & 11 \\ -7 & 3 & 2 & -5  \end{bmatrix} \right) = \begin{bmatrix} 1 & 0 & 1 & -1 \\ 0 & 1 & 3 & -4 \\ 0 & 0 & 0 & 0\end{bmatrix} $$
Since the rank is less than 3, they do not span $\IR^3$.
\end{solution}

\begin{problem}{V4} Let $W$ be the set of all polynomials of even degree.  Determine if $W$ is a subspace of the vector space of all polynomials.
\end{problem}
\begin{solution}
$W$ is closed under scalar multiplication, but not under addition.  For example, $x-x^2$ and $x^2$ are both in $W$, but $(x-x^2)+(x^2)=x \notin W$.
\end{solution}


\begin{problem}{S1}
Determine if the set of vectors  $\left\{\begin{bmatrix} 1 \\ 0 \\ 1 \end{bmatrix}, \begin{bmatrix} 1 \\ 2 \\ -1 \end{bmatrix}, \begin{bmatrix} 1 \\ 3 \\ -2 \end{bmatrix}\right\}$ is  linearly dependent or linearly independent
\end{problem}
\begin{solution}
$$\RREF\left( \begin{bmatrix} 1 &  1 & 1 \\ 0  & 2 & 3 \\ 1  & -1 & -2 \end{bmatrix} \right) = \begin{bmatrix} 1 &  0 & -\frac{1}{2} \\ 0  & 1 & \frac{3}{2} \\ 0& 0 & 0  \end{bmatrix}$$
Since there is a nonpivot column, the set is linearly dependent.
\end{solution}

\begin{problem}{S2}
Determine if the set $\left\{ x^3-3x^2+2x+2, -x^3+4x^2-x+1, -x^3+2x+1, 3x^2+3x+9 \right\}$ is a basis of $\P^3$ or not.
\end{problem}

\begin{solution}
$$\RREF \begin{bmatrix} 1 & -1 & -1 & 0 \\ -3 & 4 & 0 & 3 \\ 2 & -1 & 2 & 3 \\ 2 & 1 & 1 & 9 \end{bmatrix}=\begin{bmatrix} 1 &0 & 0 & 3 \\ 0 & 1 & 0 & 3 \\ 0 & 0 & 1 & 0 \\ 0 & 0 & 0 & 0 \end{bmatrix}$$
Since this is not the identity matrix, the set is not a basis.
\end{solution}
\begin{problem}{S3}
Let $W={\rm span}\left( \left\{ \begin{bmatrix} 1 \\ -1 \\ 3 \\ -3 \end{bmatrix},\begin{bmatrix} 2 \\ 0 \\ 1 \\ 1 \end{bmatrix}, \begin{bmatrix} 3 \\ -1 \\ 4 \\ -2 \end{bmatrix},  \begin{bmatrix} 1 \\ 1 \\ 1 \\ -7 \end{bmatrix} \right\}\right)$.  Find a basis of $W$.
\end{problem}
\begin{solution}
$$ \RREF \left( \begin{bmatrix} 1 & 2 & 3 & 1 \\ -1 & 0 & -1 & 1 \\ 3 & 1 & 4 & 1 \\ -3 & 1 & -2 & -7 \end{bmatrix} \right) =  \begin{bmatrix} 1 & 0 & 1 & 0 \\ 0 & 1 & 1 & 0 \\ 0 & 0 & 0 & 1 \\ 0 & 0 & 0 & 0\end{bmatrix}$$
Then  $\left\{ \begin{bmatrix} 1 \\ -1 \\ 3 \\ -3 \end{bmatrix},\begin{bmatrix} 2 \\ 0 \\ 1 \\ 1 \end{bmatrix},   \begin{bmatrix} 1 \\ 1 \\ 1 \\ -7 \end{bmatrix} \right\}$ is a basis for $W$.
\end{solution}


\begin{problem}{S4}
Let $W = {\rm span} \left( \left\{ \begin{bmatrix} 1 \\ 1 \\ 2 \\ 1 \end{bmatrix}, \begin{bmatrix} 3 \\ 3 \\ 6 \\ 3 \end{bmatrix}, \begin{bmatrix} 3 \\ -1 \\ 3 \\ -2 \end{bmatrix}, \begin{bmatrix} 7 \\ -1 \\ 8 \\ -3 \end{bmatrix} \right\} \right)$.  Find the dimension of $W$.
\end{problem}
\begin{solution}
$$\RREF\left(\begin{bmatrix} 1 & 3 & 3 & 7 \\ 1 & 3 & -1 & -1 \\ 2 & 6 & 3 & 8 \\ 1 & 3 & -2 & -3 \end{bmatrix}\right) = \begin{bmatrix} 1 & 3 & 0 & 1 \\ 0 & 0 & 1 & 2 \\ 0 & 0 & 0 & 0 \\  0 & 0 & 0 & 0 \end{bmatrix}$$

This has two pivot columns, so $W$ has dimension 2.
\end{solution}


\begin{problem}{A1}
Let $T: \IR^3\rightarrow \IR^4$ be the linear transformation given by $$T\left(\begin{bmatrix} x \\ y \\ z \\  \end{bmatrix} \right) = \begin{bmatrix} -3x+y \\ -8x+2y-z \\ 2y+3z \\ 7x \end{bmatrix}.$$  Write the matrix for $T$ with respect to the standard bases of $\IR^3$ and $\IR^4$.
\end{problem}
\begin{solution}
$$\begin{bmatrix} 3 & 1 & 0 \\ -8 & 2 & -1 \\ 0 & 2 & 3 \\ 7 & 0 & 0 \end{bmatrix}$$
\end{solution}

\begin{problem}{A2} 
 Determine if $T: \IR^2 \rightarrow \IR^2$ given by $T\left(\begin{bmatrix} x \\ y \end{bmatrix}\right) = \begin{bmatrix} e^{x} \\ e^y \end{bmatrix}$ is a linear transformation.
\end{problem}
\begin{solution}
It is not linear.  For example, $$\begin{bmatrix}e^2 \\ 1 \end{bmatrix} = T\left(\begin{bmatrix} 2 \\ 0 \end{bmatrix} \right) \neq 2 T\left(\begin{bmatrix} 1 \\ 0 \end{bmatrix}\right)=\begin{bmatrix} 2e \\ 1 \end{bmatrix}$$
\end{solution}

\begin{problem}{A3}
Determine if each of the following linear transformations is injective (one-to-one) and/or surjective (onto).
\begin{enumerate}[(a)]
\item $S: \IR^2 \rightarrow \IR^4$ given by the standard matrix $\begin{bmatrix} 2 & 1 \\ 1 & 2 \\ 0 & 1 \\ 3 & -3 \end{bmatrix}$.
\item $T: \IR^4 \rightarrow \IR^3$ given by the standard matrix $\begin{bmatrix} 2 & 3 & -1 & 1 \\ -1 & 1 & 1 & 1 \\ 4 & 11 & -1 & 5 \end{bmatrix}$
\end{enumerate}
\end{problem}
\begin{solution}
\begin{enumerate}[(a)]
\item $ \begin{bmatrix} 2 & 1 \\ 1 & 2 \\ 0 & 1 \\ 3 & -3 \end{bmatrix}=\begin{bmatrix}1 & 0 \\ 0 & 1 \\ 0 & 0 \\ 0 & 0  \end{bmatrix}$.  Since each column is a pivot column, $S$ is injective.  Since there a no zero row, $S$ is not surjective.
\item Since $\dim \IR^4 > \dim \IR^3$, $T$ is not injective.
$$\RREF\left(\begin{bmatrix} 2 & 3 & -1 & 1 \\ -1 & 1 & 1 & 1 \\ 4 & 11 & -1 & 5 \end{bmatrix}\right) = \begin{bmatrix} 1 & 0  & -\frac{4}{5} & -\frac{2}{5} \\ 0 & 1 & \frac{1}{5} & \frac{3}{5} \\ 0 & 0 & 0 & 0 \end{bmatrix}$$
Since there is a zero row, $T$ is not surjective.
\end{enumerate}
\end{solution}

\begin{problem}{A4}
Let $T: \IR^4\rightarrow \IR^4$ be the linear transformation given by $$T\left(\begin{bmatrix} x \\ y \\ z \\ w \end{bmatrix} \right) = \begin{bmatrix} x+3y+3z+7w \\ x+3y-z-w \\ 2x+6y+3z+8w \\ x+3y-2z-3w \end{bmatrix}$$
Compute a basis for the kernel and a basis for the image of $T$.
\end{problem}
\begin{solution}

$$\RREF\left(\begin{bmatrix} 1 & 3 & 3 & 7 \\ 1 & 3 & -1 & -1 \\ 2 & 6 & 3 & 8 \\ 1 & 3 & -2 & -3 \end{bmatrix}\right) = \begin{bmatrix} 1 & 3 & 0 & 1 \\ 0 & 0 & 1 & 2 \\ 0 & 0 & 0 & 0 \end{bmatrix}$$

Then a basis for the kernel is
$$\left\{ \begin{bmatrix} -3 \\ 1 \\ 0 \\ 0 \end{bmatrix} , \begin{bmatrix} -1\\0 \\ -2 \\ 1 \end{bmatrix} \right\}$$
and a basis for the image is
$$\left\{ \begin{bmatrix} 1 \\ 1 \\ 2 \\ 1 \end{bmatrix} , \begin{bmatrix} 3 \\ -1 \\ 3 \\ -2 \end{bmatrix}\right\} $$
\end{solution}

\begin{problem}{M1}
Let 
\begin{align*}
A &= \begin{bmatrix} 3 \\ 5 \\ -1  \end{bmatrix} & B&=\begin{bmatrix} 1 & -1 & 3 & -3 \\ 2 & 1 & -1 & 2 \end{bmatrix} & C &= \begin{bmatrix} 2 & -1 \\ 0 & 4 \\ 3 & 1 \end{bmatrix} \end{align*}
Exactly one of the six products $AB$, $AC$, $BA$, $BC$, $CA$, $CB$ can be computed.  Determine which one, and compute it.
\end{problem}
\begin{solution}
$CB$ is the only one that can be computed, and
$$CB=\begin{bmatrix} 0 & -3 & 7 & -8 \\ 8 & 4 & -4 & 8 \\ 5 & -2 & 8 & -7 \end{bmatrix}$$
\end{solution}

\begin{problem}{M2}
Determine if the matrix $\begin{bmatrix} 3 & -1 & 0 & 4 \\ 2 & 1 & 1 & -1 \\ 0 & 1 & 1 & 3 \\ 1 & -2 & 0 & 0 \end{bmatrix}$ is invertible.
\end{problem}
\begin{solution}
This matrix is row equivalent to the identity matrix, so it is invertible.
\end{solution}

\begin{problem}{M3}
  Find the inverse of the matrix
  \(\begin{bmatrix}
    2 & -1 & -3  \\
    -14 & 9 & 24  \\
    3 & -2 & -5
  \end{bmatrix}\).
\end{problem}
\begin{solution}
  \(\begin{bmatrix}[ccc|ccc]
    2 & -1 & -3 & 1 & 0 & 0 \\
    -14 & 9 & 24 & 0 & 1 & 0 \\
    3 & -2 & -5 & 0 & 0 & 1
  \end{bmatrix}\sim\begin{bmatrix}[ccc|ccc]
    1 & 0 & 0 & 3 & 1 & 3  \\
    0 & 1 & 0 & 2 & -1 & -6  \\
    0 & 0 & 1 & 1 & 1 & 4
  \end{bmatrix}\). Thus the inverse is
  \(\begin{bmatrix}
    3 & 1 & 3  \\
    2 & -1 & -6  \\
    1 & 1 & 4
  \end{bmatrix}\).
\end{solution}
\begin{problem}{G1}
Compute the determinant of the matrix $\begin{bmatrix} 8 & 5 & 3 & 0 \\ 3 & 2 & 1 & 1 \\ 5 & -3 & 1 & -2 \\ -1 & 2 & 0 & 1\end{bmatrix} $.
\end{problem}
\begin{solution}
$-1$.
\end{solution}

\begin{problem}{G2} 
Compute the eigenvalues, along with their algebraic multiplicities, of the matrix $ \begin{bmatrix} 8 & -3 & 2 \\ 15 & -5 & 5 \\ -3 & 2 & 1 \end{bmatrix}$.
\end{problem}
\begin{solution}
The eigenvalues are $0$ with multiplicity 1 and $2$, with algebraic multiplicity 2.
\end{solution}

\begin{problem}{G3}
Compute the eigenspace associated to the eigenvalue $2$ in the matrix $\begin{bmatrix} -1 & 1 & 0 \\ -9 & 5 & 0 \\ 15 & -5 & 2 \end{bmatrix}$.
\end{problem}

\begin{solution}
The eigenspace is the solution space of the system $(B-2I)X=0$.
$$\RREF(B-2I)=\RREF\left(\begin{bmatrix} -3 & 1 & 0 \\ -9 & 3 & 0 \\ 15 & - 5 & 0 \end{bmatrix} \right) = \begin{bmatrix} 1 & -\frac{1}{3} & 0 \\ 0 & 0 & 0 \\ 0 & 0 & 0 \end{bmatrix}$$
So the system simplifies to $x-\frac{y}{3}=0$, or $3x=y$.  Thus the eigenspace is $$E_2 = {\rm span}\left( \left\{ \begin{bmatrix} 1 \\ 3 \\ 0 \end{bmatrix}, \begin{bmatrix} 0 \\ 0 \\ 1\end{bmatrix} \right\} \right)$$
\end{solution}
\begin{problem}{G4}
Compute the geometric multiplicity of the eigenvalue $-1$ in the matrix $\begin{bmatrix} 4 & -2 & -1 \\ 15 & -7 & -3 \\ -5 & 2 & 0 \end{bmatrix}$.  \end{problem}
\begin{solution}
$$\RREF\left(A+I\right) = \begin{bmatrix} 1 & - \frac{2}{5} & -\frac{1}{5} \\ 0 & 0 & 0 \\ 0 & 0 & 0 \end{bmatrix}$$
So the geometric multiplicity is $2$.
\end{solution}


\end{document}