\documentclass{sbgLAsemi}

\begin{extract*}
\usepackage{amsmath,amssymb,amsthm,enumerate}
\coursetitle{Math 237}
\courselabel{Linear Algebra}
\calculatorpolicy{You may use a calculator, but you must show all relevant work to receive credit for a standard.}


\newcommand{\IR}{\mathbb{R}}
\newcommand{\IC}{\mathbb{C}}
\renewcommand{\P}{\mathcal{P}}
\renewcommand{\Im}{{\rm Im\ }}
\DeclareMathOperator{\RREF}{RREF}
\DeclareMathOperator{\vspan}{span}

\makeatletter
\renewcommand*\env@matrix[1][*\c@MaxMatrixCols c]{%
  \hskip -\arraycolsep
  \let\@ifnextchar\new@ifnextchar
  \array{#1}}
\makeatother

\title{Semifinal}
\version{4}
\end{extract*}

\begin{document}

\begin{problem}{E1}
Write an augmented matrix corresponding to the following system of linear equations.
\begin{align*}
x_1+4x_3 &= 1 \\
x_2-x_3 &= 7 \\
x_1-x_2+3x_3 &= -1
\end{align*}
\end{problem}
\begin{solution}
\[
\begin{bmatrix}[ccc|c]
1 & 0 & 4 & 1 \\
0 & 1 & -1 & 7 \\
1 & -1 & 3 & -1
\end{bmatrix}
\]
\end{solution}

\begin{problem}{E2}
Find \(\RREF A\), where
\[
  A =
  \begin{bmatrix}[ccc|c]
    2 & -1 & 5 & 4 \\
    -1 & 0 & -2 & -1 \\
    1 & 3 & -1 & -5
  \end{bmatrix}
\]
\end{problem}
\begin{solution}
\[
  \RREF A =
  \begin{bmatrix}[ccc|c]
    1 & 0 & 2 & 1 \\
    0 & 1 & -1 & -2 \\
    0 & 0 & 0 & 0
  \end{bmatrix}
\]
\end{solution}

\begin{problem}{E3}
Solve the system of equations
\begin{align*}
-3x +y &= 2\\
-8x+2y-z &= 6 \\
2y+3z &= -2
\end{align*}


\end{problem}

\begin{solution}
$$\RREF\left(\begin{bmatrix}[ccc|c] -3 & 1 & 0 & 2 \\ -8 & 2 & -1 & 6 \\ 0 & 2 & 3 & -2 \end{bmatrix} \right) = \begin{bmatrix}[ccc|c] 1 & 0 & \frac{1}{2} & -1 \\ 0 & 1 & \frac{3}{2} & -1 \\ 0 & 0 & 0 & 0 \end{bmatrix}$$
The solutions are $$\left\{ \begin{bmatrix} -1-\frac{c}{2} \\ -1-\frac{3c}{2} \\ c \end{bmatrix}\ \bigg|\ c\in \IR\right\} = \left\{ \begin{bmatrix} c-1 \\ 3c-1 \\ -2c  \end{bmatrix}\ \bigg|\ c\in \IR\right\}$$
\end{solution}

\begin{problem}{E4}
Find a basis for the solution set to the homogeneous system of equations
\begin{align*}
4x_1+4x_2+3x_3-6x_4 &= 0 \\
-2x_3-4x_4 &= 0 \\
2x_1+2x_2+x_3-4x_4 &= 0 \\
\end{align*}
\end{problem}
\begin{solution}
Let \(A =
  \begin{bmatrix}[cccc|c]
    4 & 4 & 3 & -6 & 0 \\
    0 & 0 & -2 & -4 & 0 \\
    2 & 2 & 1 & -4 & 0
  \end{bmatrix}
\), so \(\RREF A =
  \begin{bmatrix}[cccc|c]
    1 & 1 & 0 & -3 & 0 \\
    0 & 0 & 1 & 2 & 0 \\
    0 & 0 & 0 & 0 & 0
  \end{bmatrix}
\).
It follows that the basis for the solution set is given by \(\left\{
  \begin{bmatrix}
    -1 \\
    1 \\
    0 \\
    0
  \end{bmatrix},
  \begin{bmatrix}
    3 \\
    0 \\
    -2 \\
    1
  \end{bmatrix}
\right\}\).
\end{solution}

\begin{problem}{V1}
Let $V$ be the set of all pairs of real numbers with the operations, for any $(x_1,y_1), (x_2,y_2) \in V$, $c \in \IR$,
\begin{align*}
(x_1,y_1) \oplus (x_2,y_2) &= (x_1+x_2,y_1+y_2) \\
c \odot (x_1,y_1) &= (c^2x_1, c^3y_1)
\end{align*}
\begin{enumerate}[(a)]
\item Show that scalar multiplication \textbf{distributes scalars} over
      vector addition:
      \(c\odot((x_1,y_1) \oplus (x_2,y_2))=
      c\odot(x_1,y_1) \oplus c\odot(x_2,y_2)\).
\item Determine if $V$ is a vector space or not.  Justify your answer.
\end{enumerate}
\end{problem}
\begin{solution}
Let $(x_1,y_1), (x_2,y_2) \in V$ and let $c \in \IR$.
\begin{align*}
c \odot \left( (x_1,y_1) \oplus (x_2,y_2) \right) &= c \odot (x_1+x_2,y_1+y_2) \\
&= (c^2(x_1+x_2), c^3(y_1+y_2) ) \\
&= (c^2x_1,c^3y_1) \oplus (c^2x_2,c^3y_2) \\
&= c \odot (x_1,y_1) \oplus c \odot (x_2,y_2)
\end{align*}
However, $V$ is not a vector space, as the other distributive law fails:
$$(c+d) \odot (x_1,y_1) = ( (c+d)^2 x_1, (c+d)^3 y_1) \neq ((c^2+d^2)x_1, (c^3+d^3)y_1) = c \odot (x_1,y_1) \oplus d \odot (x_1,y_1).$$
\end{solution}


\begin{problem}{V2}
Determine if $\begin{bmatrix}0 \\ -1 \\ 2 \\ 6 \end{bmatrix}$ can be written as a linear combination of the vectors $\begin{bmatrix} 3 \\ -1 \\ -1 \\ 0 \end{bmatrix}$ and $\begin{bmatrix} -1 \\ 0 \\ 1 \\ 2 \end{bmatrix}$.
\end{problem}
\begin{solution}
$$\RREF\left(\left[\begin{array}{cc|c} 3 & -1 & 0 \\ -1 & 0 & -1 \\ -1 & 1 & 2 \\ 0 & 2 & 6\end{array} \right] \right)= \left[ \begin{array}{cc|c} 1 & 0 & 1 \\ 0 & 1 & 3 \\ 0 & 0 & 0 \\ 0 & 0 & 0 \end{array} \right]$$

Since this system has a solution, $\begin{bmatrix}0 \\ -1 \\ 2 \\ 6 \end{bmatrix}$ can be written as a linear combination of the vectors $\begin{bmatrix} 3 \\ -1 \\ -1 \\ 0 \end{bmatrix}$ and $\begin{bmatrix} -1 \\ 0 \\ 1 \\ 2 \end{bmatrix}$, namely
$$\begin{bmatrix}0 \\ -1 \\ 2 \\ 6 \end{bmatrix}=\begin{bmatrix} 3 \\ -1 \\ -1 \\ 0 \end{bmatrix}+3\begin{bmatrix} -1 \\ 0 \\ 1 \\ 2 \end{bmatrix}.$$
\end{solution}

\begin{problem}{V3}
Determine if the vectors $\begin{bmatrix} 1 \\ 0 \\ 2 \\1 \end{bmatrix}$, $\begin{bmatrix} 3 \\ 1 \\ 0 \\ -3 \end{bmatrix}$,$\begin{bmatrix} 0 \\ 3 \\ 0 \\ -2 \end{bmatrix}$, and $\begin{bmatrix}-1 \\ 1 \\ -1 \\ -1 \end{bmatrix}$ span $\IR^4$.
\end{problem}
\begin{solution}
$$\RREF\left(\begin{bmatrix}1 & 3 & 0 & -1 \\ 0 & 1 & 3 & 1 \\ 2 & 0 & 0 & -1 \\ 1 & -3 & -2 & -1 \end{bmatrix} \right) =\begin{bmatrix} 1 & 0 & 0 & 0 \\ 0 & 1 & 0 & 0 \\ 0 & 0 & 1 & 0 \\ 0 & 0 & 0 & 1 \end{bmatrix}$$
Since every row contains a pivot, the vectors span $\IR^4$.
\end{solution}

\begin{problem}{V4} Let $W$ be the set of all polynomials of the form
\(ax^3+bx\).  Determine if $W$ is a subspace of \(\P^3\).
\end{problem}
\begin{solution}
Yes because \(s(a_1x^3+b_1x)+t(a_2x^3+b_2x)=
(sa_1+ta_2)x^3+(sb_1+tb_2)x\) also belongs to \(W\).
Alternately, yes because \(W\) is isomorphic to \(\IR^2\).
\end{solution}


\begin{problem}{S1}
Determine if the set of matrices $\left\{\begin{bmatrix} 3 & -1 \\ 0 & 4 \end{bmatrix}, \begin{bmatrix} 1  & 2 \\ -2 & 1 \end{bmatrix}, \begin{bmatrix} 3 & -8 \\ 6 & 5 \end{bmatrix} \right\}$  is linearly dependent or linearly independent.
\end{problem}
\begin{solution}
$$\RREF\left(\begin{bmatrix} 3 & 1 & 3 \\ -1 & 2 & -8 \\ 0 & -2 & 6 \\ 4 & 1 & 5 \end{bmatrix} \right) = \begin{bmatrix} 1 & 0 & 2 \\ 0 & 1 & -3 \\ 0 & 0 & 0 \\ 0 & 0 & 0 \end{bmatrix}$$
Since the reduced row echelon form has a nonpivot column, the vectors are linearly dependent.
\end{solution}
\begin{problem}{S2}
Determine if the set $\left\{ x^3-3x^2+2x+2, -x^3+4x^2-x+1, -x^3+2x+1, 3x^2+3x+9 \right\}$ is a basis of $\P^3$ or not.
\end{problem}

\begin{solution}
$$\RREF \begin{bmatrix} 1 & -1 & -1 & 0 \\ -3 & 4 & 0 & 3 \\ 2 & -1 & 2 & 3 \\ 2 & 1 & 1 & 9 \end{bmatrix}=\begin{bmatrix} 1 &0 & 0 & 3 \\ 0 & 1 & 0 & 3 \\ 0 & 0 & 1 & 0 \\ 0 & 0 & 0 & 0 \end{bmatrix}$$
Since this is not the identity matrix, the set is not a basis.
\end{solution}
\begin{problem}{S3}
Let \(
  W={\rm span}\left\{
    \begin{bmatrix} 2 \\ 0 \\ 2 \\ 1 \end{bmatrix},
    \begin{bmatrix} 3 \\ 1 \\ -1 \\ 1 \end{bmatrix},
    \begin{bmatrix} 0 \\ 2 \\ -8 \\ -1 \end{bmatrix}
  \right\}
\). Find a basis for this vector space.
\end{problem}
\begin{solution}
\[
  \RREF\left(\begin{bmatrix}
    2 & 3 & 0 \\
    0 & 1 & 2 \\
    2 & -1 & -8 \\
    1 & 1 & -1
  \end{bmatrix} \right) =
  \begin{bmatrix}
    1 & 0 & -3 \\
    0 & 1 & 2 \\
    0 & 0 & 0 \\
    0 & 0 & 0
  \end{bmatrix}
\]
Thus \(\left\{
  \begin{bmatrix} 2 \\ 0 \\ 2 \\ 1 \end{bmatrix},
  \begin{bmatrix} 3 \\ 1 \\ -1 \\ 1 \end{bmatrix}
\right\}\) is a basis of $W$.
\end{solution}
\begin{problem}{S4}
Let $W$ be the subspace of $\P_3$ given by $W={\rm span}\left( \left\{ x^3-x^2+3x-3, 2x^3+x+1, 3x^3-x^2+4x-2, x^3+x^2+x-7\right\}\right)$.  Compute the dimension of $W$.
\end{problem}
\begin{solution}
$$ \RREF \left( \begin{bmatrix} 1 & 2 & 3 & 1 \\ -1 & 0 & -1 & 1 \\ 3 & 1 & 4 & 1 \\ -3 & 1 & -2 & -7 \end{bmatrix} \right) =  \begin{bmatrix} 1 & 0 & 1 & 0 \\ 0 & 1 & 1 & 0 \\ 0 & 0 & 0 & 1 \\ 0 & 0 & 0 & 0\end{bmatrix}$$
This has 3 pivot columns so $\dim(W)=3$.
\end{solution}


\begin{problem}{A1}
Let $T: \IR^4 \rightarrow \IR^2$ be the linear transformation given by $$T\left(\begin{bmatrix} x_1 \\ x_2 \\ x_3 \\ x_4 \end{bmatrix} \right) = \begin{bmatrix} x_1+3x_3 \\ 3x_2-5x_3 \end{bmatrix}.$$ Write the matrix for $T$ with respect to the standard bases of $\IR^4$ and $\IR^2$.
\end{problem}
\begin{solution}
$$\begin{bmatrix} 1 & 0 & 3 & 0 \\ 0 & 3 & -5 & 0 \end{bmatrix}$$
\end{solution}


\begin{problem}{A2} 
Let $T: \IR^2 \rightarrow \IR^2$ be given by $T\left(\begin{bmatrix}x \\ y  \end{bmatrix} \right) = \begin{bmatrix} x+y \\ \sqrt{x}+\sqrt{y} \end{bmatrix}$.  Determine if $T$ is a linear transformation.
\end{problem}
\begin{solution}
$$T\left(\begin{bmatrix} 0 \\ 4 \end{bmatrix}\right) = \begin{bmatrix} 4 \\ 2 \end{bmatrix} \neq \begin{bmatrix} 4 \\ 4 \end{bmatrix} = 4T\left(\begin{bmatrix} 0 \\ 1 \end{bmatrix} \right)$$
So $T$ is not a linear transformation.
\end{solution}

\begin{problem}{A3}
Determine if the following linear maps are injective (one-to-one) and/or surjective (onto).
\begin{enumerate}[(a)]
\item $S: \IR^2 \rightarrow \IR^3$ given by $S\left(\begin{bmatrix} x \\ y  \end{bmatrix} \right) = \begin{bmatrix} 3x+2y \\ x-y \\ x+4y \end{bmatrix} $
\item $T: \IR^3 \rightarrow \IR^3$ given by $T\left(\begin{bmatrix} x \\ y \\ z  \end{bmatrix} \right) = \begin{bmatrix} x+y+z \\ 2y+3z \\ x-y-2z \end{bmatrix} $
\end{enumerate}
\end{problem}

\begin{solution}
\begin{enumerate}[(a)]
\item $$\RREF\left( \begin{bmatrix} 1 &  1 & 1 \\ 0  & 2 & 3 \\ 1  & -1 & -2 \end{bmatrix} \right) = \begin{bmatrix} 1 &  0 & -\frac{1}{2} \\ 0  & 1 & \frac{3}{2} \\ 0& 0 & 0  \end{bmatrix}$$
Since there is a nonpivot column, $T$ is not injective.  Since there is a zero row, $T$ is not surjective.
\item $$\RREF \left( \begin{bmatrix} 3 & 2 \\ 1 & -1 \\ 1 & 4 \end{bmatrix} \right) = \begin{bmatrix} 1 & 0 \\ 0 & 1 \\ 0 & 0 \end{bmatrix}$$
Since all columns are pivot columns, $S$ is injective.  Since there is a zero row, $S$ is not surjective.
\end{enumerate}
\end{solution}



\begin{problem}{A4}
Let $T: \IR^3 \rightarrow \IR^3$ be the linear map given by \(
  T\left(\begin{bmatrix} x \\ y \\ z \end{bmatrix} \right) =
  \begin{bmatrix}
    8x-3y-z \\
    y+3z \\
    -7x+3y+2z
  \end{bmatrix}
\). Compute a basis for the kernel and a basis for the image of $T$.
\end{problem}
\begin{solution}
\[
  \RREF \left( \begin{bmatrix}
    8 & -3 & -1 \\
    0 & 1 & 3 \\
    -7 & 3 & 2
  \end{bmatrix} \right) = \begin{bmatrix}
    1 & 0 & 1 \\
    0 & 1 & 3 \\
    0 & 0 & 0
  \end{bmatrix}
\]

Thus \(\left\{
  \begin{bmatrix} 8 \\ 0 \\ -7 \end{bmatrix},
  \begin{bmatrix} -3 \\ 1 \\ 3 \end{bmatrix}
\right\} \) is a basis for the image, and \(\left\{
  \begin{bmatrix} -1 \\ -3 \\ 1 \end{bmatrix}
\right\} \) is a basis for the kernel.
\end{solution}


\begin{problem}{M1}
Let 
\begin{align*}
A &= \begin{bmatrix} 3 \\ 5 \\ -1  \end{bmatrix} & B&=\begin{bmatrix}  2 & 1 & -1 & 2 \\ 1 & -1 & 3 & -3  \end{bmatrix} & C &= \begin{bmatrix} 2 & -1 \\ 0 & 4 \\ 3 & 1 \end{bmatrix} \end{align*}
Exactly one of the six products $AB$, $AC$, $BA$, $BC$, $CA$, $CB$ can be computed.  Determine which one, and compute it.
\end{problem}
\begin{solution}
$CB$ is the only one that can be computed, and
$$CB=\begin{bmatrix} 3 & 3 & -5 & 7 \\ 4 & -4 & 12 & -12 \\ 7 & 2 & 0 & 3 \end{bmatrix}$$
\end{solution}
\begin{problem}{M2}
Determine if the matrix $\begin{bmatrix} 3 & -1 & 0 & 4 \\ 2 & 1 & 1 & 1 \\ 0 & 1 & 1 & -1 \\ 1 & -2 & 0 & 3 \end{bmatrix}$ is invertible.
\end{problem}
\begin{solution}
$$\RREF \begin{bmatrix} 3 & -1 & 0 & 4 \\ 2 & 1 & 1 & 1 \\ 0 & 1 & 1 & -1 \\ 1 & -2 & 0 & 3 \end{bmatrix} = \begin{bmatrix} 1 & 0 & 0 & 1 \\ 0 & 1 & 0 & -1 \\ 0 & 0 & 1 & 0 \\ 0 & 0 & 0 & 0 \end{bmatrix}$$
This matrix is not row equivalent to the identity matrix, so it is not invertible.
\end{solution}


\begin{problem}{M3}
  Find the inverse of the matrix
  \(\begin{bmatrix}
    3 & 1 & 3  \\
    2 & -1 & -6  \\
    1 & 1 & 4
  \end{bmatrix}\).
\end{problem}
\begin{solution}
\(\begin{bmatrix}[ccc|ccc]
  3 & 1 & 3 & 1 & 0 & 0 \\
  2 & -1 & -6 & 0 & 1 & 0 \\
  1 & 1 & 4 & 0 & 0 & 1
\end{bmatrix}\sim\begin{bmatrix}[ccc|ccc]
  1 & 0 & 0 & 2 & -1 & -3  \\
  0 & 1 & 0 & -14 & 9 & 24  \\
  0 & 0 & 1 & 3 & -2 & -5
\end{bmatrix}\). Thus the inverse is
\(\begin{bmatrix}
  2 & -1 & -3  \\
  -14 & 9 & 24  \\
  3 & -2 & -5
\end{bmatrix}\).
\end{solution}


\begin{problem}{G1}
Compute the determinant of the matrix $\begin{bmatrix} 3 & -1 & 0 & 4 \\ 2 & 1 & 1& -1 \\ 0 & 1 & 1 & 3 \\ 1 & -2 & 0 & 0 \end{bmatrix}$.
\end{problem}
\begin{solution}
$$\det \begin{bmatrix} 3 & -1 & 0 & 4 \\ 2 & 1 & 1 &-1 \\ 0 & 1 & 1 & 3 \\ 1 & -2 & 0 & 0 \end{bmatrix} = -\det \begin{bmatrix} -1 & 0 & 4 \\ 1 & 1 & -1 \\ 1 & 1 & 3 \end{bmatrix} +(-2) \det \begin{bmatrix} 3 & 0 & 4 \\ 2 & 1 & -1 \\ 0 & 1 & 3 \end{bmatrix} = -1(-4)+(-2)(20) = -36$$
\end{solution}

\begin{problem}{G2} 
Compute the eigenvalues, along with their algebraic multiplicities, of the matrix $ \begin{bmatrix} 8 & -3 & 2 \\ 23 & -9 & 5 \\ -7 & 2 & -3 \end{bmatrix}$.
\end{problem}
\begin{solution}
The eigenvalues are $0$ with multiplicity 1 and $-2$, with algebraic multiplicity 2.
\end{solution}

\begin{problem}{G3}
Compute the eigenspace of the eigenvalue $-1$ in the matrix $\begin{bmatrix} 4 & -2 & -1 \\ 15 & -7 & -3 \\ -5 & 2 & 0 \end{bmatrix}$. 
\end{problem}
\begin{solution}
$$\RREF\left(A+I\right) = \begin{bmatrix} 1 & - \frac{2}{5} & -\frac{1}{5} \\ 0 & 0 & 0 \\ 0 & 0 & 0 \end{bmatrix}$$
So the eigenspace is spanned by $\begin{bmatrix} 2 \\5 \\  0 \end{bmatrix}$ and $\begin{bmatrix} 1 \\ 0 \\ 5 \end{bmatrix}$.
\end{solution}


\begin{problem}{G4}
Compute the geometric multiplicity of the eigenvalue $-1$ in the matrix $\begin{bmatrix} 4 & -2 & -1 \\ 15 & -7 & -3 \\ -5 & 2 & 0 \end{bmatrix}$.  \end{problem}
\begin{solution}
$$\RREF\left(A+I\right) = \begin{bmatrix} 1 & - \frac{2}{5} & -\frac{1}{5} \\ 0 & 0 & 0 \\ 0 & 0 & 0 \end{bmatrix}$$
So the geometric multiplicity is $2$.
\end{solution}


\end{document}