\documentclass{sbgLAsemi}

\begin{extract*}
\usepackage{amsmath,amssymb,amsthm,enumerate}
\coursetitle{Math 237}
\courselabel{Linear Algebra}
\calculatorpolicy{You may use a calculator, but you must show all relevant work to receive credit for a standard.}


\newcommand{\IR}{\mathbb{R}}
\newcommand{\IC}{\mathbb{C}}
\renewcommand{\P}{\mathcal{P}}
\renewcommand{\Im}{{\rm Im\ }}
\DeclareMathOperator{\RREF}{RREF}
\DeclareMathOperator{\vspan}{span}

\makeatletter
\renewcommand*\env@matrix[1][*\c@MaxMatrixCols c]{%
  \hskip -\arraycolsep
  \let\@ifnextchar\new@ifnextchar
  \array{#1}}
\makeatother

\title{Semifinal}
\version{6}
\end{extract*}

\begin{document}

\begin{problem}{E1}
Write an augmented matrix corresponding to the following system of linear equations.
\begin{align*}
x_1+3x_2-4x_3 +x_4 &= 5 \\
3x_1+9x_2+x_3-7x_4 &= 0 \\
x_1-x_3 +x_4 &= 1
\end{align*}
\end{problem}
\begin{solution}
\[
\begin{bmatrix}[cccc|c]
1 & 3 & -4 & 1 & 5 \\
3 & 9 & 1 & -7 & 0 \\
1 & 0 & -1 & 1 &  1
\end{bmatrix}
\]
\end{solution}

\begin{problem}{E2}
Find the reduced row echelon form of the matrix below.
$$\begin{bmatrix}[cccc|c] 2 & 1 & -1 & 0 & 5 \\ 3 & -1 & 0 & -2 & 0 \\ -1 & 0 & 5 & 0 & -1 \end{bmatrix}$$
\end{problem}
\begin{solution}
$$\begin{bmatrix}[cccc|c]
 2 & 1 & -1 & 0 & 5 \\
 3 & -1 & 0 & -2 & 0 \\
 -1 & 0 & 5 & 0 & -1
\end{bmatrix} \sim
\begin{bmatrix}[cccc|c]
 -1 & 0 & 5 & 0 & -1  \\
 2 & 1 & -1 & 0 & 5 \\
 3 & -1 & 0 & -2 & 0
\end{bmatrix} \sim
\begin{bmatrix}[cccc|c]
 1 & 0 & -5 & 0 & 1 \\
 2 & 1 & -1 & 0 & 5 \\
 3 & -1 & 0 & -2 & 0
\end{bmatrix} $$
$$\sim
\begin{bmatrix}[cccc|c]
 1 & 0 & -5 & 0 & 1  \\
 0 & 1 & 9 & 0 & 3 \\
 0 & -1 & 15 & -2 & -3 \\
\end{bmatrix} \sim
\begin{bmatrix}[cccc|c]
 1 & 0 & -5 & 0 & 1  \\
 0 & 1 & 9 & 0 & 3 \\
 0 & 0 & 24 & -2 & 0 \\
\end{bmatrix} \sim
\begin{bmatrix}[cccc|c]
 1 & 0 & -5 & 0 & 1 \\
 0 & 1 & 9 & 0 & 3 \\
 0 & 0 & 1 & -\frac{1}{12} & 0 \\
\end{bmatrix} \sim
\begin{bmatrix}[cccc|c]
1 & 0 & 0 & -\frac{5}{12} & 1 \\
 0 & 1 & 0 & \frac{3}{4} & 3 \\
 0 & 0 & 1 & -\frac{1}{12} & 0
\end{bmatrix}$$
\end{solution}

\begin{problem}{E3}
Solve the system of linear equations.
\begin{align*}
2x+y-z+w &=5 \\
3x-y-2w &= 0 \\
-x+5z+3w&=-1
\end{align*}
\end{problem}
\begin{solution}
$$\RREF\left( \begin{bmatrix}[cccc|c] 2 & 1 & -1 & 0 & 5 \\ 3 & -1 & 0 & -2 & 0 \\ -1 & 0 & 5 & 0 & -1 \end{bmatrix} \right) = \begin{bmatrix}[cccc|c] 1 & 0 & 0 & -\frac{1}{12} & 1 \\ 0 & 1 & 0 & \frac{7}{4} & 3 \\ 0 & 0 & 1 & \frac{7}{12} & 0 \end{bmatrix}$$
So the solutions are $$\left\{ \begin{bmatrix} 1+a \\ 3-21a \\ -7a \\ 12a \end{bmatrix}\ \big|\ a \in \IR\right\}$$
\end{solution}

\begin{problem}{E4}
Find a basis for the solution set of the system of equations
\begin{align*}
x+3y+3z+7w &= 0 \\
 x+3y-z-w &= 0 \\
  2x+6y+3z+8w &= 0 \\
   x+3y-2z-3w &= 0
\end{align*}
\end{problem}
\begin{solution}
$$\RREF\left(\begin{bmatrix} 1 & 3 & 3 & 7 \\ 1 & 3 & -1 & -1 \\ 2 & 6 & 3 & 8 \\ 1 & 3 & -2 & -3 \end{bmatrix}\right) = \begin{bmatrix} 1 & 3 & 0 & 1 \\ 0 & 0 & 1 & 2 \\ 0 & 0 & 0 & 0 \\ 0 & 0 & 0 & 0 \end{bmatrix}$$
Then the solution set is
$$\left\{ \begin{bmatrix} -3a-b \\ a \\ -2b \\ b \end{bmatrix}\ \big|\ a,b \in \IR\right\}$$
So a basis for the solution set is $$ \left\{ \begin{bmatrix} 3 \\ -1 \\ 0 \\ 0 \end{bmatrix} , \begin{bmatrix} 1\\0 \\ 2 \\ -1 \end{bmatrix} \right\} $$
\end{solution}


\begin{problem}{V1}
Let $V$ be the set of all points on the parabola $y=x^2$ with the operations, for any $(x_1,y_1), (x_2,y_2) \in V$, $c \in \IR$,
\begin{align*}
(x_1,y_1) \oplus (x_2,y_2) &= (x_1-x_2,y_1+y_2-2x_1x_2) \\
c \odot (x_1,y_1) &= (cx_1, c^2y_1)
\end{align*}
\begin{enumerate}[(a)]
\item Show that scalar multiplication \textbf{distributes scalars} over
      vector addition:
      \(c\odot((x_1,y_1) \oplus (x_2,y_2))=
      c\odot(x_1,y_1) \oplus c\odot(x_2,y_2)\).
\item Determine if $V$ is a vector space or not.  Justify your answer.
\end{enumerate}
\end{problem}
\begin{solution}
$$ c\odot((x_1,y_1) \oplus (x_2,y_2))= c \odot (x_1-x_2,y_1+y_2-2x_1x_2) = (c(x_1-x_2), c^2(y_1+y_2-2x_1x_2) ) $$
$$ c\odot(x_1,y_1) \oplus c\odot(x_2,y_2) = (cx_1,c^2y_1) \oplus (cx_2,c^2y_2) = (cx_1-cx_2, c^2y_1+c^2y_2-2(cx_1)(cx_2) )$$

Not a vector space as addition is not commutative.
\end{solution}


\begin{problem}{V2} Determine if $\begin{bmatrix} 1 \\ 4 \\ 3 \end{bmatrix}$ is a linear combination of the vectors $\begin{bmatrix} 3 \\ 0 \\ -1 \end{bmatrix}$, $\begin{bmatrix} 1 \\ -1 \\ 4 \end{bmatrix}$, and $\begin{bmatrix} 5 \\ 1 \\  -6 \end{bmatrix}$.
\end{problem}
\begin{solution}
$$\RREF\left(\begin{bmatrix}[ccc|c] 3 & 1 & 5 & 1 \\ 0 & -1 & 1 & 4 \\ -1 & 4 & -6 & 3 \end{bmatrix} \right) = \begin{bmatrix}[ccc|c] 1 & 0 & 2 & 0 \\ 0 & 1 & -1 & 0 \\ 0 & 0 & 0 & 1 \end{bmatrix}$$
So $\begin{bmatrix} 1 \\ 4 \\ 3 \end{bmatrix}$ is not a linear combination of the three vectors.
\end{solution}


\begin{problem}{V3}
Determine if the vectors $\begin{bmatrix} 1 \\ 0 \\ 2 \\1 \end{bmatrix}$, $\begin{bmatrix} 3 \\ 1 \\ 0 \\ -3 \end{bmatrix}$,$\begin{bmatrix} 0 \\ 3 \\ 0 \\ -2 \end{bmatrix}$, and $\begin{bmatrix}-1 \\ 1 \\ -1 \\ -1 \end{bmatrix}$ span $\IR^4$.
\end{problem}
\begin{solution}
$$\RREF\left(\begin{bmatrix}1 & 3 & 0 & -1 \\ 0 & 1 & 3 & 1 \\ 2 & 0 & 0 & -1 \\ 1 & -3 & -2 & -1 \end{bmatrix} \right) =\begin{bmatrix} 1 & 0 & 0 & 0 \\ 0 & 1 & 0 & 0 \\ 0 & 0 & 1 & 0 \\ 0 & 0 & 0 & 1 \end{bmatrix}$$
Since every row contains a pivot, the vectors span $\IR^4$.
\end{solution}

\begin{problem}{V4} Let $W$ be the set of all polynomials of even degree.  Determine if $W$ is a subspace of the vector space of all polynomials.
\end{problem}
\begin{solution}
$W$ is closed under scalar multiplication, but not under addition.  For example, $x-x^2$ and $x^2$ are both in $W$, but $(x-x^2)+(x^2)=x \notin W$.
\end{solution}


\begin{problem}{S1}
Determine if the set of vectors $\left\{ \begin{bmatrix} -3 \\ -8 \\ 0 \end{bmatrix}, \begin{bmatrix} 1 \\ 2 \\ 2 \end{bmatrix}, \begin{bmatrix} 0 \\ -1 \\ 3 \end{bmatrix} \right\}$ is  linearly dependent or linearly independent
\end{problem}
\begin{solution}
$$\RREF\left( \begin{bmatrix}-3 & 1 & 0 \\ -8 & 2 & -1 \\ 0 & 2 & 3 \end{bmatrix}\right) = \begin{bmatrix} 1 & 0 & \frac{1}{2} \\ 0 & 1 & \frac{3}{2} \\ 0 & 0 & 0 \end{bmatrix}$$
This has a non pivot column, therefore the set is linearly dependent.
\end{solution}

\begin{problem}{S2}
Determine if the set $\left\{
\begin{bmatrix} 1 & -3 \\ 2 & 2 \end{bmatrix},
\begin{bmatrix} -1 & 4 \\ -1 & 1 \end{bmatrix},
\begin{bmatrix} -1 & 0 \\ 2 & 1 \end{bmatrix},
\begin{bmatrix} 0 & 3 \\ 3 & 9 \end{bmatrix}
\right\}$ is a basis of $\IR^{2 \times 2}$ or not.
\end{problem}

\begin{solution}
$$\RREF \begin{bmatrix} 1 & -1 & -1 & 0 \\ -3 & 4 & 0 & 3 \\ 2 & -1 & 2 & 3 \\ 2 & 1 & 1 & 9 \end{bmatrix}=\begin{bmatrix} 1 &0 & 0 & 3 \\ 0 & 1 & 0 & 3 \\ 0 & 0 & 1 & 0 \\ 0 & 0 & 0 & 0 \end{bmatrix}$$
Since this is not the identity matrix, the set is not a basis.
\end{solution}

\begin{problem}{S3}
Let $W = {\rm span} \left( \left\{ \begin{bmatrix} 1 \\ 1 \\ 2 \\ 1 \end{bmatrix}, \begin{bmatrix} 3 \\ 3 \\ 6 \\ 3 \end{bmatrix}, \begin{bmatrix} 3 \\ -1 \\ 3 \\ -2 \end{bmatrix}, \begin{bmatrix} 7 \\ -1 \\ 8 \\ -3 \end{bmatrix} \right\} \right)$.  Find a basis for $W$.
\end{problem}
\begin{solution}
$$\RREF\left(\begin{bmatrix} 1 & 3 & 3 & 7 \\ 1 & 3 & -1 & -1 \\ 2 & 6 & 3 & 8 \\ 1 & 3 & -2 & -3 \end{bmatrix}\right) = \begin{bmatrix} 1 & 3 & 0 & 1 \\ 0 & 0 & 1 & 2 \\ 0 & 0 & 0 & 0 \\  0 & 0 & 0 & 0 \end{bmatrix}$$

Then a basis is
$ \left\{ \begin{bmatrix} 1 \\ 1 \\ 2 \\ 1 \end{bmatrix} , \begin{bmatrix} 3 \\ -1 \\ 3 \\ -2 \end{bmatrix}\right\} $.
\end{solution}


\begin{problem}{S4}
Let $W={\rm span}\left(\left\{\begin{bmatrix} 2 \\ 0 \\ -2 \\ 0 \end{bmatrix}, \begin{bmatrix} 3 \\ 1 \\ 3 \\ 6 \end{bmatrix}, \begin{bmatrix} 0 \\ 0 \\ 1 \\ 1 \end{bmatrix}, \begin{bmatrix}1 \\ 2 \\ 0 \\ 1 \end{bmatrix}\right\}\right)$. Compute the dimension of $W$.
\end{problem}
\begin{solution}
$$\RREF\left( \begin{bmatrix} 2 & 3 & 0 & 1 \\ 0 & 1 & 0 & 2 \\ -2 & 3 & 1 & 0 \\ 0 & 6 & 1 & 1\end{bmatrix} \right) = \begin{bmatrix}1 & 0 & 0 & -\frac{5}{2} \\ 0 & 1 & 0 & 2 \\ 0 & 0 & 1 & -11\\ 0 & 0 & 0 & 0  \end{bmatrix} $$
This has 3 pivot columns so  $\dim(W) =3$.
\end{solution}


\begin{problem}{A1}
Let $T: \IR^3 \rightarrow \IR$ be the linear transformation given by $$T\left(\begin{bmatrix} x_1 \\ x_2 \\ x_3  \end{bmatrix} \right) = \begin{bmatrix} x_2+3x_3 \end{bmatrix}.$$ Write the matrix for $T$ with respect to the standard bases of $\IR^3$ and $\IR$.
\end{problem}
\begin{solution}
$$\begin{bmatrix} 0 & 1 & 3 \end{bmatrix}$$
\end{solution}


\begin{problem}{A2}
Determine if $D: \IR^{2\times 2} \rightarrow \IR$ given by $D\left(\begin{bmatrix} a & b \\ c & d \end{bmatrix} \right) = ad-bc$ is a linear transformation or not.
\end{problem}
\begin{solution}
$D(I)=1$ but $D(2I)=4 \neq 2D(I)$, so $D$ is not linear.
\end{solution}

\begin{problem}{A3}
Determine if each of the following linear transformations is injective (one-to-one) and/or surjective (onto).
\begin{enumerate}[(a)]
\item $S: \IR^2 \rightarrow \IR^2$ given by the standard matrix $\begin{bmatrix} 0 & 1 \\ -1 & 0 \end{bmatrix}$.
\item $T: \IR^4 \rightarrow \IR^3$ given by the standard matrix $\begin{bmatrix} 2 & 3 & -1 & -2 \\ 0 & 1 & 4 & 1 \\ 2 & 1 & -7 & -4 \end{bmatrix}$
\end{enumerate}
\end{problem}
\begin{solution}
\begin{enumerate}[(a)]
\item $ \RREF\begin{bmatrix} 0 & 1 \\ -1 & 0 \end{bmatrix}=\begin{bmatrix}1 & 0 \\ 0 & 1 \end{bmatrix}$.  Since each column is a pivot column, $S$ is injective.  Since there is no zero row, $S$ is surjective.
\item Since $\dim \IR^4 > \dim \IR^3$, $T$ is not injective.
$$\RREF\left(\begin{bmatrix} 2 & 3 & -1 & -2 \\ 0 & 1 & 3 & 1 \\ 2 & 1 & -7 & -4 \end{bmatrix}\right) = \begin{bmatrix} 1 & 0 & 0 & -\frac{5}{2} \\ 0 & 1 & 0 & 1 \\ 0 & 0 & 1 & 0\end{bmatrix}$$
Since there are no zero rows, $T$ is surjective.
\end{enumerate}
\end{solution}

\begin{problem}{A4}
Let $T: \IR^{2\times 2} \rightarrow \IR^3$ be the linear map given by \(
  T\left(\begin{bmatrix} a & b \\ x & y \end{bmatrix} \right) =
  \begin{bmatrix}
    a+x \\ 0 \\ b+y
  \end{bmatrix}
\). Compute a basis for the kernel and a basis for the image of $T$.
\end{problem}
\begin{solution}
Rewrite as \(
  T'\left(\begin{bmatrix} a \\ b \\ x \\ y \end{bmatrix} \right) =
  \begin{bmatrix}
    a+x \\ 0 \\ b+y
  \end{bmatrix}
\).
\[
  \RREF \left( \begin{bmatrix}
    1 & 0 & 1 & 0 \\
    0 & 0 & 0 & 0 \\
    0 & 1 & 0 & 1
  \end{bmatrix} \right) = \begin{bmatrix}
    1 & 0 & 1 & 0 \\
    0 & 0 & 0 & 0 \\
    0 & 1 & 0 & 1
  \end{bmatrix}
\]

Thus \(\left\{
  \begin{bmatrix} 1 \\ 0 \\ 0 \end{bmatrix},
  \begin{bmatrix} 0 \\ 0 \\ 1 \end{bmatrix}
\right\} \) is a basis for the image, and \(\left\{
  \begin{bmatrix} -1 & 0 \\ 1 & 0 \end{bmatrix},
  \begin{bmatrix} 0 & -1 \\ 0 & 1 \end{bmatrix}
\right\} \) is a basis for the kernel.
\end{solution}
\begin{problem}{M1}
Let 
\begin{align*}
A &= \begin{bmatrix} 0 & 0 & -1 & -1 \\ 1 & 3 & 7 & 2 \end{bmatrix} & B &= \begin{bmatrix} 0 & 1 & 7 & 7 \\ -1 & -2 & 0 & 4 \\ 0 & 0 & 1 & 5 \end{bmatrix} & C&=\begin{bmatrix} 3 & 2 \\ 0 & 1 \\ -2 & -1 \end{bmatrix}
\end{align*}
Exactly one of the six products $AB$, $AC$, $BA$, $BC$, $CA$, $CB$ can be computed.  Determine which one, and compute it.
\end{problem}
\begin{solution}
$CA$ is the only one that can be computed, and 
$$CA = \begin{bmatrix} 2 & 6 & 11 & 1 \\ 1 & 3 & 7 & 2 \\ -1 & -3 & -5 & 0 \end{bmatrix}$$
\end{solution}



\begin{problem}{M2}
Determine if the matrix $\begin{bmatrix} 3 & -1 & 0 & 4 \\ 2 & 1 & 1 & -1 \\ 0 & 1 & 1 & 3 \\ 1 & -2 & 0 & 0 \end{bmatrix}$ is invertible.
\end{problem}
\begin{solution}
This matrix is row equivalent to the identity matrix, so it is invertible.
\end{solution}

\begin{problem}{M3}
  Find the inverse of the matrix
  \(\begin{bmatrix}
    3 & 1 & 3  \\
    2 & -1 & -6  \\
    1 & 1 & 4
  \end{bmatrix}\).
\end{problem}
\begin{solution}
\(\begin{bmatrix}[ccc|ccc]
  3 & 1 & 3 & 1 & 0 & 0 \\
  2 & -1 & -6 & 0 & 1 & 0 \\
  1 & 1 & 4 & 0 & 0 & 1
\end{bmatrix}\sim\begin{bmatrix}[ccc|ccc]
  1 & 0 & 0 & 2 & -1 & -3  \\
  0 & 1 & 0 & -14 & 9 & 24  \\
  0 & 0 & 1 & 3 & -2 & -5
\end{bmatrix}\). Thus the inverse is
\(\begin{bmatrix}
  2 & -1 & -3  \\
  -14 & 9 & 24  \\
  3 & -2 & -5
\end{bmatrix}\).
\end{solution}


\begin{problem}{G1}
Compute the determinant of the matrix $\begin{bmatrix} 2 & 3 & 0 & 1 \\ -1 & 3 & 1 & 4 \\ 0 & 2 & 0 & 3 \\ 1 & -1 & 3 & 5 \end{bmatrix}$.
\end{problem}
\begin{solution}
$-60$.
\end{solution}

\begin{problem}{G2} 
Compute the eigenvalues, along with their algebraic multiplicities, of the matrix $ \begin{bmatrix} 9 & -3 & 2 \\ 19 & -6 & 5 \\ -11 & 4 & -2\end{bmatrix}$.
\end{problem}
\begin{solution}
1 with algebraic multiplicity 2, and -1 with algebraic multiplicity 1.
\end{solution}

\begin{problem}{G3}
Compute the eigenspace of the eigenvalue $-1$ in the matrix $\begin{bmatrix} 4 & -2 & -1 \\ 15 & -7 & -3 \\ -5 & 2 & 0 \end{bmatrix}$. 
\end{problem}
\begin{solution}
$$\RREF\left(A+I\right) = \begin{bmatrix} 1 & - \frac{2}{5} & -\frac{1}{5} \\ 0 & 0 & 0 \\ 0 & 0 & 0 \end{bmatrix}$$
So the eigenspace is spanned by $\begin{bmatrix} 2 \\5 \\  0 \end{bmatrix}$ and $\begin{bmatrix} 1 \\ 0 \\ 5 \end{bmatrix}$.
\end{solution}


\begin{problem}{G4}
Compute the geometric multiplicity of the eigenvalue $-1$ in the matrix $\begin{bmatrix} 4 & -2 & -1 \\ 15 & -7 & -3 \\ -5 & 2 & 0 \end{bmatrix}$.  \end{problem}
\begin{solution}
$$\RREF\left(A+I\right) = \begin{bmatrix} 1 & - \frac{2}{5} & -\frac{1}{5} \\ 0 & 0 & 0 \\ 0 & 0 & 0 \end{bmatrix}$$
So the geometric multiplicity is $2$.
\end{solution}


\end{document}