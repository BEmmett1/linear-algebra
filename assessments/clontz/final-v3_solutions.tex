\documentclass{sbgLAexam}

\begin{extract*}
\usepackage{amsmath,amssymb,amsthm,enumerate}
\coursetitle{Math 237}
\courselabel{Linear Algebra}
\calculatorpolicy{You may use a calculator, but you must show all relevant work to receive credit for a standard.}


\newcommand{\IR}{\mathbb{R}}
\newcommand{\IC}{\mathbb{C}}
\renewcommand{\P}{\mathcal{P}}
\renewcommand{\Im}{{\rm Im\ }}
\DeclareMathOperator{\RREF}{RREF}
\DeclareMathOperator{\vspan}{span}

\makeatletter
\renewcommand*\env@matrix[1][*\c@MaxMatrixCols c]{%
  \hskip -\arraycolsep
  \let\@ifnextchar\new@ifnextchar
  \array{#1}}
\makeatother

\title{Final Exam}
\standard{E1,E2,E3,E4,V1,V2,V3,V4,S1,S2,S3,S4,A1,A2,A3,A4,M1,M2,M3,G1,G2,G3,G4}
\version{3}
\end{extract*}

\begin{document}

\begin{problem}{E1}
Write an augmented matrix corresponding to the following system of linear equations.
\begin{align*}
x+3y-4z &= 5 \\
3x+9y+z &= 0 \\
x-z &= 1
\end{align*}
\end{problem}
\begin{solution}
\[
\begin{bmatrix}[ccc|c]
1 & 3 & -4 & 5 \\
3 & 9 & 1 & 0 \\
1 & 0 & -1 & 1
\end{bmatrix}
\]
\end{solution}

\begin{problem}{E2}
Find \(\RREF A\), where
\[
  A =
  \begin{bmatrix}[cccc|c]
    2 & 2 & 1 & 2 & -1 \\
    1 & 1 & 2 & 4 & 5 \\
    3 & 3 & -1 & -2 & 1
  \end{bmatrix}
\]
\end{problem}
\begin{solution}
\[
  \RREF A =
  \begin{bmatrix}[cccc|c]
    1 & 1 & 0 & 0 & 0 \\
    0 & 0 & 1 & 2 & 0 \\
    0 & 0 & 0 & 0 & 1
  \end{bmatrix}
\]
\end{solution}
\begin{extract}\newpage\end{extract}
\begin{problem}{E3}
Solve the following linear system.
\begin{align*}
3x+2y+z &= 7 \\
x+y+z &= 1 \\
-2x+3z &= -11
\end{align*}
\end{problem}
\begin{solution}
Let \(A =
  \begin{bmatrix}[ccc|c]
    3 & 2 & 1 & 7 \\
    1 & 1 & 1 & 1 \\
    -2 & 0 & 3 & 11
  \end{bmatrix}
\), so \(\RREF A =
  \begin{bmatrix}[ccc|c]
    1 & 0 & 0 & 4 \\
    0 & 1 & 0 & -2 \\
    0 & 0 & 1 & -1
  \end{bmatrix}
\). It follows that the system has exactly one solution:
\(\begin{bmatrix}
  4 & -2 & -1
\end{bmatrix}\)
\end{solution}
\begin{problem}{E4}
Find a basis for the solution set to the system of equations
\begin{align*}
x+2y-3z &= 0 \\
2x+y-4z &= 0 \\
3y -2z & = 0 \\
x -y -z &= 0
\end{align*}
\end{problem}
\begin{solution}
$$\RREF \left(\begin{bmatrix} 1 & 2 & -3 \\ 2 & 1 & -4 \\ 0 & 3 & -2 \\ 1 & -1 & -1 \end{bmatrix} \right) = \begin{bmatrix} 1 & 0 & -\frac{5}{3} \\ 0 & 1 & -\frac{2}{3} \\ 0 & 0 & 0 \\ 0 & 0 & 0 \end{bmatrix}$$
Then the solution set is
$$\left\{ \begin{bmatrix} \frac{5}{3}a \\ \frac{2}{3}a \\ a \end{bmatrix} \bigg|\ a \in \IR \right\}$$
So a basis is $\left\{ \begin{bmatrix} \frac{5}{3} \\ \frac{2}{3} \\ 1 \end{bmatrix} \right\}$ or $\left\{ \begin{bmatrix} 5 \\  2 \\ 3 \end{bmatrix} \right\}$.
\end{solution}

\begin{extract}\newpage\end{extract}
\begin{problem}{V1}
Let $V$ be the set of all pairs of real numbers with the operations, for any $(x_1,y_1), (x_2,y_2) \in V$, $c \in \IR$,
\begin{align*}
(x_1,y_1) \oplus (x_2,y_2) &= (x_1+x_2,y_1+y_2) \\
c \odot (x_1,y_1) &= (c^2x_1, c^3y_1)
\end{align*}
\begin{enumerate}[(a)]
\item Show that scalar multiplication \textbf{distributes scalars} over
      vector addition:
      \(c\odot((x_1,y_1) \oplus (x_2,y_2))=
      c\odot(x_1,y_1) \oplus c\odot(x_2,y_2)\).
\item Determine if $V$ is a vector space or not.  Justify your answer.
\end{enumerate}
\end{problem}
\begin{solution}
Let $(x_1,y_1), (x_2,y_2) \in V$ and let $c \in \IR$.
\begin{align*}
c \odot \left( (x_1,y_1) \oplus (x_2,y_2) \right) &= c \odot (x_1+x_2,y_1+y_2) \\
&= (c^2(x_1+x_2), c^3(y_1+y_2) ) \\
&= (c^2x_1,c^3y_1) \oplus (c^2x_2,c^3y_2) \\
&= c \odot (x_1,y_1) \oplus c \odot (x_2,y_2)
\end{align*}
However, $V$ is not a vector space, as the other distributive law fails:
$$(c+d) \odot (x_1,y_1) = ( (c+d)^2 x_1, (c+d)^3 y_1) \neq ((c^2+d^2)x_1, (c^3+d^3)y_1) = c \odot (x_1,y_1) \oplus d \odot (x_1,y_1).$$
\end{solution}


\begin{problem}{V2}
  Determine if
  \(\begin{bmatrix} 4 \\ -1 \\ 6 \\ -7 \end{bmatrix}\)
  belongs to the span of the set
  \(\left\{
    \begin{bmatrix} 2 \\ 0 \\ -1 \\ 5 \end{bmatrix},
    \begin{bmatrix} 4 \\ -1 \\ 4 \\ 3 \end{bmatrix}
    \right\}
  \).
\end{problem}
\begin{solution}
  Since
  \[
    \RREF\left(
      \begin{bmatrix}[cc|c]
        2 & 4 & 4 \\
        0 & -1 & -1 \\
        -1 & 4 & 6 \\
        5 & 3 & -7
      \end{bmatrix}
    \right) =
    \begin{bmatrix}[cc|c]
      1 & 0 & 0 \\
      0 & 1 & 0 \\
      0 & 0 & 1 \\
      0 & 0 & 0
    \end{bmatrix}
  \]
  contains the contradiction \(0=1\),
  \(\begin{bmatrix} 4 \\ -1 \\ 6 \\ -7 \end{bmatrix}\) is
  not a linear combination of the three vectors.
\end{solution}


\begin{extract}\newpage\end{extract}
\begin{problem}{V3}
Determine if the vectors $\begin{bmatrix} 2 \\ 0 \\ -2 \\ 0 \end{bmatrix}$, $\begin{bmatrix} 3 \\ 1 \\ 3 \\ 6 \end{bmatrix}$, $\begin{bmatrix} 0 \\ 0 \\ 1 \\ 1 \end{bmatrix}$, and $\begin{bmatrix}1 \\ 2 \\ 0 \\ 1 \end{bmatrix}$ span $\IR^4$.
\end{problem}
\begin{solution}
$$\RREF\left( \begin{bmatrix} 2 & 3 & 0 & 1 \\ 0 & 1 & 0 & 2 \\ -2 & 3 & 1 & 0 \\ 0 & 6 & 1 & 1\end{bmatrix} \right) = \begin{bmatrix}1 & 0 & 0 & -\frac{5}{2} \\ 0 & 1 & 0 & 2 \\ 0 & 0 & 1 & -11\\ 0 & 0 & 0 & 0  \end{bmatrix} $$
Since there is a zero row, the vectors do not span $\IR^4$.
\end{solution}

\begin{problem}{V4}
Determine if the set of all lattice points, i.e. $\{(x,y)\ \big|\ \text{$x$ and $y$ are integers} \}$ is a subspace of $\IR^2$.
\end{problem}
\begin{solution}
This set is closed under addition, but not under scalar multiplication so it is not a subspace.
\end{solution}

\begin{extract}\newpage\end{extract}
\begin{problem}{S1}
Determine if the vectors $\begin{bmatrix} 1 \\ 1 \\ -1 \end{bmatrix}$, $\begin{bmatrix} 3 \\ -1 \\ 1 \end{bmatrix}$, and $\begin{bmatrix} 2 \\ 0 \\ -2 \end{bmatrix}$ are linearly dependent or linearly independent
\end{problem}
\begin{solution}
$$\RREF\left(\begin{bmatrix} 1 & 3 & 2 \\ 1 & -1 & 0 \\ -1 & 1 & -2 \end{bmatrix} \right)= \begin{bmatrix} 1 & 0 &0 \\ 0 & 1 & 0 \\ 0 & 0 & 1\end{bmatrix}$$
Since each column is a pivot column, the vectors are linearly independent.
\end{solution}


\begin{problem}{S2}
  Determine if the set \(\left\{
    \begin{bmatrix} 3 \\ -1 \\ 2 \end{bmatrix},
    \begin{bmatrix} 2 \\ 0 \\ 2 \end{bmatrix},
    \begin{bmatrix} 1 \\ 4 \\ -1 \end{bmatrix}
  \right\}\) is a basis of $\IR^3$.
\end{problem}
\begin{solution}
  \[\RREF\left(
    \begin{bmatrix}
      3 & 2 & 1 \\
      -1 & 0 & 4 \\
      2 & 2 & -1
    \end{bmatrix} \right)= \begin{bmatrix}
      1 & 0 &0 \\
      0 & 1 & 0 \\
      0 & 0 & 1
    \end{bmatrix}
  \]
Since the resulting matrix is the identity matrix, it is a basis.
\end{solution}


\begin{extract}\newpage\end{extract}
\begin{problem}{S3}
Let $W = {\rm span} \left( \left\{  \begin{bmatrix} -3 \\ -8 \\ 0 \end{bmatrix}, \begin{bmatrix} 1 \\ 2 \\ 2 \end{bmatrix}, \begin{bmatrix} 0 \\ -1 \\ 3 \end{bmatrix} \right\} \right)$.   Find a basis for $W$.
\end{problem}
\begin{solution}
Let $A= \begin{bmatrix}-3 & 1 & 0 \\ -8 & 2 & -1 \\ 0 & 2 & 3\end{bmatrix}$, and compute $\RREF(A) = \begin{bmatrix} 1 & 0 & \frac{1}{2} \\ 0 & 1 & \frac{3}{2} \\ 0 & 0 & 0 \end{bmatrix}$.
Since the first two columns are pivot columns, $\left\{ \begin{bmatrix} -3 \\ -8 \\ 0 \end{bmatrix}, \begin{bmatrix} 1 \\ 2 \\ 2 \end{bmatrix} \right\} $ is a basis for $W$.
\end{solution}


\begin{problem}{S4}
Let $W$ be the subspace of $\IR^{2\times2}$ given by $W={\rm span}\left(\left\{\begin{bmatrix} 2 & 0 \\ -2 & 0 \end{bmatrix}, \begin{bmatrix} 3 & 1 \\ 3 & 6 \end{bmatrix}, \begin{bmatrix} 0 & 0 \\ 1 & 1 \end{bmatrix}, \begin{bmatrix}1 & 2 \\ 0 & 1 \end{bmatrix}\right\}\right)$. Compute the dimension of $W$.
\end{problem}
\begin{solution}
$$\RREF\left( \begin{bmatrix} 2 & 3 & 0 & 1 \\ 0 & 1 & 0 & 2 \\ -2 & 3 & 1 & 0 \\ 0 & 6 & 1 & 1\end{bmatrix} \right) = \begin{bmatrix}1 & 0 & 0 & -\frac{5}{2} \\ 0 & 1 & 0 & 2 \\ 0 & 0 & 1 & -11\\ 0 & 0 & 0 & 0  \end{bmatrix} $$
This has 3 pivot columns so  $\dim(W) =3$.
\end{solution}


\begin{extract}\newpage\end{extract}
\begin{problem}{A1}
Let $T: \IR^3\rightarrow \IR^4$ be the linear transformation given by $$T\left(\begin{bmatrix} x \\ y \\ z \\  \end{bmatrix} \right) = \begin{bmatrix} -3x+y \\ -8x+2y-z \\ 2y+3z \\ 7x \end{bmatrix}.$$  Write the matrix for $T$ with respect to the standard bases of $\IR^3$ and $\IR^4$.
\end{problem}
\begin{solution}
$$\begin{bmatrix} 3 & 1 & 0 \\ -8 & 2 & -1 \\ 0 & 2 & 3 \\ 7 & 0 & 0 \end{bmatrix}$$
\end{solution}

\begin{problem}{A2} 
 Determine if $T: \IR^2 \rightarrow \IR^2$ given by $T\left(\begin{bmatrix} x \\ y \end{bmatrix}\right) = \begin{bmatrix} e^{x} \\ e^y \end{bmatrix}$ is a linear transformation.
\end{problem}
\begin{solution}
It is not linear.  For example, $$\begin{bmatrix}e^2 \\ 1 \end{bmatrix} = T\left(\begin{bmatrix} 2 \\ 0 \end{bmatrix} \right) \neq 2 T\left(\begin{bmatrix} 1 \\ 0 \end{bmatrix}\right)=\begin{bmatrix} 2e \\ 1 \end{bmatrix}$$
\end{solution}

\begin{extract}\newpage\end{extract}
\begin{problem}{A3}
Determine if each of the following linear transformations is injective (one-to-one) and/or surjective (onto).
\begin{enumerate}[(a)]
\item $T: \IR^3 \rightarrow \IR^3$ given by $T\left(\begin{bmatrix} x \\ y \\ z  \end{bmatrix} \right) = \begin{bmatrix} x+y+z \\ 2y+3z \\ x-y-2z \end{bmatrix} $
\item $S: \IR^2 \rightarrow \IR^3$ given by $S\left(\begin{bmatrix} x \\ y  \end{bmatrix} \right) = \begin{bmatrix} 3x+2y \\ x-y \\ x+4y \end{bmatrix} $
\end{enumerate}
\end{problem}
\begin{solution}
\begin{enumerate}[(a)]
\item $$\RREF\left( \begin{bmatrix} 1 &  1 & 1 \\ 0  & 2 & 3 \\ 1  & -1 & -2 \end{bmatrix} \right) = \begin{bmatrix} 1 &  0 & -\frac{1}{2} \\ 0  & 1 & \frac{3}{2} \\ 0& 0 & 0  \end{bmatrix}$$
Since there is a nonpivot column, $T$ is not injective.  Since there is a zero row, $T$ is not surjective.
\item $$\RREF \left( \begin{bmatrix} 3 & 2 \\ 1 & -1 \\ 1 & 4 \end{bmatrix} \right) = \begin{bmatrix} 1 & 0 \\ 0 & 1 \\ 0 & 0 \end{bmatrix}$$
Since all columns are pivot columns, $S$ is injective.  Since there is a zero row, $S$ is not surjective.
\end{enumerate}
\end{solution}



\begin{problem}{A4}
Let $T: \IR^4\rightarrow \IR^4$ be the linear transformation given by $$T\left(\begin{bmatrix} x \\ y \\ z \\ w \end{bmatrix} \right) = \begin{bmatrix} x+3y+3z+7w \\ x+3y-z-w \\ 2x+6y+3z+8w \\ x+3y-2z-3w \end{bmatrix}$$
Compute a basis for the kernel and a basis for the image of $T$.
\end{problem}
\begin{solution}

$$\RREF\left(\begin{bmatrix} 1 & 3 & 3 & 7 \\ 1 & 3 & -1 & -1 \\ 2 & 6 & 3 & 8 \\ 1 & 3 & -2 & -3 \end{bmatrix}\right) = \begin{bmatrix} 1 & 3 & 0 & 1 \\ 0 & 0 & 1 & 2 \\ 0 & 0 & 0 & 0 \end{bmatrix}$$

Then a basis for the kernel is
$$\left\{ \begin{bmatrix} -3 \\ 1 \\ 0 \\ 0 \end{bmatrix} , \begin{bmatrix} -1\\0 \\ -2 \\ 1 \end{bmatrix} \right\}$$
and a basis for the image is
$$\left\{ \begin{bmatrix} 1 \\ 1 \\ 2 \\ 1 \end{bmatrix} , \begin{bmatrix} 3 \\ -1 \\ 3 \\ -2 \end{bmatrix}\right\} $$
\end{solution}

\begin{extract}\newpage\end{extract}
\begin{problem}{M1}
Let 
\begin{align*}
A &= \begin{bmatrix} 0 & 0 & -1 & -1 \\ 1 & 3 & 7 & 2 \end{bmatrix} & B &= \begin{bmatrix} 0 & 1 & 7 & 7 \\ -1 & -2 & 0 & 4 \\ 0 & 0 & 1 & 5 \end{bmatrix} & C&=\begin{bmatrix} 3 & 2 \\ 0 & 1 \\ -2 & -1 \end{bmatrix}
\end{align*}
Exactly one of the six products $AB$, $AC$, $BA$, $BC$, $CA$, $CB$ can be computed.  Determine which one, and compute it.
\end{problem}
\begin{solution}
$CA$ is the only one that can be computed, and 
$$CA = \begin{bmatrix} 2 & 6 & 11 & 1 \\ 1 & 3 & 7 & 2 \\ -1 & -3 & -5 & 0 \end{bmatrix}$$
\end{solution}



\begin{problem}{M2}
Determine if the matrix $\begin{bmatrix} 3 & -1 & 0 & 4 \\ 2 & 1 & 1 & -1 \\ 0 & 1 & 1 & 3 \\ 1 & -2 & 0 & 0 \end{bmatrix}$ is invertible.
\end{problem}
\begin{solution}
This matrix is row equivalent to the identity matrix, so it is invertible.
\end{solution}

\begin{extract}\newpage\end{extract}
\begin{problem}{M3}
  Find the inverse of the matrix
  \(\begin{bmatrix}
    3 & 1 & 3  \\
    2 & -1 & -6  \\
    1 & 1 & 4
  \end{bmatrix}\).
\end{problem}
\begin{solution}
\(\begin{bmatrix}[ccc|ccc]
  3 & 1 & 3 & 1 & 0 & 0 \\
  2 & -1 & -6 & 0 & 1 & 0 \\
  1 & 1 & 4 & 0 & 0 & 1
\end{bmatrix}\sim\begin{bmatrix}[ccc|ccc]
  1 & 0 & 0 & 2 & -1 & -3  \\
  0 & 1 & 0 & -14 & 9 & 24  \\
  0 & 0 & 1 & 3 & -2 & -5
\end{bmatrix}\). Thus the inverse is
\(\begin{bmatrix}
  2 & -1 & -3  \\
  -14 & 9 & 24  \\
  3 & -2 & -5
\end{bmatrix}\).
\end{solution}


\begin{problem}{G1}
Compute the determinant of the matrix
\[
  \begin{bmatrix}
    1 & 3 & 2 & 4 \\
    5 & 0 & -4 & 0 \\
    -2 & 3 & -1 & 1 \\
    0 & 1 & 0 & 1
  \end{bmatrix}
.\]
\end{problem}
\begin{solution}
\(15\).
\end{solution}

\begin{extract}\newpage\end{extract}
\begin{problem}{G2}
Compute the eigenvalues, along with their algebraic multiplicities, of the matrix $ \begin{bmatrix}2 & -3 & 2 \\ 8 & -9 & 5 \\ 8 & -7 & 3 \end{bmatrix}$.
\end{problem}
\begin{solution}
The eigenvalues are $0$ (with algebraic multiplicity 1)
and $-2$ (with algebraic multiplicity 2).
\end{solution}
\begin{problem}{G3}
Find the eigenspace associated to the eigenvalue $-2$ in the matrix $A=\begin{bmatrix}8 & -3 & 2 \\ 23 & -9 & 5 \\ -7 & 2 & -3\end{bmatrix}$
\end{problem}
\begin{solution}
The eigenspace is spanned by $\begin{bmatrix} 1 \\ 4 \\ 1 \end{bmatrix}$.
\end{solution}

\begin{extract}\newpage\end{extract}
\begin{problem}{G4}
Compute the geometric multiplicity of the eigenvalue $2$ in the matrix $A=\begin{bmatrix}8 & -3 & 2 \\ 15 & -5 & 5 \\ -3 & 2 & 1\end{bmatrix}$
\end{problem}
\begin{solution}
The eigenspace is spanned by $\begin{bmatrix} -\frac{1}{3} \\ 0 \\ 1 \end{bmatrix}$, so the geometric multiplicity is $1$.
\end{solution}


\end{document}