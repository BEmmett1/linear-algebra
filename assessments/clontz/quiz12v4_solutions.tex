\documentclass{sbgLAquiz}

\begin{extract*}
\usepackage{amsmath,amssymb,amsthm,enumerate}
\coursetitle{Math 237}
\courselabel{Linear Algebra}
\calculatorpolicy{You may use a calculator, but you must show all relevant work to receive credit for a standard.}


\newcommand{\IR}{\mathbb{R}}
\newcommand{\IC}{\mathbb{C}}
\renewcommand{\P}{\mathcal{P}}
\renewcommand{\Im}{{\rm Im\ }}
\DeclareMathOperator{\RREF}{RREF}
\DeclareMathOperator{\vspan}{span}

\makeatletter
\renewcommand*\env@matrix[1][*\c@MaxMatrixCols c]{%
  \hskip -\arraycolsep
  \let\@ifnextchar\new@ifnextchar
  \array{#1}}
\makeatother

\title{Mastery Quiz Day 12 }
\standard{V1,V3,V4}
\version{4}
\end{extract*}

\begin{document}

\begin{problem}{V1}
Let $V$ be the set of all polynomials with the operations, for any $f, g \in V$, $c \in \IR$,
\begin{align*}
f \oplus g &= f^\prime + g^\prime \\
c \odot f &= c f^\prime
\end{align*}
(here $f^\prime$ denotes the derivative of $f$).
\begin{enumerate}[(a)]
\item Show that scalar multiplication \textbf{distributes scalars} over
      vector addition:
      \(c\odot(f \oplus g)=
      c\odot f \oplus c\odot g\).
\item Determine if $V$ is a vector space or not.  Justify your answer.
\end{enumerate}
\end{problem}
\begin{solution}
Let $f,g \in \mathcal{P}$, and let $c \in \IR$.
$$c \odot (f \oplus g) = c \odot (f^\prime+g^\prime) =
c(f^\prime+g^\prime)^\prime = cf^{\prime\ \prime}+cg^{\prime\ \prime} =
cf^\prime\oplus cg^\prime= c \odot f \oplus c \odot g.$$
However, this is not a vector space, as there is no zero vector.  Additionally, $1 \odot f \neq f$ for any nonzero polynomial $f$.
\end{solution}


\begin{problem}{V3}
Determine if the vectors  $\begin{bmatrix} -3 \\ 1 \\ 1 \end{bmatrix}$,$\begin{bmatrix} 5 \\ -1 \\ -2 \end{bmatrix}$,$\begin{bmatrix}2 \\ 0 \\ -1 \end{bmatrix}$, and $\begin{bmatrix} 0 \\ 2 \\ -1\end{bmatrix}$ span $\IR^3$
\end{problem}
\begin{solution}
$$\RREF\left(\begin{bmatrix}
-3 & 5 & 2 & 0 \\ 1 & -1 & 0 & 2 \\ 1 & -2 & -1 & -1 \end{bmatrix}\right)=\begin{bmatrix} 1 & 0 & 1 & 5 \\ 0 & 1 & 1 & 3 \\ 0 & 0 & 0 & 0\end{bmatrix}$$
Since the resulting matrix has only two pivot columns, the vectors do not span $\IR^3$.
\end{solution}


\begin{extract}\newpage\end{extract}
\begin{problem}{V4} Let \(W\) be the set of all \(\IR^3\) vectors
\(\begin{bmatrix} x \\ y \\ z \end{bmatrix}\) satisfying \(x+y+z=0\) (this forms a plane).
Determine if \(W\) is a subspace of \(\IR^3\).
\end{problem}
\begin{solution}
Yes, because \(z=-x-y\) and
\(
  a\begin{bmatrix} x_1 \\ y_1 \\ -x_1-y_1 \end{bmatrix}+
  b\begin{bmatrix} x_2 \\ y_2 \\ -x_2-y_2 \end{bmatrix}=
  \begin{bmatrix}
    ax_1+bx_2 \\
    ay_1+by_2 \\
    -(ax_1+bx_2)-(ay_1+by_2)
  \end{bmatrix}
\).
Alternately, yes because \(W\) is isomorphic to \(\IR^2\).
\end{solution}


\end{document}