\documentclass{sbgLAexam}

\begin{extract*}
\usepackage{amsmath,amssymb,amsthm,enumerate}
\coursetitle{Math 237}
\courselabel{Linear Algebra}
\calculatorpolicy{You may use a calculator, but you must show all relevant work to receive credit for a standard.}


\newcommand{\IR}{\mathbb{R}}
\newcommand{\IC}{\mathbb{C}}
\renewcommand{\P}{\mathcal{P}}
\renewcommand{\Im}{{\rm Im\ }}
\DeclareMathOperator{\RREF}{RREF}
\DeclareMathOperator{\vspan}{span}

\makeatletter
\renewcommand*\env@matrix[1][*\c@MaxMatrixCols c]{%
  \hskip -\arraycolsep
  \let\@ifnextchar\new@ifnextchar
  \array{#1}}
\makeatother

\title{Midterm Exam}
\standard{E1,E2,E3,E4,V1,V2,V3,V4,S1,S2,S3,S4}
\version{2}
\end{extract*}

\begin{document}

\begin{problem}{E1}
Write a system of linear equations corresponding to the following
augmented matrix.
\[
\begin{bmatrix}[ccc|c]
2 & -1 & 0 & 1  \\
-1 & 4 & 1 & -7  \\
1 & 2 & -1 & 0
\end{bmatrix}
\]
\end{problem}
\begin{solution}
\begin{align*}
2x_1-x_2&=1 \\
-x_1+4x_2+x_3&=-7 \\
x_1+2x_2-x_3 &= 0
\end{align*}
\end{solution}

\begin{problem}{E2}
Put the following matrix in reduced row echelon form.
$$\begin{bmatrix}-3 & 5 & 2 & 0 \\ 1 & -1 & 0 & 2 \\ 1 & -2 & -1 & -1 \end{bmatrix}$$
\end{problem}
\begin{solution}
\begin{align*}
\begin{bmatrix}
-3 & 5 & 2 & 0 \\
 1 & -1 & 0 & 2 \\
 1 & -2 & -1 & -1
\end{bmatrix} &\sim
\begin{bmatrix}
 1 & -1 & 0 & 2 \\
-3 & 5 & 2 & 0 \\
 1 & -2 & -1 & -1
\end{bmatrix} \sim
\begin{bmatrix}
 1 & -1 & 0 & 2 \\
 0 & 2 & 2 & 6 \\
 0 & -1 & -1 & -3
\end{bmatrix} \sim
\begin{bmatrix}
 1 & -1 & 0 & 2 \\
 0 & 1 & 1 & 3 \\
 0 & -1 & -1 & -3
\end{bmatrix} \\ &\sim
\begin{bmatrix}
 1 & 0 & 1 & 5 \\
 0 & 1 & 1 & 3 \\
 0 & 0 & 0 & 0
\end{bmatrix}
\end{align*}
\end{solution}

\begin{extract}\newpage\end{extract}
\begin{problem}{E3}
Find the solution set for the following system of linear equations.
\begin{align*}
2x_1-2x_2+6x_3-x_4 &=-1 \\
3x_1+6x_3+x_4 &= 5 \\
-4x_1+x_2-9x_3+2x_4&=-7
\end{align*}
\end{problem}
\begin{solution}
Let \(A =
  \begin{bmatrix}[cccc|c]
    2 & -2 & 6 & -1 & -1 \\
    3 & 0 & 6 & 1 & 5 \\
    -4 & 1 & -9 & 2 & -7
  \end{bmatrix}
\), so \(\RREF A =
  \begin{bmatrix}[cccc|c]
    1 & 0 & 2 & 0 & 2 \\
    0 & 1 & -1 & 0 & 3 \\
    0 & 0 & 0 & 1 & -1
  \end{bmatrix}
\). It follows that the solution set is given by \(
  \begin{bmatrix}
    2 - 2a \\
    3 + a \\
    a \\
    -1
  \end{bmatrix}
\) for all real numbers \(a\).
\end{solution}

\begin{problem}{E4}
Find a basis for the solution set to the system of equations
\begin{align*}
x+2y-3z &= 0 \\
2x+y-4z &= 0 \\
3y -2z & = 0 \\
x -y -z &= 0
\end{align*}
\end{problem}
\begin{solution}
$$\RREF \left(\begin{bmatrix} 1 & 2 & -3 \\ 2 & 1 & -4 \\ 0 & 3 & -2 \\ 1 & -1 & -1 \end{bmatrix} \right) = \begin{bmatrix} 1 & 0 & -\frac{5}{3} \\ 0 & 1 & -\frac{2}{3} \\ 0 & 0 & 0 \\ 0 & 0 & 0 \end{bmatrix}$$
Then the solution set is
$$\left\{ \begin{bmatrix} \frac{5}{3}a \\ \frac{2}{3}a \\ a \end{bmatrix} \bigg|\ a \in \IR \right\}$$
So a basis is $\left\{ \begin{bmatrix} \frac{5}{3} \\ \frac{2}{3} \\ 1 \end{bmatrix} \right\}$ or $\left\{ \begin{bmatrix} 5 \\  2 \\ 3 \end{bmatrix} \right\}$.
\end{solution}

\begin{extract}\newpage\end{extract}
\begin{problem}{V1}
Let $V$ be the  set of all real numbers together with the operations $\oplus$ and $\odot$ defined by, for any $x,y \in V$ and $c \in \IR$,
\begin{align*}
x\oplus y  &= x+y-3 \\
c \odot x &= cx-3(c-1)
\end{align*}
\begin{enumerate}[(a)]
\item Show that \textbf{scalar multiplication} is
      \textbf{associative}: \(a\odot(b\odot x)=(ab)\odot x\).
\item Determine if $V$ is a vector space or not.  Justify your answer
\end{enumerate}
\end{problem}

\begin{solution}
Let $x,y \in V$, $c,d \in \IR$.
To show associativity:
\begin{align*}
c\odot \left( d \odot x\right) &= c\odot \left( dx-3(d-1) \right) \\
&= c\left(dx-3(d-1)\right)-3(c-1) \\
&= cdx-3(cd-1) \\
&= (cd) \odot x
\end{align*}

We verify the remaining 7 properties to see that $V$ is a vector space.
\begin{enumerate}[1)]
\item Real addition is associative, so $\oplus$ is associative.
\item $x\oplus 3 = x+3-3=x$, so $3$ is the additive identity.
\item $x \oplus (6-x) = x+(6-x)-3=3$, so $6-x$ is the additive inverse of $x$.
\item Real addition is commutative, so $\oplus$ is commutative.
\item Associativity shown above
\item $1 \odot x = x-3(1-1)=x$
\item \begin{align*} c \odot (x \oplus y) &=
c \odot (x+y-3) \\
&= c(x+y-3)-3(c-1) \\
&= cx-3(c-1) + cy-3(c-1) -3 \\
&= (c\odot x ) \oplus (c\odot y)
\end{align*}
\item \begin{align*} (c+d) \odot x &= (c+d)x-3(c+d-1) \\
&= cx-3(c-1)+dx-3(c-1)-3 \\
&= (c\odot x ) \oplus (d \odot x)
\end{align*}
\end{enumerate}

Therefore $V$ is a vector space.
\end{solution}

\begin{problem}{V2}
Determine if $\begin{bmatrix}0 \\ -1 \\ 2 \\ 6 \end{bmatrix}$ can be written as a linear combination of the vectors $\begin{bmatrix} 3 \\ -1 \\ -1 \\ 0 \end{bmatrix}$ and $\begin{bmatrix} -1 \\ 0 \\ 1 \\ 2 \end{bmatrix}$.
\end{problem}
\begin{solution}
$$\RREF\left(\left[\begin{array}{cc|c} 3 & -1 & 0 \\ -1 & 0 & -1 \\ -1 & 1 & 2 \\ 0 & 2 & 6\end{array} \right] \right)= \left[ \begin{array}{cc|c} 1 & 0 & 1 \\ 0 & 1 & 3 \\ 0 & 0 & 0 \\ 0 & 0 & 0 \end{array} \right]$$

Since this system has a solution, $\begin{bmatrix}0 \\ -1 \\ 2 \\ 6 \end{bmatrix}$ can be written as a linear combination of the vectors $\begin{bmatrix} 3 \\ -1 \\ -1 \\ 0 \end{bmatrix}$ and $\begin{bmatrix} -1 \\ 0 \\ 1 \\ 2 \end{bmatrix}$, namely
$$\begin{bmatrix}0 \\ -1 \\ 2 \\ 6 \end{bmatrix}=\begin{bmatrix} 3 \\ -1 \\ -1 \\ 0 \end{bmatrix}+3\begin{bmatrix} -1 \\ 0 \\ 1 \\ 2 \end{bmatrix}.$$
\end{solution}

\begin{extract}\newpage\end{extract}
\begin{problem}{V3}
Does
\(
  \operatorname{span}\left\{
    \begin{bmatrix} 2 \\ -1 \\ 4 \end{bmatrix},
    \begin{bmatrix} 3 \\ 12 \\ -9 \end{bmatrix},
    \begin{bmatrix} 1 \\ 4 \\ -3 \end{bmatrix},
    \begin{bmatrix} -4 \\ 2 \\ -8 \end{bmatrix}
  \right\} = \IR^3
\)?
\end{problem}
\begin{solution}
Since
\[
  \RREF\begin{bmatrix}
    2 & 3 & 1 & -4 \\
    -1 & 12 & 4 & 2 \\
    4 & -9 & -3 & -8
  \end{bmatrix} =
  \begin{bmatrix}
    1 & 0 & 0 & -2 \\
    0 & 1 & 1/3 & 0 \\
    0 & 0 & 0 & 0
  \end{bmatrix}
\]
has a zero row, the vectors fail to span \(\IR^3\).
\end{solution}

\begin{problem}{V4}
Determine if $\left\{ \begin{bmatrix} x \\ y \\ 0 \\ z \end{bmatrix}\  \bigg|\ x,y,z \in \IR\right\}$  a subspace of $\IR^4$.
\end{problem}
\begin{solution}
It is closed under addition and scalar multiplication, so it is a subspace.  Alternatively, it is the image of the linear transformation from $\IR^3 \rightarrow \IR^4$ given by $$\begin{bmatrix} x \\ y \\ z \end{bmatrix} \mapsto  \begin{bmatrix} x \\ y \\ 0 \\ z \end{bmatrix}.$$
\end{solution}


\begin{extract}\newpage\end{extract}
\begin{problem}{S1}
Determine if the set of polynomials  $\left\{x^2+x, x^2+2x-1, x^2+3x-2\right\}$ is  linearly dependent or linearly independent
\end{problem}
\begin{solution}
$$\RREF\left( \begin{bmatrix} 1 &  1 & 1 \\ 0  & 2 & 3 \\ 1  & -1 & -2 \end{bmatrix} \right) = \begin{bmatrix} 1 &  0 & -\frac{1}{2} \\ 0  & 1 & \frac{3}{2} \\ 0& 0 & 0  \end{bmatrix}$$
Since there is a nonpivot column, the set is linearly dependent.
\end{solution}


\begin{problem}{S2}
Determine if the set $\left\{ x^3-3x^2+2x+2, -x^3+4x^2-x+1, -x^3+2x+1, 3x^2+3x+9 \right\}$ is a basis of $\P^3$ or not.
\end{problem}

\begin{solution}
$$\RREF \begin{bmatrix} 1 & -1 & -1 & 0 \\ -3 & 4 & 0 & 3 \\ 2 & -1 & 2 & 3 \\ 2 & 1 & 1 & 9 \end{bmatrix}=\begin{bmatrix} 1 &0 & 0 & 3 \\ 0 & 1 & 0 & 3 \\ 0 & 0 & 1 & 0 \\ 0 & 0 & 0 & 0 \end{bmatrix}$$
Since this is not the identity matrix, the set is not a basis.
\end{solution}
\begin{extract}\newpage\end{extract}
\begin{problem}{S3}
Let $W$ be the subspace of $\P_2$ given by $W = {\rm span} \left( \left\{  -3x^2-8x, x^2+2x+2, -x+3\right\} \right)$.   Find a basis for $W$.
\end{problem}
\begin{solution}
Let $A= \begin{bmatrix}-3 & 1 & 0 \\ -8 & 2 & -1 \\ 0 & 2 & 3\end{bmatrix}$, and compute $\RREF(A) = \begin{bmatrix} 1 & 0 & \frac{1}{2} \\ 0 & 1 & \frac{3}{2} \\ 0 & 0 & 0 \end{bmatrix}$.
Since the first two columns are pivot columns, $\left\{ -3x^2-8x, x^2+2x+2\right\} $ is a basis for $W$.
\end{solution}


\begin{problem}{S4}  
Let $W={\rm span}\left( \left\{ \begin{bmatrix} 1 \\ -1 \\ 3 \\ -3 \end{bmatrix},\begin{bmatrix} 2 \\ 0 \\ 1 \\ 1 \end{bmatrix}, \begin{bmatrix} 3 \\ -1 \\ 4 \\ -2 \end{bmatrix},  \begin{bmatrix} 1 \\ 1 \\ 1 \\ -7 \end{bmatrix} \right\}\right)$.  Compute the dimension of $W$.
\end{problem}
\begin{solution}
$$ \RREF \left( \begin{bmatrix} 1 & 2 & 3 & 1 \\ -1 & 0 & -1 & 1 \\ 3 & 1 & 4 & 1 \\ -3 & 1 & -2 & -7 \end{bmatrix} \right) =  \begin{bmatrix} 1 & 0 & 1 & 0 \\ 0 & 1 & 1 & 0 \\ 0 & 0 & 0 & 1 \\ 0 & 0 & 0 & 0\end{bmatrix}$$
This has 3 pivot columns so $\dim(W)=3$.
\end{solution}


\end{document}