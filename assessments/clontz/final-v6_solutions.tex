\documentclass{sbgLAexam}

\begin{extract*}
\usepackage{amsmath,amssymb,amsthm,enumerate}
\coursetitle{Math 237}
\courselabel{Linear Algebra}
\calculatorpolicy{You may use a calculator, but you must show all relevant work to receive credit for a standard.}


\newcommand{\IR}{\mathbb{R}}
\newcommand{\IC}{\mathbb{C}}
\renewcommand{\P}{\mathcal{P}}
\renewcommand{\Im}{{\rm Im\ }}
\DeclareMathOperator{\RREF}{RREF}
\DeclareMathOperator{\vspan}{span}

\makeatletter
\renewcommand*\env@matrix[1][*\c@MaxMatrixCols c]{%
  \hskip -\arraycolsep
  \let\@ifnextchar\new@ifnextchar
  \array{#1}}
\makeatother

\title{Final Exam}
\standard{E1,E2,E3,E4,V1,V2,V3,V4,S1,S2,S3,S4,A1,A2,A3,A4,M1,M2,M3,G1,G2,G3,G4}
\version{6}
\end{extract*}

\begin{document}

\begin{problem}{E1}
Write a system of linear equations corresponding to the following
augmented matrix.
\[
\begin{bmatrix}[cccc|c]
3 & -1 & 0 & 1 & 5 \\
-1 & 9 & 1 & -7 & 0 \\
1 & 0 & -1 & 0 & -3
\end{bmatrix}
\]
\end{problem}
\begin{solution}
\begin{align*}
3x_1-x_2+x_4 &= 5 \\
-x_1+9x_2+x_3-7x_4 &= 0 \\
x_1-x_3 &= -3
\end{align*}
\end{solution}

\begin{problem}{E2}
Find \(\RREF A\), where
\[
  A =
  \begin{bmatrix}[cc|c]
    2 & -7 & 4 \\
    1 & -3 & 2 \\
    3 & 0 & 3
  \end{bmatrix}
\]
\end{problem}
\begin{solution}
\[
  \RREF A =
  \begin{bmatrix}[cc|c]
    1 & 0 & 0 \\
    0 & 1 & 0 \\
    0 & 0 & 1
  \end{bmatrix}
\]
\end{solution}

\begin{extract}\newpage\end{extract}
\begin{problem}{E3}
Solve the system of linear equations.
\begin{align*}
2x+y-z+w &=5 \\
3x-y-2w &= 0 \\
-x+5z+3w&=-1
\end{align*}
\end{problem}
\begin{solution}
$$\RREF\left( \begin{bmatrix}[cccc|c] 2 & 1 & -1 & 0 & 5 \\ 3 & -1 & 0 & -2 & 0 \\ -1 & 0 & 5 & 0 & -1 \end{bmatrix} \right) = \begin{bmatrix}[cccc|c] 1 & 0 & 0 & -\frac{1}{12} & 1 \\ 0 & 1 & 0 & \frac{7}{4} & 3 \\ 0 & 0 & 1 & \frac{7}{12} & 0 \end{bmatrix}$$
So the solutions are $$\left\{ \begin{bmatrix} 1+a \\ 3-21a \\ -7a \\ 12a \end{bmatrix}\ \big|\ a \in \IR\right\}$$
\end{solution}

\begin{problem}{E4}
Find a basis for the solution set to the homogeneous system of equations
given by
\begin{align*}
3x+2y+z &= 0 \\
x+y+z &= 0
\end{align*}
\end{problem}
\begin{solution}
Let \(A =
  \begin{bmatrix}[ccc|c]
    3 & 2 & 1 & 0 \\
    1 & 1 & 1 & 0
  \end{bmatrix}
\), so \(\RREF A =
  \begin{bmatrix}[ccc|c]
    1 & 0 & -1 & 0 \\
    0 & 1 & 2 & 0
  \end{bmatrix}
\).
It follows that the basis for the solution set is given by \(\left\{
  \begin{bmatrix}
    1 \\
    -2 \\
    1
  \end{bmatrix}
\right\}\).
\end{solution}

\begin{extract}\newpage\end{extract}
\begin{problem}{V1}
Let $V$ be the  set of all real numbers together with the operations $\oplus$ and $\odot$ defined by, for any $x,y \in V$ and $c \in \IR$,
\begin{align*}
x\oplus y  &= x+y-3 \\
c \odot x &= cx-3(c-1)
\end{align*}
\begin{enumerate}[(a)]
\item Show that \textbf{scalar multiplication} is
      \textbf{associative}: \(a\odot(b\odot x)=(ab)\odot x\).
\item Determine if $V$ is a vector space or not.  Justify your answer
\end{enumerate}
\end{problem}

\begin{solution}
Let $x,y \in V$, $c,d \in \IR$.
To show associativity:
\begin{align*}
c\odot \left( d \odot x\right) &= c\odot \left( dx-3(d-1) \right) \\
&= c\left(dx-3(d-1)\right)-3(c-1) \\
&= cdx-3(cd-1) \\
&= (cd) \odot x
\end{align*}

We verify the remaining 7 properties to see that $V$ is a vector space.
\begin{enumerate}[1)]
\item Real addition is associative, so $\oplus$ is associative.
\item $x\oplus 3 = x+3-3=x$, so $3$ is the additive identity.
\item $x \oplus (6-x) = x+(6-x)-3=3$, so $6-x$ is the additive inverse of $x$.
\item Real addition is commutative, so $\oplus$ is commutative.
\item Associativity shown above
\item $1 \odot x = x-3(1-1)=x$
\item \begin{align*} c \odot (x \oplus y) &=
c \odot (x+y-3) \\
&= c(x+y-3)-3(c-1) \\
&= cx-3(c-1) + cy-3(c-1) -3 \\
&= (c\odot x ) \oplus (c\odot y)
\end{align*}
\item \begin{align*} (c+d) \odot x &= (c+d)x-3(c+d-1) \\
&= cx-3(c-1)+dx-3(c-1)-3 \\
&= (c\odot x ) \oplus (d \odot x)
\end{align*}
\end{enumerate}

Therefore $V$ is a vector space.
\end{solution}


\begin{problem}{V2}
Determine if $\begin{bmatrix} 0 \\ 1 \\ -2 \\ 1 \end{bmatrix}$ can be written as a linear combination of the vectors $\begin{bmatrix} 5 \\ 2 \\ -3 \\ 2 \end{bmatrix}$, $\begin{bmatrix} 3 \\ 1 \\ 1 \\ 0 \end{bmatrix}$, and $\begin{bmatrix} 8 \\ 3 \\ 5 \\ -1 \end{bmatrix}$.
\end{problem}
\begin{solution}

$$\RREF \left(\begin{bmatrix}[ccc|c] 8 & 5 & 3 & 0\\ 3 & 2 & 1 & 1 \\ 5 & -3 & 1 & -2  \\ -1 & 2 & 0 & 1 \end{bmatrix} \right) = \begin{bmatrix}[ccc|c] 1 & 0 & 0 & 0  \\ 0 &  1 & 0 & 0  \\ 0 & 0 & 1 & 0 \\ 0 & 0 & 0 & 1  \end{bmatrix}$$
The system has no solution, so $\begin{bmatrix} 0 \\ 1 \\ -2 \\ 1 \end{bmatrix}$ is not a linear combination of the three other vectors.
\end{solution}


\begin{extract}\newpage\end{extract}
\begin{problem}{V3}
Determine if the vectors $\begin{bmatrix} 1 \\ 0 \\ 2 \\1 \end{bmatrix}$, $\begin{bmatrix} 3 \\ 1 \\ 0 \\ -3 \end{bmatrix}$,$\begin{bmatrix} 0 \\ 3 \\ 0 \\ -2 \end{bmatrix}$, and $\begin{bmatrix}-1 \\ 1 \\ -1 \\ -1 \end{bmatrix}$ span $\IR^4$.
\end{problem}
\begin{solution}
$$\RREF\left(\begin{bmatrix}1 & 3 & 0 & -1 \\ 0 & 1 & 3 & 1 \\ 2 & 0 & 0 & -1 \\ 1 & -3 & -2 & -1 \end{bmatrix} \right) =\begin{bmatrix} 1 & 0 & 0 & 0 \\ 0 & 1 & 0 & 0 \\ 0 & 0 & 1 & 0 \\ 0 & 0 & 0 & 1 \end{bmatrix}$$
Since every row contains a pivot, the vectors span $\IR^4$.
\end{solution}

\begin{problem}{V4} Let \(W\) be the set of all complex numbers \(a+bi\)
satisfying  \(a=2b\).
Determine if \(W\) is a subspace of \(\IC\).
\end{problem}
\begin{solution}
Yes, because \(c(2b_1+b_1i)+d(2b_2+b_2i)=2(cb_1+db_2)+(cb_1+db_2)i\) belongs to
\(W\). Alternately, yes because \(W\) is isomorphic to \(\IR\).
\end{solution}
\begin{extract}\newpage\end{extract}
\begin{problem}{S1}
Determine if the set of vectors  $\left\{\begin{bmatrix} 1 \\ 0 \\ 1 \end{bmatrix}, \begin{bmatrix} 1 \\ 2 \\ -1 \end{bmatrix}, \begin{bmatrix} 1 \\ 3 \\ -2 \end{bmatrix}\right\}$ is  linearly dependent or linearly independent
\end{problem}
\begin{solution}
$$\RREF\left( \begin{bmatrix} 1 &  1 & 1 \\ 0  & 2 & 3 \\ 1  & -1 & -2 \end{bmatrix} \right) = \begin{bmatrix} 1 &  0 & -\frac{1}{2} \\ 0  & 1 & \frac{3}{2} \\ 0& 0 & 0  \end{bmatrix}$$
Since there is a nonpivot column, the set is linearly dependent.
\end{solution}

\begin{problem}{S2}
Determine if the set $\left\{ \begin{bmatrix} 0 \\ 1 \\ 1 \\ 1 \end{bmatrix}, \begin{bmatrix} 1 \\ -1 \\ 0 \\ 2 \end{bmatrix}, \begin{bmatrix} 1 \\ 0 \\ -1 \\ 0 \end{bmatrix}, \begin{bmatrix}0 \\ 2 \\ 0 \\ -1 \end{bmatrix} \right\}$ is a basis of $\IR^4$.
\end{problem}
\begin{solution}
$$\RREF\left(\begin{bmatrix} 0 & 1 & 1 & 0 \\ 1 & -1 & 0 & 2   \\ 1 & 0 & -1 & 0  \\ 1 & 2 & 0 & -1  \end{bmatrix} \right) = \begin{bmatrix} 1 & 0 & 0 & 1  \\ 0 & 1 & 0 & -1  \\ 0 & 0 & 1 & 1  \\ 0 & 0 & 0 & 0 \end{bmatrix} $$
Since this is not the identity matrix, the set is not a basis.
\end{solution}



\begin{extract}\newpage\end{extract}
\begin{problem}{S3}
Let $W={\rm span}\left(\left\{\begin{bmatrix} 2 \\ 0 \\ -2 \\ 0 \end{bmatrix}, \begin{bmatrix} 3 \\ 1 \\ 3 \\ 6 \end{bmatrix}, \begin{bmatrix} 0 \\ 0 \\ 1 \\ 1 \end{bmatrix}, \begin{bmatrix}1 \\ 2 \\ 0 \\ 1 \end{bmatrix}\right\}\right)$.  Find a basis of $W$.
\end{problem}
\begin{solution}
$$\RREF\left( \begin{bmatrix} 2 & 3 & 0 & 1 \\ 0 & 1 & 0 & 2 \\ -2 & 3 & 1 & 0 \\ 0 & 6 & 1 & 1\end{bmatrix} \right) = \begin{bmatrix}1 & 0 & 0 & -\frac{5}{2} \\ 0 & 1 & 0 & 2 \\ 0 & 0 & 1 & -11\\ 0 & 0 & 0 & 0  \end{bmatrix} $$
Then $\left\{\begin{bmatrix} 2 \\ 0 \\ -2 \\ 0 \end{bmatrix}, \begin{bmatrix} 3 \\ 1 \\ 3 \\ 6 \end{bmatrix}, \begin{bmatrix} 0 \\ 0 \\ 1 \\ 1 \end{bmatrix}\right\}$ is a basis of $W$.
\end{solution}


\begin{problem}{S4}
Let $W$ be the subspace of $\IR^{2\times2}$ given by $W={\rm span}\left(\left\{\begin{bmatrix} 2 & 0 \\ -2 & 0 \end{bmatrix}, \begin{bmatrix} 3 & 1 \\ 3 & 6 \end{bmatrix}, \begin{bmatrix} 0 & 0 \\ 1 & 1 \end{bmatrix}, \begin{bmatrix}1 & 2 \\ 0 & 1 \end{bmatrix}\right\}\right)$. Compute the dimension of $W$.
\end{problem}
\begin{solution}
$$\RREF\left( \begin{bmatrix} 2 & 3 & 0 & 1 \\ 0 & 1 & 0 & 2 \\ -2 & 3 & 1 & 0 \\ 0 & 6 & 1 & 1\end{bmatrix} \right) = \begin{bmatrix}1 & 0 & 0 & -\frac{5}{2} \\ 0 & 1 & 0 & 2 \\ 0 & 0 & 1 & -11\\ 0 & 0 & 0 & 0  \end{bmatrix} $$
This has 3 pivot columns so  $\dim(W) =3$.
\end{solution}


\begin{extract}\newpage\end{extract}
\begin{problem}{A1}
Let $T: \IR^4 \rightarrow \IR^2$ be the linear transformation given by $$T\left(\begin{bmatrix} x_1 \\ x_2 \\ x_3 \\ x_4 \end{bmatrix} \right) = \begin{bmatrix} x_1+3x_3 \\ 3x_2-5x_3 \end{bmatrix}.$$ Write the matrix for $T$ with respect to the standard bases of $\IR^4$ and $\IR^2$.
\end{problem}
\begin{solution}
$$\begin{bmatrix} 1 & 0 & 3 & 0 \\ 0 & 3 & -5 & 0 \end{bmatrix}$$
\end{solution}


\begin{problem}{A2}
Determine if $D: \IR^{2\times 2} \rightarrow \IR$ given by $D\left(\begin{bmatrix} a & b \\ c & d \end{bmatrix} \right) = ad-bc$ is a linear transformation or not.
\end{problem}
\begin{solution}
$D(I)=1$ but $D(2I)=4 \neq 2D(I)$, so $D$ is not linear.
\end{solution}

\begin{extract}\newpage\end{extract}
\begin{problem}{A3}
Determine if each of the following linear transformations is injective (one-to-one) and/or surjective (onto).
\begin{enumerate}[(a)]
\item
  \(S: \IR^2 \rightarrow \IR^3\) where
  \(S(\vec e_1)=\begin{bmatrix}
    2 \\
    1 \\
    0
  \end{bmatrix}\) and
  \(S(\vec e_2)=\begin{bmatrix}
    1 \\
    2 \\
    1
  \end{bmatrix}\).
\item
  \(T: \IR^3 \rightarrow \IR^2\) where
  \(T(\vec e_1)=\begin{bmatrix}
    2 \\
    2
  \end{bmatrix}\),
  \(T(\vec e_2)=\begin{bmatrix}
   1  \\
   0
  \end{bmatrix}\), and
  \(T(\vec e_3)=\begin{bmatrix}
    1 \\
    1
  \end{bmatrix}\).
\end{enumerate}
\end{problem}
\begin{solution}
\begin{enumerate}[(a)]
\item
  \(\RREF\begin{bmatrix}
    2 & 1 \\
    1 & 2 \\
    0 & 1
  \end{bmatrix}=\begin{bmatrix}
    1 & 0 \\
    0 & 1 \\
    0 & 0
  \end{bmatrix}\).
  The map is injective since every column has a pivot, but is not surjective
  because there is a row without a pivot.
\item
  \(\RREF\begin{bmatrix}
    2 & 1 & 1 \\
    2 & 0 & 1
  \end{bmatrix}=\begin{bmatrix}
    1 & 0 & 1/2 \\
    0 & 1 & 1/2
  \end{bmatrix}\).
  The map is not injective since there is a column without a pivot,
  but it is surjective because every row has a pivot.
\end{enumerate}
\end{solution}

\begin{problem}{A4}
Let $T: \IR^3 \rightarrow \IR^3$ be the linear map given by \(
  T\left(\begin{bmatrix} x \\ y \\ z \end{bmatrix} \right) =
  \begin{bmatrix}
    8x-3y-z \\
    y+3z \\
    -7x+3y+2z
  \end{bmatrix}
\). Compute a basis for the kernel and a basis for the image of $T$.
\end{problem}
\begin{solution}
\[
  \RREF \left( \begin{bmatrix}
    8 & -3 & -1 \\
    0 & 1 & 3 \\
    -7 & 3 & 2
  \end{bmatrix} \right) = \begin{bmatrix}
    1 & 0 & 1 \\
    0 & 1 & 3 \\
    0 & 0 & 0
  \end{bmatrix}
\]

Thus \(\left\{
  \begin{bmatrix} 8 \\ 0 \\ -7 \end{bmatrix},
  \begin{bmatrix} -3 \\ 1 \\ 3 \end{bmatrix}
\right\} \) is a basis for the image, and \(\left\{
  \begin{bmatrix} -1 \\ -3 \\ 1 \end{bmatrix}
\right\} \) is a basis for the kernel.
\end{solution}


\begin{extract}\newpage\end{extract}
\begin{problem}{M1}
Let 
\begin{align*}
A &= \begin{bmatrix} 3 \\ 5 \\ -1  \end{bmatrix} & B&=\begin{bmatrix}  2 & 1 & -1 & 2 \\ 1 & -1 & 3 & -3  \end{bmatrix} & C &= \begin{bmatrix} 2 & -1 \\ 0 & 4 \\ 3 & 1 \end{bmatrix} \end{align*}
Exactly one of the six products $AB$, $AC$, $BA$, $BC$, $CA$, $CB$ can be computed.  Determine which one, and compute it.
\end{problem}
\begin{solution}
$CB$ is the only one that can be computed, and
$$CB=\begin{bmatrix} 3 & 3 & -5 & 7 \\ 4 & -4 & 12 & -12 \\ 7 & 2 & 0 & 3 \end{bmatrix}$$
\end{solution}
\begin{problem}{M2}
Determine if the matrix $\begin{bmatrix} 3 & -1 & 0 & 4 \\ 2 & 1 & 1 & -1 \\ 0 & 1 & 1 & 3 \\ 1 & -2 & 0 & 0 \end{bmatrix}$ is invertible.
\end{problem}
\begin{solution}
This matrix is row equivalent to the identity matrix, so it is invertible.
\end{solution}

\begin{extract}\newpage\end{extract}
\begin{problem}{M3}
Find the inverse of the matrix $\begin{bmatrix} 8 & 5 & 3 & 0 \\ 3 & 2 & 1 & 1 \\ 5 & -3 & 1 & -2 \\ -1 & 2 & 0 & 1\end{bmatrix} $.
\end{problem}
\begin{solution}
$$\RREF \left(\begin{bmatrix}[cccc|cccc] 8 & 5 & 3 & 0 & 1 & 0 & 0 & 0\\ 3 & 2 & 1 & 1 & 0 & 1 & 0 & 0 \\ 5 & -3 & 1 & -2 & 0 & 0 & 1 & 0 \\ -1 & 2 & 0 & 1 & 0 & 0 & 0 & 1 \end{bmatrix} \right) = \begin{bmatrix}[cccc|cccc] 1 & 0 & 0 & 0 & 1 & 2 & -5 & 12 \\ 0 &  1 & 0 & 0 & 1 & 1 & -4 & -9 \\ 0 & 0 & 1 & 0 & -4 & -7 & 20 & 47 \\ 0 & 0 & 0 & 1 & -1 & 0 & 3 & 7 \end{bmatrix}$$

So the inverse is $\begin{bmatrix} 1 & 2 & -5 & 12 \\  1 & 1 & -4 & -9 \\  -4 & -7 & 20 & 47 \\-1 & 0 & 3 & 7 \end{bmatrix}$.
\end{solution}


\begin{problem}{G1}
Compute the determinant of the matrix
\[
  \begin{bmatrix}
    1 & 3 & 2 & 4 \\
    5 & 0 & -4 & 0 \\
    -2 & 3 & -1 & 1 \\
    0 & 1 & 0 & 1
  \end{bmatrix}
.\]
\end{problem}
\begin{solution}
\(15\).
\end{solution}

\begin{extract}\newpage\end{extract}
\begin{problem}{G2} 
Compute the eigenvalues, along with their algebraic multiplicities, of the matrix $ \begin{bmatrix} 8 & -3 & -1 \\ 21 & -8 & -3 \\ -7 & 3 & 2\end{bmatrix}$.
\end{problem}
\begin{solution}
\begin{align*}
\det(A-\lambda I) &= (8-\lambda) \det \begin{bmatrix} -8-\lambda & -3 \\ 3 & 2-\lambda \end{bmatrix}-(-3) \det \begin{bmatrix} 21 & -3 \\ -7 & 2-\lambda \end{bmatrix} +(-1) \det \begin{bmatrix} 21 & -8-\lambda \\ -7 & 3 \end{bmatrix} \\
&=(8-\lambda)\left(\lambda ^2+6\lambda-7\right)+3\left(-21\lambda + 21 \right)-\left(-7\lambda +7 \right) \\
&=(\lambda -1) \left( (8-\lambda)(\lambda+7)-63+7 \right) \\
&= (\lambda-1)(\lambda -\lambda ^2) \\
&= -\lambda (\lambda-1)^2
\end{align*}
So the eigenvalues are $0$ (with algebraic multiplicity 1) and $1$ (with algebraic multiplicity 2).
\end{solution}


\begin{problem}{G3}
Find the eigenspace associated to the eigenvalue $1$ in the matrix $A=\begin{bmatrix}9 & -3 & 2 \\ 19 & -6 & 5 \\ -11 & 4 & -2 \end{bmatrix}$
\end{problem}
\begin{solution}
The eigenspace is spanned by $\begin{bmatrix} -1 \\ -2 \\ 1 \end{bmatrix}$.
\end{solution}

\begin{extract}\newpage\end{extract}
\begin{problem}{G4}
Compute the geometric multiplicity of the eigenvalue $-1$ in the matrix $\begin{bmatrix} 4 & -2 & -1 \\ 15 & -7 & -3 \\ -5 & 2 & 0 \end{bmatrix}$.  \end{problem}
\begin{solution}
$$\RREF\left(A+I\right) = \begin{bmatrix} 1 & - \frac{2}{5} & -\frac{1}{5} \\ 0 & 0 & 0 \\ 0 & 0 & 0 \end{bmatrix}$$
So the geometric multiplicity is $2$.
\end{solution}


\end{document}