\documentclass{sbgLAquiz}

\begin{extract*}
\usepackage{amsmath,amssymb,amsthm,enumerate}
\coursetitle{Math 237}
\courselabel{Linear Algebra}
\calculatorpolicy{You may use a calculator, but you must show all relevant work to receive credit for a standard.}


\newcommand{\IR}{\mathbb{R}}
\newcommand{\IC}{\mathbb{C}}
\renewcommand{\P}{\mathcal{P}}
\renewcommand{\Im}{{\rm Im\ }}
\DeclareMathOperator{\RREF}{RREF}
\DeclareMathOperator{\vspan}{span}

\makeatletter
\renewcommand*\env@matrix[1][*\c@MaxMatrixCols c]{%
  \hskip -\arraycolsep
  \let\@ifnextchar\new@ifnextchar
  \array{#1}}
\makeatother

\title{Mastery Quiz Day 14 }
\standard{V1,V3,V4,S2}
\version{5}
\end{extract*}

\begin{document}

\begin{problem}{V1}
Let $V$ be the set of all real numbers with the operations, for any $x, y \in V$, $c \in \IR$,
\begin{align*}
x \oplus y &= \sqrt{x^2+y^2} \\
c \odot x &= c x
\end{align*}
\begin{enumerate}[(a)]
\item Show that the vector \textbf{addition} $\oplus$ is \textbf{associative}:
      \(x \oplus (y \oplus z)=(x\oplus y)\oplus z\).
\item Determine if $V$ is a vector space or not.  Justify your answer.
\end{enumerate}
\end{problem}
\begin{solution}
Let $x,y,z \in \IR$.  Then
\begin{align*}
(x\oplus y) \oplus z &= \sqrt{x^2+y^2} \oplus z \\&= \sqrt{ (\sqrt{x^2+y^2})^2+z^2} \\&= \sqrt{x^2+y^2+z^2} \\
&= \sqrt{x^2+(\sqrt{y^2+z^2})^2} \\ &= x \oplus \sqrt{y^2+z^2} \\ &= x \oplus (y \oplus z)
\end{align*}
However, this is not a vector space, as there is no zero vector.
\end{solution}
\begin{problem}{V3}
Does
\(
  \operatorname{span}\left\{
    \begin{bmatrix} 2 \\ -1 \\ 4 \\ 2 \\ 1 \end{bmatrix},
    \begin{bmatrix} -1 \\ 3 \\ 5 \\ 2 \\ 0 \end{bmatrix},
    \begin{bmatrix} 1 \\ 0 \\ 5 \\ 1 \\ -3 \end{bmatrix}
  \right\} = \IR^5
\)?
\end{problem}
\begin{solution}
Since there are only three vectors, they cannot span \(\IR^5\).
\end{solution}
\begin{extract}\newpage\end{extract}
\begin{problem}{V4} Let $W$ be the set of all polynomials of the form
\(ax^3+bx\).  Determine if $W$ is a subspace of \(\P^3\).
\end{problem}
\begin{solution}
Yes because \(s(a_1x^3+b_1x)+t(a_2x^3+b_2x)=
(sa_1+ta_2)x^3+(sb_1+tb_2)x\) also belongs to \(W\).
Alternately, yes because \(W\) is isomorphic to \(\IR^2\).
\end{solution}


\begin{problem}{S2}
Determine if the set $\left\{ x^3-x, x^2+x+1, x^3-x^2+2, 2x^2-1 \right\}$ is a basis of $\P_3$
\end{problem}
\begin{solution}
$$\RREF\left(\begin{bmatrix} 1 & 0 & 1 & 0 \\ 0 & 1 & -1 & 2 \\ -1 & 1 & 0 & 0 \\ 0 & 1 & 2 & -1 \end{bmatrix} \right)= \begin{bmatrix} 1 & 0 &0 & 1 \\ 0 & 1 & 0 & 1 \\ 0 & 0 & 1 & -1 \\ 0 & 0 & 0 & 0 \end{bmatrix}$$
Since the resulting matrix is not the identity matrix, it is not a basis.
\end{solution}


\end{document}