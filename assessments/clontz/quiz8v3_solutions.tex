\documentclass{sbgLAquiz}

\begin{extract*}
\usepackage{amsmath,amssymb,amsthm,enumerate}
\coursetitle{Math 237}
\courselabel{Linear Algebra}
\calculatorpolicy{You may use a calculator, but you must show all relevant work to receive credit for a standard.}


\newcommand{\IR}{\mathbb{R}}
\newcommand{\IC}{\mathbb{C}}
\renewcommand{\P}{\mathcal{P}}
\renewcommand{\Im}{{\rm Im\ }}
\DeclareMathOperator{\RREF}{RREF}
\DeclareMathOperator{\vspan}{span}

\makeatletter
\renewcommand*\env@matrix[1][*\c@MaxMatrixCols c]{%
  \hskip -\arraycolsep
  \let\@ifnextchar\new@ifnextchar
  \array{#1}}
\makeatother

\title{Mastery Quiz Day 8 }
\standard{E1,E3,E4,V1,E2}
\version{3}
\end{extract*}

\begin{document}

\begin{problem}{E1}
Write a system of linear equations corresponding to the following
augmented matrix.
\[
\begin{bmatrix}[cccc|c]
3 & -1 & 0 & 1 & 5 \\
-1 & 9 & 1 & -7 & 0 \\
1 & 0 & -1 & 0 & -3
\end{bmatrix}
\]
\end{problem}
\begin{solution}
\begin{align*}
3x_1-x_2+x_4 &= 5 \\
-x_1+9x_2+x_3-7x_4 &= 0 \\
x_1-x_3 &= -3
\end{align*}
\end{solution}

\begin{problem}{E3}
Find the solution set for the following system of linear equations.
\begin{align*}
2x_1+3x_2-5x_3+14x_4 &= 8 \\
x_1+x_2-x_3+5x_4&= 3
\end{align*}
\end{problem}
\begin{solution}
Let \(A =
  \begin{bmatrix}[cccc|c]
    2 & 3 & -5 & 14 & 8 \\
    1 & 1 & -1 & 5 & 3
  \end{bmatrix}
\), so \(\RREF A =
  \begin{bmatrix}[cccc|c]
    1 & 0 & 2 & 1 & 1 \\
    0 & 1 & -3 & 4 & 2 \\
  \end{bmatrix}
\). It follows that the solution set is given by \(
  \begin{bmatrix}
    2 - 2a - b \\
    2 + 3a - 4b \\
    a \\
    b
  \end{bmatrix}
\) for all real numbers \(a,b\).
\end{solution}

\begin{extract}\newpage\end{extract}
\begin{problem}{E4}
Find a basis for the solution set to the homogeneous system of equations
\begin{align*}
2x_1+3x_2-5x_3+14x_4 &= 0\\
x_1+x_2-x_3+5x_4 &= 0
\end{align*}
\end{problem}
\begin{solution}
Let \(A =
  \begin{bmatrix}[cccc|c]
    2 & 3 & -5 & 14 & 0 \\
    1 & 1 & -1 & 5 & 0
  \end{bmatrix}
\), so \(\RREF A =
  \begin{bmatrix}[cccc|c]
    1 & 0 & 2 & 1 & 1 \\
    0 & 1 & -3 & 4 & 2 \\
  \end{bmatrix}
\).
It follows that the basis for the solution set is given by \(\left\{
  \begin{bmatrix}
    - 2 \\
    3 \\
    1 \\
    0
  \end{bmatrix},
  \begin{bmatrix}
    -1 \\
    - 4 \\
    0 \\
    1
  \end{bmatrix}
\right\}\).
\end{solution}

\begin{problem}{V1}
Let $V$ be the  set of all real numbers together with the operations $\oplus$ and $\odot$ defined by, for any $x,y \in V$ and $c \in \IR$,
\begin{align*}
x\oplus y  &= x+y-3 \\
c \odot x &= cx-3(c-1)
\end{align*}
\begin{enumerate}[(a)]
\item Show that this scalar multiplication $\odot$ is associative.
\item Determine if $V$ is a vector space or not.  Justify your answer
\end{enumerate}
\end{problem}

\begin{solution}
Let $x,y \in V$, $c,d \in \IR$.
To show associativity:
\begin{align*}
c\odot \left( d \odot x\right) &= c\odot \left( dx-3(d-1) \right) \\
&= c\left(dx-3(d-1)\right)-3(c-1) \\
&= cdx-3(cd-1) \\
&= (cd) \odot x
\end{align*}

We verify the remaining 7 properties to see that $V$ is a vector space.
\begin{enumerate}[1)]
\item Real addition is associative, so $\oplus$ is associative.
\item $x\oplus 3 = x+3-3=x$, so $3$ is the additive identity.
\item $x \oplus (6-x) = x+(6-x)-3=3$, so $6-x$ is the additive inverse of $x$.
\item Real addition is commutative, so $\oplus$ is commutative.
\item Associativity shown above
\item $1 \odot x = x-3(1-1)=x$
\item \begin{align*} c \odot (x \oplus y) &=
c \odot (x+y-3) \\
&= c(x+y-3)-3(c-1) \\
&= cx-3(c-1) + cy-3(c-1) -3 \\
&= (c\odot x ) \oplus (c\odot y)
\end{align*}
\item \begin{align*} (c+d) \odot x &= (c+d)x-3(c+d-1) \\
&= cx-3(c-1)+dx-3(c-1)-3 \\
&= (c\odot x ) \oplus (d \odot x)
\end{align*}
\end{enumerate}

Therefore $V$ is a vector space.
\end{solution}

\end{document}