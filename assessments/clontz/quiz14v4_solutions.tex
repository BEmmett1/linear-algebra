\documentclass{sbgLAquiz}

\begin{extract*}
\usepackage{amsmath,amssymb,amsthm,enumerate}
\coursetitle{Math 237}
\courselabel{Linear Algebra}
\calculatorpolicy{You may use a calculator, but you must show all relevant work to receive credit for a standard.}


\newcommand{\IR}{\mathbb{R}}
\newcommand{\IC}{\mathbb{C}}
\renewcommand{\P}{\mathcal{P}}
\renewcommand{\Im}{{\rm Im\ }}
\DeclareMathOperator{\RREF}{RREF}
\DeclareMathOperator{\vspan}{span}

\makeatletter
\renewcommand*\env@matrix[1][*\c@MaxMatrixCols c]{%
  \hskip -\arraycolsep
  \let\@ifnextchar\new@ifnextchar
  \array{#1}}
\makeatother

\title{Mastery Quiz Day 14 }
\standard{V1,V3,V4,S2}
\version{4}
\end{extract*}

\begin{document}

\begin{problem}{V1}
Let $V$ be the set of all points on the line $x+y=2$ with the operations, for any $(x_1,y_1), (x_2,y_2) \in V$, $c \in \IR$,
\begin{align*}
(x_1,y_1) \oplus (x_2,y_2) &= (x_1+x_2-1,y_1+y_2-1) \\
c \odot (x_1,y_1) &= (cx_1-(c-1), cy_1-(c-1))
\end{align*}
\begin{enumerate}[(a)]
\item Show that this vector space has an \textbf{additive identity} element
      \(\mathbf{0}\) satisfying \((x,y)\oplus\mathbf{0}=(x,y)\).
\item Determine if $V$ is a vector space or not.  Justify your answer.
\end{enumerate}
\end{problem}
\begin{solution}
Let $(x_1,y_1) \in V$; then $(x_1,y_1) \oplus (1,1) = (x_1,y_1)$, so $(1,1)$ is an additive identity element.

Now we will show the other seven properties.  Let $(x_1,y_1), (x_2,y_2) \in V$, and let $c,d \in \IR$.
\begin{enumerate}[1)]
\item Since real addition is associative, $\oplus$ is associative.
\item Since real addition is commutative, $\oplus$ is commutative.
\item The additive identity is $(1,1)$.
\item $(x_1,y_1) \oplus (2-x_1,2-y_1) = (1,1)$, so $(2-x_1,2-y_1)$ is the additive inverse of $(x_1,y_1)$.
\item \begin{align*} c\odot \left(d \odot (x_1,y_1) \right) &=c\odot \left( dx_1-(d-1),dy_1-(d-1)\right) \\
&= \left( c\left(dx_1-(d-1) \right)-(c-1), c\left(dy_1-(d-1) \right) \right) \\
&= \left(cdx_1-cd+c-(c-1), cdy_1-cd+c-(c-1) \right) \\
&= \left(cdx_1-(cd-1), cdy_1-(cd-1) \right) \\
&= (cd) \odot (x_1,y_1)
\end{align*}
\item $1 \odot (x_1,y_1) = (x_1-(1-1),y_1-(1-1)=(x_1,y_1)$
\item \begin{align*} c \odot \left( (x_1,y_1)\oplus(x_2,y_2) \right) &=
c\odot \left( x_1+y_1-1,x_2+y_2-1 \right) \\
&= \left( c(x_1+y_1-1)-(c-1), c(x_2+y_2-1)-(c-1) \right) \\
&= (cx_1+cx_2-2c+1, cy_1+cy_2-2c+1) \\
&= \left(cx_1-(c-1),cy_1-(c-1) \right) \oplus (cx_2-(c-1),cy_2-(c-1)) \\
&=c \odot (x_1,y_1) \oplus c\odot (x_2,y_2)
\end{align*}
\item \begin{align*} (c+d) \odot (x_1,y_1) &=
\left( (c+d)x_1-(c+d-1), (c+d)y_1-(c+d-1) \right) \\
&= \left( cx_1-(c-1), cy_1-(c-1) \right) \oplus (dx_1-(d-1), dy_1-(d-1) ) \\
&= c\odot (x_1,y_1) \oplus c \odot (x_2,y_2)
\end{align*}
\end{enumerate}
Therefore $V$ is a vector space.
\end{solution}

\begin{problem}{V3}
Determine if the vectors  $\begin{bmatrix} 8 \\ 21 \\ -7 \end{bmatrix}$, $\begin{bmatrix} -3 \\ -8 \\ 3 \end{bmatrix}$, $\begin{bmatrix} -1 \\ -3 \\ 2 \end{bmatrix}$, and $\begin{bmatrix} 4 \\ 11 \\ -5 \end{bmatrix}$ span $\IR^3$.
\end{problem}
\begin{solution}
$$\RREF\left(\begin{bmatrix} 8 & -3 & -1 & 4 \\ 21 & -8 & -3 & 11 \\ -7 & 3 & 2 & -5  \end{bmatrix} \right) = \begin{bmatrix} 1 & 0 & 1 & -1 \\ 0 & 1 & 3 & -4 \\ 0 & 0 & 0 & 0\end{bmatrix} $$
Since the rank is less than 3, they do not span $\IR^3$.
\end{solution}

\begin{extract}\newpage\end{extract}
\begin{problem}{V4} Let \(W\) be the set of all complex numbers
that are purely real (i.e of the form $a+0i$)  or purely imaginary (i.e. of the form $0+bi$).
Determine if \(W\) is a subspace of \(\IC\).
\end{problem}
\begin{solution}
No, because \(1\) is purely real and \(i\) is purely imaginary, but
the linear combination \(1+i\) is neither.
\end{solution}


\begin{problem}{S2}
Determine if the set $\left\{ \begin{bmatrix} 0 \\ 1 \\ 1 \\ 1 \end{bmatrix}, \begin{bmatrix} 1 \\ -1 \\ 0 \\ 2 \end{bmatrix}, \begin{bmatrix} 1 \\ 0 \\ -1 \\ 0 \end{bmatrix}, \begin{bmatrix}0 \\ 2 \\ 0 \\ -1 \end{bmatrix} \right\}$ is a basis of $\IR^4$.
\end{problem}
\begin{solution}
$$\RREF\left(\begin{bmatrix} 0 & 1 & 1 & 0 \\ 1 & -1 & 0 & 2   \\ 1 & 0 & -1 & 0  \\ 1 & 2 & 0 & -1  \end{bmatrix} \right) = \begin{bmatrix} 1 & 0 & 0 & 1  \\ 0 & 1 & 0 & -1  \\ 0 & 0 & 1 & 1  \\ 0 & 0 & 0 & 0 \end{bmatrix} $$
Since this is not the identity matrix, the set is not a basis.
\end{solution}



\end{document}