\documentclass{sbgLAsemi}

\begin{extract*}
\usepackage{amsmath,amssymb,amsthm,enumerate}
\coursetitle{Math 237}
\courselabel{Linear Algebra}
\calculatorpolicy{You may use a calculator, but you must show all relevant work to receive credit for a standard.}


\newcommand{\IR}{\mathbb{R}}
\newcommand{\IC}{\mathbb{C}}
\renewcommand{\P}{\mathcal{P}}
\renewcommand{\Im}{{\rm Im\ }}
\DeclareMathOperator{\RREF}{RREF}
\DeclareMathOperator{\vspan}{span}

\makeatletter
\renewcommand*\env@matrix[1][*\c@MaxMatrixCols c]{%
  \hskip -\arraycolsep
  \let\@ifnextchar\new@ifnextchar
  \array{#1}}
\makeatother

\title{Semifinal}
\version{3}
\end{extract*}

\begin{document}

\begin{problem}{E1}
Write a system of linear equations corresponding to the following
augmented matrix.
\[
\begin{bmatrix}[ccc|c]
1 & 0 & 4 & 1 \\
0 & 1 & -1 & 7 \\
1 & -1 & 3 & -1
\end{bmatrix}
\]
\end{problem}
\begin{solution}
\begin{align*}
x_1+4x_3 &= 1 \\
x_2-x_3 &= 7 \\
x_1-x_2+3x_3 &= -1
\end{align*}
\end{solution}
\begin{problem}{E2}
Find \(\RREF A\), where
\[
  A =
  \begin{bmatrix}[cccc|c]
    2 & 2 & 1 & 2 & -1 \\
    1 & 1 & 2 & 4 & 5 \\
    3 & 3 & -1 & -2 & 1
  \end{bmatrix}
\]
\end{problem}
\begin{solution}
\[
  \RREF A =
  \begin{bmatrix}[cccc|c]
    1 & 1 & 0 & 0 & 0 \\
    0 & 0 & 1 & 2 & 0 \\
    0 & 0 & 0 & 0 & 1
  \end{bmatrix}
\]
\end{solution}
\begin{problem}{E3}
Find the solution set for the following system of linear equations.
\begin{align*}
2x_1-2x_2+6x_3-x_4 &=-1 \\
3x_1+6x_3+x_4 &= 5 \\
-4x_1+x_2-9x_3+2x_4&=-7
\end{align*}
\end{problem}
\begin{solution}
Let \(A =
  \begin{bmatrix}[cccc|c]
    2 & -2 & 6 & -1 & -1 \\
    3 & 0 & 6 & 1 & 5 \\
    -4 & 1 & -9 & 2 & -7
  \end{bmatrix}
\), so \(\RREF A =
  \begin{bmatrix}[cccc|c]
    1 & 0 & 2 & 0 & 2 \\
    0 & 1 & -1 & 0 & 3 \\
    0 & 0 & 0 & 1 & -1
  \end{bmatrix}
\). It follows that the solution set is given by \(
  \begin{bmatrix}
    2 - 2a \\
    3 + a \\
    a \\
    -1
  \end{bmatrix}
\) for all real numbers \(a\).
\end{solution}

\begin{problem}{E4}
Find a basis for the solution set to the homogeneous system of equations
\begin{align*}
4x_1+4x_2+3x_3-6x_4 &= 0 \\
-2x_3-4x_4 &= 0 \\
2x_1+2x_2+x_3-4x_4 &= 0 \\
\end{align*}
\end{problem}
\begin{solution}
Let \(A =
  \begin{bmatrix}[cccc|c]
    4 & 4 & 3 & -6 & 0 \\
    0 & 0 & -2 & -4 & 0 \\
    2 & 2 & 1 & -4 & 0
  \end{bmatrix}
\), so \(\RREF A =
  \begin{bmatrix}[cccc|c]
    1 & 1 & 0 & -3 & 0 \\
    0 & 0 & 1 & 2 & 0 \\
    0 & 0 & 0 & 0 & 0
  \end{bmatrix}
\).
It follows that the basis for the solution set is given by \(\left\{
  \begin{bmatrix}
    -1 \\
    1 \\
    0 \\
    0
  \end{bmatrix},
  \begin{bmatrix}
    3 \\
    0 \\
    -2 \\
    1
  \end{bmatrix}
\right\}\).
\end{solution}

\begin{problem}{V1}
Let $V$ be the set of all points on the parabola $y=x^2$ with the operations, for any $(x_1,y_1), (x_2,y_2) \in V$, $c \in \IR$,
\begin{align*}
(x_1,y_1) \oplus (x_2,y_2) &= (x_1-x_2,y_1+y_2-2x_1x_2) \\
c \odot (x_1,y_1) &= (cx_1, c^2y_1)
\end{align*}
\begin{enumerate}[(a)]
\item Show that scalar multiplication \textbf{distributes scalars} over
      vector addition:
      \(c\odot((x_1,y_1) \oplus (x_2,y_2))=
      c\odot(x_1,y_1) \oplus c\odot(x_2,y_2)\).
\item Determine if $V$ is a vector space or not.  Justify your answer.
\end{enumerate}
\end{problem}
\begin{solution}
$$ c\odot((x_1,y_1) \oplus (x_2,y_2))= c \odot (x_1-x_2,y_1+y_2-2x_1x_2) = (c(x_1-x_2), c^2(y_1+y_2-2x_1x_2) ) $$
$$ c\odot(x_1,y_1) \oplus c\odot(x_2,y_2) = (cx_1,c^2y_1) \oplus (cx_2,c^2y_2) = (cx_1-cx_2, c^2y_1+c^2y_2-2(cx_1)(cx_2) )$$

Not a vector space as addition is not commutative.
\end{solution}


\begin{problem}{V2}
  Determine if
  \(\begin{bmatrix} 3 \\ -2 \\ 4 \end{bmatrix}\)
  belongs to the span of the set
  \(\left\{
    \begin{bmatrix} 1 \\ 2 \\ -3 \end{bmatrix},
    \begin{bmatrix} 2 \\ 4 \\ -6 \end{bmatrix},
    \begin{bmatrix} 0 \\ 0 \\ 0 \end{bmatrix}
    \right\}
  \).
\end{problem}
\begin{solution}
  Since
  \[
    \RREF\left(
      \begin{bmatrix}[ccc|c]
        1 & 2 & 0 & 3 \\
        2 & 4 & 0 & -2 \\
        -3 & -6 & 0 & 4
      \end{bmatrix}
    \right) =
    \begin{bmatrix}[ccc|c]
      1 & 2 & 0 & 0 \\
      0 & 0 & 0 & 1 \\
      0 & 0 & 0 & 0
    \end{bmatrix}
  \]
  contains the contradiction \(0=1\),
  \(\begin{bmatrix} 3 \\ -2 \\ 4 \end{bmatrix}\) is
  not a linear combination of the three vectors.
\end{solution}
\begin{problem}{V3}
Determine if the vectors $\begin{bmatrix} 2 \\ 0 \\ -2 \\ 0 \end{bmatrix}$, $\begin{bmatrix} 3 \\ 1 \\ 3 \\ 6 \end{bmatrix}$, $\begin{bmatrix} 0 \\ 0 \\ 1 \\ 1 \end{bmatrix}$, and $\begin{bmatrix}1 \\ 2 \\ 0 \\ 1 \end{bmatrix}$ span $\IR^4$.
\end{problem}
\begin{solution}
$$\RREF\left( \begin{bmatrix} 2 & 3 & 0 & 1 \\ 0 & 1 & 0 & 2 \\ -2 & 3 & 1 & 0 \\ 0 & 6 & 1 & 1\end{bmatrix} \right) = \begin{bmatrix}1 & 0 & 0 & -\frac{5}{2} \\ 0 & 1 & 0 & 2 \\ 0 & 0 & 1 & -11\\ 0 & 0 & 0 & 0  \end{bmatrix} $$
Since there is a zero row, the vectors do not span $\IR^4$.
\end{solution}

\begin{problem}{V4} Let \(W\) be the set of all \(\IR^3\) vectors
\(\begin{bmatrix} x \\ y \\ z \end{bmatrix}\) satisfying \(x+y+z=0\) (this forms a plane).
Determine if \(W\) is a subspace of \(\IR^3\).
\end{problem}
\begin{solution}
Yes, because \(z=-x-y\) and
\(
  a\begin{bmatrix} x_1 \\ y_1 \\ -x_1-y_1 \end{bmatrix}+
  b\begin{bmatrix} x_2 \\ y_2 \\ -x_2-y_2 \end{bmatrix}=
  \begin{bmatrix}
    ax_1+bx_2 \\
    ay_1+by_2 \\
    -(ax_1+bx_2)-(ay_1+by_2)
  \end{bmatrix}
\).
Alternately, yes because \(W\) is isomorphic to \(\IR^2\).
\end{solution}


\begin{problem}{S1}
Determine if the set of vectors  $\left\{\begin{bmatrix} 1 \\ 0 \\ 1 \end{bmatrix}, \begin{bmatrix} 1 \\ 2 \\ -1 \end{bmatrix}, \begin{bmatrix} 1 \\ 3 \\ -2 \end{bmatrix}\right\}$ is  linearly dependent or linearly independent
\end{problem}
\begin{solution}
$$\RREF\left( \begin{bmatrix} 1 &  1 & 1 \\ 0  & 2 & 3 \\ 1  & -1 & -2 \end{bmatrix} \right) = \begin{bmatrix} 1 &  0 & -\frac{1}{2} \\ 0  & 1 & \frac{3}{2} \\ 0& 0 & 0  \end{bmatrix}$$
Since there is a nonpivot column, the set is linearly dependent.
\end{solution}

\begin{problem}{S2}
  Determine if the set $\left\{ 2x^2-x+3, 2x^2+2, -x^2+4x+1 \right\}$
  is a basis of $\P^2$.
\end{problem}
\begin{solution}
  \[\RREF\left(
    \begin{bmatrix}
      2 & 2 & -1 \\
      -1 & 0 & 4 \\
      3 & 2 & 1
    \end{bmatrix} \right)= \begin{bmatrix}
      1 & 0 &0 \\
      0 & 1 & 0 \\
      0 & 0 & 1
    \end{bmatrix}
  \]
Since the resulting matrix is the identity matrix, it is a basis.
\end{solution}


\begin{problem}{S3}
Let \(
  W={\rm span}\left\{
    \begin{bmatrix} 2 \\ 0 \\ 2 \\ 1 \end{bmatrix},
    \begin{bmatrix} 3 \\ 1 \\ -1 \\ 1 \end{bmatrix},
    \begin{bmatrix} 0 \\ 2 \\ -8 \\ -1 \end{bmatrix}
  \right\}
\). Find a basis for this vector space.
\end{problem}
\begin{solution}
\[
  \RREF\left(\begin{bmatrix}
    2 & 3 & 0 \\
    0 & 1 & 2 \\
    2 & -1 & -8 \\
    1 & 1 & -1
  \end{bmatrix} \right) =
  \begin{bmatrix}
    1 & 0 & -3 \\
    0 & 1 & 2 \\
    0 & 0 & 0 \\
    0 & 0 & 0
  \end{bmatrix}
\]
Thus \(\left\{
  \begin{bmatrix} 2 \\ 0 \\ 2 \\ 1 \end{bmatrix},
  \begin{bmatrix} 3 \\ 1 \\ -1 \\ 1 \end{bmatrix}
\right\}\) is a basis of $W$.
\end{solution}
\begin{problem}{S4}
Let $W = {\rm span} \left( \left\{ \begin{bmatrix} 1 \\ 1 \\ 2 \\ 1 \end{bmatrix}, \begin{bmatrix} 3 \\ 3 \\ 6 \\ 3 \end{bmatrix}, \begin{bmatrix} 3 \\ -1 \\ 3 \\ -2 \end{bmatrix}, \begin{bmatrix} 7 \\ -1 \\ 8 \\ -3 \end{bmatrix} \right\} \right)$.  Find the dimension of $W$.
\end{problem}
\begin{solution}
$$\RREF\left(\begin{bmatrix} 1 & 3 & 3 & 7 \\ 1 & 3 & -1 & -1 \\ 2 & 6 & 3 & 8 \\ 1 & 3 & -2 & -3 \end{bmatrix}\right) = \begin{bmatrix} 1 & 3 & 0 & 1 \\ 0 & 0 & 1 & 2 \\ 0 & 0 & 0 & 0 \\  0 & 0 & 0 & 0 \end{bmatrix}$$

This has two pivot columns, so $W$ has dimension 2.
\end{solution}


\begin{problem}{A1}
Let $T: \IR^4 \rightarrow \IR^2$ be the linear transformation given by $$T\left(\begin{bmatrix} x_1 \\ x_2 \\ x_3 \\ x_4 \end{bmatrix} \right) = \begin{bmatrix} x_1+3x_3 \\ 3x_2-5x_3 \end{bmatrix}.$$ Write the matrix for $T$ with respect to the standard bases of $\IR^4$ and $\IR^2$.
\end{problem}
\begin{solution}
$$\begin{bmatrix} 1 & 0 & 3 & 0 \\ 0 & 3 & -5 & 0 \end{bmatrix}$$
\end{solution}


\begin{problem}{A2}
Determine if the map $T: \P^6  \rightarrow \P^6$ given by $T(f) = f(x)-f(0)$ is a linear transformation or not.
\end{problem}

\begin{problem}{A3}
Determine if the following linear maps are injective (one-to-one) and/or surjective (onto).
\begin{enumerate}[(a)]
\item $S: \IR^2 \rightarrow \IR^3$ given by $S\left(\begin{bmatrix} x \\ y  \end{bmatrix} \right) = \begin{bmatrix} 3x+2y \\ x-y \\ x+4y \end{bmatrix} $
\item $T: \IR^3 \rightarrow \IR^3$ given by $T\left(\begin{bmatrix} x \\ y \\ z  \end{bmatrix} \right) = \begin{bmatrix} x+y+z \\ 2y+3z \\ x-y-2z \end{bmatrix} $
\end{enumerate}
\end{problem}

\begin{solution}
\begin{enumerate}[(a)]
\item $$\RREF\left( \begin{bmatrix} 1 &  1 & 1 \\ 0  & 2 & 3 \\ 1  & -1 & -2 \end{bmatrix} \right) = \begin{bmatrix} 1 &  0 & -\frac{1}{2} \\ 0  & 1 & \frac{3}{2} \\ 0& 0 & 0  \end{bmatrix}$$
Since there is a nonpivot column, $T$ is not injective.  Since there is a zero row, $T$ is not surjective.
\item $$\RREF \left( \begin{bmatrix} 3 & 2 \\ 1 & -1 \\ 1 & 4 \end{bmatrix} \right) = \begin{bmatrix} 1 & 0 \\ 0 & 1 \\ 0 & 0 \end{bmatrix}$$
Since all columns are pivot columns, $S$ is injective.  Since there is a zero row, $S$ is not surjective.
\end{enumerate}
\end{solution}



\begin{problem}{A4}
Let $T: \IR^{2\times 2} \rightarrow \IR^3$ be the linear map given by $T\left(\begin{bmatrix} x & y \\ z & w \end{bmatrix} \right) = \begin{bmatrix}  8x-3y-z+4w \\ y+3z-4w \\ -7x+3y+2z-5w\end{bmatrix} $.
Compute a basis for the kernel and a basis for the image of $T$.
\end{problem}
\begin{solution}
$$\RREF \left( \begin{bmatrix} 8 & -3 & -1 & 4 \\ 0 & 1 & 3 & -4 \\ -7 & 3 & 2 & -5 \end{bmatrix} \right) = \begin{bmatrix} 1 & 0 & 1 & -1 \\ 0 & 1 & 3 & -4 \\ 0 & 0 & 0 & 0 \end{bmatrix}$$

Thus \(\left\{ \begin{bmatrix} 8 \\ 0 \\ -7 \end{bmatrix}, \begin{bmatrix} -3 \\ 1 \\ 3 \end{bmatrix} \right\}\) is a basis for the image, and \( \left\{ \begin{bmatrix} -1 & -3 \\ 1 & 0 \end{bmatrix}, \begin{bmatrix} 1 & 4 \\ 0 & 1 \end{bmatrix} \right\} \) is a basis for the kernel.
\end{solution}


\begin{problem}{M1}
Let 
\begin{align*}
A &= \begin{bmatrix} 2 & 3 \\ 0 & 1 \end{bmatrix} & B&= \begin{bmatrix} 3 & 1 & 0 \end{bmatrix} & C&= \begin{bmatrix} 3 & -1 & 4 \\ 1 & 0 & 2 \end{bmatrix}
\end{align*}

Exactly one of the six products $AB$, $AC$, $BA$, $BC$, $CA$, $CB$ can be computed.  Determine which one, and compute it.
\end{problem}
\begin{solution}
$AC$ is the only one that can be computed, and 
$$AC = \begin{bmatrix} 9 & -2 & 14 \\ 1 & 0 & 2 \end{bmatrix}$$
\end{solution}


\begin{problem}{M2}
Determine if the matrix $\begin{bmatrix} 2 & 1 & 0 & 3 \\ 1 & -1 & 3 & 1 \\ 3 & 2 & -1 & 7 \\ 4 & 1 & 2 & 0 \end{bmatrix}$ is invertible.
\end{problem}
\begin{solution}
$$\RREF \begin{bmatrix} 2 & 1 & 0 & 3 \\ 1 & -1 & 3 & 1 \\ 3 & 2 & -1 & 7 \\ 4 & 1 & 2 & 0 \end{bmatrix}=\begin{bmatrix} 1 & 0 & 1 & 0 \\ 0 & 1 & -2 & 0 \\ 0 & 0 & 0 & 1 \\ 0 & 0 & 0 & 0 \end{bmatrix}$$
Since it is not row equivalent to the identity matrix, it is not invertible.
\end{solution}

\begin{problem}{M3}
Compute the inverse of the matrix $\begin{bmatrix} 1 & 2 & 3 & 0 \\ 0 & -1 & 4 & -2 \\ 0 & 0 & 1 & 3 \\ 0 & 0 & 0 & 1 \end{bmatrix}$.
\end{problem}
\begin{solution}
$$\RREF(A|I) = \begin{bmatrix}[cccc|cccc] 1 & 0 & 0 &0 & 1 & 2 & -11 & 37 \\ 0 & 1 & 0 & 0 & 0 & -1 & 4 & -14 \\ 0 & 0 & 1 & 0 & 0 & 0 & 1 & -3 \\ 0 & 0 & 0 & 1 & 0 & 0 & 0 & 1 \end{bmatrix}$$
So the inverse is $\begin{bmatrix}  1 & 2 & -11 & 37 \\ 0 & -1 & 4 & -14 \\  0 & 0 & 1 & -3 \\ 0 & 0 & 0 & 1 \end{bmatrix}$
\end{solution}


\begin{problem}{G1}
Compute the determinant of the matrix $\begin{bmatrix} 3 & -1 & 0 & 4 \\ 2 & 1 & 1& -1 \\ 0 & 1 & 1 & 3 \\ 1 & -2 & 0 & 0 \end{bmatrix}$.
\end{problem}
\begin{solution}
$$\det \begin{bmatrix} 3 & -1 & 0 & 4 \\ 2 & 1 & 1 &-1 \\ 0 & 1 & 1 & 3 \\ 1 & -2 & 0 & 0 \end{bmatrix} = -\det \begin{bmatrix} -1 & 0 & 4 \\ 1 & 1 & -1 \\ 1 & 1 & 3 \end{bmatrix} +(-2) \det \begin{bmatrix} 3 & 0 & 4 \\ 2 & 1 & -1 \\ 0 & 1 & 3 \end{bmatrix} = -1(-4)+(-2)(20) = -36$$
\end{solution}

\begin{problem}{G2}
Compute the eigenvalues, along with their algebraic multiplicities, of the matrix $ \begin{bmatrix} 9 & -3 & 2 \\ 23 & -8 & 5 \\  -1 & 0 & 0 \end{bmatrix}$.
\end{problem}
\begin{solution}
1 (with algebraic multiplicy 2), and -1 (with algebraic multiplicity 1).
\end{solution}

\begin{problem}{G3}
Find the eigenspace associated to the eigenvalue $2$ in the matrix $A=\begin{bmatrix} 0 & 1 & 0 & 0 \\ -4 & 4 & 0 & 0 \\ 11 & -6 & 1 & -1 \\ -9 & 5 & 1 & 3 \end{bmatrix}$.
\end{problem}
\begin{solution}
The eigenspace is spanned by $\begin{bmatrix} -1 \\ -2 \\ 1 \\ 0 \end{bmatrix}$ and $\begin{bmatrix} -1 \\ -2 \\ 0 \\ 1 \end{bmatrix}$.
\end{solution}

\begin{problem}{G4}
Compute the geometric multiplicity of the eigenvalue $2$ in the matrix $A=\begin{bmatrix} 0 & 1 & 0 & 0 \\ -4 & 4 & 0 & 0 \\ 11 & -6 & 1 & -1 \\ -9 & 5 & 1 & 3 \end{bmatrix}$.
\end{problem}
\begin{solution}
The eigenspace is spanned by $\begin{bmatrix} -1 \\ -2 \\ 1 \\ 0 \end{bmatrix}$ and $\begin{bmatrix} -1 \\ -2 \\ 0 \\ 1 \end{bmatrix}$, so the geometric multiplicity is $2$.
\end{solution}

\end{document}