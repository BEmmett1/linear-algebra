\documentclass{sbgLAsemi}

\begin{extract*}
\usepackage{amsmath,amssymb,amsthm,enumerate}
\coursetitle{Math 237}
\courselabel{Linear Algebra}
\calculatorpolicy{You may use a calculator, but you must show all relevant work to receive credit for a standard.}


\newcommand{\IR}{\mathbb{R}}
\newcommand{\IC}{\mathbb{C}}
\renewcommand{\P}{\mathcal{P}}
\renewcommand{\Im}{{\rm Im\ }}
\DeclareMathOperator{\RREF}{RREF}
\DeclareMathOperator{\vspan}{span}

\makeatletter
\renewcommand*\env@matrix[1][*\c@MaxMatrixCols c]{%
  \hskip -\arraycolsep
  \let\@ifnextchar\new@ifnextchar
  \array{#1}}
\makeatother

\title{Semifinal}
\version{3}
\end{extract*}

\begin{document}

\begin{problem}{E1}
Write an augmented matrix corresponding to the following system of linear equations.
\begin{align*}
x_1+4x_3 &= 1 \\
x_2-x_3 &= 7 \\
x_1-x_2+3x_4 &= -1
\end{align*}
\end{problem}
\begin{solution}
\[
\begin{bmatrix}[cccc|c]
1 & 0 & 4 & 0 & 1 \\
0 & 1 & -1 & 0 & 7 \\
1 & -1 & 0 & 3 & -1
\end{bmatrix}
\]
\end{solution}

\begin{problem}{E2}
Find the reduced row echelon form of the matrix below.
$$\begin{bmatrix}[cccc|c] 2 & 1 & -1 & 0 & 5 \\ 3 & -1 & 0 & -2 & 0 \\ -1 & 0 & 5 & 0 & -1 \end{bmatrix}$$
\end{problem}
\begin{solution}
$$\begin{bmatrix}[cccc|c]
 2 & 1 & -1 & 0 & 5 \\
 3 & -1 & 0 & -2 & 0 \\
 -1 & 0 & 5 & 0 & -1
\end{bmatrix} \sim
\begin{bmatrix}[cccc|c]
 -1 & 0 & 5 & 0 & -1  \\
 2 & 1 & -1 & 0 & 5 \\
 3 & -1 & 0 & -2 & 0
\end{bmatrix} \sim
\begin{bmatrix}[cccc|c]
 1 & 0 & -5 & 0 & 1 \\
 2 & 1 & -1 & 0 & 5 \\
 3 & -1 & 0 & -2 & 0
\end{bmatrix} $$
$$\sim
\begin{bmatrix}[cccc|c]
 1 & 0 & -5 & 0 & 1  \\
 0 & 1 & 9 & 0 & 3 \\
 0 & -1 & 15 & -2 & -3 \\
\end{bmatrix} \sim
\begin{bmatrix}[cccc|c]
 1 & 0 & -5 & 0 & 1  \\
 0 & 1 & 9 & 0 & 3 \\
 0 & 0 & 24 & -2 & 0 \\
\end{bmatrix} \sim
\begin{bmatrix}[cccc|c]
 1 & 0 & -5 & 0 & 1 \\
 0 & 1 & 9 & 0 & 3 \\
 0 & 0 & 1 & -\frac{1}{12} & 0 \\
\end{bmatrix} \sim
\begin{bmatrix}[cccc|c]
1 & 0 & 0 & -\frac{5}{12} & 1 \\
 0 & 1 & 0 & \frac{3}{4} & 3 \\
 0 & 0 & 1 & -\frac{1}{12} & 0
\end{bmatrix}$$
\end{solution}

\begin{problem}{E3}
Find the solution set for the following system of linear equations.
\begin{align*}
2x_1+3x_2-5x_3+14x_4 &= 8 \\
x_1+x_2-x_3+5x_4&= 3
\end{align*}
\end{problem}
\begin{solution}
Let \(A =
  \begin{bmatrix}[cccc|c]
    2 & 3 & -5 & 14 & 8 \\
    1 & 1 & -1 & 5 & 3
  \end{bmatrix}
\), so \(\RREF A =
  \begin{bmatrix}[cccc|c]
    1 & 0 & 2 & 1 & 1 \\
    0 & 1 & -3 & 4 & 2 \\
  \end{bmatrix}
\). It follows that the solution set is given by \(
  \begin{bmatrix}
    1 - 2a - b \\
    2 + 3a - 4b \\
    a \\
    b
  \end{bmatrix}
\) for all real numbers \(a,b\).
\end{solution}

\begin{problem}{E4}
Find a basis for the solution set to the homogeneous system of equations
\begin{align*}
4x_1+4x_2+3x_3-6x_4 &= 0 \\
-2x_3-4x_4 &= 0 \\
2x_1+2x_2+x_3-4x_4 &= 0 \\
\end{align*}
\end{problem}
\begin{solution}
Let \(A =
  \begin{bmatrix}[cccc|c]
    4 & 4 & 3 & -6 & 0 \\
    0 & 0 & -2 & -4 & 0 \\
    2 & 2 & 1 & -4 & 0
  \end{bmatrix}
\), so \(\RREF A =
  \begin{bmatrix}[cccc|c]
    1 & 1 & 0 & -3 & 0 \\
    0 & 0 & 1 & 2 & 0 \\
    0 & 0 & 0 & 0 & 0
  \end{bmatrix}
\).
It follows that the basis for the solution set is given by \(\left\{
  \begin{bmatrix}
    -1 \\
    1 \\
    0 \\
    0
  \end{bmatrix},
  \begin{bmatrix}
    3 \\
    0 \\
    -2 \\
    1
  \end{bmatrix}
\right\}\).
\end{solution}

\begin{problem}{V1}
Let $V$ be the set of all pairs of real numbers with the operations, for any $(x_1,y_1), (x_2,y_2) \in V$, $c \in \IR$,
\begin{align*}
(x_1,y_1) \oplus (x_2,y_2) &= (x_1+x_2,y_1+y_2) \\
c \odot (x_1,y_1) &= (c^2x_1, c^3y_1)
\end{align*}
\begin{enumerate}[(a)]
\item Show that scalar multiplication \textbf{distributes scalars} over
      vector addition:
      \(c\odot((x_1,y_1) \oplus (x_2,y_2))=
      c\odot(x_1,y_1) \oplus c\odot(x_2,y_2)\).
\item Determine if $V$ is a vector space or not.  Justify your answer.
\end{enumerate}
\end{problem}
\begin{solution}
Let $(x_1,y_1), (x_2,y_2) \in V$ and let $c \in \IR$.
\begin{align*}
c \odot \left( (x_1,y_1) \oplus (x_2,y_2) \right) &= c \odot (x_1+x_2,y_1+y_2) \\
&= (c^2(x_1+x_2), c^3(y_1+y_2) ) \\
&= (c^2x_1,c^3y_1) \oplus (c^2x_2,c^3y_2) \\
&= c \odot (x_1,y_1) \oplus c \odot (x_2,y_2)
\end{align*}
However, $V$ is not a vector space, as the other distributive law fails:
$$(c+d) \odot (x_1,y_1) = ( (c+d)^2 x_1, (c+d)^3 y_1) \neq ((c^2+d^2)x_1, (c^3+d^3)y_1) = c \odot (x_1,y_1) \oplus d \odot (x_1,y_1).$$
\end{solution}


\begin{problem}{V2}
  Determine if
  \(\begin{bmatrix} 3 \\ -2 \\ 4 \end{bmatrix}\)
  belongs to the span of the set
  \(\left\{
    \begin{bmatrix} 1 \\ 2 \\ -3 \end{bmatrix},
    \begin{bmatrix} 2 \\ 4 \\ -6 \end{bmatrix},
    \begin{bmatrix} 0 \\ 0 \\ 0 \end{bmatrix}
    \right\}
  \).
\end{problem}
\begin{solution}
  Since
  \[
    \RREF\left(
      \begin{bmatrix}[ccc|c]
        1 & 2 & 0 & 3 \\
        2 & 4 & 0 & -2 \\
        -3 & -6 & 0 & 4
      \end{bmatrix}
    \right) =
    \begin{bmatrix}[ccc|c]
      1 & 2 & 0 & 0 \\
      0 & 0 & 0 & 1 \\
      0 & 0 & 0 & 0
    \end{bmatrix}
  \]
  contains the contradiction \(0=1\),
  \(\begin{bmatrix} 3 \\ -2 \\ 4 \end{bmatrix}\) is
  not a linear combination of the three vectors.
\end{solution}
\begin{problem}{V3}
Determine if the vectors  $\begin{bmatrix} -3 \\ 1 \\ 1 \end{bmatrix}$,$\begin{bmatrix} 5 \\ -1 \\ -2 \end{bmatrix}$,$\begin{bmatrix}2 \\ 0 \\ -1 \end{bmatrix}$, and $\begin{bmatrix} 0 \\ 2 \\ -1\end{bmatrix}$ span $\IR^3$
\end{problem}
\begin{solution}
$$\RREF\left(\begin{bmatrix}
-3 & 5 & 2 & 0 \\ 1 & -1 & 0 & 2 \\ 1 & -2 & -1 & -1 \end{bmatrix}\right)=\begin{bmatrix} 1 & 0 & 1 & 5 \\ 0 & 1 & 1 & 3 \\ 0 & 0 & 0 & 0\end{bmatrix}$$
Since the resulting matrix has only two pivot columns, the vectors do not span $\IR^3$.
\end{solution}


\begin{problem}{V4} Let \(W\) be the set of all \(\IR^3\) vectors
\(\begin{bmatrix} x \\ y \\ z \end{bmatrix}\)
satisfying \(x+y+z=1\) (this forms a plane).
Determine if \(W\) is a subspace of \(\IR^3\).
\end{problem}
\begin{solution}
No, because \(\mathbf{0}\) does not belong to \(W\).
\end{solution}


\begin{problem}{S1}
Determine if the set of matrices $\left\{\begin{bmatrix} 3 & -1 \\ 0 & 4 \end{bmatrix}, \begin{bmatrix} 1  & 2 \\ -2 & 1 \end{bmatrix}, \begin{bmatrix} 3 & -8 \\ 6 & 5 \end{bmatrix} \right\}$  is linearly dependent or linearly independent.
\end{problem}
\begin{solution}
$$\RREF\left(\begin{bmatrix} 3 & 1 & 3 \\ -1 & 2 & -8 \\ 0 & -2 & 6 \\ 4 & 1 & 5 \end{bmatrix} \right) = \begin{bmatrix} 1 & 0 & 2 \\ 0 & 1 & -3 \\ 0 & 0 & 0 \\ 0 & 0 & 0 \end{bmatrix}$$
Since the reduced row echelon form has a nonpivot column, the vectors are linearly dependent.
\end{solution}
\begin{problem}{S2}
  Determine if the set \(\left\{
    \begin{bmatrix} 3 & -1 \\ 2 &3 \end{bmatrix},
    \begin{bmatrix} 2 & 0 \\ 2 & 4\end{bmatrix},
    \begin{bmatrix} 1 & 4 \\ -1 & 8\end{bmatrix},
    \begin{bmatrix} -1 & 3 \\ 0 & 4\end{bmatrix}
  \right\}\) is a basis of $\IR^{2\times 2}$.
\end{problem}
\begin{solution}
  \[\RREF\left(
    \begin{bmatrix}
      3 & 2 & 1 & -1\\
      -1 & 0 & 4 & 3\\
      2 & 2 & -1 & 0\\
      3 & 4 & 8 & 4\\
    \end{bmatrix} \right)= \begin{bmatrix}
      1 & 0 & 0 & 0 \\
      0 & 1 & 0 & 0 \\
      0 & 0 & 1 & 0 \\
      0 & 0 & 0 & 1
    \end{bmatrix}
  \]
Since the resulting matrix is the identity matrix, it is a basis.
\end{solution}


\begin{problem}{S3}
Let $W={\rm span}\left( \left\{ \begin{bmatrix} 1 \\ -1 \\ 3 \\ -3 \end{bmatrix},\begin{bmatrix} 2 \\ 0 \\ 1 \\ 1 \end{bmatrix}, \begin{bmatrix} 3 \\ -1 \\ 4 \\ -2 \end{bmatrix},  \begin{bmatrix} 1 \\ 1 \\ 1 \\ -7 \end{bmatrix} \right\}\right)$.  Find a basis of $W$.
\end{problem}
\begin{solution}
$$ \RREF \left( \begin{bmatrix} 1 & 2 & 3 & 1 \\ -1 & 0 & -1 & 1 \\ 3 & 1 & 4 & 1 \\ -3 & 1 & -2 & -7 \end{bmatrix} \right) =  \begin{bmatrix} 1 & 0 & 1 & 0 \\ 0 & 1 & 1 & 0 \\ 0 & 0 & 0 & 1 \\ 0 & 0 & 0 & 0\end{bmatrix}$$
Then  $\left\{ \begin{bmatrix} 1 \\ -1 \\ 3 \\ -3 \end{bmatrix},\begin{bmatrix} 2 \\ 0 \\ 1 \\ 1 \end{bmatrix},   \begin{bmatrix} 1 \\ 1 \\ 1 \\ -7 \end{bmatrix} \right\}$ is a basis for $W$.
\end{solution}


\begin{problem}{S4}
Let $W$ be the subspace of $\IR^{2\times2}$ given by $W={\rm span}\left(\left\{\begin{bmatrix} 2 & 0 \\ -2 & 0 \end{bmatrix}, \begin{bmatrix} 3 & 1 \\ 3 & 6 \end{bmatrix}, \begin{bmatrix} 0 & 0 \\ 1 & 1 \end{bmatrix}, \begin{bmatrix}1 & 2 \\ 0 & 1 \end{bmatrix}\right\}\right)$. Compute the dimension of $W$.
\end{problem}
\begin{solution}
$$\RREF\left( \begin{bmatrix} 2 & 3 & 0 & 1 \\ 0 & 1 & 0 & 2 \\ -2 & 3 & 1 & 0 \\ 0 & 6 & 1 & 1\end{bmatrix} \right) = \begin{bmatrix}1 & 0 & 0 & -\frac{5}{2} \\ 0 & 1 & 0 & 2 \\ 0 & 0 & 1 & -11\\ 0 & 0 & 0 & 0  \end{bmatrix} $$
This has 3 pivot columns so  $\dim(W) =3$.
\end{solution}


\begin{problem}{A1}
Let $T: \IR^3 \rightarrow \IR$ be the linear transformation given by $$T\left(\begin{bmatrix} x_1 \\ x_2 \\ x_3  \end{bmatrix} \right) = \begin{bmatrix} x_2+3x_3 \end{bmatrix}.$$ Write the matrix for $T$ with respect to the standard bases of $\IR^3$ and $\IR$.
\end{problem}
\begin{solution}
$$\begin{bmatrix} 0 & 1 & 3 \end{bmatrix}$$
\end{solution}


\begin{problem}{A2}
Determine if the map $T: \P^6  \rightarrow \P^7$ given by $T(f) = xf(x)-f(1)$ is a linear transformation or not.
\end{problem}

\begin{problem}{A3}
Determine if each of the following linear transformations is injective (one-to-one) and/or surjective (onto).
\begin{enumerate}[(a)]
\item
  \(S: \IR^4 \rightarrow \IR^3\) where
  \(S(\vec e_1)=\begin{bmatrix}
    2  \\
    1 \\
    0
  \end{bmatrix}\),
  \(S(\vec e_2)=\begin{bmatrix}
    1  \\
    2\\
    1
  \end{bmatrix}\),
  \(S(\vec e_3)=\begin{bmatrix}
    0  \\
    -1 \\
    0
  \end{bmatrix}\), and
  \(S(\vec e_4)=\begin{bmatrix}
    3 \\
    2\\
    1
  \end{bmatrix}\),
\item
  \(T: \IR^3 \rightarrow \IR^3\) where
  \(T(\vec e_1)=\begin{bmatrix}
    2  \\
    2 \\
    1\\
  \end{bmatrix}\),
  \(T(\vec e_2)=\begin{bmatrix}
     1 \\
     0 \\
     4 \\
  \end{bmatrix}\), and
  \(T(\vec e_3)=\begin{bmatrix}
     1 \\
     2 \\
     -3 \\
  \end{bmatrix}\).
\end{enumerate}
\end{problem}
\begin{solution}
\begin{enumerate}[(a)]
\item
  \(\RREF\begin{bmatrix}
    2 & 1 & 0 & 3 \\
    1 & 2 & -1 & 2\\
    0 & 1 & 0 & 1
  \end{bmatrix}=\begin{bmatrix}
    1 & 0 & 0 & 1\\
    0 & 1 & 0 & 1\\
    0 & 0 & 1 & 1
  \end{bmatrix}\).
  The map is not injective since it has a column without pivot,
  but it is surjective because every row has a pivot.
\item
  \(\RREF\begin{bmatrix}
    2 & 1 & 1 \\
    2 & 0 & 2 \\
    1 & 4 & -3
  \end{bmatrix}=\begin{bmatrix}
    1 & 0 & 1 \\
    0 & 1 & -1 \\
    0 & 0 & 0
  \end{bmatrix}\).
  The map is not injective since there is a column without a pivot,
  and it is not surjective because there is a row without a pivot.
\end{enumerate}
\end{solution}
\begin{problem}{A4}
Let $T: \IR^3\rightarrow \IR^3$ be the linear transformation given by $$T\left(\begin{bmatrix} x \\ y \\ z \\  \end{bmatrix} \right) = \begin{bmatrix} -3x+y \\ -8x+2y-z \\ 2y+3z \end{bmatrix}$$
Compute a basis for the kernel and a basis for the image of $T$.
\end{problem}
\begin{solution}
Let $A= \begin{bmatrix}-3 & 1 & 0 \\ -8 & 2 & -1 \\ 0 & 2 & 3\end{bmatrix}$, and compute $\RREF(A) = \begin{bmatrix} 1 & 0 & \frac{1}{2} \\ 0 & 1 & \frac{3}{2} \\ 0 & 0 & 0 \end{bmatrix}$.  Then a basis for the image is its columns, $$\left\{ \begin{bmatrix} -3 \\ -8 \\ 0 \end{bmatrix}, \begin{bmatrix} 1 \\ 2 \\ 2 \end{bmatrix} \right\}$$
And the kernel is the solution set of $AX=0$, so a basis would be
$$\left\{\begin{bmatrix} 1 \\ 3 \\ -2 \end{bmatrix} \right\}$$
\end{solution}


\begin{problem}{M1}
Let 
\begin{align*}
A &= \begin{bmatrix} 1 & 3 & -1  \\ 0 & 0 & 7  \end{bmatrix} & B &= \begin{bmatrix} 0 & 1 & 7 & 7 \\ -1 & -2 & 0 & 4 \\ 0 & 0 & 1 & 5 \end{bmatrix} & C&=\begin{bmatrix} 3  \\  1 \end{bmatrix}
\end{align*}
Exactly one of the six products $AB$, $AC$, $BA$, $BC$, $CA$, $CB$ can be computed.  Determine which one, and compute it.
\end{problem}
\begin{solution}
$AB$ is the only ones that can be computed, and 
$$AB = \begin{bmatrix} -3 & -5 & 6 & 14 \\ 0 & 0 & 7 & 35 \end{bmatrix}$$
\end{solution}


\begin{problem}{M2}
Determine if the matrix $\begin{bmatrix}-3 & 1 & 0 \\ -8 & 2 & -1 \\ 0 & 2 & 3\end{bmatrix}$ is invertible.
\end{problem}
\begin{solution}
$$\RREF \begin{bmatrix}-3 & 1 & 0 \\ -8 & 2 & -1 \\ 0 & 2 & 3\end{bmatrix} = \begin{bmatrix} 1 & 0 & \frac{1}{2} \\ 0 & 1 & \frac{3}{2} \\ 0 & 0 & 0 \end{bmatrix}$$
Since it is not equivalent to the identity matrix, it is not invertible.
\end{solution}


\begin{problem}{M3}
  Find the inverse of the matrix
  \(\begin{bmatrix}
    3 & 1 & 3  \\
    2 & -1 & -6  \\
    1 & 1 & 4
  \end{bmatrix}\).
\end{problem}
\begin{solution}
\(\begin{bmatrix}[ccc|ccc]
  3 & 1 & 3 & 1 & 0 & 0 \\
  2 & -1 & -6 & 0 & 1 & 0 \\
  1 & 1 & 4 & 0 & 0 & 1
\end{bmatrix}\sim\begin{bmatrix}[ccc|ccc]
  1 & 0 & 0 & 2 & -1 & -3  \\
  0 & 1 & 0 & -14 & 9 & 24  \\
  0 & 0 & 1 & 3 & -2 & -5
\end{bmatrix}\). Thus the inverse is
\(\begin{bmatrix}
  2 & -1 & -3  \\
  -14 & 9 & 24  \\
  3 & -2 & -5
\end{bmatrix}\).
\end{solution}


\begin{problem}{G1}
Compute the determinant of the matrix $\begin{bmatrix} 3 & -1 & 0 & 4 \\ 2 & 1 & 1& -1 \\ 0 & 1 & 1 & 3 \\ 1 & -2 & 0 & 0 \end{bmatrix}$.
\end{problem}
\begin{solution}
$$\det \begin{bmatrix} 3 & -1 & 0 & 4 \\ 2 & 1 & 1 &-1 \\ 0 & 1 & 1 & 3 \\ 1 & -2 & 0 & 0 \end{bmatrix} = -\det \begin{bmatrix} -1 & 0 & 4 \\ 1 & 1 & -1 \\ 1 & 1 & 3 \end{bmatrix} +(-2) \det \begin{bmatrix} 3 & 0 & 4 \\ 2 & 1 & -1 \\ 0 & 1 & 3 \end{bmatrix} = -1(-4)+(-2)(20) = -36$$
\end{solution}

\begin{problem}{G2} 
Compute the eigenvalues, along with their algebraic multiplicities, of the matrix $ \begin{bmatrix} 9 & -3 & 2 \\ 23 & -8 & 5 \\  2 & -1 & 1 \end{bmatrix}$.
\end{problem}
\begin{solution}
The eigenvalues are $-1$, $1$, and $2$, each with multiplicity 1.
\end{solution}

\begin{problem}{G3}
Compute the eigenspace associated to the eigenvalue $2$ in the matrix $\begin{bmatrix} -1 & 1 & 0 \\ -9 & 5 & 0 \\ 15 & -5 & 2 \end{bmatrix}$.
\end{problem}

\begin{solution}
The eigenspace is the solution space of the system $(B-2I)X=0$.
$$\RREF(B-2I)=\RREF\left(\begin{bmatrix} -3 & 1 & 0 \\ -9 & 3 & 0 \\ 15 & - 5 & 0 \end{bmatrix} \right) = \begin{bmatrix} 1 & -\frac{1}{3} & 0 \\ 0 & 0 & 0 \\ 0 & 0 & 0 \end{bmatrix}$$
So the system simplifies to $x-\frac{y}{3}=0$, or $3x=y$.  Thus the eigenspace is $$E_2 = {\rm span}\left( \left\{ \begin{bmatrix} 1 \\ 3 \\ 0 \end{bmatrix}, \begin{bmatrix} 0 \\ 0 \\ 1\end{bmatrix} \right\} \right)$$
\end{solution}
\begin{problem}{G4}
Compute the geometric multiplicity of the eigenvalue $-1$ in the matrix $\begin{bmatrix} 4 & -2 & -1 \\ 15 & -7 & -3 \\ -5 & 2 & 0 \end{bmatrix}$.  \end{problem}
\begin{solution}
$$\RREF\left(A+I\right) = \begin{bmatrix} 1 & - \frac{2}{5} & -\frac{1}{5} \\ 0 & 0 & 0 \\ 0 & 0 & 0 \end{bmatrix}$$
So the geometric multiplicity is $2$.
\end{solution}


\end{document}