%% 
%% This is file, `2-v-rat.tex',
%% generated with the extract package.
%% 
%% Generated on :  2017/08/31,15:20
%% From source  :  readiness.tex
%% Using options:  active,generate=rat/2-V-RAT.tex,extract-env={readinessAssuranceTest}
%% 
\documentclass{article}
 \usepackage{tbil-la,course-notes}
\usepackage{xstring}
%\let\Xfile\empty
\StrBefore*{\jobname}{-rat}[\Xfile]
\renewcommand{\header}{\expandafter\MakeUppercase\expandafter\Xfile}

\begin{document}

\begin{readinessAssuranceTest}
\setcounter{enumi}{10}
%Correct answer C
\item Simplify the following vector expression.
  \[
  2
  \begin{bmatrix}
    3 \\ -1 \\ 0
  \end{bmatrix}-
  3
  \begin{bmatrix}
    0 \\ 2 \\ 1
  \end{bmatrix}
  \]

\begin{multicols}{4}
\begin{readinessAssuranceTestChoices}
\item \(
        \begin{bmatrix}
          0 \\ 4 \\ -8
        \end{bmatrix}
      \)
\item \(
        \begin{bmatrix}
          3 \\ 2 \\ -5
        \end{bmatrix}
      \)
\item \(
        \begin{bmatrix}
          6 \\ -8 \\ -3
        \end{bmatrix}
      \) % correct
\item \(
        \begin{bmatrix}
          -2 \\ 0 \\ 1
        \end{bmatrix}
      \)
\end{readinessAssuranceTestChoices}
\end{multicols}

%Correct answer C
\item Simplify the following vector expression.
  \[
  2\left(
  \begin{bmatrix}
    1 \\ 1 \\ -1
  \end{bmatrix}+
  \begin{bmatrix}
    -1 \\ 1 \\ -3
  \end{bmatrix}\right)
  \]

\begin{multicols}{4}
\begin{readinessAssuranceTestChoices}
\item \(
        \begin{bmatrix}
          6 \\ -8 \\ -3
        \end{bmatrix}
      \)
\item \(
        \begin{bmatrix}
          3 \\ 2 \\ -5
        \end{bmatrix}
      \)
\item \(
        \begin{bmatrix}
          0 \\ 4 \\ -8
        \end{bmatrix}
      \) % correct
\item \(
        \begin{bmatrix}
          -2 \\ 0 \\ 1
        \end{bmatrix}
      \)
\end{readinessAssuranceTestChoices}
\end{multicols}

%Correct answer A
\item Simplify the complex number expression
      \(-4(3-2i)+2(5+i)\).

\begin{multicols}{4}
\begin{readinessAssuranceTestChoices}
\item \(-2+10i\) % correct
\item \(-1-5i\)
\item \(3-7i\)
\item \(4+i\)
\end{readinessAssuranceTestChoices}
\end{multicols}

%Correct answer C
\item Which of these complex numbers might be represented
      by the following vector
      plotted on the complex plane (where the horizontal axis gives the
      real part and the vertical axis gives the imaginary part)?

      \begin{center}
        \begin{tikzpicture}[scale=0.25]
        \draw[thin,gray,<->] (-5,0) -- (5,0);
        \draw[thin,gray,<->] (0,-5) -- (0,5);
        \draw[thick,blue,->] (0,0) -- (-2,3);
        \end{tikzpicture}
      \end{center}

\begin{multicols}{4}
\begin{readinessAssuranceTestChoices}
\item \(5+i\)
\item \(-3-9i\)
\item \(-2+3i\) % correct
\item \(4i\)
\end{readinessAssuranceTestChoices}
\end{multicols}

%Correct answer A
\item Simplify \(3f(x)-2g(x)\) where
      \(f(x)=7-x^2\) and
      \(g(x)=2x^3+x-1\).

\begin{multicols}{4}
\begin{readinessAssuranceTestChoices}
\item \(-4x^3-3x^2-2x+23\) % correct
\item \(-x^3+19x^2-4\)
\item \(x^3+4x-5\)
\item \(3x^3+5x^2-3x+17\)
\end{readinessAssuranceTestChoices}
\end{multicols}

%Correct answer C
\item Express the following system of linear equations as an augmented matrix.
\begin{alignat*}{5}
  x_1 &\,+\,& 2x_2 &\, \,&     &\,-\,&  x_4 &\,=\,& 3 \\
      &\, \,&      &\, \,& x_3 &\,+\,& 4x_4 &\,=\,& -2
\end{alignat*}

\begin{multicols}{4}
\begin{readinessAssuranceTestChoices}
\item \(
      \begin{bmatrix}[c|c]
        1 & 0 \\
        2 & 0 \\
        0 & 1 \\
        -1 & 4 \\
        -2 & 3
      \end{bmatrix}
    \)
    \item \(
          \begin{bmatrix}[c|c]
            1 & 2 \\
            0 & -1 \\
            3 & 0 \\
            0 & 1 \\
            4 & -2
          \end{bmatrix}
        \)
\item \(
      \begin{bmatrix}[cccc|c]
        1 & 2 & 0 & -1 & 3 \\
        0 & 0 & 1 & 4 & -2
      \end{bmatrix}
    \) % correct
\vspace{0.15in}
\item \(
      \begin{bmatrix}[cccc|c]
        1 & 2 & 1 & 4 & 3 \\
        -2 & 1 & 3 & 4 & 5
      \end{bmatrix}
    \)
\end{readinessAssuranceTestChoices}
\end{multicols}

%Correct answer B
\item Which of the following matrices is equivalent to the following matrix?
  \[
    \begin{bmatrix}[cc|c]
      1 & 2 & 3 \\
      0 & 4 & -1 \\
      2 & 3 & 2 \\
    \end{bmatrix}
  \]
  (Hint: The correct answer was obtained from a single row operation.)

  \begin{multicols}{4}
  \begin{readinessAssuranceTestChoices}
  \item
    \(
      \begin{bmatrix}[cc|c]
        1 & 2 & 3 \\
        0 & 4 & -1 \\
        0 & 0 & 1 \\
      \end{bmatrix}
    \)
\item
    \(
      \begin{bmatrix}[cc|c]
        1 & 2 & 3 \\
        0 & 4 & -1 \\
        0 & -1 & -4 \\
      \end{bmatrix}
    \) % correct
  \item
    \(
      \begin{bmatrix}[cc|c]
        1 & 2 & 3 \\
        1 & 3 & 4 \\
        2 & 3 & 2 \\
      \end{bmatrix}
    \)
  \item
    \(
      \begin{bmatrix}[cc|c]
        1 & 2 & 3 \\
        0 & 1 & 1 \\
        2 & 3 & 2 \\
      \end{bmatrix}
    \)

  \end{readinessAssuranceTestChoices}
  \end{multicols}

%Correct answer B
\item Find
  \(\RREF
    \begin{bmatrix}[cc|c]
      1 & 2 & 3 \\
      0 & 4 & -1 \\
      2 & 3 & 2 \\
    \end{bmatrix}
  \).

  \begin{multicols}{4}
  \begin{readinessAssuranceTestChoices}
  \item
    \(
      \begin{bmatrix}[cc|c]
        1 & 0 & 3 \\
        0 & 1 & -1 \\
        0 & 0 & 0 \\
      \end{bmatrix}
    \)
\item
    \(
      \begin{bmatrix}[cc|c]
        1 & 0 & 0 \\
        0 & 1 & 0 \\
        0 & 0 & 1 \\
      \end{bmatrix}
    \) % correct
  \item
    \(
      \begin{bmatrix}[cc|c]
        1 & 2 & 3 \\
        1 & 3 & 4 \\
        0 & 0 & 0 \\
      \end{bmatrix}
    \)
  \item
    \(
      \begin{bmatrix}[cc|c]
        1 & 2 & 3 \\
        0 & 1 & 1 \\
        0 & 0 & 0 \\
      \end{bmatrix}
    \)

  \end{readinessAssuranceTestChoices}
  \end{multicols}

%Correct answer B
  \item Solve the following system of linear equations.
  \begin{alignat*}{4}
    2x_1 &\,+\,& x_2 &\,+\,& 4x_3 &\,=\,& 0 \\
     x_1 &\,+\,& x_2 &\,+\,&  x_3 &\,=\,& 1 \\
   -3x_1 &\,+\,& 4x_2 &\,+\,& x_3 &\,=\,& -7 \\
  \end{alignat*}

  \begin{multicols}{2}
  \begin{readinessAssuranceTestChoices}
  \item \(
          \begin{bmatrix}
            x_1 \\ x_2 \\ x_3
          \end{bmatrix}=
          \begin{bmatrix}
            1 \\ -6 \\ 1
          \end{bmatrix}
        \)
  \item \(
          \begin{bmatrix}
            x_1 \\ x_2 \\ x_3
          \end{bmatrix}=
          \begin{bmatrix}
            2 \\ 0 \\ -1
          \end{bmatrix}
        \) % correct
  \item \(
          \begin{bmatrix}
            x_1 \\ x_2 \\ x_3
          \end{bmatrix}=
          \begin{bmatrix}
            1 \\ -2 \\ 1
          \end{bmatrix}+
          a
          \begin{bmatrix}
            1 \\ 0 \\ 3
          \end{bmatrix}
        \) for all real numbers \(a\)
  \item No solutions
  \end{readinessAssuranceTestChoices}
  \end{multicols}

%Correct answer C
  \item Solve the following system of linear equations.
  \begin{alignat*}{4}
    2x_1 &\,+\,& x_2 &\,+\,& 4x_3 &\,=\,& 0 \\
     x_1 &\,+\,& x_2 &\,+\,&  x_3 &\,=\,& 0 \\
  \end{alignat*}

  \begin{multicols}{2}
  \begin{readinessAssuranceTestChoices}
  \item \(
          \begin{bmatrix}
            x_1 \\ x_2 \\ x_3
          \end{bmatrix}=
          \begin{bmatrix}
            4 \\ 1 \\ -5
          \end{bmatrix}
        \)
  \item \(
          \begin{bmatrix}
            x_1 \\ x_2 \\ x_3
          \end{bmatrix}=
          \begin{bmatrix}
            1 \\ 2 \\ -1
          \end{bmatrix}
        \)
  \item \(
          \begin{bmatrix}
            x_1 \\ x_2 \\ x_3
          \end{bmatrix}=
          a
          \begin{bmatrix}
            -3 \\ 2 \\ 1
          \end{bmatrix}
        \) for all real numbers \(a\) % correct
  \item No solutions
  \end{readinessAssuranceTestChoices}
  \end{multicols}



\end{readinessAssuranceTest}

\end{document}
