%% 
%% This is file, `4-a-rat.tex',
%% generated with the extract package.
%% 
%% Generated on :  2017/09/01,12:46
%% From source  :  readiness.tex
%% Using options:  active,generate=rat/4-A-RAT.tex,extract-env={readinessAssuranceTest}
%% 
\documentclass{article}
 \usepackage{tbil-la,course-notes}
\usepackage{xstring}
%\let\Xfile\empty
\StrBefore*{\jobname}{-rat}[\Xfile]
\renewcommand{\header}{\expandafter\MakeUppercase\expandafter\Xfile}

\begin{document}

\begin{readinessAssuranceTest}
\setcounter{enumi}{30}
\item Which of the following is a solution to the system of linear equations
      \begin{align*}
      x+3y-z    &=   2 \\
      2x+8y+3z  &=  -1 \\
      -x-y+9z   &= -10
      \end{align*}

\begin{multicols}{4}
\begin{readinessAssuranceTestChoices}
\item $\begin{bmatrix} 1 \\ 1 \\ 0 \end{bmatrix}$
\item $\begin{bmatrix} 0 \\ 1 \\ -1 \end{bmatrix}$
\item $\begin{bmatrix} 1 \\ 0 \\ -1 \end{bmatrix}$
\item $\begin{bmatrix} 1 \\ -1 \\ 1 \end{bmatrix}$
\end{readinessAssuranceTestChoices}
\end{multicols}


\item Find a basis for the solution set of the following homogeneous system of
      linear equations
      \begin{align*}
      x+2y+-z-w    &= 0 \\
      -2x-4y+3z+5w &= 0
      \end{align*}

\begin{multicols}{4}
\begin{readinessAssuranceTestChoices}
\item $\left\{ \begin{bmatrix} 2 \\ 1 \\ 0 \\ 0 \end{bmatrix}, \begin{bmatrix} 2 \\ 0 \\ 3 \\ 1 \end{bmatrix} \right\}$
\item $\left\{ \begin{bmatrix} 2 \\ 2 \\ 0 \\ 0 \end{bmatrix}, \begin{bmatrix} 0 \\ 0 \\ 3 \\ 0 \end{bmatrix} \right\}$
\item $\left\{ \begin{bmatrix} 2 \\ 1 \\ 3 \\ 1 \end{bmatrix} \right\}$
\item None of these are a basis.
\end{readinessAssuranceTestChoices}
\end{multicols}


\item Determine which property applies to the set of vectors $$\left\{ \begin{bmatrix}  1 \\ 0 \\ 0 \end{bmatrix}, \begin{bmatrix} 0 \\ 1 \\ 0 \end{bmatrix} \right\} \subset \IR^3.$$

\begin{readinessAssuranceTestChoices}
\item It does not span and is linearly dependent
\item It does not span and is linearly independent
\item It spans but it is linearly dependent
\item It is a basis of $\IR^3$.
\end{readinessAssuranceTestChoices}


\item Determine which property applies to the set of vectors $$\left\{ \begin{bmatrix}  1 \\ 0 \\ 0 \end{bmatrix}, \begin{bmatrix} 2 \\ 1 \\ 0 \end{bmatrix} , \begin{bmatrix} 1 \\ 1 \\ 3 \end{bmatrix} \right\}\subset \IR^3.$$

\begin{readinessAssuranceTestChoices}
\item It does not span and is linearly dependent
\item It does not span and is linearly independent
\item It spans but it is linearly dependent
\item It is a basis of $\IR^3$.
\end{readinessAssuranceTestChoices}


\item Determine which property applies to the set of vectors $$\left\{ \begin{bmatrix}  1 \\ 0 \\ 0 \end{bmatrix}, \begin{bmatrix} -2 \\ 0 \\ -2 \end{bmatrix} , \begin{bmatrix} 1 \\ 1 \\ 0 \end{bmatrix} , \begin{bmatrix} 3 \\ 3 \\ -3 \end{bmatrix}\right\}\subset \IR^3.$$

\begin{readinessAssuranceTestChoices}
\item It does not span and is linearly dependent
\item It does not span and is linearly independent
\item It spans but it is linearly dependent
\item It is a basis of $\IR^3$.
\end{readinessAssuranceTestChoices}


\item Determine which property applies to the set of vectors $$\left\{ \begin{bmatrix}  2 \\ 2 \\ -1 \end{bmatrix}, \begin{bmatrix} -3 \\ 1 \\ -2 \end{bmatrix} , \begin{bmatrix} 1 \\ 5 \\ -4 \end{bmatrix}\right\}\subset \IR^3.$$

\begin{readinessAssuranceTestChoices}
\item It does not span and is linearly dependent
\item It does not span and is linearly independent
\item It spans but it is linearly dependent
\item It is a basis of $\IR^3$.
\end{readinessAssuranceTestChoices}


\item Find a basis for the subspace of $\IR^4$ spanned by the vectors ...


\item Suppose you know that every vector in $\IR^5$ can be written as a linear combination of the vectors $\{\vec{v}_1, \ldots, \vec{v}_n\}$.  What can you conclude about $n$?

\begin{readinessAssuranceTestChoices}
\item $n \leq 5$
\item $n=5$
\item $n \geq 5$
\item $n$ could be any positive integer
\end{readinessAssuranceTestChoices}

\item Suppose you know that every vector in $\IR^5$ can be written uniquely as a linear combination of the vectors $\{\vec{v}_1, \ldots, \vec{v}_n\}$.  What can you conclude about $n$?

\begin{readinessAssuranceTestChoices}
\item $n \leq 5$
\item $n=5$
\item $n \geq 5$
\item $n$ could be any positive integer
\end{readinessAssuranceTestChoices}

\item Suppose you know that every vector in $\IR^5$ can be written uniquely as a linear combination of the vectors $\{\vec{v}_1, \ldots, \vec{v}_n\}$.  What can you conclude about the set $\{\vec{v}_1, \ldots, \vec{v}_n\}$?

\begin{readinessAssuranceTestChoices}
\item It does not span and is linearly dependent
\item It does not span and is linearly independent
\item It spans but it is linearly dependent
\item It is a basis of $\IR^3$.
\end{readinessAssuranceTestChoices}

\end{readinessAssuranceTest}

\end{document}
