%% 
%% This is file, `5-m-rat.tex',
%% generated with the extract package.
%% 
%% Generated on :  2017/10/30,10:06
%% From source  :  readiness.tex
%% Using options:  active,generate=rat/5-M-RAT.tex,extract-env={readinessAssuranceTest}
%% 
\documentclass{article}
 \usepackage{tbil-la,course-notes}
\usepackage{xstring}
%\let\Xfile\empty
\StrBefore*{\jobname}{-rat}[\Xfile]
\renewcommand{\header}{\expandafter\MakeUppercase\expandafter\Xfile}

\begin{document}

\begin{readinessAssuranceTest}
\setcounter{enumi}{40}
%B
\item Suppose $f(x)$ and $g(x)$ are real-valued functions satisfying
\begin{align*}
f(2) &= 4 & g(2) & = 4 \\
f(3) &= 5 & g(3) &= 5 \\
f(4) &= 3 & g(4) &= 2
\end{align*}
Compute $(f \circ g)(2)$.
\begin{multicols}{4}
\begin{readinessAssuranceTestChoices}
\item $2$
\item $3$ %Correct
\item $4$
\item $5$
\end{readinessAssuranceTestChoices}
\end{multicols}

%B
\item Let $f(x) = x^2-2$ and $g(x)= x^2+1$.  Compute the composition function $(f \circ g)(x)$.
\begin{multicols}{4}
\begin{readinessAssuranceTestChoices}
\item $x^2-1$
\item $x^4+2x^2-1$ %Correct
\item $x^4-4x^2+5$
\item $x^4-x^2-2$
\end{readinessAssuranceTestChoices}
\end{multicols}



%C
\item Solve the system of linear equations
\begin{align*}
x+3y &= -2 \\
2x-7y &= 9
\end{align*}

\begin{multicols}{4}
\begin{readinessAssuranceTestChoices}
\item $\begin{bmatrix} -2 \\ 9 \end{bmatrix}$
\item $\begin{bmatrix} 2 \\ 3 \end{bmatrix}$
\item $\begin{bmatrix} 1 \\ -1 \end{bmatrix}$ %Correct
\item $\begin{bmatrix} 3 \\ 1 \end{bmatrix}$
\end{readinessAssuranceTestChoices}
\end{multicols}

%B
\item Let $a, b, c$ be fixed real numbers.  How many solutions does the system of linear equations below have?
\begin{align*}
x+2y+3z &= a \\
y-z &= b \\
y+z &= c
\end{align*}

\begin{multicols}{4}
\begin{readinessAssuranceTestChoices}
\item $0$
\item $1$ %Correct
\item Infinitely many
\item It depends on the values of $a$, $b$, and $c$.
\end{readinessAssuranceTestChoices}
\end{multicols}


%C
\item What is the standard matrix corresponding to the linear transformation $T: \IR^3 \rightarrow \IR^3$ given by $T\left( \begin{bmatrix} x \\ y \\ z \end{bmatrix}\right) = \begin{bmatrix} x+2y-z \\ y+3z \\x+7y \end{bmatrix}$?
\begin{multicols}{4}
\begin{readinessAssuranceTestChoices}
\item $\begin{bmatrix} 1 & 2 & -1 \\ 1 & 3 & 0 \\ 1 & 7 & 0 \end{bmatrix}$
\item $\begin{bmatrix} 1 & 1 & 1 \\ 2 & 3 & 7 \\ -1 & 0 & 0 \end{bmatrix}$
\item $\begin{bmatrix} 1 & 2 & -1 \\ 0 & 1 & 3 \\ 1 & 7 & 0 \end{bmatrix}$ %Correct
\item $\begin{bmatrix}  1 & 0 & 1 \\ 2 & 1 & 7 \\ -1 & 3 & 0 \end{bmatrix}$
\end{readinessAssuranceTestChoices}
\end{multicols}

\newpage
%A
\item Let $T: \IR^2 \rightarrow \IR^3$ be the linear transformation with standard matrix $A=\begin{bmatrix} 2 & 3 \\ -1 & -1 \\ 0 & 4 \end{bmatrix}$.  Compute $T\left(\begin{bmatrix} 2 \\ -1 \end{bmatrix}\right)$.
\begin{multicols}{4}
\begin{readinessAssuranceTestChoices}
\item $\begin{bmatrix} 1 \\ -1 \\ -4 \end{bmatrix}$ %Correct
\item $\begin{bmatrix} 2 \\ -1 \\ 0 \end{bmatrix}$
\item $\begin{bmatrix} 5 \\ 7 \\ 4\end{bmatrix}$
\item $\begin{bmatrix} 4 \\ -1 \\ 8 \end{bmatrix}$
\end{readinessAssuranceTestChoices}
\end{multicols}


%D
\item Which of the following is true of the linear transformation $T: \IR^3 \rightarrow \IR^3$ given by $$T\left(\begin{bmatrix} x \\ y \\ z \end{bmatrix} \right) = \begin{bmatrix} x+3y-4z \\ x+y \\ 3z \end{bmatrix}?$$
\begin{readinessAssuranceTestChoices}
\item $T$ is neither injective nor surjective
\item $T$ is injective but not surjective
\item $T$ is surjective but not injective
\item $T$ is both injective and surjective %Correct
\end{readinessAssuranceTestChoices}


%A
\item Which of the following is true of the linear transformation $T:\IR^3 \rightarrow \IR^2$ given by $$T\left(\begin{bmatrix} x \\ y \\ z \end{bmatrix} \right) = \begin{bmatrix} x-y \\ x+z \end{bmatrix} ?$$
\begin{readinessAssuranceTestChoices}
\item $T$ is surjective but not injective %Correct
\item $T$ is injective but not surjective
\item $T$ is both injective and surjective
\item $T$ is neither injective nor surjective
\end{readinessAssuranceTestChoices}


%C
\item Let $T: \IR^n \rightarrow \IR^m$ be a linear transformation with standard matrix $A$.  Which of the following is {\bf not} a characterization of the statement ``$T$ is injective''?
\begin{readinessAssuranceTestChoices}
\item If $T(\vec{v})=T(\vec{w})$ for some $\vec{v}, \vec{w} \in \IR^n$, then $\vec{v}=\vec{w}$.
\item The columns of $A$ are linearly independent
\item $T$ has a non-trivial kernel %Correct
\item $\RREF(A)$ has only pivot columns
\end{readinessAssuranceTestChoices}


%B
\item Let $T: \IR^n \rightarrow \IR^m$ be a linear transformation with standard matrix $A$.  Which of the following is {\bf not} a characterization of the statement ``$T$ is surjective''?
\begin{readinessAssuranceTestChoices}
\item $\RREF(A)$ has a pivot in every row
\item $\RREF(A)$ has has a pivot in every column %Correct
\item $\Im T = \IR^m$
\item The columns of $A$ span $\IR^m$
\end{readinessAssuranceTestChoices}


\end{readinessAssuranceTest}

\end{document}
