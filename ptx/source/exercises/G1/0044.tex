
\begin{exerciseStatement}


Let \(A\) be a \(4 \times 4\) matrix with determinant \( -2 \).


\begin{enumerate}[(a)]
\item Let \(Q\) be the matrix obtained from \(A\) by applying the row operation \( R_1 \leftrightarrow R_4 \). What is \(\operatorname{det}\ Q\)?
\item Let \(M\) be the matrix obtained from \(A\) by applying the row operation \( R_3 \to R_3 + -3R_4 \). What is \(\operatorname{det}\ M\)?
\item Let \(B\) be the matrix obtained from \(A\) by applying the row operation \( R_1 \to -3R_1 \). What is \(\operatorname{det}\ B\)?
\end{enumerate}
    
\end{exerciseStatement}
    
\begin{exerciseAnswer} 

\begin{enumerate}[(a)]
\item \(\operatorname{det}\ Q= 2 \)
\item \(\operatorname{det}\ M= -2 \)
\item \(\operatorname{det}\ B= 6 \)
\end{enumerate}
    
\end{exerciseAnswer}
    
