
\begin{exerciseStatement}


Consider the following maps of polynomials \(S:\mathcal{P}\rightarrow\mathcal{P}\) and \(T:\mathcal{P}\rightarrow\mathcal{P}\) defined by 
\begin{align*} S(g(x))= -4 \, g\left(x\right)^{3} + 3 \, g'\left(2\right)  & \text{and} & T(g)= -2 \, g\left(-1\right) - 4 \, g'\left(x\right) . \\ \end{align*}
             Explain why one these maps is a linear transformation and why the other map is not. 


\end{exerciseStatement}
    
\begin{exerciseAnswer} 


\(S\) is not linear and \(T\) is linear.


\end{exerciseAnswer}
    
