
\begin{exerciseStatement}
 Let \(T:\mathbb{R}^ 3  \to \mathbb{R}^ 4 \) be the linear transformation given by \[T\left(  \left[\begin{array}{c}
x \\
y \\
z
\end{array}\right]  \right) =  \left[\begin{array}{c}
x + z \\
y + 2 \, z \\
-2 \, x - 3 \, y - 8 \, z \\
3 \, y + 6 \, z
\end{array}\right] .\]
\begin{enumerate}[(a)]
\item Explain how to find the image of \(T\) and the kernel of \(T\).
\item Explain how to find a basis of the image of \(T\) and a basis of the kernel of \(T\).
\item Explain how to find the rank and nullity of \(T\), and why the rank-nullity theorem holds for \(T\).
\end{enumerate}
    
\end{exerciseStatement}
    
\begin{exerciseAnswer} 


\[\operatorname{RREF} \left[\begin{array}{ccc}
1 & 0 & 1 \\
0 & 1 & 2 \\
-2 & -3 & -8 \\
0 & 3 & 6
\end{array}\right] = \left[\begin{array}{ccc}
1 & 0 & 1 \\
0 & 1 & 2 \\
0 & 0 & 0 \\
0 & 0 & 0
\end{array}\right] \]


\begin{enumerate}[(a)]
\item \[\operatorname{Im}\ T = \operatorname{span}\  \left\{ \left[\begin{array}{c}
1 \\
0 \\
-2 \\
0
\end{array}\right] , \left[\begin{array}{c}
0 \\
1 \\
-3 \\
3
\end{array}\right] \right\} \]\[\operatorname{ker}\ T =  \left\{ \left[\begin{array}{c}
-a \\
-2 \, a \\
a
\end{array}\right] \middle|\,a\in\mathbb{R}\right\} \]
\item  A basis of \(\operatorname{Im}\ T\) is \( \left\{ \left[\begin{array}{c}
1 \\
0 \\
-2 \\
0
\end{array}\right] , \left[\begin{array}{c}
0 \\
1 \\
-3 \\
3
\end{array}\right] \right\} \). A basis of \(\operatorname{ker}\ T\) is \( \left\{ \left[\begin{array}{c}
-1 \\
-2 \\
1
\end{array}\right] \right\} \)
\item  The rank of \(T\) is \( 2 \), the nullity of \(T\) is \( 1 \), and the dimension of the domain of \(T\) is \( 3 \). The rank-nullity theorem asserts that \( 2 + 1 = 3 \), which we see to be true. 
\end{enumerate}
    
\end{exerciseAnswer}
    
