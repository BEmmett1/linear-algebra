
\begin{exerciseStatement}


Consider the statement 
\begin{center}\begin{minipage}{0.8\textwidth}
 The set of vectors \( \left\{ \left[\begin{array}{c}
2 \\
-4 \\
-3 \\
1
\end{array}\right] , \left[\begin{array}{c}
4 \\
-1 \\
-3 \\
-1
\end{array}\right] , \left[\begin{array}{c}
0 \\
-2 \\
1 \\
2
\end{array}\right] , \left[\begin{array}{c}
-5 \\
3 \\
-5 \\
-1
\end{array}\right] , \left[\begin{array}{c}
-3 \\
-5 \\
-2 \\
-4
\end{array}\right] , \left[\begin{array}{c}
3 \\
-1 \\
-5 \\
0
\end{array}\right] \right\} \)spans \(\mathbb{R}^4\). 
\end{minipage}\end{center}
    


\begin{enumerate}[(a)]
\item  Write an equivalent statement using a vector equation.
\item  Explain why your statement is true or false.
\end{enumerate}
    
\end{exerciseStatement}
    
\begin{exerciseAnswer} 


\[\operatorname{RREF} \left[\begin{array}{cccccc}
2 & 4 & 0 & -5 & -3 & 3 \\
-4 & -1 & -2 & 3 & -5 & -1 \\
-3 & -3 & 1 & -5 & -2 & -5 \\
1 & -1 & 2 & -1 & -4 & 0
\end{array}\right] = \left[\begin{array}{cccccc}
1 & 0 & 0 & 0 & \frac{633}{113} & -\frac{43}{113} \\
0 & 1 & 0 & 0 & -\frac{600}{113} & \frac{180}{113} \\
0 & 0 & 1 & 0 & -\frac{922}{113} & \frac{141}{113} \\
0 & 0 & 0 & 1 & -\frac{159}{113} & \frac{59}{113}
\end{array}\right] \]


\begin{enumerate}[(a)]
\item The statement 
\begin{center}\begin{minipage}{0.8\textwidth}
 The set of vectors \( \left\{ \left[\begin{array}{c}
2 \\
-4 \\
-3 \\
1
\end{array}\right] , \left[\begin{array}{c}
4 \\
-1 \\
-3 \\
-1
\end{array}\right] , \left[\begin{array}{c}
0 \\
-2 \\
1 \\
2
\end{array}\right] , \left[\begin{array}{c}
-5 \\
3 \\
-5 \\
-1
\end{array}\right] , \left[\begin{array}{c}
-3 \\
-5 \\
-2 \\
-4
\end{array}\right] , \left[\begin{array}{c}
3 \\
-1 \\
-5 \\
0
\end{array}\right] \right\} \) spans \(\mathbb{R}^4\). 
\end{minipage}\end{center}
     is equivalent to the statement 
\begin{center}\begin{minipage}{0.8\textwidth}
 The vector equation \( x_{1} \left[\begin{array}{c}
2 \\
-4 \\
-3 \\
1
\end{array}\right] + x_{2} \left[\begin{array}{c}
4 \\
-1 \\
-3 \\
-1
\end{array}\right] + x_{3} \left[\begin{array}{c}
0 \\
-2 \\
1 \\
2
\end{array}\right] + x_{4} \left[\begin{array}{c}
-5 \\
3 \\
-5 \\
-1
\end{array}\right] + x_{5} \left[\begin{array}{c}
-3 \\
-5 \\
-2 \\
-4
\end{array}\right] + x_{6} \left[\begin{array}{c}
3 \\
-1 \\
-5 \\
0
\end{array}\right] =\vec{v}\) has a solution for every vector \(\vec{v}\) in \(\mathbb{R}^4\). 
\end{minipage}\end{center}
    
\item  The set of vectors \( \left\{ \left[\begin{array}{c}
2 \\
-4 \\
-3 \\
1
\end{array}\right] , \left[\begin{array}{c}
4 \\
-1 \\
-3 \\
-1
\end{array}\right] , \left[\begin{array}{c}
0 \\
-2 \\
1 \\
2
\end{array}\right] , \left[\begin{array}{c}
-5 \\
3 \\
-5 \\
-1
\end{array}\right] , \left[\begin{array}{c}
-3 \\
-5 \\
-2 \\
-4
\end{array}\right] , \left[\begin{array}{c}
3 \\
-1 \\
-5 \\
0
\end{array}\right] \right\} \) spans \(\mathbb{R}^4\). 
\end{enumerate}
    
\end{exerciseAnswer}
    
