\begin{module}{V}{Vector Spaces}
 
\moduleQuestion{What is a vector space?}
\begin{moduleStandards}
  \standard{Vector property verification}{VecPrp}{
    show why an example satisfies a given vector space property, but does not satisfy another given property.
  }
  \standard{Vector space identification}{VecId}{
    list the eight defining properties of a vector space, infer which of these properties a given example satisfies, and thus determine if the example is a vector space.
  }
  \standard{Linear combinations}{LinCmb}{
    determine if a Euclidean vector can be written as a linear combination of a given set of Euclidean vectors.
  }
  \standard{Spanning sets}{Span}{
    determine if a set of Euclidean vectors spans \(\IR^n\).
  }
  \standard{Subspaces}{Subsp}{
    determine if a subset of \(\IR^n\) is a subspace or not.
  }
\end{moduleStandards}

\begin{readinessAssuranceOutcomes}
\item Use set builder notation to describe sets of vectors.
\item Add Euclidean vectors and multiply Euclidean vectors by scalars.
\item Add polynomials and multiply polynomials by scalars.
\item Perform basic manipulations of augmented matrices and linear
systems \listOfStandards{E1,E2,E3}.
\end{readinessAssuranceOutcomes}

\begin{readinessAssuranceResources}
\item \url{https://www.khanacademy.org/math/algebra2/manipulating-functions/funciton-composition/v/function-composition}
\end{readinessAssuranceResources}



\CatchFileDef{\activities}{modules/2-V/activities.tex}

\newModuleSection
\begin{goals}
Today we will explore what properties are shared by \(\IR^1\) and \(\IR^2\) with an eye towards generalizing to higher dimensions.
\end{goals}
\includeActivities{1}{7}{\activities}
\begin{summary}
Today we came up with a set of properties shared by all \(\IR^n\) and used these to define \term{vector spaces} (Standard V1). Next class, we wll practice determining  when other sets are or are not vector spaces.
\end{summary}

\newModuleSection
\begin{goals}
Today we will practice showing when a set with given operations is or is not a vector space (Standard V1).
\end{goals}
\includeActivities{8}{12}{\activities}
\begin{summary}
Today we practiced showing when a set with given operations is or is not a vector space (Standard V1).
\end{summary}

\newModuleSection
\begin{goals}
Today we will begin exploring the notions of \term{linear combinations} and \term{span}.
\end{goals}
\includeActivities{13}{17}{\activities}
\begin{summary}
Today we begin exploring \term{linear combinations} and \term{span}.
\vfill
Next class, we will learn how to check if a vector is a linear combination of a set of vectors or not (Standard V2).
\end{summary}

\newModuleSection
\begin{goals}
Today we will learn how to check if a vector is a linear combination of a set of vectors or not (Standard V2).
\end{goals}
\includeActivities{18}{26}{\activities}
\begin{summary}
Today we learned how to check if a vector is a linear combination of a set of vectors or not (Standard V2).
\end{summary}

\newModuleSection
\begin{goals}
Today we will learn how to determine when a set of vectors spans the entire vector space (Standard V3).
\end{goals}
\includeActivities{27}{36}{\activities}
\begin{summary}
Today we learned how to determine when a set of vectors spans the entire vector space (Standard V3).
\end{summary}

\newModuleSection
\begin{goals}
Today we will learn about \term{subspaces} (Standard V4)
\end{goals}
\includeActivities{37}{44}{\activities}
\begin{summary}
Today we learned how to determine if a set of vectors forms a  {subspace} of \(\IR^n\) (Standard V4).
\end{summary}

\newModuleSection
\begin{goals}
Today we will learn about an important concept called \term{linear independence} (Standard V5).
\end{goals}
\includeActivities{45}{53}{\activities}
\begin{summary}
Today we learned how to determine if a set of vectors is \term{linearly independent} or \term{linearly dependent} (Standard V5).
\end{summary}

\newModuleSection
\begin{goals}
Today we will learn about a \term{basis} of a vector space (Standard V6).
\end{goals}
\includeActivities{54}{59}{\activities}
\begin{summary}
Today we learned how to determine if a set of vectors form a \term{basis} of a vector space (Standard V6).
\end{summary}

\newModuleSection
\begin{goals}
Today we will learn how to find a basis of a subspace of \(\IR^n\).
\end{goals}
\includeActivities{62}{67}{\activities}
\begin{summary}
Today we learned how to find a basis of a subspace of \(\IR^n\).
\end{summary}

\newModuleSection
\begin{goals}
Today we will learn how to find the dimension of a subspace of \(\IR^n\).
\end{goals}
\includeActivities{68}{82}{\activities}
\begin{summary}
Today we learned how to find the dimension of a subspace of \(\IR^n\). We also started learning how to find a basis of the solution space of a \term{homogeneous} system of equations (Standard V9).
\end{summary}

\newModuleSection
\begin{goals}
Today we will practice finding a basis of the solution space of a \term{homogeneous} system of equations (Standard V9).  This will come up again later when we find the \term{kernel} of a \term{linear transformation}.
\end{goals}
\includeActivities{83}{86}{\activities}
\begin{summary}
Today we learned how to find a basis of the solution space of a \term{homogeneous} system of equations (Standard V9).
\end{summary}


\end{module}
