\usetheme{Hannover}
\usecolortheme{rose}

%Center frame titles
\setbeamertemplate{frametitle}[default][center]

\title{Welcome to Linear Algebra}
\author{\prof}

\date{\ifbool{TR}{August 17, 2017}{August 16,2017}}


\begin{document}

\begin{frame}
\titlepage
\end{frame}

\begin{frame} \frametitle{What is Linear Algebra? }
Linear algebra is the study of {\bf linear maps}.
\begin{itemize}
\item In Calculus, you learn how to approximate any function by a linear function.
\item In Linear Algebra, we learn about how linear maps behave.
\item Combining the two, we can approximate how any function behaves.
\end{itemize}
\end{frame}

\begin{frame} \frametitle{What is Linear Algebra good for?}
\begin{itemize}
\item In an abstract sense, linear algebra is arguably the most used tool in higher math.
\item In computer graphics, linear algebra is used to help represent 3-dimensional objects in a two dimensional grid of pixels.
\item Differential equations are often very difficult (or impossible) to solve exactly; we use linear algebra to understand approximate solutions in a vast number of engineering applications such as fluid flows, vibrations, heat transfer, etc.
\item Google's famed Page Rank algorithm is based on linear algebra
\end{itemize}
\end{frame}

\begin{frame} \frametitle{Learning Outcomes }
By the end of this class, you will be able to
\begin{itemize}
\item Solve systems of linear equations.
\pause \item Determine whether or not a set with given operations is a vector space or a subspace of another vector space.
\pause \item Determine properties of sets of vectors such as whether they are linearly independent, whether they span, and whether they are a basis.
\pause \item Perform fundamental operations in the algebra of matrices, including multiplying and inverting matrices.
\pause \item Use and apply algebraic properties of a linear transformation.
\pause \item Determine geometric information about a linear transformation, including computing determinants, eigenvalues, and eigenvectors.
\end{itemize}
\end{frame}

\section{Standards Based Grading (SBG)}
\begin{frame}\frametitle{Standards Based Grading}
Your main job in this course is to \textbf{master the covered material}
and \textbf{demonstrate that mastery to me}.

\vspace{0.2in}
\pause

You will be given several opportunities to demonstrate mastery throughout
the semester, and if
at first you don't succeed, you can try again without any penalty.
\end{frame}

\begin{frame}\frametitle{SBG}
The course material is broken down into 23 learning \textbf{standards}.
\begin{itemize}
\item Each attempted exercise will be simply marked according to whether or not
      your solution demonstrates mastery of the relevant standards.
\item Each solution that demonstrates complete mastery counts as a
      \textbf{checkmark} for that standard.
\item Up to two checkmarks may be earned for each standard. Your grade depends
      on the total number of checkmarks you earn this semester (up to 46).
\item Standards will be assessed several times, and there's no penalty for
      incorrect solutions. So, if you don't succeed the first time,
      keep studying and try again!
\end{itemize}
\end{frame}

\begin{frame}\frametitle{Assessment Opportunities}
Checkmarks may be earned as follows.
\begin{itemize}
\item {\bf Quizzes}: Each day at the end of class we will have a quiz. This
      is how you should earn most of your checkmarks.
\item {\bf Midterm}: There will be a single midterm exam the week of Fall Break
      to give you the chance to catch up on missed standards.
\item {\bf Final Exam}: Your final opportunity to demonstrate mastery,
      cumulative over the entire course.
\item {\bf Out-of-class Reassessments}: A limited number of oppotunities
      will be provided to earn checkmarks outside of class.
\end{itemize}

\pause

\vspace{0.2in}

The assessment method (quiz/exam/etc.) you used to earn a checkmark
isn't important: I only care that you
learn the material and demonstrate that mastery to me before the end of the
semester!
\end{frame}

\begin{frame}\frametitle{Interpreting Feedback}
On each assessment, for each standard you will receive one of the following marks.
\begin{itemize}
\item \masteryMark{} means you demonstrated \textbf{Mastery} of that standard.
      Great job!  Check off another box on your progress sheet.
\item \minorMark{} means you have a minor mistake, but if you can correct it,
      this mark will be changed to \masteryMark{}.
\item \reattemptMark{} means you made a good faith effort and demonstrated
      partial understanding, but not complete mastery. You are eligible to
      \textbf{Reattempt} the standard outside of class.
\item \noMark{} means there was \textbf{No Significant Evidence} of understanding.
\end{itemize}

\pause

\vspace{0.2in}

Marks other than \masteryMark{} do not improve your course letter grade, but
they don't hurt you either.
\end{frame}

\begin{frame}\frametitle{Course Grades}

\begin{tabular}{llll}
A & \begin{minipage}{0.6\textwidth}
\begin{itemize}
\item Earn 40 mastery checkmarks.
\item Complete 10 homework reports.
\item \ifbool{TBL}{Have a 90\% Class Participation Score.}{Have an 80\% attendence record.} \\
\end{itemize}
\vspace{0.1em}\end{minipage} & OR & 45 \checkmark{}s
\\\hline
B & \begin{minipage}{0.6\textwidth}~\vspace{0.1em}

\begin{itemize}
\item Earn 35 mastery checkmarks.
\item Complete 8 homework reports.
\item \ifbool{TBL}{Have a 80\% Class Participation Score.}{Have an 80\% attendence record.} \\
\end{itemize}
\vspace{0.1em}\end{minipage} & OR & 40 \checkmark{}s
\\\hline
C 	& \begin{minipage}{0.6\textwidth}~\vspace{0.1em}

\begin{itemize}
\item Earn 30 mastery checkmarks.
\item Complete 6 homework reports.
\item \ifbool{TBL}{Have a 70\% Class Participation Score.}{Have an 80\% attendence record.} \\
\end{itemize}
\vspace{0.1em}\end{minipage} & OR & 35 \checkmark{}s
% \\\hline
% D 	& \begin{minipage}{0.6\textwidth}~\vspace{0.1em}
%
% \begin{itemize}
% \item Earn 20 mastery checkmarks.
% \item Complete 4 homework reports.
% \item \ifbool{TBL}{Have a 50\% Class Participation Score.}{Have an 50\% attendence record.} \\
% \end{itemize}
% \vspace{0.1em}\end{minipage} & OR & 25 \checkmark{}s
\\

\end{tabular}

\end{frame}

\begin{frame}\frametitle{Homework}
Homework is practice.
\begin{itemize}
\item I will not collect or grade homework problems.
\item A list of suggested exercise for practice is in USAOnline, sorted by standard.  You should work as many or as few of these as you need to master the material.
\item \ifbool{TR}{\textbf{Caveat discipulus}:}{\textbf{Heads up!}} Most students do not work as many homework exercises as they should.
\item If you need help or feedback, come to my office hours.
\item I will collect homework reports each week (blank form in USAOnline).
\end{itemize}
\end{frame}

\ifbool{TBL}{
  \section{Team-Based Learning (TBL)}

  \begin{frame}\frametitle{Team-Based Learning}
  In this class we will use {\bf Team-Based Learning}.
  \begin{itemize}
  \item The course is divided into six modules, each lasting about 2 weeks.
  \item At the beginning of each module is the {\bf Readiness Assurance Process}.  The first day of the module will consist of individual and team Readiness Assurance Tests
  \item The next 3-4 class days will consist of guided activities with you working in your team.
  \item Research in other STEM disciplines show that TBL leads to improved student learning.
  \end{itemize}
  \end{frame}

  \begin{frame}\frametitle{Readiness Assurance Process}
  \begin{itemize}
  \item In USAOnline, you will find a list of the skills you should have {\bf before each module starts}, along with a list of resources to help you prepare.
  \begin{itemize}
  \item Sometimes these skills are from previous courses.
  \item Sometimes these skills are standards from earlier in this course.
  \end{itemize}
  \pause \item On the first day of the module, the Readiness Assurance Tests will ensure you have these skills.
  \begin{itemize}
  \item First, you will individually take the RAT
  \item After everyone is done, you will take the RAT again collaboratively as a team.
  \end{itemize}
  \item {\bf The first Readiness Assurance day is \ifbool{TR}{Tuesday}{Monday}!}
  \end{itemize}
  \end{frame}

  \begin{frame}\frametitle{Teams}
  Stand up.  Line up in alphabetical order by last name, with A at the front left of the room.
  \vspace{4in}
  \end{frame}

  \begin{frame} \frametitle{What makes a good team member?}
  Create a list of criteria that make an effective team member.
  \vspace{4in}
  \end{frame}

  \begin{frame} \frametitle{Peer Evaluation Questions}
  Create a list of questions your team thinks should be on the peer evaluation surveys.  Answers to the questions should be on a scale from 1 to 5.
  \vspace{4in}
  \end{frame}

  \begin{frame} \frametitle{Class Participation Score}
  There will be four components to your participation score

  \begin{center}
  \begin{tabular}{l|l}
  \hline
  iRAT (individual) & \phantom{xxxx}\%   \\ \hline
  tRAT (team) & \phantom{xxxx}\%   \\ \hline
  Peer Evaluation & \phantom{xxxx}\%   \\ \hline
  Attendence & \phantom{xxxx}\%   \\ \hline
  \end{tabular}
  \end{center}


  In your teams, decide what percentage each of the four components should have.  They should add to 100\%.
  \end{frame}
}{
  \section{Lectures}
  \begin{frame} \frametitle{Class structure}
  \begin{itemize}
  \item The course is divided into six modules, each lasting about 2 weeks.
  \item At the beginning of each module is one review day working through
        examples from previous courses and standards.
  \item The next 3-4 class days will consist of lectures covering several
        exercises/activities that will help you master the course standards.
  \end{itemize}
  \end{frame}
}


\ifbool{TR}{
\begin{frame}\frametitle{Office Hours}
Choose up 3 one-hour long periods your team would like me to have an office hour during.  Rank them in order of your preference.\\
\ \\

I have the following constraints:
\begin{itemize}
\item They must be during business hours, i.e. 8-5.
\item I teach another class from 12:30-1:45 on TR
\item I have departmental meetings/seminars 3:30-5 on Thursdays
\end{itemize}
\end{frame}
}{}



\end{document}
