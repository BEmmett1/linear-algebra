\documentclass{article}
\usepackage{enumerate,amssymb,tikz, tikz-cd,amsmath,amsthm,multicol,hyperref,environ,etoolbox,graphicx,catchfile}

\usepackage[left=1in,right=1in,top=0.5in,bottom=0.7in,includeheadfoot]{geometry}
\usepackage{fancyhdr}

%%%Page layout setup%%%
\newcommand{\header}{Linear Algebra}
\newcommand{\setHeader}[1]{\renewcommand{\header}{#1}}

\newcommand{\moduleLetter}{X}

\pagestyle{fancy}
\fancyhf{}
\chead{\header}
\rfoot{Page \thepage}
\renewcommand{\headrulewidth}{1pt}
\renewcommand{\footrulewidth}{1pt}


\NewEnviron{module}[2]{
  \renewcommand{\moduleLetter}{#1}
  \setHeader{Module \moduleLetter{}: #2}
  \BODY
}
\NewEnviron{readinessAssuranceTest}{
  \newpage
  \subsection*{Readiness Check}
  Choose the most appropriate response for each question.
  \begin{enumerate}[1)]
    \BODY
  \end{enumerate}
}


\NewEnviron{readinessAssuranceTestChoices}{
  \begin{enumerate}[(a)]
    \BODY
  \end{enumerate}
}

%%%New Commands/Math operators%%%
\newcommand{\IR}{\mathbb{R}}
\newcommand{\IC}{\mathbb{C}}
\renewcommand{\P}{\mathcal{P}}
\renewcommand{\Im}{\operatorname{Im}}
\newcommand{\RREF}{\operatorname{RREF}}
\newcommand{\vspan}{\operatorname{span}}
\newcommand{\setList}[1]{\left\{#1\right\}}
\newcommand{\setBuilder}[2]{\left\{#1\,\middle|\,#2\right\}}
%This code allows for augmented matrices, e.g. by \begin{bmatrix}[cc|c] ...
\makeatletter
\renewcommand*\env@matrix[1][*\c@MaxMatrixCols c]{%
  \hskip -\arraycolsep
  \let\@ifnextchar\new@ifnextchar
  \array{#1}}
\makeatother


\begin{document}
\begin{module}{MX}{}
\begin{readinessAssuranceTest}
\setcounter{enumi}{30}

%B011
%A C B A C B D C B D


%A
\item Suppose $f(x)$ and $g(x)$ are real-valued functions satisfying
\begin{align*}
f(2) &= 4 & g(2) & = 4 \\
f(3) &= 5 & g(3) &= 5 \\
f(4) &= 3 & g(4) &= 2
\end{align*}
Compute $(f \circ g)(2)$.
\begin{multicols}{4}
\begin{readinessAssuranceTestChoices}
\item $3$ %Correct
\item $4$
\item $5$
\item $6$
\end{readinessAssuranceTestChoices}
\end{multicols}

%C
\item Let $f(x) = x^2-2$ and $g(x)= x^2+1$.  Compute the composition function $(f \circ g)(x)$.
\begin{multicols}{4}
\begin{readinessAssuranceTestChoices}
\item $x^2-1$
\item $x^4-4x^2+5$
\item $x^4+2x^2-1$ %Correct
\item $x^4-x^2-2$
\end{readinessAssuranceTestChoices}
\end{multicols}



%B
\item What is the standard matrix corresponding to the linear transformation $T: \IR^3 \rightarrow \IR^3$ given by $T\left( \begin{bmatrix} x \\ y \\ z \end{bmatrix}\right) = \begin{bmatrix} x+2y-z \\ y+3z \\x+7y \end{bmatrix}$?
\begin{multicols}{4}
\begin{readinessAssuranceTestChoices}
\item $\begin{bmatrix} 1 & 2 & -1 \\ 1 & 3 & 0 \\ 1 & 7 & 0 \end{bmatrix}$
\item $\begin{bmatrix} 1 & 2 & -1 \\ 0 & 1 & 3 \\ 1 & 7 & 0 \end{bmatrix}$ %Correct
\item $\begin{bmatrix}  1 & 0 & 1 \\ 2 & 1 & 7 \\ -1 & 3 & 0 \end{bmatrix}$
\item $\begin{bmatrix} 1 & 1 & 1 \\ 2 & 3 & 7 \\ -1 & 0 & 0 \end{bmatrix}$
\end{readinessAssuranceTestChoices}
\end{multicols}

%A
\item Let \(T: \IR^3 \rightarrow \IR^2\) be the linear map corresponding to the standard matrix \(\begin{bmatrix} 2 & 1 & -1 \\ 0 & 1 & 3  \end{bmatrix} \).  Compute \(T\left(\begin{bmatrix} 1 \\ -1 \\ 3 \end{bmatrix} \right)\).
\begin{multicols}{4}
\begin{readinessAssuranceTestChoices}
\item $\begin{bmatrix}  -2 \\ 8 \end{bmatrix}$ %Correct
\item $\begin{bmatrix} -2 \\ -8  \end{bmatrix}$
\item $\begin{bmatrix} 2 \\ -4 \end{bmatrix}$
\item $\begin{bmatrix}  2 \\ 4 \end{bmatrix}$ 
\end{readinessAssuranceTestChoices}
\end{multicols}

%C
\item Let $T: \IR^2 \rightarrow \IR^3$ be the linear transformation corresponding to the standard matrix $A=\begin{bmatrix} 2 & 3 \\ -1 & -1 \\ 0 & 4 \end{bmatrix}$.  Compute $T\left(\begin{bmatrix} 2 \\ -1 \end{bmatrix}\right)$.
\begin{multicols}{4}
\begin{readinessAssuranceTestChoices}
\item $\begin{bmatrix} 2 \\ -1 \\ 0 \end{bmatrix}$
\item $\begin{bmatrix} 5 \\ 7 \\ 4\end{bmatrix}$
\item $\begin{bmatrix} 1 \\ -1 \\ -4 \end{bmatrix}$ %Correct
\item $\begin{bmatrix} 4 \\ -1 \\ 8 \end{bmatrix}$
\end{readinessAssuranceTestChoices}
\end{multicols}

\newpage
%B
\item Let \(T: \IR^4 \rightarrow \IR^2\) be the linear transformation corresponding to the standard matrix \(\begin{bmatrix} 3 & -1 & 0 & 2 \\ -2 & -4 & -1 & 1 \end{bmatrix} \).  What are the domain and codomain of \(T\)?
\begin{readinessAssuranceTestChoices}
\item The domain is \(\IR^2\) and the codomain is \(\IR^4\)
\item The domain is \(\IR^4\) and the codomain is \(\IR^2\) %correct
\item The domain and codomain are both \(\IR^4\)
\item The domain and codomain are both \(\IR^2\)
\end{readinessAssuranceTestChoices}

%D
\item Which of the following is true of the linear transformation $T: \IR^3 \rightarrow \IR^3$ given by $$T\left(\begin{bmatrix} x \\ y \\ z \end{bmatrix} \right) = \begin{bmatrix} x+3y-4z \\ x+y \\ 3z \end{bmatrix}?$$
\begin{readinessAssuranceTestChoices}
\item $T$ is neither injective nor surjective
\item $T$ is injective but not surjective
\item $T$ is surjective but not injective
\item $T$ is both injective and surjective %Correct
\end{readinessAssuranceTestChoices}


%C
\item Which of the following is true of the linear transformation $T:\IR^3 \rightarrow \IR^2$ given by $$T\left(\begin{bmatrix} x \\ y \\ z \end{bmatrix} \right) = \begin{bmatrix} x-y \\ x+z \end{bmatrix} ?$$
\begin{readinessAssuranceTestChoices}
\item $T$ is injective but not surjective
\item $T$ is both injective and surjective
\item $T$ is surjective but not injective %Correct
\item $T$ is neither injective nor surjective
\end{readinessAssuranceTestChoices}


%B
\item Let $T: \IR^n \rightarrow \IR^m$ be a linear transformation with standard matrix $A$.  Which of the following is {\bf not} a characterization of the statement ``$T$ is injective''?
\begin{readinessAssuranceTestChoices}
\item If $T(\vec{v})=T(\vec{w})$ for some $\vec{v}, \vec{w} \in \IR^n$, then $\vec{v}=\vec{w}$.
\item $T$ has a non-trivial kernel, i.e. \(\ker T \neq \left\{ \vec{0}\right\}\) %Correct
\item The columns of $A$ are linearly independent
\item $\RREF(A)$ has a pivot in every column
\end{readinessAssuranceTestChoices}


%D
\item Let $T: \IR^n \rightarrow \IR^m$ be a linear transformation with standard matrix $A$.  Which of the following is {\bf not} a characterization of the statement ``$T$ is surjective''?
\begin{readinessAssuranceTestChoices}
\item $\RREF(A)$ has a pivot in every row
\item $\Im T = \IR^m$
\item The columns of $A$ span $\IR^m$
\item $\RREF(A)$ has a pivot in every column %Correct
\end{readinessAssuranceTestChoices}





\end{readinessAssuranceTest}
\end{module}
\end{document}
