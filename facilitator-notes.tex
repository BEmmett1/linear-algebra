\documentclass{article}
\usepackage[top=1in,bottom=1in,right=1in,left=1in]{geometry}
\usepackage{enumerate,multicol}

\begin{document}

\begin{center}
Facilitator Notes
\end{center}

\subsection*{Introduction}

These notes are for a one semester, 3-hour sophomore-level Linear Algebra course.  Teams should have at minimum 5 people to ensure a sufficiently sized team factoring in absences.  Each team should be provided with a vertical, ideally non-permanent surface for their work, such as a whiteboard or chalkboard.  A vertical surface encourages collaboration and discussion, and discourages individual work on paper.  These notes have been field-tested at a large regional public university in classes with 25-35 students, with each ``part'' designed to fill one 50-60 minute class period.  

The course consists of 6 modules, listed below; here they are optimized for 29 75-minute class days with 60 minutes each day devoted to application activity, and the remaining 15 minutes for assessment.  

\begin{enumerate}
\item [1-E] Systems of Equations
\item [2-V] Vector Spaces
\item [3-S] Structure of Vector Spaces
\item [4-A] Algebraic Properties of Linear Maps
\item [5-M] Algebraic Structure of Matrices
\item [6-G] Geometric Properties of Linear Maps
\end{enumerate}

\subsection*{Module 6-G: Geometric properties of linear maps}

This module explores the geometric properties of linear transformations, focusing on determinants, eigenvalues, and eigenvectors.  The readiness assurance outcomes include some basic high school geometry and algebra, as well as some outcomes of earlier modules concerning the algebraic properties of linear transformations.  The first two days are devoted to determinants, while the next two explore eigenvalues and eigenvectors.  The last day focuses on on a popular application of eigenvectors, namely Google's PageRank algorithm.

RAT answers (IF-AT card D012):
\begin{multicols}{10}
\begin{enumerate}[1)]
\item A
\item A
\item D
\item C
\item A
\item D
\item B
\item B
\item A
\item A
\end{enumerate}
\end{multicols}

\subsubsection*{Part 1}
The goal of this day is to get students to understand that a determinant is measuring how area/volume changes under a linear transformation.  Activities 25.1 through 25.6 require the students to make a specific construction of a graph and the resulting area transformation.  Students sometimes have a little trouble withe first of these, but then make short work of the remainder, requiring little class wide discussion.  Activity 25.10 invites the first incorrect responses after simultaneous reporting; the instructor is advised to emphasize the multiplicative nature of the stretching factor here.  Activity 25.11 can be made a little more concrete by distributing parallelograms and triangles cut out of card stock that invite the students to see the equality geometrically.  Activity 25.16 sometimes invites teams to settle on the incorrect answer choice of 7; once again, the instructor is advised to emphasize that the stretching factors (determinants) are multiplicative.

\subsubsection*{Part 2}
The goal of this day is to help students understand how to actually compute a determinant.  Activity 26.4 gives one straightforward but tedious method to motivate the search for better techniques.  Activity 26.6 begins an inquiry towards Laplace expansion.  Activities 26.8 through 26.11 are carefully scaffolded to guide students toward understanding how linearity of the determinant leads to Laplace expansion.  In practice, students will find a mixture of row operations and Laplace expansion the most efficient way to actually perform the computations.

\subsubsection*{Part 3}
The focus now shifts towards eigenvalues and eigenvectors.  Activity 27.8 can be a little tedious for students still learning determinants.  Activities 27.9 and 27.11 should be done one part at a time, levelling the class between parts.

\subsubsection*{Part 4}
This day is oriented towards an understanding of geometric multiplicity.  Activity 28.2 is a quick observation about complex eigenvalues (a situation that arises frequently in computer graphics applications).   Activity 28.5 and 28.6 give two examples with the same characteristic polynomial, but different geometric multiplicities for the eigenvalue $2$.  Activity 28.10 is reflective of how they might use these skills in subsequent courses, taking full advantage of technology to perform the tedious calculations.  Depending on their comfort level, students may need to be guided through how to use their preferred technology to compute determinants and solve systems of equations.

\subsubsection*{Part 5}
This sequence of activities is ideal for the last day of the semester.  The goal is to use what students have learned about eigenvalues and eigenvectors to explain the basics behind Google's PageRank algorithm (the \$700,000,000,000 references Google's market cap as of this writing).  In activity 29.1, students should use whatever reasoning they like to decide on a ranking; many will choose 4 or 7 as the most important.  Activity 29.5 emphasizes how this is an application of what has been learned the last two class days.  Activity 29.6 asks them to compute a very simple example, while activity 29.8 returns to the example from 29.1, showing that webpage 2 is actually most important by this metric.

\end{document}
